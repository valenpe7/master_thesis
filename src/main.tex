\documentclass[12pt, twoside, a4paper, openright]{report}

%packages-----------------------------------------------------------------------

\usepackage[utf8]{inputenc}
\usepackage[english]{babel}
\usepackage[T1]{fontenc}
\usepackage{lmodern}
\usepackage{array}
%\usepackage{times}
%\usepackage{charter}
\usepackage{graphicx}
\usepackage{color}
\usepackage{amsmath}
\usepackage{amssymb}
\usepackage{amsthm}
\usepackage{verbatim}
\usepackage{pdfpages}
\usepackage[unicode]{hyperref}
\usepackage[small, bf]{caption}
\usepackage{enumerate}
\usepackage{epstopdf}
\usepackage{epsfig}
\usepackage{setspace}
\usepackage{esdiff}
\usepackage{enumitem}
\usepackage[inner=2.7cm, outer=1.8cm]{geometry}
\usepackage{emptypage}
\usepackage{caption}
%\usepackage{subcaption}
\usepackage{listings}
\usepackage{color}
\usepackage{multicol}
\usepackage{multirow}
\usepackage{lipsum}
\usepackage{mathabx}
\usepackage{floatrow}
\usepackage{courier}
\usepackage{cite}
\usepackage[position=top]{subfig}

%fancystyles--------------------------------------------------------------------

\usepackage{fancyhdr}

\fancypagestyle{myfancy}{

\fancyhead[LE]{\nouppercase{\leftmark}}
\fancyhead[LO]{}
\fancyhead[CO]{}
\fancyhead[CE]{}
\fancyhead[RE]{}
\fancyhead[RO]{\nouppercase{\rightmark}}

\fancyfoot[LE]{\thepage}
\fancyfoot[LO]{}
\fancyfoot[CO]{}
\fancyfoot[CE]{}
\fancyfoot[RE]{}
\fancyfoot[RO]{\thepage}

\renewcommand{\headrulewidth}{0.4pt}
\renewcommand{\footrulewidth}{0.0pt}
}

\fancypagestyle{plain}{

\fancyhead[LE]{}
\fancyhead[LO]{}
\fancyhead[CO]{}
\fancyhead[CE]{}
\fancyhead[RE]{}
\fancyhead[RO]{}

\fancyfoot[LE]{\thepage}
\fancyfoot[LO]{}
\fancyfoot[CO]{}
\fancyfoot[CE]{}
\fancyfoot[RE]{}
\fancyfoot[RO]{\thepage}

\renewcommand{\headrulewidth}{0.0pt}
\renewcommand{\footrulewidth}{0.0pt}
}

%tikz---------------------------------------------------------------------------

\usepackage{tikz}
\usetikzlibrary{shapes, arrows}

%url----------------------------------------------------------------------------

\usepackage{url}
\DeclareUrlCommand\url{\def\UrlLeft{<}\def\UrlRight{>} \urlstyle{tt}}

%color--------------------------------------------------------------------------

\definecolor{darkred}{rgb}{0.6,0,0}
\definecolor{darkgreen}{rgb}{0,0.6,0}
\definecolor{darkblue}{rgb}{0,0,0.6}
\definecolor{darkgrey}{rgb}{0.3,0.3,0.3}
\definecolor{grey}{rgb}{0.6,0.6,0.6}
\definecolor{lightgrey}{rgb}{0.95,0.95,0.95}
\definecolor{lightred}{rgb}{0.99,0.85,0.85}
\definecolor{violet}{rgb}{0.65,0.45,0.75}

%listings-----------------------------------------------------------------------

\definecolor{mygreen}{rgb}{0,0.6,0}
\definecolor{mygray}{rgb}{0.5,0.5,0.5}
\definecolor{light}{rgb}{0.96, 0.96, 0.96}
\definecolor{mymauve}{rgb}{0.58,0,0.82}

\lstdefinestyle{CXX} {
	language=C++,
	backgroundcolor=\color{light},
	basicstyle=\scriptsize\ttfamily,
	breakatwhitespace=false,
	breaklines=true,
	captionpos=t,
	%commentstyle=\color{mygreen},
	deletekeywords={},
	escapeinside={\%*}{*)},
	extendedchars=true,
	frame=single,
	keepspaces=true,
	keywordstyle=\color{blue},
	otherkeywords={},
	numbers=left,
	numbersep=5pt,
	numberstyle=\tiny\color{mygray},
	rulecolor=\color{black},
	showspaces=false,
	showstringspaces=false, 
	showtabs=false,
	stepnumber=1,
	stringstyle=\color{mymauve},
	tabsize=3,
	title=\lstname   
}

\lstdefinestyle{FORTRAN} {
	language=[90]Fortran,
	backgroundcolor=\color{light},
	basicstyle=\scriptsize\ttfamily,
	keywordstyle=\color{blue},
	%commentstyle=\color{mygreen},
	breakatwhitespace=false,
	breaklines=true,
	captionpos=t,
	deletekeywords={},
	escapeinside={\%*}{*)},
	extendedchars=true,
	frame=single,
	keepspaces=true,
	otherkeywords={},
	numbers=left,
	numbersep=5pt,
	numberstyle=\tiny\color{mygray},
	rulecolor=\color{black},
	showspaces=false,
	showstringspaces=false, 
	showtabs=false,
	stepnumber=1,
	stringstyle=\color{mymauve},
	tabsize=3,
	title=\lstname 
}

%settings-----------------------------------------------------------------------

\setlength{\parindent}{20pt}
\setlength{\parskip}{0pt}
\renewcommand{\baselinestretch}{1.0}
\pagenumbering{arabic}
\frenchspacing

\DeclareFontFamily{U}{mathx}{\hyphenchar\font45}
\DeclareFontShape{U}{mathx}{m}{n}{<-> mathx10}{}
\DeclareSymbolFont{mathx}{U}{mathx}{m}{n}
\DeclareMathAccent{\widebar}{0}{mathx}{"73}

%commands----------------------------------------------------------------------

\newcommand{\ctu}{Czech Technical University in Prague }
\newcommand{\fnspe}{Faculty of Nuclear Sciences and Physical Engineering }
\newcommand{\dpe}{Department of Physical Electronics }
\newcommand{\branch}{Computational Physics }
\newcommand{\projecttitle}{Tight-focusing of short intense laser pulses in particle-in-cell simulations of laser-plasma~interaction }
\newcommand{\projecttitlecz}{Zaostření krátkého intenzivního laserového impulsu do velmi malého ohniska v částicových simulacích interakce s plazmatem }
\newcommand{\valenta}{Bc. Petr Valenta }
\newcommand{\klimo}{doc. Ing. Ondřej Klimo, Ph.D. }
\newcommand{\weber}{Dr. Stefan Andreas Weber }
\newcommand{\academicyear}{2016/2017 }
\newcommand{\keywords}{tight-focusing, PIC simulations, laser-plasma interaction, plasma optics}
\newcommand{\keywordscz}{fokuzace, PIC simulace, laser-plazma interakce, optika}

%macros-------------------------------------------------------------------------

\newcommand{\nucl}[3]{
\ensuremath{
\phantom{
\ensuremath{^{#1}_{#2}}}
\llap{\ensuremath{^{\rule{0pt}{0pt}#1}}}
\llap{\ensuremath{_{\rule{0pt}{7pt}#2}}}
\mbox{#3}}}

\newcommand{\norm}[1]{\lVert#1\rVert}
\newcommand{\abs}[1]{\lvert#1\rvert}

\renewcommand{\vec}[1]{\mathbf{#1}}

\newcommand{\rot}[1]{\nabla \times #1}
\newcommand{\grad}[1]{\nabla #1}
\renewcommand{\div}[1]{\nabla \cdot #1}
\newcommand{\laplace}[1]{\Delta #1}
\newcommand{\dalembert}[1]{\Box #1}

\newcommand{\e}[0]{\mathrm{e}}
\renewcommand{\i}[0]{\mathrm{i}}
\renewcommand{\d}[0]{\mathrm{d}}

%changecountering---------------------------------------------------------------

\usepackage{chngcntr}
%\counterwithout{equation}{chapter}
\counterwithout{figure}{chapter}
\counterwithout{table}{chapter}

%document-----------------------------------------------------------------------

\begin{document}

\pagestyle{empty}

%1------------------------------------------------------------------------------

%\mbox{}
%\newpage

\begin{titlepage}

\begin{center}
{\Large \bf \ctu}\\[4mm]
{\Large \bf \fnspe}\\[4mm]
{\Large \bf \dpe}\\[16mm]
\epsfysize=45mm \epsffile{img/logo/ctu.pdf}\\[12mm]
\begin{spacing}{2.0}
{\LARGE \bf \projecttitle}\\[2mm]
\end{spacing}
{\Large (Master's thesis)} \\
\vfill
\end{center}

\begin{flushleft}
\begin{tabular}{rl}
Author: & \valenta \\[1mm]
Supervisor: & \klimo \\[1mm]
Consultant: & \weber \\[1mm]
Academic year: & \academicyear \\[1mm]
\end{tabular}
\end{flushleft}

\end{titlepage}

%2------------------------------------------------------------------------------

\newpage
\mbox{}

%3------------------------------------------------------------------------------

\newpage
\includepdf[pages=1]{dat/guidelines.pdf}

%4------------------------------------------------------------------------------

\newpage
\includepdf[pages=2]{dat/guidelines.pdf}

%5------------------------------------------------------------------------------

\newpage
\null
\vfill
{\bf \noindent Prohlášení/Declaration} \\[5mm]
Prohlašuji, že jsem předloženou práci vypracoval samostatně a že jsem uvedl veškerou
použitou literaturu.\\[2mm]
I hereby declare that I carried out this work independently, and only with the cited sources, literature and other professional sources.\\
\vspace{5mm}V Praze dne/In Prague on .............................\hfill
\begin{tabular}{c}
........................................\\
\valenta
\end{tabular}

%6------------------------------------------------------------------------------

\newpage
\thispagestyle{empty}
\mbox{}

%7------------------------------------------------------------------------------

\newpage
\begin{flushleft}
	\renewcommand{\arraystretch}{1.3}
	\begin{tabular}{r p{12cm}}
		Název práce:
		~ & \bf \projecttitlecz \\
		Autor:
		~ & \valenta \\
		Druh práce:
		~ & Diplomová práce \\
		Studijní program:
		~ & (N3913) Aplikace přírodních věd \\
		Obor:
		~ & (3901T065) Informatická fyzika \\
		Vedoucí práce:
		~ & \klimo \newline Katedra fyzikální elektroniky, Fakulta jaderná a fyzikálně inženýrská, České vysoké učení technické v Praze \\
		Konzultant:
		~ & \weber \newline Projekt ELI-Beamlines, Fyzikální ústav Akademie věd České republiky, v. v. i. \\
	\end{tabular}
\end{flushleft}

\begin{center}
\textbf{Abstrakt}\\
\end{center}

Úzká fokuzace s využitím plazmové optiky může vést ke zvýšení intenzity a zlepšení časového i prostorového kontrastu laserových svazků. Vzhledem k tomu, že pro popis těchto impulsů neplatí paraxiální aproximace, je zapotřebí použít vhodnější model. V rámci této práce byly do částicového kódu implementovány a důkladně otestovány nové okrajové podmínky pro výpočet časového vývoje laserového impulsu na hranici simulační obasti v souladu s Maxwellovými rovnicemi. Upravený kód byl použit pro simulace laserových svazků zaostřených do velmi malého ohniska. Výsledky simulací byly analyzovány z hlediska vlivu velikosti ohniska na průběh interakce laserových svazků s pevnými terči. Ukazuje se, že trajektorie horkých elektronů a absorpční procesy během interakce jsou silně ovlivněny příčnou složkou ponderomotorické síly, která je velmi vysoká v případě ohniska menšího než je vlnová délka laseru. V tomoto případě ostře narůstá účinnost absorpce laserové energie v plazmatu, distribuční funkce energie elektronů jsou kvalitativně rozdílné a teplota horkých elektronů se výrazně zvyšuje. \\

\noindent Klíčová slova: \keywordscz


%8------------------------------------------------------------------------------

\newpage
\begin{flushleft}
	\renewcommand{\arraystretch}{1.3}
	\begin{tabular}{r p{12cm}}
		Title:
		~ & \bf \projecttitle \\
		Author:
		~ & \valenta \\
		Type of work:
		~ & Master's thesis \\
		Study programme:
		~ & (N3913) Aplications of Natural Sciences	 \\
		Branch of study:
		~ & (3901T065) Computational Physics \\
		Supervisor:
		~ & \klimo \newline Department of Physical Electronics, Faculty of Nuclear Sciences and Physical Engineering, Czech Technical University in Prague \\
		Consultant:
		~ & \weber \newline Project ELI-Beamlines, Institute of Physics of the Czech Academy of Sciences \\
	\end{tabular}
\end{flushleft}

\begin{center}
	\textbf{Abstract}\\
\end{center}

Tight-focusing using the plasma-based optics could be an effective way to increase the peak intensity and enhance the spatial and temporal contrast ratio of the laser pulse. Since the paraxial approximation is not valid for description of tightly focused laser beams, the appropriate model has to be employed. Within this work, the laser boundary conditions based on the approach consistent with the Maxwell's equations have been implemented into the particle-in-cell simulation code and thoroughly tested. The instrumented code has been exploited for series of large-scale simulations employing tightly focused laser beams interacting with solid targets. The emphasis was placed mainly on identifying the effects of the focal spot size on the laser-matter interaction. We show that the hot electron trajectories and the absorption processes which take place during the interaction are strongly influenced by the transverse component of ponderomotive force which becomes significant in the case of sub-wavelength focus. The laser energy absorption efficiency sharply increases, the energy distribution functions are qualitatively different and the temperature of hot electrons is significantly higher in this case. \\

\noindent Keywords: \keywords

%9------------------------------------------------------------------------------

\tableofcontents
\addtocontents{toc}{\protect\thispagestyle{empty}}
\thispagestyle{empty}

%-------------------------------------------------------------------------------

\pagestyle{myfancy}

\chapter*{Introduction\markboth{Introduction}{Introduction}}
\addcontentsline{toc}{chapter}{Introduction}
Since the first demonstration of pulsed laser in 1960 \cite{Maiman1960}, the intensive research and development in the field of laser technology have seen a tremendous progress. Pulse compression and amplification techniques, such as CPA \cite{StricklandMourou1985}, OPCPA \cite{Dubietis1992} and lately RBA \cite{Malkin1999}, have enabled a generation of ultra-short laser pulses with intensities exceeding $ 10^{22} \ \mathrm{W/cm^{2}} $. Pulses in this regime provide an unprecedented capability for basic research (e.g. high-energy density physics, warm dense matter, plasma optics, laboratory astrophysics) as well as a broad range of groundbreaking applications in diverse fields (e.g. coherent diffractive imaging, X-ray diffraction and spectroscopy, production of compact sources for radiotherapy, inertial confinement fusion).

The peak laser intensities are typically increased by enhancing the produced laser properties, either by lowering the pulse duration or increasing the pulse energy. This approach comes at great cost since it requires a higher level of complexity for the laser chain \cite{Fuchs2014}. More effective way to increase the laser intensity is to reduce the focal spot size. However, the conventional solid state optics are inappropriate in the case of tight-focusing (expensive, susceptible to damage from solid target debris, sensitive to small misalignments). Nevertheless, it seems that many drawbacks might be in future overcome by using a plasma-based focusing optics. Therefore, the interaction of tightly focused laser beams with matter is currently attracting much attention \cite{Popov2008, Popov2009, Lifschitz2016, Yan2005}. 

This work is structured as follows: the first chapter provides a brief introduction to the classical electromagnetic field theory, including the mathematical derivation of the paraxial Gaussian beam formula. The second chapter summarizes the elementary knowledge of plasma physics and physics of laser-plasma interaction. In the third chapter, one of the most popular numerical methods in plasma physics, particle-in-cell (PIC), is thoroughly discussed. The characteristics and features of the code EPOCH \cite{bennett}, which has been used for the simulations within this work, can be found in the last section of this chapter. The fourth chapter is devoted to the tight-focusing of laser pulses, including a description of algorithm for rigorous calculation of electromagnetic fields at boundaries of simulation domain \cite{Thiele2016}. This chapter also contains the details about the implementation of the algorithm as well as thorough evaluation of its correctness and the correctness of the implementation. At the end, one may find the overview of currently used experimental methods for tight-focusing. The last chapter demonstrates the results of several large-scale two-dimensional simulations employing tightly focused laser beams interacting with solid targets.

Although the most convenient unit system for most plasma applications is the Gaussian cgs system, throughout this work the SI (System Internationale) units are used, unless explicitly stated. Symbols in bold represent vector quantities, and symbols in italics represent scalar quantities, unless otherwise indicated.


%-------------------------------------------------------------------------------

\chapter{Electromagnetic field}
Since the laser beam is nothing but the superposition of electromagnetic waves with high spatial and temporal coherence, the opening chapter has to be logically devoted to the fundamental physical aspects of the classical electromagnetic field theory based on the Maxwell's equations. The chapter contains a brief description of the microscopic as well as the macroscopic variant of the Maxwell's equations and their general solutions exploiting electrodynamic potentials and Hertz vectors. A short part is devoted also to energy and momentum of the electromagnetic waves. In the last part, one finds the simplest mathematical description of the focused laser beam and corresponding conditions of its validity.

\section{Maxwell's equations}
In order to achieve fusion, the reacting nuclei have to get very close to each other to activate the strong nuclear force. However, there is a large electrostatic repulsion between them as they come together because the protons in nuclei are positively charged. There are several possibilities of overcoming this barrier. One can for example utilize the initial kinetic energy of the chaotic thermal motion of the particles. In thermal equilibrium, the Coulomb collisions redistribute the kinetic energy among the plasma particles, and fusion reactions will eventually occur after a sequence of collisions. This approach is called thermonuclear fusion.

Reactions which could be usable in the case of the energy production on Earth must have high cross-section values at relatively low required incident energies. Furthermore, the economic viability and competitiveness is crucial for commercial use of fusion power. In other words, the reaction must be able to cover obviously no less than the energetic costs incurred for its ignition. The word ignition refers to the moment when a controlled fusion reaction generates as much or more energy than is needed to spark the reaction. In regards to each of these attributes, the best reaction candidate appears to be the reaction of two hydrogen isotopes - deuterium and tritium,
\begin{equation}
\nucl{2}{1}{H} + \nucl{3}{1}{H} \rightarrow \nucl{4}{2}{He} \ \mathrm{(3{,}5 \ MeV)} + \nucl{1}{0}{n} \ \mathrm{(14{,}1 \ MeV)}.
\end{equation}

Deuterium has quite rich natural abundance in Earth's oceans, thus it is commonly available in sufficient quantity. The efficient extraction from seawater is supposed to be technically ready \cite{bradshaw}. On the contrary, naturally occurring tritium is extremely rare due to its short half-life, therefore the fuel cycle requires the breeding of tritium. The production is possible with irradiation of lithium by fusion neutrons using the following reaction,
\begin{equation}
\nucl{6}{3}{Li} + \nucl{1}{0}{n} \rightarrow \nucl{4}{2}{He} \ \mathrm{(2{,}1 \ MeV)} + \nucl{3}{1}{H} \ \mathrm{(2{,}7 \ MeV)}.
\end{equation}

Lithium is widely distributed in the Earth's crust and seawater. Consequently, both fusion fuel resources are relatively easily accessible, uniformly geographically distributed and have near unlimited availability. The product of their reaction is a neutron and a helium nucleus, of which the latter product is not radioactive. These attributes and the potential to produce large amounts of carbon-free energy with almost no environmental impact predetermine nuclear fusion to become a global energy source.

\section{Electrodynamic potentials}
The first-order partial differential Maxwell's equations can be effectively converted to a smaller number of second-order equations by introducing electrodynamic potentials. Hence, one can express the electric and magnetic field as follows \cite{Thide2011},
\begin{equation}
\label{1.12}
\vec{E} = -\grad{\Phi} - \diffp{\vec{A}}{t},
\end{equation}
\begin{equation}
\label{1.13}
\vec{B} = \rot{\vec{A}},
\end{equation}
where $ \Phi\left(\vec{r}, t \right) $ is the scalar potential and $ \vec{A}\left(\vec{r}, t \right) $ is the vector potential of the corresponding fields. One can clearly see that using the definitions \ref{1.12}, \ref{1.13}, six vector components are replaced by only four potential functions and two Maxwell's homogeneous equations (\ref{1.8}, \ref{1.9}) are fulfilled identically. 

However, by definitions \ref{1.12}, \ref{1.13}, $ \Phi\left(\vec{r}, t \right) $ and $ \vec{A}\left(\vec{r}, t \right) $ are not defined uniquely, thus an infinite number of potentials which lead to the same fields may be constructed. To avoid that, one has to impose a supplementary condition, for example
\begin{equation}
\label{1.14}
\div{\vec{A}} + \mu \varepsilon \diffp{\Phi}{t} = 0.
\end{equation}
The condition \ref{1.14} is called the Lorenz gauge \cite{Thide2011}. Lorenz gauge is commonly used in electromagnetism because of its independence of the coordinate system. Furthermore, it leads to the following uncoupled equations,
\begin{equation}
\label{1.15}
\laplace{\Phi} - \mu \varepsilon \diffp[2]{\Phi}{t} = -\frac{\rho}{\varepsilon},
\end{equation}
\begin{equation}
\label{1.16}
\laplace{\vec{A}} - \mu \varepsilon \diffp[2]{\vec{A}}{t} = -\mu \vec{J},
\end{equation}
that are in all respects equivalent to the Maxwell's equations and in many situations much simpler to solve.

Equations \ref{1.15}, \ref{1.16} correspond to the inhomogeneous wave equations for scalar potential $ \Phi\left(\vec{r}, t \right) $ and vector potential $ \vec{A}\left(\vec{r}, t \right) $. Their general solutions are given by the following expressions \cite{Thide2011},
\begin{equation}
\label{1.17}
\Phi\left(\vec{r}, t \right) = \frac{1}{4 \pi \varepsilon} \int \frac{\rho\left(\vec{r^{\: \prime}}, t^{\: \prime} \right)}{\norm{\vec{r} - \vec{r^{\: \prime}}}} \mathrm{d} V,
\end{equation}
\begin{equation}
\label{1.18}
\vec{A}\left(\vec{r}, t \right) = \frac{\mu}{4 \pi} \int \frac{\vec{J}\left(\vec{r^{\: \prime}}, t^{\: \prime} \right)}{\norm{\vec{r} - \vec{r^{\: \prime}}}} \mathrm{d} V,
\end{equation}
where $ \mathrm{d} V $ is a volume element and $ \norm{.} $ stands for the standard Euclidean norm. Note that the solutions \ref{1.17}, \ref{1.18} are dependent only on charge and current densities at position $ \vec{r^{\: \prime}} $ at so-called retarded time $ t^{\: \prime} = t - \sqrt{\mu \epsilon} \norm{\vec{r} - \vec{r^{\: \prime}}} $ which takes into account the finite velocity of the wave \cite{Feynman1963}. In other words, the fields at the observation point $ \vec{r} $ at the time $ t $ are proportional to the sum of all the electromagnetic waves that leave the source elements at point $ \vec{r^{\: \prime}} $ at the retarded time $ t^{\: \prime} $.

\section{Hertz vectors}
There exists also other possibilities how to express the electromagnetic field. Under ordinary conditions, an arbitrary electromagnetic field may be defined in terms of a single vector function \cite{Essex1977}. This may be helpful for solving of many problems of classical electromagnetic theory, particularly the wave propagation.

First, let us introduce the electric Hertz vector $ {\vec{\Pi_e}}\left(\vec{r}, t \right) $ in terms of the scalar and vector potentials \cite{Stratton2007},
\begin{equation}
\label{1.19}
\Phi = - \div{\vec{\Pi_e}},
\end{equation}
\begin{equation}
\label{1.20}
\vec{A} = \mu \varepsilon \diffp{\vec{\Pi_e}}{t}.
\end{equation}

Note that the definitions \ref{1.19}, \ref{1.20} are consistent with the Lorenz gauge condition \ref{1.14}. In the absence of magnetization, it might be easily shown that $ \vec{J} = \partial{\vec{P}}/\partial{t} $ and the electric Hertz vector $ {\vec{\Pi_e}}\left(\vec{r}, t \right) $ is governed by an inhomogeneous wave equation
\begin{equation}
\label{1.21}
\laplace{\vec{\Pi_e}} - \mu \varepsilon \diffp[2]{\vec{\Pi_e}}{t} = -\frac{\vec{P}}{\varepsilon}.
\end{equation}

The equation \ref{1.21} is of the same type as the equations \ref{1.15}, \ref{1.16} and has therefore the familiar general solution 
\begin{equation}
\label{1.22}
\vec{\Pi_e}\left(\vec{r}, t \right) = \frac{1}{4 \pi \varepsilon} \int \frac{\vec{P}\left(\vec{r^{\: \prime}}, t^{\: \prime} \right)}{\norm{\vec{r} - \vec{r^{\: \prime}}}} \mathrm{d} V.
\end{equation}

As might be seen form \ref{1.22}, the fields derived from the electric Hertz vector $ {\vec{\Pi_e}}\left(\vec{r}, t \right) $ can be interpreted as being due to a density distribution of electric dipoles \cite{Essex1977}. Every solution of \ref{1.22} then uniquely determines the electromagnetic field through
\begin{equation}
\label{1.23}
\vec{E} = \grad{\left(\div{\vec{\Pi_e}}\right)} - \mu \epsilon \diffp[2]{\vec{\Pi_e}}{t},
\end{equation}
\begin{equation}
\label{1.24}
\vec{B} = \mu \varepsilon \left(\rot{\diffp{\vec{\Pi_e}}{t}}\right).
\end{equation}

Second, one may introduce the magnetic Hertz vector $ {\vec{\Pi_m}}\left(\vec{r}, t \right) $ in terms of the scalar and vector potentials by the following expressions \cite{Stratton2007},
\begin{equation}
\label{1.25}
\Phi = 0,
\end{equation}
\begin{equation}
\label{1.26}
\vec{A} = \rot{\vec{\Pi_m}}.
\end{equation}
In the absence of polarization, $ \vec{J} = \rot{\vec{M}} $ and the magnetic Hertz vector $ {\vec{\Pi_m}}\left(\vec{r}, t \right) $ defined by \ref{1.25} and \ref{1.26} fulfills an inhomogeneous wave equation
\begin{equation}
\label{1.27}
\laplace{\vec{\Pi_m}} - \mu \varepsilon \diffp[2]{\vec{\Pi_m}}{t} = -\mu \vec{M}.
\end{equation}
As for the previous cases, one may easily find the solution of \ref{1.27},
\begin{equation}
\label{1.28}
\vec{\Pi_m}\left(\vec{r}, t \right) = \frac{\mu}{4 \pi} \int \frac{\vec{M}\left(\vec{r^{\: \prime}}, t^{\: \prime} \right)}{\norm{\vec{r} - \vec{r^{\: \prime}}}} \mathrm{d} V,
\end{equation}
thus the fields derived from the magnetic Hertz vector $ {\vec{\Pi_m}}\left(\vec{r}, t \right) $ may be imagined to be due to a density distribution of magnetic dipoles \cite{Essex1977}. Again, every solution of \ref{1.28} uniquely determines the electromagnetic field via
\begin{equation}
\label{1.29}
\vec{E} = \rot{\diffp{\vec{\Pi_m}}{t}},
\end{equation}
\begin{equation}
\label{1.30}
\vec{B} = \rot{\left(\rot{\vec{\Pi_m}}\right)}.
\end{equation}

Note that the above derivations considered electric and magnetic Hertz vectors as a separate quantities. It is also possible, however, to introduce them together in the form of one six-vector \cite{Nisbet1955}.

\section{Energy and momentum}
To be able to describe the interaction of electromagnetic field with matter, one has to know the energy distribution throughout the field as well as the momentum balance.

By scalar multiplications of \ref{1.9} by $ \vec{H}\left( \vec{r}, t \right) $, of \ref{1.10} by $ \vec{E}\left( \vec{r}, t \right) $, following subtraction of both obtained equations and using standard vector identities, one gets the expression 
\begin{equation}
\label{1.31}
\vec{E} \cdot \diffp{\vec{D}}{t} + \vec{H} \cdot \diffp{\vec{B}}{t} + \div{\left(\vec{E} \times \vec{H} \right)} = -\vec{E} \cdot \vec{J}.
\end{equation}
The equation \ref{1.31} can be rewritten in the form of conservation law,
\begin{equation}
\label{1.32}
\diffp{u}{t} + \div{\vec{S}} = - \vec{E} \cdot \vec{J},
\end{equation}
where
\begin{equation}
\label{1.33}
u = \frac{1}{2} \left(\vec{E} \cdot \vec{D} + \vec{H} \cdot \vec{B} \right), \quad \vec{S} = \vec{E} \times \vec{H}.
\end{equation}
The quantity $ u\left( \vec{r}, t \right) $ in \ref{1.33} describes the total energy density in the field and $ \vec{S}\left( \vec{r}, t \right) $ is so-called Poynting vector which represents the energy flow of the field.

The important statement \ref{1.32}, also referred to as the Poynting theorem, expresses the conservation of energy for the electromagnetic field. In other words, the time rate of change of the field energy within a certain region and the energy flowing out of that region is balanced by the conversion of the electromagnetic energy into mechanical or heat energy and vice-versa.

+momentum...

\noindent
Lorentz force:
\begin{equation}
\vec{F} = q \left(E + v \times B \right) 
\end{equation}
Ohm's law:
\begin{equation}
\vec{J} = \sigma \vec{E}
\end{equation}

\section{Electromagnetic waves and Gaussian beam}
In this section, the simplest mathematical description of a focused laser beam based on approximations to the wave equation is deduced. Since in numerical codes it is a common practice to prescribe the laser beams by their propagation in free space, the set of the microscopic Maxwell's equations \ref{1.1} - \ref{1.4} will be exploited.

In the absence of external sources, it might be easily shown that the equations \ref{1.1} - \ref{1.4} may be alternatively formulated as an uncoupled homogeneous wave equations for electric field $ \vec{E}\left( \vec{r}, t \right) $ and magnetic field $ \vec{B}\left( \vec{r}, t \right) $ \cite{Feynman1963},
\begin{equation}
\label{1.34}
\laplace{\vec{E}} - \frac{1}{c^{2}} \diffp[2]{\vec{E}}{t} = 0,
\end{equation}
\begin{equation}
\label{1.35}
\laplace{\vec{B}} - \frac{1}{c^{2}} \diffp[2]{\vec{B}}{t} = 0,
\end{equation}
where the universal constant $ c = 1/\sqrt{\mu_0 \varepsilon_0} $ is the speed of light in vacuum, which leads to the essential fact, that the electromagnetic waves propagate in vacuum with the velocity of light $ c $. However, the wave equations \ref{1.34}, \ref{1.35} do not provide all the information about the electric and magnetic field of the wave. There are further constraints due to Maxwell's equations restricting the orientation and proportional magnitudes of the fields. From the set \ref{1.1} - \ref{1.4}, it might be clearly seen that $ \vec{E}\left( \vec{r}, t \right) $ and $ \vec{B}\left( \vec{r}, t \right) $ must be mutually perpendicular to each other as well as to the direction of the wave propagation \cite{Stratton2007}. 

Without any loss of generality, consider the laser beam as a monochromatic electromagnetic wave propagating toward the positive direction of the z-axis. In many standard references, the description of such a wave is given by the evolution of a single electric field component linearly polarized along the x-axis of the Cartesian coordinate system \cite{Siegman1986, Milonni1988, Yariv1989, Svelto2010, Galvez2006} (although the more proper way would be to use the vector potential \cite{Davis1979, Davis1981, Guenther1990, Salamin2006, Vaveliuk2007}), therefore one has to look for the solution of the equation \ref{1.34}. 

According to the previous assumptions, the solution is expected to be in the form of the following plane wave,
\begin{equation}
\label{1.36}
\vec{E}\left(\vec{r_\bot}, z, t \right)  = \Re \ E_0 \Psi \left(\vec{r_\bot}, z \right) \e^{\i \left(k_z z - \omega t \right)} \mathrm{\vec{\hat{e}_x}},
\end{equation}
where $ \vec{r_\bot} = (x, y)^{\mathrm{T}} $ is the vector of transverse Cartesian coordinates, symbol $ \Re $ stands for the real part of the complex quantity, $ E_0 $ is a constant amplitude, $ \Psi \left(\vec{r_\bot}, z \right) $ is the part of the wave function which is dependent only on the spatial coordinates, $ \omega $ denotes the angular frequency, $ k_z $ is the z-component of the wave vector $ \vec{k}\left(\omega \right) $, $ \mathrm{i} $ denotes the imaginary unit and $ \mathrm{\vec{\hat{e}_x}} $ is the unit vector pointing in the direction of the x-axis.

Direct substitution of expression \ref{1.36} into the equation \ref{1.34} yields the time-independent form of the scalar wave equation
\begin{equation}
\label{1.37}
\laplace{\Psi \left(\vec{r_\bot}, z \right)} + 2 \i k_z \diffp{\Psi \left(\vec{r_\bot}, z \right)}{z} = 0.
\end{equation}
The equation \ref{1.37} is called the Helmholtz equation \cite{Born2013}. Note that it is sufficient to seek solutions to the equation \ref{1.37} since the wave \ref{1.36} is monochromatic.

It turned out, that the geometry of the focused laser beam can be expressed in terms of the laser wavelength $ \lambda $ and the following three parameters,
\begin{equation}
\label{1.38}
w_0, \qquad z_{\mathrm{R}} = \frac{k_z w_0^2}{2} = \frac{\pi w_0^2}{\lambda}, \qquad \Theta = \frac{w_0}{z_\mathrm{R}} = \frac{\lambda}{\pi w_0}.
\end{equation}
The parameter $ w_0 $ in \ref{1.38} is the beam waist, defined as a radius at which the laser intensity fall to $ 1/\e^2 $ of its axial value at the focal spot. The second parameter, $ z_\mathrm{R} $, is so-called Rayleigh range which is a distance in the longitudinal direction from the focal spot to the point where the beam radius is $ \sqrt{2} $ larger than the beam waist $ w_0 $. And the last parameter, $ \Theta $, is the divergence angle of the beam that represents the ratio of transverse and longitudinal extent.

Because of the symmetry about the longitudinal axis of the equation \ref{1.37}, the following calculations may be made simpler by introducing a dimensionless cylindrical coordinates that use the parameters \ref{1.38},
\begin{equation}
\label{1.39}
\rho = \frac{\norm{\vec{r_\bot}}}{w_0}, \qquad \zeta = \frac{z}{z_{\mathrm{R}}}.
\end{equation}
After performing a transformation of coordinates, the Helmholtz equation \ref{1.37} becomes 
\begin{equation}
\label{1.40}
\frac{1}{\rho} \diffp{}{\rho}\left(\rho \diffp{\Psi \left(\rho, \zeta \right)}{\rho} \right) + 4 \i \diffp{\Psi \left(\rho, \zeta \right)}{\zeta}  = - \Theta^2 \diffp[2]{\Psi \left(\rho, \zeta \right)}{\zeta}.
\end{equation}
In the following calculations, the beam divergence angle $ \Theta $ is assumed to be small ($ \Theta \ll 1 $), thus it can be used as an expansion parameter for $ \Psi $ and the solution of \ref{1.40} will always be consistent,
\begin{equation}
\label{1.41}
\Psi = \sum_{n = 0}^{+\infty} \Theta^{2n} \Psi_{2n}.
\end{equation}
Next, one shall insert \ref{1.41} into \ref{1.40} and collect the terms with the same power of $ \Theta $. Then the zeroth-order function $ \Psi_0 $ obeys the following equation,
\begin{equation}
\label{1.42}
\frac{1}{\rho} \diffp{}{\rho}\left(\rho \diffp{\Psi_0 \left(\rho, \zeta \right)}{\rho} \right) + 4 \i \diffp{\Psi_0\left(\rho, \zeta \right)}{\zeta} = 0.
\end{equation}

The equation \ref{1.42}, which is called the paraxial Helmholtz equation, is the starting point of traditional Gaussian beam theory \cite{Davis1979}. One can expect the solution of \ref{1.42} in the form of a Gaussian function with a width varying along the longitudinal direction, thus 
\begin{equation}
\label{1.43}
\Psi_0 \left(\rho, \zeta \right) = h\left(\zeta \right)\e^{-f\left(\zeta \right) \rho^2},
\end{equation}
where $ f\left(\zeta \right) $ and $ h\left(\zeta \right) $ are unknown complex functions that have to satisfy a condition $ f\left(0 \right) = h\left(0 \right) = 1 $. After plugging \ref{1.43} into \ref{1.42}, one gets the following equation,
\begin{equation}
\label{1.44}
-f\left(\zeta \right) h\left(\zeta \right) + \i \diff{h\left(\zeta \right)}{\zeta} + \rho^2 h\left(\zeta \right) \left(f\left(\zeta \right)^2 - \i \diff{f\left(\zeta \right)}{\zeta} \right) = 0.
\end{equation}
Since the equation \ref{1.44} has to hold for arbitrary value of $ \rho $, one may find two independent equations that are equivalent to \ref{1.44}
\begin{equation}
\label{1.45}
\frac{1}{f\left(\zeta \right)^2} \diff{f\left(\zeta \right)}{\zeta} + \i = 0, \qquad \frac{1}{f\left(\zeta \right) h\left(\zeta \right)} \diff{h\left(\zeta \right)}{\zeta} + \i = 0.
\end{equation}
It might be easily shown, that under specified conditions the solutions of equations \ref{1.45} have to be
\begin{equation}
\label{1.46}
h\left(\zeta \right) = f\left(\zeta \right), \qquad f\left(\zeta \right) = \frac{1}{\sqrt{1 + \zeta^2}} \e^{-\i \arctan{\zeta}},
\end{equation}
and therefore the complete expression for the zeroth-order wave function $ \Psi_0 \left(\rho, \zeta \right) $ is
\begin{equation}
\label{1.47}
\Psi_0 \left(\rho, \zeta \right) = \frac{1}{\sqrt{1 + \zeta^2}} \exp{\left[- \frac{\rho^2}{1 + \zeta^2} + \i \left(\frac{\rho^2 \zeta}{1 + \zeta^2} - \arctan{\zeta} \right) \right]}.
\end{equation}
In many situations, it is also useful to evaluate the expression \ref{1.47} in terms of Cartesian coordinates, in which the zeroth-order wave function $ \Psi_0 \left(\vec{r_\bot}, z \right) $ is
\begin{equation}
\label{1.48}
\Psi_0 \left(\vec{r_\bot}, z \right) = \frac{w_0}{w\left(z\right)} \exp{\left[- \frac{\vec{r_\bot}^2}{w\left(z \right)^2} + \i \left( k_z \frac{\vec{r_\bot}^2}{2 R\left(z \right)} - \varphi_\mathrm{G} \left( z\right) \right) \right]},
\end{equation}
where the parameters used to simplify the expression \ref{1.48} are defined as
\begin{equation}
\label{1.49}
w\left(z\right) = w_0 \sqrt{1 + \left(\frac{z}{z_\mathrm{R}}\right)^2}, \quad R\left(z \right) = z \left[1 + \left(\frac{z_\mathrm{R}}{z} \right)^2\right], \quad \varphi_\mathrm{G}\left(z\right) = \arctan{\left(\frac{z}{z_\mathrm{R}}\right)}.
\end{equation}
One shall discuss the physical meaning of the three parameters \ref{1.49}. The function $ w\left(z\right) $ represents the spot size parameter of the beam, that is the radius at which the laser intensity fall to $ 1/\e^2 $ of its axial value at any position $ z $ along the beam propagation. Note that the minimum of the spot size $ w(0) = w_0 $, consequently the focal spot is stationary and located at the origin of a Cartesian coordinate system. The second parameter, $ R\left(z \right) $ is known to be the radius of curvature of the beam's wavefront at any position $ z $ along the beam propagation. Note that $ \lim_{z \to 0^{\pm}} R(z) = \pm \infty $, therefore the beam behaves like a plane wave at focus as required. The last parameter, $ \varphi_\mathrm{G}\left(z\right) $, is the so-called Gouy phase \cite{Pang2012} of the beam at any position $ z $ along the beam propagation, which describes a phase shift in the wave as it passes through the focal spot.

Finally, by substituting \ref{1.48} for $ \Psi \left(\vec{r_\bot}, z \right) $ in \ref{1.36} and taking the real part of that complex quantity, one obtains the electric field of the so-called paraxial Gaussian beam,
\begin{equation}
\label{1.50}
\vec{E}\left(\vec{r_\bot}, z, t \right) = E_0 \frac{w_0}{w(z)} \exp\left(-\frac{\vec{r_\bot}^2}{w(z)^2}\right) \cos\left(\omega t - k_z \left(z + \frac{\vec{r_\bot}^2}{2 R(z)} \right) + \varphi_\mathrm{G}\left(z\right) \right) \mathrm{\vec{\hat{e}_x}}.
\end{equation}
Although the electric field \ref{1.50} describes the main features of the focused laser beam, it might be clearly seen that it does not satisfy Gauss's law (\ref{1.1}). The correct electric field cannot vary with the direction of its polarization or has to have at least two non-zero vector components \cite{Davis1979}. To fix that, one would have to solve the wave equation for the vector potential \ref{1.16} and afterwards exploit the solution to deduce all components of the electric and magnetic fields.   

In addition, since one assumed $ \Theta \ll 1 $, the solution \ref{1.50} is not accurate for strongly diverging beams. Since the divergence angle is inversely proportional to the beam waist, the previous condition yields $ w_0 \gg \lambda $. In other words, it means that \ref{1.50} is not valid for tightly focused laser beams and the need may arise for higher-order corrections \cite{Thiele2016}. 

%-------------------------------------------------------------------------------

\chapter{Laser-plasma interaction}
When a high-power laser pulse is focused onto the surface of a solid target, a high density plasma layer is produced almost immediately due to the presence of strong electromagnetic fields. The dense plasma expanding from the target surface forms a density profile and the laser-plasma interaction takes place in its lower density part. Many different and often non-linear processes are involved in the laser-plasma interaction. A brief introduction to this field of research, which is rich both in physics and in applications, is provided in this chapter.

\section{Basic plasma parameters}
A plasma, one of the four fundamental states of matter, is a quasi-neutral gas of charged and neutral particles which exhibits collective behavior. It is necessary to explain some terms used in this definition.

By collective behavior one means motions that depend not only on local conditions but on the state of the plasma in remote regions as well. As charged particles move around, they can generate local concentrations of positive or negative charge, which give rise to electric fields. Motion of charges also generates currents, and hence magnetic fields. These fields affect the motion of other charged particles far away. Thus, the plasma gets a wide range of possible motions.

Quasi-neutrality describes the apparent charge neutrality of a plasma over large volumes, while at smaller scales, there can be charge imbalance, which may give rise to electric fields. This fact can be expressed mathematically as
\begin{equation}
\label{2.1.1}
\sum_{s} q_s n_s \approx 0,
\end{equation}
where $ q_s $ and $ n_s $ is, respectively, the charge and density of particles of species $ s $. The index of summation is taken over all the particle species in plasma.

One of the most important parameters, which allows us to more accurately predict the behavior of plasmas, is the degree of its ionization. For a gas containing only single atomic species in thermodynamic equilibrium, the ionization can be clearly recognized from the Saha-Langmuir equation, which can be written in the following form,
\begin{equation}
\label{2.1.2}
\frac{n_{k+1}}{n_k} = \frac{2}{n_e h^3}\left(2\pi m_e k_B T\right)^{\frac{3}{2}} \frac{g_{k+1}}{g_k} \exp\left(-\frac{\varepsilon_{k+1} - \varepsilon_{k}}{k_{B} T} \right).
\end{equation}
Here $ n_k $ is the density of atoms in the k-th state of ionization, $ n_e $ is the electron density, $ m_e $ stands for the mass of electron, $ k_B $ is Boltzmann's constant, $ T $ is the gas temperature, $ h $ is Planck's constant, $ g_k $ is the degeneracy of the energy level for ions in the k-th state, $ \varepsilon_k $ is the ionization energy of the k-th level. It can be clearly seen from equation (\ref{2.1.2}) that the fully ionized plasmas exist only at high temperatures. That is the reason why plasmas do not occur naturally on Earth (with a few exceptions).

A fundamental characteristic of the behavior of a plasma is its ability to shield out electric potentials that are applied to it. Therefore, another important quantity $ \lambda_{Ds} $ which is called the Debye length of species $ s $ is established,
\begin{equation}
\label{2.1.3}
\lambda_{Ds} = \sqrt{\frac{\varepsilon_0 k_B T_s}{q_s^2 \, n_s}}.
\end{equation}
The physical constant $ \varepsilon_0 $ is commonly called the permittivity of free space and $ T_s $ denotes the temperature of particles of species $ s $. It often happens that the different species of particles in plasma have separate distributions with different temperatures, however each species can be in its own thermal equilibrium. The Debye length is a measure of the shielding distance or thickness of the sheath.

In plasma, each particle tries to gather its own shielding cloud. The previously mentioned concept of Debye shielding is valid only if there are enough particles in that cloud. Therefore, another important dimensionless number $ N_{Ds} $, which is called plasma parameter of species $ s $, is established. Definition of this parameter is given by the average number of particles of species $ s $ in a plasma contained within a sphere of radius of the Debye length, thus
\begin{equation}
\label{2.1.4}
N_{Ds} = \frac{4}{3} \pi n_s \lambda_{Ds}^3. 
\end{equation}

Consider an electrically neutral plasma in equilibrium. Suppose an amount of electrons is displaced with respect to the ions, for example by intense laser pulse, and then allowed to move freely. An electric field will be set up, causing the electrons to be pulled back toward ions. Thus, the net result is a harmonic oscillation. The frequency of the oscillation is called the electron plasma frequency $ \omega_{pe} $,
\begin{equation}
\label{2.1.5}
\omega_{pe} = \sqrt{\frac{e^{2}\,n_e}{\varepsilon_0 m_e}}.
\end{equation}

By analogy with the electron plasma frequency (\ref{2.1.5}) one could define the ion plasma frequency $ \omega_{pi} $ for a general ion species. However, the ions are much heavier than electrons, so they do not response to high frequency oscillation of electromagnetic field. It is often possible to treat the massive ions as an immobile, uniform, neutralizing background. However, if the frequency of external radiation source or the waves induced in plasmas is close to this frequency, the ion motion must also be included. An example may be stimulated Brillouin scattering, which derivation can be found at the end of this chapter.

A typical charged particle in a plasma simultaneously undergo Coulomb collisions with all of the other particles in the plasma. The importance of collisions is contained in an expression called the collision frequency $ \nu_c $, which is defined as the inverse of the mean time that it takes for a particle to suffer a collision. Relatively accurate calculation of electron-ion collision frequency $ \nu_{ei} $ can be obtained from the following relation,
\begin{equation}
\label{2.1.6}
\nu_{ei} = \frac{Z e^4 n_e}{4 \pi \varepsilon_0^2 m_e^2 v^3} \ln{\Lambda}, \qquad \Lambda = \frac{\lambda_D}{b_0}.
\end{equation}
The coefficient $ Z $ denotes the charge number, $ v $ is relative velocity of colliding particles and $ \ln \Lambda $ is the so-called Coulomb logarithm. It is ratio of the Debye to Landau length. Landau length $ b_0 $ is the impact parameter at which the scattering angle in the center of mass frame is $ 90^\circ $. For many plasmas of interest Coulomb logarithm takes on values between $ 5 - 15 $. In a plasma a Coulomb collision rarely results in a large deflection. The cumulative effect of the many random small angle collisions that it suffers, however, is often larger than the effect of the few large angle collisions.

In a constant and uniform magnetic field one can find that a charged particle spirals in a helix about the line of force. This helix, however, defines a fundamental time unit and distance scale,
\begin{equation}
\label{2.1.7}
\omega_{cs} = \frac{\abs{q_{s}} \norm{\vec{B}}}{m_{s}}, \qquad r_{Ls} = \frac{v_\perp}{\omega_{cs}}.
\end{equation}
These are called the cyclotron frequency $ \omega_{cs} $ and the Larmor radius $ r_{Ls} $ of species $ s $. Here $ \vec{B} $ is a magnetic field and $ v_\perp $ is a positive constant denoting the speed in the plane perpendicular to $ \vec{B} $. Symbol $ \norm{.} $ stands for the Euclidean norm.


\section{Plasma description}
There are basically three different approaches to plasma physics: the hydrodynamic theory, the kinetic theory and the particle theory. Each approach has some advantages and limitations which stems from simplified assumptions appropriate only for certain phenomena and time scales.

The plasma kinetic theory takes into account the motion of all of the particles. This can be done in an exact way, using Klimontovich equation. However, one is not usually interested in the exact motion of the particles, but rather in certain average characteristics. Thus, this equation can be good starting point for the derivation of approximate equations.

The kinetic theory is based on a set of equations for the distribution functions $ f_s \left(\vec{x}, \vec{v}, t \right) $ of each plasma particle species $ s $, together with Maxwell equations. Here $ \vec{x} $ is the vector of coordinates for all the degrees of freedom, $ \vec{v} $ is the corresponding vector of velocities and $ t $ is time. The distribution function is a statistical description of a very large number of interacting particles. If collisions can be neglected (for example in hot plasmas), the evolution of such a system can be described by the collisionless Vlasov equation,
\begin{equation}
\label{2.2.1}
\diffp[]{f_{s}}{t} + \vec{v} \cdot \nabla f_s + \frac{q_{s}}{m_{s}}\left( \vec{E} + \vec{v} \times \vec{B} \right) \cdot \diffp[]{f_s}{\vec{v}} = 0.
\end{equation}
Here $ \vec{E} $ and $ \vec{B} $ are macroscopic electric and magnetic fields acting on the particles.

The equation (\ref{2.2.1}) is obtained only by making the assumption that the particle density is conserved, such that the rate of change in a phase-space volume is equal to the flux of particles into that volume. Because of its comparative simplicity, this equation is most commonly used in kinetic theory. However, the assumption to neglect collisions in a plasma is not generally valid. If it is necessary to take them into account, the collision term can be approximated under certain conditions.

The second approach is hydrodynamic theory. In this model, the conservation laws of mass, momentum and energy are coupled to Maxwell equations. The fluid theory is the simplest description of a plasma, however this approximation is sufficiently accurate to describe the majority of observed phenomena. The velocity distribution of each species is assumed to be Maxwellian everywhere, so the dependent variables are functions of only space coordinates and time. The fluid equations are simply the first three moments of the Vlasov equation. These yield the following fluid equations for the density, the momentum and the energy,
\begin{equation}
\label{2.2.2}
\diffp{n_s}{t} + \nabla \cdot \left(n_s \vec{u}_s \right) = 0,
\end{equation}
\begin{equation}
\label{2.2.3}
m_s n_s \left[ \diffp{\vec{u}_s}{t} + \left(\vec{u}_s \cdot \nabla \right) \vec{u}_s \right] + \nabla \cdot \mathbb{P}_s = q_s n_s \left(\vec{E} + \vec{u}_s \times \vec{B} \, \right),
\end{equation}
\begin{equation}
\label{2.2.4}
\diffp{}{t} \left(\frac{1}{2} n_s m_s u_s^2 + e_{s} \right) + \nabla \cdot \left(\frac{1}{2} n_s m_s u_s^2 \vec{u}_s + e_{s} \vec{u}_s + \mathbb{P}_s \vec{u}_s + \vec{Q}_s \right) = q_s n_s \vec{u}_s \cdot \vec{E}.
\end{equation}

The zeroth-order moment (\ref{2.2.2}) gives the continuity equation, where $ \vec{u}_s\left(\vec{x}, t \right) $ is the velocity of the fluid of species $ s $. This equation essentially states that the total number of particles is conserved. The first-order moment (\ref{2.2.3}) leads to a momentum equation. Here $ \mathbb{P}_s\left(\vec{x}, t \right) $ is the pressure tensor. This comes about by separating the particle velocity into the fluid and a thermal component of velocity. The thermal velocity then leads to the pressure term. Finally, the second-order moment (\ref{2.2.4}) corresponds to the energy equation, where $ e_{s} $ is the density of the internal energy and $ \vec{Q}_s $ describes the heat flux density.

The moment equations are an infinite set of equations and a truncation is required in order to solve these equations. For this equation system to be complete it has to be supplemented by an equation of state, which describes the relation between pressure and density in the plasma. However, the equations of state are well defined only in local thermodynamic equilibrium. Otherwise, the system cannot be described by fluid equations.

The last possible description is the particle theory approach. The plasma is described by electrons and ions moving under the influence of the external (e.g. laser) fields and electromagnetic fields due to their own charge. The basic equation of motion for a charged particle in an electromagnetic field is given by the Newton equations of motion with the Lorentz force,
\begin{equation}
\label{2.2.5}
\diff{\vec{x}}{t} = \vec{v}, \qquad \diff{\vec{v}}{t} = \frac{q_s}{m_s} \left(\vec{E} + \vec{v} \times \vec{B} \, \right).
\end{equation}
However, plasmas typically consist of an extremely large number of particles that interact in self-consistent fields, so the analysis can be applied only with the help of powerful computing infrastructure and particle simulation codes.

\section{Electromagnetic waves in plasmas}
In laser-plasma interaction, the knowledge how the laser beam propagates through plasma is essential. Therefore, a general properties of the electromagnetic wave propagation in plasmas are closer described in this section. One can find the characteristics of propagation in both, unmagnetized as well as magnetized plasma. Particularly, in the case of magnetized plasmas, the waves traveling parallel to and perpendicular to a constant magnetic field are discussed in greater detail.

\subsection{Unmagnetized plasmas}
To derive the dispersion relation of electromagnetic wave propagating through unmagnetized plasmas, the hydrodynamic approach is exploited. Since one assumes plasma response to a high frequency field, the ions are treated as a stationary, neutralizing background. Thermal motion of particles is also ignored, thus the pressure term in \ref{2.2.3} can be neglected. One also needs the wave equation for the electric field \ref{1.34}. However, it is now necessary to include the current density due to motion of charged particles in plasma. The hydrodynamic equations are coupled with the Maxwell's equations via $ \vec{J} = q_s n_s \vec{u}_s $, thus one shall solve the following set of equation,
\begin{equation}
\label{2.3.1.2}
m_e n_e \left[ \diffp{\vec{u}_e}{t} + \left(\vec{u}_e \cdot \nabla \right) \vec{u}_e \right] = - e n_e \vec{E},
\end{equation}
\begin{equation}
\label{2.3.1.3}
\laplace{\vec{E}} - \frac{1}{c^{2}} \diffp[2]{\vec{E}}{t} = - \mu_0 e n_e \diffp{\vec{u}_e}{t}.
\end{equation}
The system of equations above will be linearized using the methods of perturbation theory. Consider a small perturbations from the stationary state,
\begin{equation}
\label{2.3.1.4}
\vec{u}_{e} = \vec{u}_{e0} + \delta \vec{u}_{e}, \quad \vec{E} = \vec{E}_{0} + \delta \vec{E},
\end{equation}
where $ \vec{u}_{e0} $ and $ \vec{E}_{0} $ are obviously identically equal to zero vector. After substituting perturbed quantities \ref{2.3.1.4} into initial system of equations and performing Fourier transform one obtains
\begin{equation}
\label{2.3.1.6}
\delta \vec{u}_e = -\i \frac{e}{\omega m_e} \delta \vec{E},
\end{equation}
\begin{equation}
\label{2.3.1.7}
\delta \vec{E} = \i \frac{\omega \, e \, n_{e0}}{\varepsilon_0 \left(\omega^2 - c^2 k^2 \right)} \delta \vec{u}_e.
\end{equation}

Eliminating $ \delta \vec{u}_{e} $ from the equation \ref{2.3.1.7} one gets the equation for perturbation of the electric field,
\begin{equation}
\label{2.3.1.8}
\left(\omega^2 - \omega_{pe}^2 - c^2 k^2 \right) \delta \vec{E} = 0. 
\end{equation}
The equation \ref{2.3.1.8} is the dispersion relation of the electromagnetic wave in plasma. To describe how the electromagnetic waves propagates through given medium, it is useful to introduce the index of refraction $ N \left( \omega \right) =  c k \left( \omega \right) / \omega $. Consequently, the dispersion relation may be rewritten as
\begin{equation}
\label{2.3.1.9}
N^{2} = 1 - \left(\frac{\omega_{pe}}{\omega}\right)^2.
\end{equation} 
The important properties of the waves are distinguished by their cut-offs ($ N \rightarrow 0 $) and resonances ($ N \rightarrow \infty $). In the vicinity of the resonance there is a total absorption, at a cut-off frequency there is a total reflection of incident waves.

In the case of the electromagnetic wave propagating through unmagnetized plasma, it might be clearly seen that there are no resonances. On the other hand, the equation \ref{2.3.1.9} exhibits cut-off and the corresponding frequency (including ions) is given by the following expression,
\begin{equation}
\label{2.3.1.10}
\omega = \sqrt{\omega_{pe}^{2} + \omega_{pi}^{2}}
\end{equation}
The condition \ref{2.3.1.10} occurs at the so-called critical plasma density $ n_c \left( \omega\right) $. Note that the electromagnetic wave with frequency $ \omega $ passing through plasma with densities larger than $ n_c \left( \omega\right) $ is exponentially damped.

\subsection{Magnetized plasmas}
\input{dat/2.3.2.tex}

\section{Ponderomotive force}
Absorption of laser energy in laser-plasma interactions is an important issue, which has been closely related to the applications including the inertial fusion research ever since the invention of laser. Intense laser pulse can be absorbed in plasma by different non-linear mechanisms. In the following, the three main mechanisms of absorption of laser radiation are briefly described.


\subsection{Non-relativistic case}
One of the most important mechanism of laser light absorption in plasma is collisional absorption, also called inverse bremsstrahlung. It is a process in which an electron absorbs a photon while colliding with an ion or with another electron, and it leads to the heating of all particles in the interaction region. This way electromagnetic energy of the laser wave is transferred into the kinetic energy of plasma.

The change in laser intensity I, passing through plasma in the direction of x-axis, is given by
\begin{equation}
\label{2.4.1.1}
\diff{I}{x} = -\kappa I,
\end{equation}
where $ \kappa $ is the spatial damping rate of the laser energy caused by collisional absorption. For a slab of plasma of length $ L $, the absorption coefficient $ \alpha_{\mathrm{abs}} $ is given by
\begin{equation}
\label{2.4.1.2}
\alpha_{\mathrm{abs}} = 1 - \frac{I_{\mathrm{out}}}{I_{\mathrm{in}}} = 1 - \exp \left(-\int\limits_0^L \kappa \: \mathrm{d} x \right).
\end{equation}
Here $ I_{\mathrm{out}} $ and $I_{\mathrm{in}} $ are the outgoing and the incoming laser intensities, respectively. The absorption coefficient for a linear and exponential density profiles is given by solving the integral (\ref{2.4.1.2}) \cite{eliezer},
\begin{equation}
\label{2.4.1.3}
\alpha_{\mathrm{abs}} = 1 - \exp\left(-\frac{32}{15} \frac{\nu_{ei}(n_c) L}{c}\right) \qquad \mathrm{for} \qquad n_e = n_c \left(1 - \frac{x}{L} \right),
\end{equation}
\begin{equation}
\label{2.4.1.4}
\alpha_{\mathrm{abs}} = 1 - \exp\left(-\frac{8}{3} \frac{\nu_{ei}(n_c) L}{c}\right) \qquad \mathrm{for} \qquad n_e = n_c \exp\left(- \frac{x}{L} \right).
\end{equation}
where $ \nu_{ei}(n_c) $ is the collision frequency evaluated at the critical density $ n_c $, which is given by formula
\begin{equation}
n_{c} = \frac{\varepsilon_{0} m_{e} \omega}{e^{2}}.
\end{equation}
Notice that a significant fraction of the collisional absorption is from the region near the critical density. Collisional absorption is the preferred absorption mechanism for driving matter ablatively with laser beams. In the case of long laser pulse duration with relatively low intensity, collisional absorption can be very efficient.

\subsection{Relativistic case}
Laser light in the plasma can be also absorbed by resonance absorption. It is a linear process in which an incident laser wave is partially absorbed by conversion into an electron wave at the critical density of plasma.

Resonance absorption takes place when a p-polarized laser pulse is obliquely incident on a plasma with an inhomogeneous density profile. A component of the laser wave electric field perpendicular to the target surface then resonantly excites an electron plasma wave also along the plasma density gradient, thus a part of the laser wave energy is transferred into the electrostatic energy of the electron plasma wave. This wave propagates into the underdense plasma and it is damped either by collision or collisionless damping mechanisms. Consequently, energy is further converted into thermal energy which heats the plasma.

In contrast to collisional absorption, resonance absorption is the main absorption process for high laser intensities and long wavelengths. The efficiency of resonance absorption can also be higher for hot plasma, low critical density, or short plasma scale-length.


\section{Self-induced transparency (SIT)}
Plasma is strongly non-linear medium. This means that it may be accompanied by very strong electric fields and its non-linearities can be excited easily. When laser pulse with intensity above certain threshold irradiates the plasma, a number of collective non-linear processes may occur which can either enhance or reduce the energy absorption. As discussed before, the knowledge of the penetration depth and absorbed fraction of the incident energy is of vital importance for inertial confinement fusion research. 

The non-linear interaction is conveniently described in terms of light pressure and the ponderomotive force, which is introduced first. In the next section, the parametric instabilities are discussed as a consequence of non-linear coupling of electromagnetic laser waves to the plasma. The last section describes the laser beam filamentation and self-focusing.

\section{Laser absorption and electron heating mechanisms}
As mentioned in the section 2.3, the electromagnetic wave cannot propagate through plasma with densities greater than the critical density $ n_c $. However, in the case of intense laser beam, one has to perform a relativistic generalization of the dispersion relation, thus the corresponding cut-off frequency is given by
\begin{equation}
\omega = \sqrt{\frac{\omega_{pe}^2 + \omega_{pi}^2}{\gamma}}, \quad \gamma = \left( 1 - \frac{v^{2}}{c} \right)^{-\frac{1}{2}}.
\end{equation}
Therefore, one may immediately see that in the relativistic case, the laser beam actually propagates up to densities $ n_c^{\: \prime} = n_c \gamma > n_c $. This increase of the effective critical density is called relativistic self-induced transparency [source].

In the physical picture, relativistic transparency occurs when the increased mass of the electrons slows their motion such that they can no longer shield the plasma from the incident laser. As a result of relativistic transparency dynamics, both the reflected and transmitted laser intensity profiles evidence temporal chopping of the original laser intensity profile.


\subsection{Resonance absorption}
The energy of laser beam may be absorbed by resonance absorption in the plasma. It is a linear process in which an incident laser wave is partially absorbed by conversion into an electron wave at the critical density of plasma $ n_c (\omega) $.

Resonance absorption takes place when a linearly p-polarized laser pulse is obliquely incident on a plasma with an inhomogeneous density profile. A component of the electric laser field perpendicular to the target surface then resonantly excites an electron plasma wave also along the plasma density gradient, thus a part of the laser wave energy is transferred into the electrostatic energy of the electron plasma wave. This wave propagates into the underdense plasma and is damped either by collision or collisionless absorption mechanisms. Consequently, energy is further converted into the thermal energy which heats the plasma and may possibly produce hot electrons.

Particularly, resonance absorption is the main absorption process for the laser beams of higher intensities and longer wavelengths. The efficiency of the resonance absorption may also be higher in plasma with high temperature, low critical density, or short scale length of the density profile.


\subsection{Brunel's vaccum heating}
Similarly as for the resonance absorption, Brunel's vaccum heating takes place when a linearly p-polarized laser pulse is obliquely incident on a plasma. In this case, however, the incident angle must be relatively large. Moreover, the plasma density profile has to be steep, so the amplitude of the oscillating electrons driven by the electric laser field is larger than the density scale length. Consequently, the energy of laser pulse carried by electrons is transfered into mechanical or heat energy in the overdense plasma region \cite{Gibbon2005}.

The energy absorbed via Brunel's vacuum heating is carried by hot electrons in the bunches
ejected once per laser period. To give a physical picture, in the first half of laser cycle, the electrons are pushed inside the plasma gaining only a small amount of energies because the electric laser field is shielded. On the other hand, in the second half of laser cycle, the electrons gain very high energies while they are ejected into vacuum. 

The trajectory of charged particles is strongly influenced by the time of their expulsion. Furthermore, a self-consistent electric field may be generated if a large amount of charged particles is ejected at the same time. The majority of the charged particles, however, returns into the plasma and propagates behind the skin layer where the laser electric field is screened so they can travel virtually unhindered into the target. Note that since the charged particles are accelerated by different phases of the laser field, the distribution of such particles can be in most cases considered as a Maxwellian \cite{Gibbon2005}.

\subsection{Relativistic $ \vec{J} \times \vec{B} $ heating}
For the relativistic laser beams, the high-frequency $ \vec{v} \times \vec{B} $ component of the Lorentz force becomes important and may contribute to plasma heating. Similarly to the Brunel's vacuum heating, discussed in the previous subsection, $ \vec{J} \times \vec{B} $ heating requires very steep plasma density gradients. On the contrary, this heating scenario works for any polarization apart from circular and is most efficient for the laser pulses normally incident onto the target surface, so there is no oscillating component of the electric field perpendicular to plasma \cite{Gibbon2005}.

In the case of $ \vec{J} \times \vec{B} $ heating, the absorbed energy is carried by bunches of hot electrons that are ejected twice per laser period \cite{Gibbon2005}. This fact may help to distinguish between the effects of Brunel's and $ \vec{J} \times \vec{B} $ heating. Again, the ejected electrons are pushed back into the plasma by self-consistent electric field generated by charged particles without restoring forces behind the skin layer.

\section{Mechanisms of laser-driven ion acceleration}
Because of the relatively large ion mass, currently achievable laser intensities are not strong enough to accelerate protons or heavier ions directly to sufficiently high energies. However, ions typically respond on slowly varying electric fields in plasma arising from the strong charge separations induced by various phenomena that take place during the interaction of intense laser beam with matter.

As one may see later, ions can be either accelerated in the vicinity of the laser focal spot at the front side of the target as well as in the vicinity of the target-vacuum boundary at the rear side. In the last two subsections of this chapter, the two main mechanisms for ion acceleration in laser-plasma interactions are briefly described.

\subsection{Radiation pressure acceleration (RPA)}
Radiation pressure acceleration (RPA) stands for the mechanism in which the ions are accelerated from the target front side in the vicinity of the laser focal spot. The acceleration is driven by the ponderomotive force (see the section 2.4) which expels the electrons into the regions of a lower laser intensities and consequently generates a strong electrostatic fields as a result of a charge separation. In the case of intense laser beams, the radiation pressure is strong enough to push an overdense target inwards whilst changing the shape of its surface and correspondingly the density profile. This process is commonly named hole boring.

For a plane, monochromatic wave at a normal incidence onto a target at rest, the balance between the electrostatic pressure and the radiation pressure at the target surface can be expressed as follows,
\begin{equation}
\label{2.7.1.1}
\frac{1}{2} \varepsilon_0 E_{\mathrm{es}}^2 = \frac{\left( 1 + R - T \right)}{c} I,
\end{equation}
where $ \vec{E}_{\mathrm{es}} \left(\vec{r}, t \right) $ is the electrostatic field generated by the charge separation, $ R $ and $ T $ are the reflection and transmission coefficients of the target, respectively, and $ I $ is the intensity of incident laser pulse.

The formula \ref{2.7.1.1} determines the extension of a charge depletion layer, which is established at the front side of the target. Ions in the depletion layer are accelerated by the electrostatic field $ \vec{E}_{\mathrm{es}} \left(\vec{r}, t \right) $, which amplitude can be obtained by solving the Poisson equation. The maximum energy, that the ions may gain, is then estimated as follows [source],
\begin{equation}
E_{max} = \frac{Z m_e c^2 a_0^2}{m_i \gamma}, \quad \gamma = \sqrt{1 + \frac{a_0^2}{2}}.
\end{equation}
 
The RPA regime starts to dominate for a laser beams with peak intensity higher than $ 10^{21} \  \mathrm{W/cm}^2 $. Also, RPA could be more efficient for circularly polarized laser beams normally incident onto a target, because of their ability to suppress the effects of electron heating mechanisms mentioned in the previous section. Note that the RPA mechanism typically produce an ion beam of a large divergence because the critical density interface where the charge separation occurs is curved by the shape of the laser beam.



\subsection{Target normal sheath acceleration (TNSA)}
The interaction of intense laser beam with matter typically produce a population of hot electrons that may pass through the target and propagate beyond its rear side. This leads to a generation of a strong electrostatic potential which accelerates the ionized particles in the direction normal to the target rear surface. Therefore, this mechanism is commonly known as a target normal sheath acceleration (TNSA). TNSA can be also observed on the target front, because the majority of expelled electrons is often attracted back to the target and may even reach the front side.

The approximation for TNSA mechanism can be provided either by the static or dynamic model of the sheath. In the following, a dynamic model based on a free isothermal expansion of electron-ion plasma into vacuum is briefly described.

To find a non-linear solution that corresponds to the isothermal rarefaction wave, one may exploit the hydrodynamic equations for an unique ion component,
\begin{equation}
\label{2.7.2.2}
\diffp{n_i}{t} + \div{\left(n_i \vec{u}_i\right)} = 0,
\end{equation}
\begin{equation}
\label{2.7.2.3}
\diffp{\vec{u}_i}{t} + \left(\vec{u}_i \cdot \grad{} \right) \vec{u}_i = \frac{Z e}{m_i} \vec{E}. 
\end{equation}

Assume that at time $ t = 0 $, the electron density satisfies the Boltzmann distribution and the ion density occupies a half space with an infinitely sharp boundary, thus
\begin{equation}
\label{2.7.2.1}
n_e = n_{e0} \exp{\left( \frac{e \Phi}{k_B T_e} \right)}, \quad n_i = n_{i0} H(x).
\end{equation}
Assume further that the plasma is locally neutral on the scale length larger than the Debye radius, therefore $ n_e = Z n_i $. However, one shall be aware that the condition for the local neutrality does not imply that there is no electrostatic field. The electrostatic field may arise due to the sources coming from the regions where the condition for quasi-neutrality does not hold. Therefore, the electric field  $ \vec{E} \left(\vec{r}, t\right) $ in the equation \ref{2.7.2.3} may be replaced by the electrostatic field $ \vec{E}_{es} \left(\vec{r}, t\right) = -\grad{\Phi} \left(\vec{r}, t\right) $, which one can obtain by taking the gradient of the electron density \ref{2.7.2.1}.

After replacing the electric field $ \vec{E} \left(\vec{r}, t\right) $ in the equation \ref{2.7.2.3} one gets
\begin{equation}
\label{2.7.2.5}
\diffp{\vec{u}_i}{t} + \left(\vec{u}_i \cdot \grad{} \right) \vec{u}_i = - \frac{c_s^2}{n_i} \grad{n_i}, \quad c_s = \sqrt{\frac{Z k_B T_e}{m_i}},
\end{equation}
where $ c_s \left(\vec{r}, t\right) $ is the ion-acoustic velocity. By performing the Fourier transforms of the equations \ref{2.7.2.2} and \ref{2.7.2.5}, one easily obtain the dispersion relation of the ion-acoustic waves. However, the system has also a non-linear self-similar solution which describes the rarefaction wave. Thus, define a self-similar variable in 1D geometry as $ \xi = x/t $. The ion fluid density and velocity then become $ n_i \left(x, t\right) = N \left( \xi \right) $ and $ u_i \left(x, t\right) = V \left( \xi \right) $, respectively. After transformation, the system of equations \ref{2.7.2.2} and \ref{2.7.2.5} yields
\begin{equation}
\label{2.7.2.6}
\frac{1}{N} \diff{N}{\xi} = - \frac{V - \xi}{c_s},
\end{equation}
\begin{equation}
\label{2.7.2.7}
\diff{V}{\xi} = - \frac{1}{N} \diff{N}{\xi} \left(V - \xi \right).
\end{equation}
Now, the set of equations \ref{2.7.2.6}, \ref{2.7.2.7} can be solved easily. The solution in the original coordinates can be found below,
\begin{equation}
\label{2.7.2.8}
n_i = n_{i0} \exp \left( -\frac{x}{c_s t} - 1 \right), \quad u_i = c_s + \frac{x}{t}.
\end{equation}
Note that the solution \ref{2.7.2.8} is valid only for $ x > -c_s t $ where $ u_i > 0 $. The condition $ x = - c_s t $ describes the rarefaction front propagating backwards into the target at the speed $ c_s $. Finally, the electric field at the ion front $ x = - c_s t $ can be expressed as
\begin{equation}
\label{2.7.2.9}
E_x = \frac{2 E_0}{\omega_{pi} t}, \quad E_0 = c_s \sqrt{\frac{n_i m_i}{\varepsilon_0}}.
\end{equation}

The apparent drawback of formula \ref{2.7.2.9} is that it has a singularity at time $ t = 0 $. However, one may find a simple expression for the electric field at $ t = 0 $ by integration of the Poisson's equation from $ x = 0 $ to $ x = +\infty $,
\begin{equation}
\label{2.7.2.10}
E_x = \sqrt{\frac{2}{e}} E_0.
\end{equation}
A very precise interpolation of the electric field at the ion front, which is valid for any time, has been already found [source],
\begin{equation}
\label{2.7.2.11}
E_x \cong \frac{2 E_0}{\sqrt{2e + \omega_{pi}^2 t^2}}.
\end{equation}
As a consequence, one may find relatively accurate prediction of the corresponding ion front velocity by solving the equation of motion \ref{2.7.2.3} with the electric field \ref{2.7.2.11},
\begin{equation}
\label{2.7.2.12}
u_i \cong 2 c_s \ln \left[ \frac{\omega_{pi} t}{\sqrt{2 e}} + \left(\frac{\omega_{pi}^2 t^2}{2 e} + 1 \right)^{\frac{1}{2}} \right].
\end{equation}
The next drawback of the model is that the derived formula for maximum velocity of ions \ref{2.7.2.12} diverges logarithmically with time, therefore the system accelerates the ions infinitely. This result leads from the isothermal assumption. This might be overcome by assuming the finite thickness of plasma or by introducing a maximum acceleration time constant at which the ion acceleration stops [source].

%-------------------------------------------------------------------------------

\chapter{Particle-in-cell (PIC) method}
This chapter is devoted to numerical simulations, particularly to the particle-in-cell (PIC) method, which represents one of the most popular numerical algorithms in plasma physics. The mathematical background of this method, description of the individual steps of the computational cycle as well as the stability conditions are all discussed. The last section provides a brief overview of the particle-in-cell code EPOCH \cite{bennett}, which has been used for simulations within this work.

\section{Mathematical derivation}
The particle-in-cell (PIC) method refers to a technique used to solve a certain class of partial differential equations. The method was proposed in the mid-fifties and it gained a great popularity in plasma physics applications early. It is based on the particle description approach, thus the evolution of the system is conducted in principle by following the trajectory of each particle.

However, the real systems are often extremely large in terms of the number of particles they contain. In order to make simulations efficient or at all possible, so-called macro-particles are used. A macro-particle is a finite-sized computational particle that represents a group of physical particles that are near each other in the phase space. It is allowed to rescale the number of particles, because the Lorentz force depends only on the charge to mass ratio, which is invariant to this transformation. Thus, a macro-particle will follow the same trajectory as the corresponding real particles would \cite{hockney}.

Although this approach significantly reduces the number of computational particles, the binary interactions for every pair of a system cannot be taken into account. The cost would scales quadratically, as the number of particles increases, which makes the computational effort unmanageable in the case of larger systems. Many of the phenomena occur in high-temperature plasmas where collisional effects are very weak, thus one can neglect them. Otherwise, one may use other techniques to include collisional effects \cite{lapenta}.

Here, the procedure for deriving the PIC method is considered. The phase space distribution function  $ f_{s} \left(\vec{x}, \vec{v}, t\right) $ for a given species $ s $ is governed by the Vlasov equation (\ref{2.2.1}) in the collisionless plasma. The PIC method can be regarded as a finite element approach but with finite elements that are themselves moving and overlapping. The mathematical formulation of the PIC method is obtained by assuming that the distribution function of each species is given by the sum of distribution functions for macro-particles,
\begin{equation}
\label{3.1.1}
f_{s} \left(\vec{x}, \vec{v}, t \right) =  \sum_{p} f_{p}\left(\vec{x}, \vec{v}, t \right).
\end{equation}
Index $ p $ denotes hereafter the quantities attributable to macro-particles. The distribution function for each macro-particle is further assumed to be
\begin{equation}
\label{3.1.2}
f_{p}\left(\vec{x}, \vec{v}, t \right) = N_{p} S_{x}\left(\vec{x} - \vec{x}_{p}\left(t\right) \right)  S_{v}\left(\vec{v} - \vec{v}_{p}\left( t\right) \right),
\end{equation}
where $ N_{p} $ is the number of physical particles that are represented by each macro-particle, and $ S_{x} $, $ S_{v} $ are the so-called shape functions.

The shape functions cannot be chosen arbitrarily. They have to fulfill a several special properties. Let $ S_{\xi} $ be the shape function of the phase space coordinate $ \vec{\xi} $. Then:
\begin{enumerate}[nolistsep, topsep=5pt]
\item The support of the shape function is compact, $ \exists R > 0, \: \mathrm{supp} \: S_{\xi} \subset \left(-R, R\right) $.
\item Integral of the shape function is unitary, $ \int\limits_{-\infty}^{+\infty} S_{\xi}\left(\vec{\xi}\right)  \mathrm{d} \xi = 1 $.
\item The shape function is symmetrical, $ S_{\xi}\left(\vec{\xi}\right) = S_{\xi}\left(-\vec{\xi}\right) $.
\end{enumerate}

While these restrictive conditions still offer a wide range of possibilities, the standard PIC method is essentially determined by the choice of the shape function in the velocity direction as a Dirac $ \delta $-function and in the spatial direction as a m-th order b-spline basis function $ b_{m} $,
\begin{equation}
\label{3.1.3}
S_{v}\left(\vec{v} - \vec{v}_{p}\left(t\right)\right) = \delta\left(\vec{v} - \vec{v}_{p}\left(t\right)\right), \qquad S_{x}\left(\vec{x} - \vec{x}_{p}\left(t\right)\right)  = b_{m}\left(\frac{\vec{x} - \vec{x}_{p}\left(t\right)}{\Delta p}\right),
\end{equation}
where $ \Delta p $ is the size of the support of the computational particles, typically the same as simulation grid cell. Stability and accuracy of the simulation strongly depend on the choice of the shape functions. The choice of higher-order basis functions results in less numerical noise interpolation of density and field quantities and reduces non-physical phenomena in simulations, obviously at the cost of increased computational time.

The computational cycle of the PIC method is shown in Figure \ref{3.1.4}. Individual steps are closer described in several following sections. The influence of the choice of the time and spatial step on the stability and accuracy of the PIC method is demonstrated in the last section.

\begin{figure}[h!]
\centering
\tikzstyle{empty} = [rectangle, fill=white, text width=12em, text badly centered, node distance=4em]
\tikzstyle{block} = [rectangle, draw, thick, fill=white, text width=12em, text centered, rounded corners, minimum height=4.5em]
\tikzstyle{line} = [draw, -triangle 45]
\begin{tikzpicture}[node distance = 2cm, auto]

\node [block] (nahore) {Particle mover\\[2mm] $ \vec{F}_p \rightarrow \left(\vec{x}, \vec{v}\,\right)_{p} $};
    
\node [empty, below of=nahore, node distance=2.5cm] (uprostred) {};
    
\node [block, right of=uprostred, node distance=4.5cm] (vpravo) {Particle weighting\\[2mm] $ \left(\vec{x}, \vec{v}\right)_{p} \rightarrow \left(\rho, \vec{J}\,\right) _{i, j, k} $ };
    
\node [block, left of=uprostred, node distance=4.5cm] (vlevo) {Field weighting \\[2mm] $ \left(\vec{E}, \vec{B}\,\right)_{i, j, k} \rightarrow \vec{F}_p $};
    
\node [block, below of=uprostred, node distance=2.5cm] (dole) {Field solver\\[2mm] $ \left(\rho, \vec{J}\,\right) _{i, j, k} \rightarrow \left(\vec{E}, \vec{B}\,\right)_{i, j, k} $};

\path [line] (dole) -| (vlevo);
\path [line] (vlevo) |- (nahore);
\path [line] (nahore) -| (vpravo);
\path [line] (vpravo) |- (dole);
\end{tikzpicture}
\caption{Computational cycle of the particle-in-cell method}
\label{3.1.4}
\end{figure}

\section{Particle pusher}
The abbreviation EPOCH refers to an Extendable PIC Open Collaboration project \cite{bennett}. EPOCH is a multi-dimensional, relativistic, electromagnetic code designed for plasma physics simulations based on the PIC method. The code, which has been developed at University of Warwick, is written in FORTRAN and parallelized using MPI library. EPOCH is explicit and is able to achieve second-order accuracy. The entire core of the code uses SI units.

The main features include dynamic load balancing option for making optimal use of all processors when run in parallel, allowing restart on an arbitrary number of processors. The setup of EPOCH is controlled through a customizable input deck. An input deck is a text file which can be used to set simulation parameters for EPOCH without necessity to edit or recompile the source code. Most aspects of a simulation can be controlled, such as the number of grid points in the simulation domain, the initial distribution of particles and the initial electromagnetic field configuration. In addition, EPOCH has been written to add more modern features and to structure the code in such a way that the future expansion of the code may be made as easily as possible.

By default, EPOCH uses triangular particle shape functions with the peak located at the position of computational particle and a width of two cells, which provides relatively clean and fast solution. However, user can select higher order particle shape functions based on a spline interpolation by enabling compile-time option in the makefile.

The electromagnetic field solver uses a FDTD scheme with second order of accuracy. The field components are spatially staggered on a standard Cartesian Yee cell. The solver is directly based on the scheme derived by Hartmut Ruhl \cite{ruhl}. The particle pusher is relativistic, Birdsall and Landon type \cite{birdsall} and uses Villasenor and Buneman current weighting \cite{villasenor}.

EPOCH offers several types of boundary conditions for fields and particles, such as periodic, transmissive, reflecting and also Convolutional Perfectly Matched Layer (CPML) boundary conditions. Laser beams can be attached to arbitrary boundary via special boundary condition as well.

As a side project within this work, the code EPOCH has been instrumented to enable in situ diagnostics and visualization of the electromagnetic fields using ParaView Catalyst [source].
The increasing demands of the simulations need more data to be stored on a disk and analysed. However, the capabilities of computing environment which is responsible for transferring the data and communication have not grown up as rapid as the computational power. Dumping and processing of all the data calculated during the simulation would take too much time, so in practice this usually means that they are stored only at several time steps or at much coarser resolution than the original data. The rest is just discarded and the significant part of information may be potentially lost.

In situ visualization describes techniques where data can be visualized in real-time as it is generated during a simulation and without it being stored on a storage resource. By coupling the visualization and simulation, the data transfer bottleneck can be overcome. Furthermore, this approach allows scientists to monitor and interact with a running simulation, allowing for its parameters to be modified and allowing to immediately view the effects of these changes.

While a simulation is running, a user can see the size of the datasets that a simulation produces. But none of this data is physically stored on a storage system. The computationally expensive operations are carried out using ParaView’s graphical interface. So, the user can select data structures and analyze them in the same way as in post-processing, which requires the saving of datasets onto a file system. But there is one difference, the simulation is in progress so a user can observe the data as it is being generated. With Catalyst, it is also possible to pause the simulation or specify a break-point at a selected time step. This can be helpful if a user expects some interesting behavior of investigated phenomena or for identifying regions where numerical instability arises. For the implementation details, see Appendix - .

The main goal of this work has been to implement a solution that would enable to simulate tightly focused laser beams using simulation code EPOCH. This will be closer described in the following chapter.



\section{Field solver}
\input{dat/3.3.tex}

\section{Particle and field weighting}
In order to solve the Maxwell's equations, as shown in the previous section of this chapter, one has to know the source terms produced by the motion of the charged particles. In other words, it is necessary to assign charge and current densities from the continuous macro-particle positions to the discrete grid points. This simulation step is usually referred to as particle weighting and it involves some form of interpolation.
 
According to the kinetic theory, charge density $ \rho\left(\vec{x}, t \right) $ and current density $ \vec{J}\left(\vec{x}, t \right) $ are given by the following integrals over the velocity space,
\begin{equation}
\label{3.1.2.1}
\rho\left(\vec{x}, t \right) = \sum_s q_s \int f_s \left(\vec{x}, \vec{v}, t \right) \mathrm{d} \vec{v}, \qquad \vec{J}\left(\vec{x}, t \right) = \sum_s q_s \int f_s \left(\vec{x}, \vec{v}, t \right) \vec{v} \, \mathrm{d} \vec{v}.
\end{equation}
After discretization of \ref{3.1.2.1} using macro-particles and exploiting the properties of the shape functions, one gets immediately
\begin{equation}
\label{3.1.2.2}
\rho_{ijk}^{\,n} = \sum_{p} q_p S_{r}\left(\vec{r}_{ijk} - \vec{r}_{p}^{\,n}\right), \qquad \vec{J}_{ijk}^{\,n} = \sum_{p} q_p \vec{v}_p^{\,n} S_{r}\left(\vec{r}_{ijk} - \vec{r}_{p}^{\,n}\right),
\end{equation}
where $ q_p = q_s w_p $. However, using the formulas \ref{3.1.2.2} for charge and current deposition in PIC codes may violate the discrete continuity equation (\ref{3.1.1.13}) and in turn cause errors in Gauss's law (\ref{3.1.1.6}). In this case, one would have to solve the Poisson's equation for the correction of the electric field at every simulation time step or use a numerical scheme that satisfies the continuity equation exactly. These schemes are referred to as a charge conservation methods (\cite{Eastwood1991, Eastwood1996, villasenor, esirkepov, Morse1971, umeda}).

Similarly, to advance macro-particle positions, as shown in the second section of this chapter, one has to know the force acting on them. Hence, it is necessary to assign electric and magnetic fields that are calculated at the discrete grid points to the continuous macro-particle positions. This simulation step is usually referred to as field weighting.

By analogy to the particle weighting, one may exploit the shape functions to calculate the spatial averages of the electric and magnetic field components,
\begin{equation}
\vec{E}_{p}^{\,n} = \sum_{ijk} \vec{E}_{ijk}^{\,n} S_{r}\left(\vec{r}_{ijk} - \vec{r}_{p}^{\,n}\right), \qquad \vec{B}_{p}^{\,n} = \sum_{ijk} \vec{B}_{ijk}^{\,n} S_{r}\left(\vec{r}_{ijk} - \vec{r}_{p}^{\,n}\right).
\end{equation}
Note that it is recommended to use the same weighting for both, particles and fields, in order to eliminate a self-force and ensure the conservation of momentum \cite{fehske}.



\section{Stability and accuracy}
The stability and accuracy of the standard PIC method is directly dependent on the size of the spatial and temporal simulation steps. In order to find correct parameters, one has to know the absolute accuracy and corresponding stability conditions.

The effect of the spatial grid is the smoothing of the interaction forces and the coupling of plasma perturbations to perturbations at other wavelengths, called aliases. It may lead to non-physical instabilities and numerical heating. To avoid these effects, the spatial step should approximately resolve the Debye length (see \ref{2.1.3}). Thus, it is desirable to fulfill the following condition,
\begin{equation}
\Delta x, \Delta y, \Delta z \leq \lambda_{D}.
\end{equation}
In the general electromagnetic case, the time step has to satisfy the Courant--Fridrichs--Levy (CFL) condition \cite{CFL1967},
\begin{equation}
\label{3.1.4.1}
C = c^{2} \Delta t^{2} \left(\frac{1}{\Delta x^{2}} + \frac{1}{\Delta y^{2}} + \frac{1}{\Delta z^{2}}\right),
\end{equation}
where the dimensionless number $ C \leq 1 $ is called the CFL number. This condition limits the range of motion of all objects in the simulation during one time step. It ensures that the particles would not cross more than one cell in one simulation time step. When this condition is violated, the growth of non-physical effects can be very rapid. Another condition on the time step comes from the requirement to avoid aliasing, i.e. the time step must resolve all important frequencies according to the Nyquist-Shannon theorem \cite{Shannon1949} $ \omega \Delta t \leq 2 $ ($ \omega = \max\left( \omega_p, \omega_L \right) $).

The leap-frog scheme, used to solve the field equations and equations of motion, is second-order accurate in both, time and space. In addition, this scheme is explicit and time-reversible.

\section{Extendable PIC Open Collaboration (EPOCH)}
\input{dat/3.6.tex}

%-------------------------------------------------------------------------------

\chapter{Tight-focusing of laser pulses}
The investigation of laser-matter interactions also involves exploring of specific themes of the ultra-relativistic regime, which requires extremely high intensities of the external field. Tight-focusing and the plasma-based optics could potentially increase the laser pulse intensity by an order of magnitude in comparison with the conventional focusing approaches and thus the laser intensities of the order $ 10^{22} \ \mathrm{W/cm^2} $ could be reached by Petawatt laser systems. Such intensities would then allow a broad spectrum of many new physics discoveries and applications.

As mentioned in the previous chapter, various aspects of the electromagnetic interaction are usually studied using sophisticated numerical simulation codes. Vast majority of these codes, however, use a paraxial approximation (closer described in chapter 1) to prescribe the laser fields at the boundaries, and afterwards, a field solver propagates the beam across the simulation domain. As already mentioned, the paraxial approximation is valid only if the angular spectrum of the laser pulse is sufficiently narrow. Therefore, it is not possible to simulate tightly focused laser beams using this approach. As will be seen later, the paraxial approximation in this case leads to a distorted field profiles which would have strong impact on the laser-matter interaction results.

Several interesting solutions, how to simulate strongly focused beams, have been already proposed [source]. Within this work, a simple and efficient algorithm for a Maxwell consistent calculation of the electromagnetic fields at the boundaries of the computational domain [source] (also called laser boundary conditions) has been used and implemented into the PIC code EPOCH [source]. Note that this algorithm is able to describe laser beams with an arbitrary shape in the focal spot.

\section{Laser boundary conditions}
In this section, another mathematical description of a focused laser beam based on a rigorous solution of the wave equation \ref{1.34} is presented. The following calculations are reproduced from the work of Illia Thiele et al. \cite{Thiele2016}. Assume that the laser beam propagates in vacuum without external sources along the z-axis of the Cartesian coordinate system. The wave equation \ref{1.34} in temporal Fourier space has the following form,
\begin{equation}
\label{4.1}
\laplace{\hat{\vec{E}}} \left(\vec{r}, \omega \right) + \frac{\omega^2}{c^2} \hat{\vec{E}} \left(\vec{r}, \omega \right) = 0,
\end{equation}
where the hat symbol placed on the top of a variable denotes the Fourier transform with respect to time. Next, one shall perform a spatial Fourier transform of \ref{4.1} with respect to transverse coordinates $ \vec{r_\bot} $ only,
\begin{equation}
\label{4.2}
\left(- k^2_x - k^2_y + \diffp[2]{}{z} \right) \bar{\vec{E}} \left(k_x, k_y, z, \omega \right) + \frac{\omega^2}{c^2} \bar{\vec{E}} \left(k_x, k_y, z, \omega \right) = 0,
\end{equation}
where the bar symbol placed on the top of a variable denotes the Fourier transform with respect to time and spatial transverse coordinates. The equation \ref{4.2} can be simplified as follows,
\begin{equation}
\label{4.3}
k^2_z \left(\vec{k}_\bot, \omega \right) \bar{\vec{E}}(\vec{k}_\bot, z, \omega) + \diffp[2]{}{z} \bar{\vec{E}} \left(\vec{k}_\bot, z, \omega \right) = 0.
\end{equation}
where $ k_z \left(\vec{k}_\bot, \omega \right) = \sqrt{-\vec{k}_\bot^2 + \omega^2/c^2} $ and $ \vec{k}_\bot = (k_x, k_y)^{\mathrm{T}} $. The fundamental solution of the equation \ref{4.3} consists of the forward $ (+) $ and backward $ (-) $ propagating waves,
\begin{equation}
\label{4.4}
\bar{\vec{E}}^{\pm} \left(\vec{k}_\bot, z, \omega \right) = \bar{\vec{E}}_{0}^{\pm} \left(\vec{k}_\bot, \omega \right) \e^{\pm \i k_z \left(\vec{k}_\bot, \omega \right) \left(z - z_0 \right)}.
\end{equation}
Where $ \bar{\vec{E}}_{0}^{\pm}\left(\vec{k}_\bot, \omega \right) $ is the electric laser field at some plane $ z = z_0 $. It might be clearly seen, that only two out of six vector components of the electric and magnetic fields are independent, therefore one may prescribe for example the transverse components $ \bar{\vec{E}}_{0, \bot}^{\pm}\left(\vec{k}_\bot, \omega \right) $ at the plane $ z = z_0 $ and all other components can be derived from the Maxwell's equations \ref{1.1}, \ref{1.3},
\begin{equation}
\label{4.5}
\bar{\vec{E}}^{\pm}_{\bot} \left(\vec{k}_\bot, z, \omega \right) = \bar{\vec{E}}^{\pm}_{0, \bot} \left(\vec{k}_\bot, \omega \right) \e^{\pm \i k_z \left(\vec{k}_\bot, \omega \right) \left(z - z_0 \right)},
\end{equation}
\begin{equation}
\label{4.6}
\bar{E}^{\pm}_z \left(\vec{k}_\bot, z, \omega \right) = \mp \frac{\vec{k}_\bot \cdot \bar{\vec{E}}^{\pm}_{\bot}(\vec{k}_\bot, z, \omega)}{k_z \left(\vec{k}_\bot, \omega \right)},
\end{equation}
\begin{equation}
\label{4.7}
\bar{\vec{B}}^{\pm}_{\bot} \left(\vec{k}_\bot, z, \omega \right) = \frac{1}{\omega k_z \left(\vec{k}_\bot, \omega \right)} \mathbb{R}^{\pm} \left(\vec{k}_\bot, \omega \right) \bar{\vec{E}}^{\pm} \left(\vec{k}_\bot, z, \omega \right),
\end{equation}
where
\begingroup
\renewcommand*{\arraystretch}{1.7}
\begin{equation}
\label{4.8}
\mathbb{R}^{\pm} \left(\vec{k}_\bot, \omega \right) =  \begin{pmatrix}
\mp k_x k_y & \mp \left[ k_z^2 \left(\vec{k}_\bot, \omega \right) + k_y^2 \right] & 0 \\
\pm \left[ k_z^2 \left(\vec{k}_\bot, \omega \right) + k_x^2 \right] & \pm k_x k_y & 0 \\
- k_y k_z \left(\vec{k}_\bot, \omega \right) & - k_x k_z \left(\vec{k}_\bot, \omega \right) & 0
\end{pmatrix}.
\end{equation} 
\endgroup

Analogically, one could solve the wave equation for the magnetic field \ref{1.35}, prescribe two transverse components of $ \bar{\vec{B}}_{0, \bot}^{\pm}\left(\vec{k}_\bot, \omega \right) $ at the plane $ z = z_0 $ and afterwards calculate all other fields using Maxwell's equations \ref{1.2}, \ref{1.4}. The complete proof, that the fields \ref{4.5} - \ref{4.7} are consistent with the Maxwell's equations in vacuum \ref{1.1} - \ref{1.4} can be found in the original paper \cite{Thiele2016}.

Note that for $ k_\bot^2 > \omega^2/c^2 $, $ k_z \left(\vec{k}_\bot, \omega \right) $ becomes imaginary and equation \ref{4.4} describes evanescent waves that are unphysical in free space. Thus the Fourier spectrum of laser waves has to be filtered in the transverse Fourier space. On the other hand, if the spatial Fourier spectrum contains only components with $ k_\bot^2 \ll \omega^2/c^2 $, then $ k_z \left(\vec{k}_\bot, \omega \right) $ can be approximated using the first few terms of a Taylor series,
\begin{equation}
\label{4.9}
k_z \left(\vec{k}_\bot, \omega \right) \approx \frac{\abs{\omega}}{c} - \frac{c}{2 \abs{\omega}} k_\bot^2.
\end{equation}
Note that by plugging \ref{4.9} into equations \ref{4.5} - \ref{4.7} one gets the paraxial approximation.

In the last part of this section, the practical algorithm for implementation of the boundary conditions based on the previously derived solution of the Maxwell's equations is presented. Assume that the laser beam propagates in a forward direction along the z-axis. In the beginning, it is necessary to prescribe the electric laser field $ \vec{E}_{0, \bot} (\vec{r}_{\bot}, t) $ in the plane $ \mathcal{P} $ at $ z = z_0 $. Note, that it can be defined by arbitrary function of space and time. The goal is then to find the fields $ \vec{E}_{\mathrm{B}} (\vec{r}_{\bot}, t) $ and $ \vec{B}_{\mathrm{B}} (\vec{r}_{\bot}, t) $ at the corresponding boundary $ z = z_\mathrm{B} $.

Consider that the transverse part of simulation domain is made of equidistant rectangular grid described by $ x^{i} $, $ y^{j} $, where $ i, j \in \left\lbrace 1, \ldots, N_{x, y} \right\rbrace $, and the grid steps $ \delta x $, $\delta y $. The simulation time $ t^{n} $, where $ n \in \left\lbrace 1, \ldots, N_{t} \right\rbrace $, is also divided into equidistant time steps of size $ \delta t $.

The algorithm allows to calculate fields $ \vec{E}_{\mathrm{B}}^{ij} (t) $ and $ \vec{B}_{\mathrm{B}}^{ij} (t) $ for any given time $ t $ from the interval $ \left[ t^{1} - \frac{z_\mathrm{B} - z_0}{c}, t^{N_t} - \frac{z_\mathrm{B} - z_0}{c} \right] $. In order to preserve clarity, the algorithm below is given in the exact form as in the original paper \cite{Thiele2016}.

\begin{enumerate}
	\item Calculate $ \hat{\vec{E}}_{0, \bot}^{ijn} $ via discrete Fourier transforms in time:
	\begin{equation}
	\omega^n = \frac{2 \pi}{N_t \delta t} \left( -\frac{N_t}{2} + n \right),
	\end{equation}
	\begin{equation}
	\hat{\vec{E}}_{0, \bot}^{ijn} = \frac{\delta t}{2 \pi} \sum_{l=1}^{N_t} \vec{E}_{0, \bot}^{ijl} \e^{\i \omega^n t^l}, \quad n \in \left\lbrace 1, \dots, N_t \right\rbrace.
	\end{equation}
	\item Calculate $ \bar{\vec{E}}_{0, \bot}^{ijn} $ via two-dimensional discrete Fourier transforms in transverse space:
	\begin{equation}
	k_x^i = \frac{2 \pi}{N_x \delta x} \left( - \frac{N_x}{2} + i\right), \quad k_y^j = \frac{2 \pi}{N_y \delta y} \left( - \frac{N_y}{2} + j\right),
	\end{equation}
	\begin{equation}
	\bar{\vec{E}}_{0, \bot}^{ijn} = \frac{\delta x \delta y}{(2 \pi)^2} \sum_{l, m = 1}^{N_x, N_y} \hat{\vec{E}}_{0, \bot}^{lmn} \e^{- \i \: \left(k_x^i x^l + k_y^j y^m \right)}, \quad i, j \in \left\lbrace 1, \dots, N_{x, y} \right\rbrace.
	\end{equation}
	\item Calculate transverse electric field components at the boundary $ z = z_\mathrm{B} $:
	\begin{equation}
	k_z^{ijn} = \Re \sqrt{\frac{(\omega^n)^2}{c^2} - (k_x^i)^2 - (k_y^j)^2},
	\end{equation}
	\begin{equation}
	\bar{\vec{E}}_{\mathrm{B}, \bot}^{ijn} =
	\begin{cases} \bar{\vec{E}}_{0, \bot}^{ijn} \e^{\i k_z^{ijn}(z_\mathrm{B} - z_0)} & \text{for} \ k_z^{ijn} > 0 \\ 0 & \text{for} \ k_z^{ijn} = 0 \end{cases}.
	\end{equation}
	\item Calculate longitudinal electric field components at the boundary $ z = z_\mathrm{B} $:
	\begin{equation}
	\bar{E}_{\mathrm{B}, z}^{ijn} = \begin{cases} -\frac{k_x^i \bar{E}_{\mathrm{B}, x}^{ijn} + k_y^j \bar{E}_{\mathrm{B}, y}^{ijn}}{k_z^{ijn}} & \text{for} \ k_z^{ijn} > 0 \\ 0 & \text{for} \ k_z^{ijn} = 0 \end{cases}.
	\end{equation}
	\item Calculate the magnetic field at the boundary $ z = z_\mathrm{B} $:
	\begingroup
	\renewcommand*{\arraystretch}{1.7}
	\begin{equation}
	\mathbb{R}^{ijn} =  \begin{pmatrix}
	-k_x^i k_y^j & (k_x^i)^2 - (\omega^n)^2/c^2 \\
	(\omega^n)^2/c^2 - (k_y^j)^2 & k_x^i k_y^j \\
	-k_y^j k_z^{ijn} & k_x^i k_z^{ijn} 
	\end{pmatrix},
	\end{equation} 
	\endgroup
	\begin{equation}
	\bar{\vec{B}}_{\mathrm{B}}^{ijn} = \begin{cases} (\omega^n k_z^{ijn})^{-1} \mathbb{R}^{ijn} \bar{\vec{E}}_{\mathrm{B}, \bot}^{ijn} & \text{for} \ k_z^{ijn} > 0 \\ 0 & \text{for} \ k_z^{ijn} = 0 \end{cases}.
	\end{equation}
	\item Calculate $ \hat{\vec{E}}_{\mathrm{B}}^{ijn} $, $ \hat{\vec{B}}_{\mathrm{B}}^{ijn} $ via two-dimensional inverse discrete Fourier transforms:
	\begin{equation}
	\hat{\vec{E}}_{\mathrm{B}}^{ijn} = \frac{(2 \pi)^2}{N_x N_y \delta x \delta y} \sum_{l, m = 1}^{Nx, Ny} \bar{\vec{E}}_{\mathrm{B}}^{lmn} \e^{\i(k_x^l x^i + k_y^m y^j)},
	\end{equation}
	\begin{equation}
	\hat{\vec{B}}_{\mathrm{B}}^{ijn} = \frac{(2 \pi)^2}{N_x N_y \delta x \delta y}  \sum_{l, m = 1}^{Nx, Ny} \bar{\vec{B}}_{\mathrm{B}}^{lmn} \e^{\i(k_x^l x^i + k_y^m y^j)}.
	\end{equation}
	\item Calculate $ \vec{E}_{\mathrm{B}}^{ij}(t) $, $ \vec{B}_{\mathrm{B}}^{ij}(t) $ for any given time $ t \in [t^{1} - \frac{z_{\mathrm{B}} - z_{0}}{c}, t^{N_{t}}  - \frac{z_{\mathrm{B}} - z_{0}}{c}] $.
	\begin{equation}
	\vec{E}_{\mathrm{B}}^{ij} (t) = \frac{2 \pi}{N_t \delta t} \sum_{n = 1}^{N_t} \hat{\vec{E}}_{\mathrm{B}}^{ijn} \e^{-\i \omega^n t},
	\end{equation}
	\begin{equation}
	\vec{B}_{\mathrm{B}}^{ij} (t) = \frac{2 \pi}{N_t \delta t} \sum_{n = 1}^{N_t} \hat{\vec{B}}_{\mathrm{B}}^{ijn} \e^{-\i \omega^n t}.
	\end{equation}
\end{enumerate}

\section{Implementation}
One of the main goals of this work has been to implement the algorithm mentioned in the previous section, to evaluate its correctness in several test simulations and finally, to exploit resulting implementation for simulations of tightly focused Gaussian beams in laser-matter interaction. The main requirement on implementation was fast integration with the 2D version of particle-in-cell simulation code EPOCH \cite{bennett}. For this reason, several possible solutions have been taken into account.

The final decision was to create a static library, which will be able to compute desired quantities and will provide functions for communication with the main simulation code. The essential advantage is that it could be basically linked with any laser-plasma simulation code. Also, since it is necessary to call only two additional functions, the instrumentation will be fast, easy and the main simulation code will not be excessively disturbed. Furthermore, the implementation itself come with the CMake \cite{Cmake2012} support, which simplify the compilation process using platform and compiler independent configuration files.

The library has been written in C++ language in the object-oriented style, so the algorithm can be easily extended to the three dimensional geometry. In order to speedup the whole underlying computation, the algorithm has been parallelized using hybrid techniques. The time domain has been decomposed into the stripes corresponding to individual computational processes, the communication between these processes is ensured by MPI library \cite{MPI1994}. Furthermore, the computationally most expensive cycles are parallelized using OpenMP implementation of multi-threading \cite{OpenMP1998}. Later on, the speedup and parallel scaling performance will be briefly discussed.

Fourier transforms form the core of the computational process and their performance is crucial for the overall performance of the code. For this reason, many currently available libraries have been considered. Eventually, the Fourier transforms in the algorithm can be computed using FFTW \cite{Frigo1998} library, Intel$ ^{\scriptsize \textregistered} $ MKL \cite{MKL2009} library or it is also possible to directly evaluate the formulas without using any additional library. User specifies his option before the compilation of the code. Regarding both libraries, a threaded versions of 1D in-place complex fast Fourier transforms have been used throughout the code. According to several measures, there is no significant difference between the speed of both libraries. On Intel$ ^{\scriptsize \textregistered} $ architectures, the MKL library \cite{MKL2009} has slightly better performance though.

One potential bottleneck could happen during the computation of spatial Fourier transforms since the arrays with spatial data are decomposed into different processes. The cluster versions of functions performing the Fourier transforms have been tested, however they did not bring any significant speedup. The reason is as follows. They require to have the global array in memory and use its own distribution which involves overlapping. Since the size of global arrays is usually not so large and since it is necessary to perform a lot of different Fourier transforms, the majority of computational time is spent rather for communication, mainly if many of computational cores are used.

This issue has been solved by gathering the data on master process, performing the Fourier transforms in space by only one processor and scattering the data back to corresponding processes. This is the reason why the code does not scale well, however, the time to compute all desired quantities is in most cases negligible in comparison with the time required by the main simulation cycle. Nevertheless, this could be a suggestion for future improvement.

Since it is necessary to compute the whole time evolution of the laser field at boundary for each grid point before the simulation starts, the resulting amount of data can be significantly large and does not have to fit in a computer RAM. Thus, it is inevitable to dump the data into a file, which will be then accessed by the main simulation code. Due to the performance purposes, each computational process stores its data into a shared file with corresponding offset and in binary coding. Therefore, the output operations are as fast as possible and save the storage resources. Library then provide a function which allows to seek an arbitrary position in a file. This function is then called each time step of the main simulation loop to fill the laser source arrays with all the relevant data. This way of accessing data does not cause any significant slowdown or memory overhead.

The EPOCH \cite{bennett} code require only transverse components of the laser electric field, all other quantities are computed by the FDTD solver \cite{ruhl}. The implementation of the library allows full connection with EPOCH \cite{bennett}. In practice, if user wants to simulate tight-focusing, it is necessary to enable the corresponding flag as a compile-time option and then to specify all required parameters in the input file. The code then automatically computes all necessary data. It works generally regardless the number of lasers in the simulation or boundaries that they are attached to.

The current version of library does not work for obliquely incident laser pulses, because in this case one cannot exploit the advantage of an efficient computation with fast Fourier transforms. However, the code allows to compute the laser fields at boundary by evaluating Fourier integrals directly, so it could be easily extended. Second, it is at the moment possible to simulate only Gaussian laser pulses. However, user can prescribe its own shape and position of the beam in focal plane by modifying corresponding part of the code.

Several most important data structures, functions and methods that form the core of the library for tight-focusing can be seen in appendix B.

\section{Evaluation}
\floatsetup[figure]{style=plain, subcapbesideposition=top}
\begin{figure}[h!]
	\centering
	\sidesubfloat[]{{\includegraphics[width=0.44\linewidth]{./img/parax/Ey_focus.pdf}}}
	\hspace{2mm}
	\sidesubfloat[]{{\includegraphics[width=0.44\linewidth]{./img/parax/Ex_focus.pdf}}}\\
	\sidesubfloat[]{{\includegraphics[width=0.44\linewidth]{./img/lbcs/Ey_focus.pdf}}}
	\hspace{2mm}
	\sidesubfloat[]{{\includegraphics[width=0.44\linewidth]{./img/lbcs/Ex_focus.pdf}}}
	\caption{Transverse ($ E_{y} $) and longitudinal ($ E_{x} $) electric laser field components captured at the time step of their maximal intensity at the focal spot. The cases \textbf{(a)}, \textbf{(b)} correspond to the laser pulse initialized using the paraxial approximation, whilst \textbf{(c)}, \textbf{(d)} come from the simulation where the beam at the boundary is given by the Maxwell consistent approach. In the case of paraxial approximation, both components reveal strong distortions and asymmetry, their focal spot is located about $ \mathrm{1 \lambda} $ closer to the left boundary than specified and the corresponding amplitude is significantly lower. The laser has been attached to the left hand side boundary.}
	\label{fig:1}
\end{figure}

To evaluate the correctness of the algorithm presented in the previous section of this chapter as well as to demonstrate the drawbacks of the paraxial approximation, several test simulations in 2D geometry have been performed. In the following text, a two limit cases are presented. The first pair of simulations employs tightly focused Gaussian laser beam with the size at focus comparable with the center laser wavelength, whilst the second one shows the case of the Gaussian beam with the size at focus one order of magnitude larger than the center laser wavelength, where both approaches should return identical results. Note, that all the simulations have been computed using 2D version of PIC code EPOCH [source] instrumented with library for tight-focusing.

\floatsetup[figure]{style=plain, subcapbesideposition=top}
\begin{figure}[h!]
	\centering
	\sidesubfloat[]{{\includegraphics[width=0.4\linewidth]{./img/parax/Ey_focus_trans.pdf}}}
	\hspace{2mm}
	\sidesubfloat[]{{\includegraphics[width=0.4\linewidth]{./img/parax/Ey_focus_long.pdf}}}\\
	\sidesubfloat[]{{\includegraphics[width=0.4\linewidth]{./img/lbcs/Ey_focus_trans.pdf}}}
	\hspace{2mm}
	\sidesubfloat[]{{\includegraphics[width=0.4\linewidth]{./img/lbcs/Ey_focus_long.pdf}}}
	\caption{Transverse \textbf{(a)}, \textbf{(c)} and longitudinal \textbf{(b)}, \textbf{(d)} slices of the transverse electric laser field ($ E_{y} $) at the time step when it reaches maximal intensity at the focal spot. The cases \textbf{(a)}, \textbf{(b)} correspond to the laser pulse initialized using the paraxial approximation, whilst \textbf{(c)}, \textbf{(d)} come from the simulation where the beam at the boundary is given by the Maxwell consistent approach. In the case of paraxial approximation, one can clearly see strong side-wings in the spatial beam profile \textbf{(a)} as well as the asymmetry of the field in the longitudinal line-out \textbf{(b)}.}
	\label{fig:2}
\end{figure}

The simulation of a tightly focused Gaussian beam is considered first. The p-polarized laser pulse with center wavelength $ \lambda = 1 \ \mathrm{\mu m} $ propagates from left hand side boundary to the right. Its duration has been chosen to $ \tau = 20 \ \mathrm{fs} $ in FWHM and amplitude $ E_0 = 1 \cdot 10^{15} \ \mathrm{V/m} $. The beam waist $ w_0 = 0.7 \ \mathrm{\mu m} $ is shorter than the laser wavelength, which implies that non-negligible parts of $ \bar{\vec{E}}_{0, \bot}(k_x, \omega) $ are evanescent. The focus is located at a distance $ x = 8 \ \mathrm{\mu m} $ from the boundary that the laser is attached to.

The size of the simulation domain is $ 16 \lambda \times 48 \lambda $, with 100 cells per laser wavelength in both directions, thus $ \Delta x = \Delta y = \lambda/100 = 10 \ \mathrm{nm} $. The simulation time step is chosen to satisfy the CFL condition [source] $ \Delta t = 0.95 \sqrt{2} \lambda/ 100 c \approx 0.05 \ \mathrm{fs} $ and the whole simulation time is $ t = 150 \ \mathrm{fs} $. The pulse propagates in vacuum in order to get rid of all effects that could be potentially caused by plasma. All the simulation parameters can be found in the attached input files in the appendix A.

\floatsetup[figure]{style=plain, subcapbesideposition=top}
\begin{figure}[h!]
	\centering
	\sidesubfloat[]{{\includegraphics[width=0.44\linewidth]{./img/lbcs/Ey_boundary_time.pdf}}}
	\hspace{2mm}
	\sidesubfloat[]{{\includegraphics[width=0.44\linewidth]{./img/lbcs/Ex_boundary_time.pdf}}}
	\caption{The time evolution of transverse ($ E_{y} $) \textbf{(a)} and longitudinal ($ E_{x} $) \textbf{(b)} electric laser field components at the boundary that the laser is attached to. Both components has been calculated according to the Maxwell consistent approach.}
	\label{fig:3}
\end{figure}

\begin{figure}[h!]
	\centering
	\sidesubfloat[]{{\includegraphics[width=0.45\linewidth]{./img/lbcs/divergence.pdf}}}
	\hspace{2mm}
	\sidesubfloat[]{{\includegraphics[width=0.43\linewidth]{./img/parax/focus_shift.pdf}}}
	\caption{\textbf{(a)} Graph of the spot size parameter $ w(x) $ of the beams with $ w_0 = 1 \: \lambda $ and $ 0.5 \: \lambda $ calculated using Maxwell consistent approach. From the plotted lines, one can roughly estimate the beam divergence angle $ \Theta $. The divergence angles for both beams are surprisingly in a good accordance with the formula for divergence angles of corresponding Gaussian beams. \textbf{(b)} Dependency of the absolute focal point shift on the beam waist $ w_0 $ for the laser beam initialized using the paraxial approximation. One can see sharp increase of the error for $ w_0 < \lambda $. In both cases, values have been dumped at discrete time intervals and interpolated.}
	\label{fig:7}
\end{figure}

\floatsetup[figure]{style=plain, subcapbesideposition=top}
\begin{figure}[h!]
	\centering
	\sidesubfloat[]{{\includegraphics[width=0.44\linewidth]{./img/lbcs/Ey_boundary_trans.pdf}}}
	\hspace{1mm}
	\sidesubfloat[]{{\includegraphics[width=0.44\linewidth]{./img/lbcs/Ey_boundary_long.pdf}}}
	\caption{Transverse \textbf{(a)} and longitudinal \textbf{(b)} slice of the transverse electric laser field ($ E_{y} $) when it reaches its maximal intensity at the front (blue) and rear (red) boundary. The results come from the simulation where the laser beam at the boundary has been calculated using the Maxwell consistent approach. For better comparison, the field at the rear boundary in \textbf{(b)} has been horizontally flipped. The exact match between the field shapes at a different time steps of simulation proves the correctness of the laser beam propagation.}
	\label{fig:4}
\end{figure}

In the following paragraph, the results of the first simulation are discussed in more details. Fig. \ref{fig:1} shows the transverse and longitudinal electric field components at their maximal intensity at focus for two cases. First, for the laser beam which has been initialized using the paraxial approximation (fig. \ref{fig:1} - a, b), i.e. the electric field at the boundary is given by the equation \ref{1.50}, and second, according to the approach consistent with the Maxwell's equations (fig. \ref{fig:1} - c, d). In the case of paraxial approximation, one can clearly see strong distortions and asymmetry in the shape of both electric field components. In addition, the focus location is shifted about $ 1 \ \mathrm{\mu m} $ closer to the left boundary and the corresponding amplitude at focus is less than half the required value. In contrast, the fields produced by the simulation using Maxwell consistent calculation of laser fields at boundary are symmetric with respect to the focal spot and without any distortions. Furthermore, the focus location as well as the amplitude fulfills the initial requirements precisely.

Fig. \ref{fig:2} shows transverse and longitudinal slices of transverse electric field component at focus for the case of the laser beam initialized using the paraxial approximation (fig. \ref{fig:2} - a, b) as well as for the case where the beam at the boundary is given by the Maxwell consistent approach (fig. \ref{fig:2} - c, d). For the case of paraxial approximation, one can clearly see the asymmetry of the field shape in the longitudinal slice (fig. \ref{fig:2} - b), which consequently leads to a decrease of the amplitude at focus and to the strong side-wings in the spatial beam profile, as might be better seen from the transverse slice (fig. \ref{fig:2} - a). On the other hand, Maxwell consistent approach calculates fields of perfect symmetry with respect to the focal spot (fig. \ref{fig:2} - c) and no side-wings or distortions are present (fig. \ref{fig:2} - d).

\begin{figure}[h!]
	\centering
	\sidesubfloat[]{{\includegraphics[width=0.445\linewidth]{./img/lbcs/amplitude.pdf}}}
	\hspace{2mm}
	\sidesubfloat[]{{\includegraphics[width=0.435\linewidth]{./img/lbcs/intensity.pdf}}}
	\caption{\textbf{(a)} Graph of the transverse ($ E_{y} $) electric laser field amplitude with respect to the distance from the focal spot. The values on vertical axis are given in a percentage of the amplitude at focus $ E_{y, 0} $. \textbf{(b)} Graph of the maximal instantaneous laser intensity with respect to the distance from focal spot. The values on vertical axis are given in a percentage of the laser intensity at focus $ I_{0} $. In both cases, values have been computed using the Maxwell consistent approach, dumped at discrete time intervals and interpolated.}
	\label{fig:8}
\end{figure}

\floatsetup[figure]{style=plain, subcapbesideposition=top}
\begin{figure}[h!]
	\centering
	\sidesubfloat[]{{\includegraphics[width=0.45\linewidth]{./img/parax/Ey_focus_5mic.pdf}}}
	\hspace{2mm}
	\sidesubfloat[]{{\includegraphics[width=0.44\linewidth]{./img/parax/Ex_focus_5mic.pdf}}}\\
	\sidesubfloat[]{{\includegraphics[width=0.44\linewidth]{./img/lbcs/Ey_focus_5mic.pdf}}}
	\hspace{2mm}
	\sidesubfloat[]{{\includegraphics[width=0.43\linewidth]{./img/lbcs/Ex_focus_5mic.pdf}}}
	\caption{Transverse ($ E_{y} $) and longitudinal ($ E_{x} $) electric laser field components captured at the time step of their maximal intensity at the focal spot. The cases \textbf{(a)}, \textbf{(b)} correspond to the laser pulse initialized using the paraxial approximation, whilst \textbf{(c)}, \textbf{(d)} come from the simulation where the beam at the boundary is given by the Maxwell consistent approach. The size of the focus has been chosen to be one order of the magnitude larger than the center laser wavelength. One can clearly see, that there is no significant difference between the shapes of the electric field components.}
	\label{fig:5}
\end{figure}

\floatsetup[figure]{style=plain, subcapbesideposition=top}
\begin{figure}[h!]
	\centering
	\sidesubfloat[]{{\includegraphics[width=0.44\linewidth]{./img/parax/Ey_focus_trans_comparison.pdf}}}
	\hspace{2mm}
	\sidesubfloat[]{{\includegraphics[width=0.44\linewidth]{./img/parax/Ey_focus_long_comparison.pdf}}}
	\caption{Transverse \textbf{(a)} and longitudinal \textbf{(b)} slices of the transverse electric laser field ($ E_{y} $) at the time step when it reaches maximal intensity at focus. Red lines correspond to the laser pulse initialized using the paraxial approximation, whilst blue lines come from the simulation where the beam at the boundary is calculated by the Maxwell consistent approach (LBCS). The size of the focus has been chosen to be one order of magnitude larger than the center laser wavelength. In the case of paraxial approximation, the focus is slightly shifted closer to the left boundary \textbf{(b)}, otherwise the size of the focus as well as the amplitude is correct for both cases \textbf{(a)}.}
	\label{fig:6}
\end{figure}

In fig. \ref{fig:3} one can look through the time evolution of transverse (fig. \ref{fig:3} - a) and longitudinal (fig. \ref{fig:3} - b) electric field components at the boundary as computed using the Maxwell consistent approach. Note that one must chose carefully the transverse size of the domain, since the beam width at boundary may be much larger than at focus because of a diffraction. To estimate the beam width at the boundary, one has to know the beam divergence angle $ \Theta $. In fig. \ref{fig:7} - a, one can find a graph of the spot size parameter $ w $ as calculated using the Maxwell consistent approach for the beams focused to $ w_0 = 1 \ \lambda $ and $ 0.5 \ \lambda $. From the plotted lines, one can roughly estimate the beam divergence angle $ \Theta $. For the beam focused to $ w_0 = 1 \ \lambda $, the beam divergence angle has been around $ \Theta = 18.2 ^{\circ} $ which is almost identical value as for the Gaussian beam of the same parameters. For the beam with $ w_0 = 0.5 \ \lambda $ the divergence has been estimated to $ \Theta = 32.9 ^{\circ} $, while the Gaussian beam with the same size at focus has divergence $ \Theta = 36.5 ^{\circ} $. Consequently, in spite of the fact that the divergence of the tightly focused beams calculated using the Maxwell consistent approach is a little bit lower, both approaches are in quite a good accordance regarding this parameter and one can use the angle $ \Theta $ given by the equation \ref{1.38} as a rough estimate even for tight-focusing.

To evaluate a correctness of the propagation of the beam prescribed using the Maxwell consistent approach, several criteria has been defined. The correctness of the amplitude and beam waist as well as the right focus location has already been verified. Additional criteria has been set on a beam symmetry. In fig. \ref{fig:4}, one can find a comparison of the transverse electric laser field component when it achieves its maximal intensity at front and rear boundary. One can clearly see the exact match between the field shapes at different time steps of the simulation in transverse (fig. \ref{fig:4} - a) and longitudinal (fig. \ref{fig:4} - b) slices. Moreover, all the aforementioned criteria has been fulfilled also in other simulations with different input parameters that are not presented here. Consequently, these observations prove the correctness of the simulation at least for the tightly focused Gaussian laser beams.

In fig. \ref{fig:7} - b, one may find the estimate of the absolute focal point shift with respect to the prescribed position of the beam waist. The interpolated values come from the simulations where the laser beam has been initialized using paraxial approximation. One can clearly see the sharp increase of the error for $ w_0 < \lambda $. On the other hand, for $ w_0 = 5 \ \lambda $ the shift is around $ 0.2 \ \lambda $ which is negligible in comparison with the Rayleigh range.

In fig. \ref{fig:8} - a, one can see the amplitude of the transverse electric laser field with respect to the distance from the focal spot calculated using Maxwell consistent approach. This could be particularly useful for experiments since it is not always easy to focus the beam onto the target surface perfectly. Sometimes, one might be more interested in the laser intensity rather than the amplitude of the electric field. For this reason, fig. \ref{fig:8} - b shows the instantaneous maximal laser intensity in the narrower spatial interval calculated from the values of the electric field amplitude.

Up to now, we have demonstrated the results of simulations with tight-focusing. In the following, comparison of the two numerical methods for the beam focused to a larger spot size characterized by $ w_0 = 5 \ \lambda $ is presented. Except for the beam waist, all the input parameters remained the same. The parameter $ w_0 = 5 \ \lambda $ is about the limit case for the beam initialized using the paraxial approximation. Thus, one expects that the simulation results of both methods should be almost identical.

Similarly as in fig. \ref{fig:1}, fig. \ref{fig:5} shows again transverse and longitudinal electric field components at their maximal intensity at focus for both cases, laser beam initialized using the paraxial approximation (fig. \ref{fig:5} - a, b) and according to the approach consistent with the Maxwell's equations (fig. \ref{fig:5} - c, d). Here, one cannot register any difference between the results corresponding to both approaches. Also, the transverse slice of the transverse electric laser field component at focus (fig. \ref{fig:6} - a) shows the correct shape and amplitude for both cases. The longitudinal slice (fig. \ref{fig:6} - b), however, points out the fact that the location of the focal spot is still a little bit shifted closer to the left hand side boundary. Nevertheless, this difference could be neglected in practice. At the end of the day, the beam diameter at focus should be at least one order of magnitude larger than the center laser wavelength for the Gaussian beams propagating in the paraxial approximation.

In conclusion, one should be aware that the propagation of tightly focused laser pulses cannot be described by paraxial approximation. It has been shown above that for the beams focused to a spot with the size comparable to a center laser wavelength, paraxial approximation leads to a shifted location of the focus, asymmetric laser field profiles with distortions and lower amplitude. These deviations are far from negligible and have without doubt a strong impact on the laser-matter or laser-plasma interaction results. On the other hand, the propagation of tightly focused Gaussian laser beams prescribed at boundaries according to the Maxwell consistent approach has been proven to be correct.

It has been also verified that the paraxial approximation can be safely used when the beam size at focus is about one order of magnitude larger than the center laser wavelength, such that $ w_0 \geq 5 \ \lambda $. In this case, the laser fields are symmetric with respect to the focal point, the laser intensity fulfills specified requirements precisely, only the location of the focus is shifted about $ 0.2 \ \lambda $ closer to the boundary at which the laser is attached to. However, in this case, the simulations generally produce physically relevant results, therefore the effect of the focal point shift can be neglected in practice.

Regarding the validity and trustworthiness of simulation results, no exact threshold for the size of the focal spot for laser beams initialized using the paraxial approximation has been determined. The rate of unphysical effects also depends on investigated phenomena and other laser pulse properties, such as intensity or duration. Thus the frequently used simulation set-up for beams with the waist $ w_0 = 3 \ \lambda $ can be in many situations justified.

\section{Overview of experimental methods}
In this section, basic approaches how to focus a laser beam in experiments are discussed. Before providing a brief overview of currently used methods, it is necessary to introduce several most fundamental parameters that describe any optical system. First, one defines a numerical aperture of an optical system $ N\!A $, which is a dimensionless number characterizing the range of angles over which the system can accept or emit light. Second characteristic of any optical system is a focal length $ f $ describing a distance between the center of the aperture and the point in space over which collimated light rays are brought to a focus. Finally, one defines a f-number $ f/\# $ as a ratio of the focal length $ f $ to the size of the aperture $ D $. The f-number is dimensionless and stands for a quantitative measure of a speed of the optical system.

In experiments, focusing of laser beams is usually achieved by means of an optical system, such as a lens or a curved mirror. However, when dealing with a short pulses, lenses are not preferred because they may affect the beam by frequency-dependent effects, such as chromatic aberration and other nonlinear optical distortions. On the other hand, reflective optics are generally able to produce a diffraction limited image without chromatic effects.

An off-axis parabolic mirror is a frequently used tool to focus an incoming collimated beam. It is made by cutting out a small section from a full parabolic mirror and thus it allows to deviate the beam path off the optical axis. Therefore, the focal point is at more accessible location and the target is not blocking the incoming collimated laser beam as in the case of a complete parabolic mirror. Obviously, the off-axis parabolic mirror is able to work reversibly, so it can take the light coming from a point source and produce a collimated beam. These physical properties make the off-axis parabolic mirror a valuable tool for many different optical purposes.

However, as soon as the laser intensity exceeds $ 10^{13} \ \mathrm{W/cm}^{2} $, the surface of any material becomes strongly ionized. Therefore, in order to keep the energy density on the optical components below the damage threshold, the beam diameter has to be increased. This approach inherently imposes limits upon the geometrical characteristics of the focused beam such as the size of the focus. The focal spot can be alternatively decreased by implementing a small f-number focusing optic. However, since their focal length is inevitably very short and therefore they must be placed in close proximity to the interaction region, additional care has to be taken to protect the optical components because they can be easily damaged from debris induced by the exploded target flow. Note that such optics are typically very expensive.

\floatsetup[figure]{style=plain, subcapbesideposition=top}
\begin{figure}[h!]
	\centering
	\sidesubfloat[]{{\includegraphics[width=0.35\linewidth]{./img/exp/diagram.pdf}}}
	\hspace{5mm}
	\sidesubfloat[]{{\includegraphics[width=0.4\linewidth]{./img/exp/photo2.jpg}}}
	\caption{\textbf{(a)} Schematic diagram showing the experimental set-up for tight-focusing. The incoming laser beam is focused by a conventional off-axis parabolic mirror to the first focal point of the ellipsoidal plasma mirror, beyond the light diverges and is reflected from the dense plasma created on the surface of plasma mirror. The beam is then imaged to the second focal spot where the target should be placed. \textbf{(b)} Photo taken at Extreme Light Infrastructure (ELI) laser facility capturing a very long and frustrating procedure of aligning an off-axis parabolic mirror.}
	\label{fig:9}
\end{figure}

In the case of tight-focusing, the conventional solid state optics seems to be inappropriate. Nevertheless, many of the drawbacks mentioned above might be overcome by using a plasma-based focusing optics. As already mentioned in the chapter 2, when the laser pulse with intensity higher than $ 10^{16} \ \mathrm{W/cm}^{2} $ hits a solid target, a thin dense plasma layer is immediately formed on its surface. Under certain conditions, the plasma becomes dense enough and the reflectivity of otherwise transparent target strongly increases. The incident laser light is reflected at the critical plasma density, thus the dense plasma acts as a mirror. These optical components are known as plasma mirrors.

Plasma mirrors are able to operate under much higher energy density than the conventional solid state optics, therefore the optics aperture diameter can be more than one order of magnitude smaller. Consequently, plasma mirrors can be mass-produced at much lower cost, which is crucial since they are single-use and thus have to be replaced after each shot. In addition, the concerns about damage from target debris are naturally eliminated.

To induce focusing, the surface of plasma mirror has to be curved similarly as in conventional optics. However, since the distance between the mirror and the interaction region can be significantly reduced, the corresponding f-number can be extremely small enabling to focus an incident laser beam to a spot size smaller than the center laser wavelength. In addition, plasma mirrors are attractive also due to their ability to improve the temporal and spatial contrast ratio of the laser pulse, which is crucial for many applications of the laser-matter interaction.

The diagram in Fig. \ref{fig:9} - a schematically shows an experimental configuration for tight-focusing. A conventional off-axis parabolic mirror is aligned in such a way that the focus coincides with the first focal point of the ellipsoidal focusing plasma mirror. The ellipsoidal shape of the mirror is chosen because it enables point-to-point imaging, possesses no spherical aberration and allows relatively simple alignment procedure. The light reflected from the plasma mirror is then imaged to the second focal spot. The Fig. \ref{fig:9} - b shows the alignment procedure of the off-axis parabolic mirror in practice.

Alternative to plasma mirrors could be a plasma lens. This concept is based on creating a short plasma channel which is able to guide an intense laser pulse. Similarly as for the plasma mirrors, plasma lens would tolerate laser intensities above the damage threshold for conventional solid state optics and allow to manipulate with the laser beam in close proximity to the interaction region. The focal length of the plasma lens is expected to be independent of the laser intensity as long as the interaction regime is non-relativistic. The plasma lens may, in principle, be tuned by controlling the plasma density. Consequently, it would be possible to change the position of the focal point without physically moving any optical element [source].

The last approach reviewed in this work,

%Lens, off-axis parabola mirror - conventional solid state optic
%ellipsoid plasma mirrors, plasma lens
%wedge targets

\subsection{Off-axis parabolic mirror}
An off-axis parabolic mirror is a frequently used tool to focus an incoming collimated beam. It is made by cutting out a small section from a full parabolic mirror and thus it allows to deviate the beam path off the optical axis. Therefore, the focal point is at more accessible location and the target is not blocking the incoming collimated laser beam as in the case of a complete parabolic mirror. Obviously, the off-axis parabolic mirror is able to work reversibly, so it can take the light coming from a point source and produce a collimated beam. These physical properties make the off-axis parabolic mirror a valuable tool for many different optical purposes.

As the laser intensities exceed $ 10^{13} \ \mathrm{W/cm}^{2} $, the surface of any material becomes strongly ionized. Therefore, in order to keep the energy density on the optical components below the damage threshold, the beam diameter has to be increased. The beam diameter is usually constrained by the cost of the optical components.

The focal spot can be decreased by implementing a small f-number focusing optic. However, since their focal length is inevitably very short and therefore they must be placed in close proximity to the interaction region, additional care has to be taken to protect the optical components because they can be easily damaged from debris induced by the exploded target flow. Note that such optics are typically very expensive.

\subsection{Focusing plasma mirror}
In the case of tight-focusing, the conventional solid state optics seems to be inappropriate. Nevertheless, many of the drawbacks mentioned above might be overcome by using a plasma-based focusing optics. As already mentioned in the chapter 2, when the laser pulse with intensity higher than $ 10^{16} \ \mathrm{W/cm}^{2} $ hits a solid target, a thin dense plasma layer is immediately formed on its surface. Under certain conditions, the plasma becomes dense enough and the reflectivity of otherwise transparent target strongly increases. The incident laser light is reflected at the critical plasma density, thus the dense plasma acts as a mirror. For this reason, such optical components are called plasma mirrors.

Plasma mirrors are able to operate under much higher energy density than the conventional solid state optics, therefore the diameter of the optics aperture can be more than one order of magnitude smaller. Consequently, plasma mirrors can be mass-produced at much lower cost, which is crucial since they are single-use and thus have to be replaced after each shot. In addition, the concerns about damage from target debris are naturally eliminated.

To induce focusing, the surface of plasma mirror has to be curved similarly as in the case of conventional optics. However, since the distance between the mirror and the interaction region is in practice unlimited, the corresponding f-number can be extremely small enabling to focus an incident laser beam theoretically to a spot size smaller than the laser wavelength. In addition, plasma mirrors are attractive also due to their ability to improve the temporal and spatial contrast ratio of the laser pulse, which is crucial for many applications of the laser-matter interaction.

The diagram in Fig. \ref{fig:9} - a schematically shows an experimental configuration for tight-focusing. A conventional off-axis parabolic mirror is aligned in such a way that the focus coincides with the first focal point of the ellipsoidal focusing plasma mirror. The ellipsoidal shape of the mirror is usually chosen because it enables point-to-point imaging, possesses no spherical aberration and allows relatively simple alignment procedure. The light reflected from the plasma mirror is then imaged to the second focal spot, where the target should be placed. The Fig. \ref{fig:9} - b shows the alignment procedure of the off-axis parabolic mirror in practice.

\subsection{Plasma lens}
Alternative to a plasma mirror could be a plasma lens. This concept is based on creating a short plasma channel which is able to guide an intense laser pulse. The plasma channel can be formed by variations in either the laser intensity across the beam regions or a plasma density. In the higher laser intensity region, plasma is pushed aside in the radial direction of the beam due to the ponderomotive force (see chapter 2). This effect reduces the plasma density locally and consequently increases the index of refraction of the plasma. The resulting index of refraction is seen by the laser pulse as a focusing lens, thus prevents it from further spreading. Note that apart from ponderomotive force, there is a variety of other mechanisms that lead to a change of the refractive index of plasma. These include collisions, a thermal, or relativistic effects.

Similarly as for the plasma mirrors, plasma lenses would tolerate energy densities above the damage threshold for conventional solid state optics and allow to manipulate with the laser beam in close proximity to the interaction region. The focal length of the plasma lens is expected to be independent of the laser intensity as long as the interaction regime is non-relativistic. The plasma lens may, in principle, be tuned by controlling the plasma density. Consequently, it would be possible to change the position of the focal point without physically moving any optical element [source].

\subsection{Conical targets}
Recently, it has been also shown by PIC simulations that a tightly focused laser beams could be potentially obtained by using cone-shaped targets. More specifically, the surface of the hollow cone that is open at both ends interacts with the outer parts of incoming collimated laser beam which energy is gradually squeezed due to the multiple reflections. In spite of the laser energy absorption by the target walls, the laser pulse is nonlinearly guided to the cone tip that can result in a highly localized spot of around wavelength radius with the peak intensity amplified by an order of magnitude.

The cone-focusing effect is mainly characterized by the material, opening angle and cone tip size of the target. By controlling these parameters, a laser pulse can be focused efficiently and the quality of the focal spot can be significantly enhanced. However, the tip of the cone channel has to be of a dimension comparable to the laser wavelength, making the manufacturing process difficult at present. Such a small conical channel should be realizable in the near future with the rapid advances in nanofabrication.

%-------------------------------------------------------------------------------

\chapter{2D simulations of focused laser beams}
In this chapter, the results of several large-scale simulations are presented. 

More specifically, 2D3V PIC simulations of tightly-focused Gaussian laser beams interacting with solid targets have been performed within this work. Simulations have been computed using code EPOCH (see chapter 3). 

The results have been processed and thoroughly analyzed. Emphasis has been placed mainly on identifying the influence of the laser beam focal spot size on the energy absorption efficiency and other qualitative as well as quantitative differences.

\section{Simulation model and initial conditions}

\section{Simulation results} 

\begingroup
\renewcommand*{\arraystretch}{1.5}
\begin{table}[h!]
	\centering
	\begin{tabular}{c | c | c | c}
		\multirow{2}{*}{$ w_0 $ [$ \mu\mathrm{m} $]} & \multicolumn{3}{c}{Total absorption of laser energy [\%]} \\ \cline{2-4}
		 & const. E = $ 2.83 \cdot 10^{4} $ J & const. I = $ 10^{20} $ W/cm$^2$ & const. I = $ 10^{21} $ W/cm$^2$ \\ \hline \hline
		0.5 & 20.12 & 16.23 & 31.77 \\ \hline
		1.0 & 9.59 & 9.59 & 23.76 \\ \hline
		2.0 & 5.27 & 8.26 & 23.82 \\ \hline
		4.0 & 3.49 & 8.29 & 23.71 \\
	\end{tabular}
	\caption{A summary of the values of the total laser light absorption in plasma for several different sizes of the focal spot and different laser intensities.}
	\label{table:4}
\end{table}
\endgroup

\floatsetup[figure]{style=plain, subcapbesideposition=top}
\begin{figure}[h!]
	\centering
	\sidesubfloat[]{{\includegraphics[width=0.45\linewidth]{./img/results/i1e20/05/absorp.pdf}}}
	\sidesubfloat[]{{\includegraphics[width=0.45\linewidth]{./img/results/i1e20/2/absorp.pdf}}}\\
	\sidesubfloat[]{{\includegraphics[width=0.45\linewidth]{./img/results/i1e21/05/absorp.pdf}}}
	\sidesubfloat[]{{\includegraphics[width=0.45\linewidth]{./img/results/i1e21/2/absorp.pdf}}}
	\caption{Laser energy absorption in time for the case of simulations with the laser intensity I = $ 10^{20} $ W/cm$^2$ and with the beam waist \textbf{(a)} $ w_0 = 0.5 \ \mu\mathrm{m} $, \textbf{(b)} $ w_0 = 2.0 \ \mu\mathrm{m} $ and for the case of simulations with the laser intensity I = $ 10^{21} $ W/cm$^2$ and with the beam waist \textbf{(c)} $ w_0 = 0.5 \ \mu\mathrm{m} $, \textbf{(d)} $ w_0 = 2.0 \ \mu\mathrm{m} $.}
	\label{fig:10}
\end{figure}

\floatsetup[figure]{style=plain, subcapbesideposition=top}
\begin{figure}[h!]
	\centering
	\sidesubfloat[]{{\includegraphics[width=0.45\linewidth]{./img/results/i1e20/05/x_px.pdf}}}
	\sidesubfloat[]{{\includegraphics[width=0.45\linewidth]{./img/results/i1e20/2/x_px.pdf}}}\\
	\sidesubfloat[]{{\includegraphics[width=0.45\linewidth]{./img/results/i1e21/05/x_px.pdf}}}
	\sidesubfloat[]{{\includegraphics[width=0.45\linewidth]{./img/results/i1e21/2/x_px.pdf}}}
	\caption{Dependency of the x-component of the momentum of electrons on the x-coordinate at the time $ t = 100 \ \mathrm{fs} $ for the case of simulations with the laser intensity I = $ 10^{20} $ W/cm$^2$ and with the beam waist \textbf{(a)} $ w_0 = 0.5 \ \mu\mathrm{m} $, \textbf{(b)} $ w_0 = 2.0 \ \mu\mathrm{m} $ and for the case of simulations with the laser intensity I = $ 10^{21} $ W/cm$^2$ and with the beam waist \textbf{(c)} $ w_0 = 0.5 \ \mu\mathrm{m} $, \textbf{(d)} $ w_0 = 2.0 \ \mu\mathrm{m} $. The colorbars are in logarithmic scale.}
	\label{fig:11}
\end{figure}

\floatsetup[figure]{style=plain, subcapbesideposition=top}
\begin{figure}[h!]
	\centering
	\sidesubfloat[]{{\includegraphics[width=0.45\linewidth]{./img/results/i1e20/05/px_py.pdf}}}
	\sidesubfloat[]{{\includegraphics[width=0.45\linewidth]{./img/results/i1e20/2/px_py.pdf}}}\\
	\sidesubfloat[]{{\includegraphics[width=0.45\linewidth]{./img/results/i1e21/05/px_py.pdf}}}
	\sidesubfloat[]{{\includegraphics[width=0.45\linewidth]{./img/results/i1e21/2/px_py.pdf}}}
	\caption{Dependency of the x-component of the momentum of electrons on the y-component of the  momentum of electrons at the time $ t = 100 \ \mathrm{fs} $ for the case of simulations with the laser intensity I = $ 10^{20} $ W/cm$^2$ and with the beam waist \textbf{(a)} $ w_0 = 0.5 \ \mu\mathrm{m} $, \textbf{(b)} $ w_0 = 2.0 \ \mu\mathrm{m} $ and for the case of simulations with the laser intensity I = $ 10^{21} $ W/cm$^2$ and with the beam waist \textbf{(c)} $ w_0 = 0.5 \ \mu\mathrm{m} $, \textbf{(d)} $ w_0 = 2.0 \ \mu\mathrm{m} $. The colorbars are in logarithmic scale.}
	\label{fig:12}
\end{figure}

\floatsetup[figure]{style=plain, subcapbesideposition=top}
\begin{figure}[h!]
	\centering
	\sidesubfloat[]{{\includegraphics[width=0.445\linewidth]{./img/results/i1e20/05/angles.pdf}}}
	\hspace{1mm}
	\sidesubfloat[]{{\includegraphics[width=0.445\linewidth]{./img/results/i1e20/2/angles.pdf}}}\\
	\sidesubfloat[]{{\includegraphics[width=0.445\linewidth]{./img/results/i1e21/05/angles.pdf}}}
	\hspace{1mm}
	\sidesubfloat[]{{\includegraphics[width=0.445\linewidth]{./img/results/i1e21/2/angles.pdf}}}
	\caption{Distribution of angles determining the direction of movement of electrons (the angle 0 stands for the motion forward) at the time $ t = 100 \ \mathrm{fs} $ for the case of simulations with the laser intensity I = $ 10^{20} $ W/cm$^2$ and with the beam waist \textbf{(a)} $ w_0 = 0.5 \ \mu\mathrm{m} $, \textbf{(b)} $ w_0 = 2.0 \ \mu\mathrm{m} $ and for the case of simulations with the laser intensity I = $ 10^{21} $ W/cm$^2$ and with the beam waist \textbf{(c)} $ w_0 = 0.5 \ \mu\mathrm{m} $, \textbf{(d)} $ w_0 = 2.0 \ \mu\mathrm{m} $. Electrons are divided into three energetic intervals.}
	\label{fig:13}
\end{figure}

\floatsetup[figure]{style=plain, subcapbesideposition=top}
\begin{figure}[h!]
	\centering
	\sidesubfloat[]{{\includegraphics[width=0.445\linewidth]{./img/results/i1e20/dist_e.pdf}}}
	\hspace{1mm}
	\sidesubfloat[]{{\includegraphics[width=0.445\linewidth]{./img/results/i1e20/dist_p.pdf}}}\\[2mm]
	\sidesubfloat[]{{\includegraphics[width=0.445\linewidth]{./img/results/i1e21/dist_e.pdf}}}
	\hspace{1mm}
	\sidesubfloat[]{{\includegraphics[width=0.445\linewidth]{./img/results/i1e21/dist_p.pdf}}}
	\caption{Energy distribution functions of electrons for several different beam waists at the time $ t = 100 \ \mathrm{fs} $ for the case of simulations with the laser intensity \textbf{(a)} I = $ 10^{20} $ W/cm$^2$ and \textbf{(c)} I = $ 10^{21} $ W/cm$^2$. Energy distribution functions of ions for several different beam waists at the time $ t = 150 \ \mathrm{fs} $ for the case of simulations with the laser intensity \textbf{(b)} I = $ 10^{20} $ W/cm$^2$ and \textbf{(d)} I = $ 10^{21} $ W/cm$^2$.}
	\label{fig:14}
\end{figure}

\floatsetup[figure]{style=plain, subcapbesideposition=top}
\begin{figure}[h!]
	\centering
	\sidesubfloat[]{{\includegraphics[width=0.445\linewidth]{./img/results/i1e20/dens.pdf}}}
	\hspace{1mm}
	\sidesubfloat[]{{\includegraphics[width=0.445\linewidth]{./img/results/i1e21/dens.pdf}}}
	\caption{Contours of ion critical density for several different beam waists at the time $ t = 100 \ \mathrm{fs} $ for the case of simulations with the laser intensity \textbf{(a)} I = $ 10^{20} $ W/cm$^2$ and \textbf{(b)} I = $ 10^{21} $ W/cm$^2$.}
	\label{fig:15}
\end{figure}

\floatsetup[figure]{style=plain, subcapbesideposition=top}
\begin{figure}[h!]
	\centering
	\sidesubfloat[]{{\includegraphics[width=0.45\linewidth]{./img/results/i1e20/05/ekbar.pdf}}}
	\sidesubfloat[]{{\includegraphics[width=0.45\linewidth]{./img/results/i1e20/2/ekbar.pdf}}}\\[2mm]
	\sidesubfloat[]{{\includegraphics[width=0.45\linewidth]{./img/results/i1e21/05/ekbar.pdf}}}
	\sidesubfloat[]{{\includegraphics[width=0.45\linewidth]{./img/results/i1e21/2/ekbar.pdf}}}
	\caption{Mean kinetic energy of electrons at the time $ t = 120 \ \mathrm{fs} $ for the case of simulations with the laser intensity I = $ 10^{20} $ W/cm$^2$ and with the beam waist \textbf{(a)} $ w_0 = 0.5 \ \mu\mathrm{m} $, \textbf{(b)} $ w_0 = 2.0 \ \mu\mathrm{m} $ and for the case of simulations with the laser intensity I = $ 10^{21} $ W/cm$^2$ and with the beam waist \textbf{(c)} $ w_0 = 0.5 \ \mu\mathrm{m} $, \textbf{(d)} $ w_0 = 2.0 \ \mu\mathrm{m} $. The colorbars are in logarithmic scale.}
	\label{fig:16}
\end{figure}

\floatsetup[figure]{style=plain, subcapbesideposition=top}
\begin{figure}[h!]
	\centering
	\sidesubfloat[]{{\includegraphics[width=0.45\linewidth]{./img/results/i1e21/05/ekbar_2.pdf}}}
	\sidesubfloat[]{{\includegraphics[width=0.45\linewidth]{./img/results/i1e21/2/ekbar_2.pdf}}}\\[2mm]
	\caption{Mean kinetic energy of electrons at the time $ t = 80 \ \mathrm{fs} $ for the case of simulations with the laser intensity I = $ 10^{21} $ W/cm$^2$ and with the beam waist \textbf{(a)} $ w_0 = 0.5 \ \mu\mathrm{m} $, \textbf{(b)} $ w_0 = 2.0 \ \mu\mathrm{m} $. The colorbars are in logarithmic scale.}
	\label{fig:17}
\end{figure}

\floatsetup[figure]{style=plain, subcapbesideposition=top}
\begin{figure}[h!]
	\centering
	\sidesubfloat[]{{\includegraphics[width=0.45\linewidth]{./img/results/i1e21/05/jy.pdf}}}
	\sidesubfloat[]{{\includegraphics[width=0.45\linewidth]{./img/results/i1e21/2/jy.pdf}}}\\[2mm]
	\caption{The y-component of the current density $ J_{y} $ at the time $ t = 100 \ \mathrm{fs} $ for the case of simulations with the laser intensity I = $ 10^{21} $ W/cm$^2$ and with the beam waist \textbf{(a)} $ w_0 = 0.5 \ \mu\mathrm{m} $, \textbf{(b)} $ w_0 = 2.0 \ \mu\mathrm{m} $.}
	\label{fig:18}
\end{figure}

\floatsetup[figure]{style=plain, subcapbesideposition=top}
\begin{figure}[h!]
	\centering
	\sidesubfloat[]{{\includegraphics[width=0.4\linewidth]{./img/results/i1e20/05/traj_1.pdf}}}
	\sidesubfloat[]{{\includegraphics[width=0.4\linewidth]{./img/results/i1e20/05/traj_2.pdf}}}\\[2mm]
	\sidesubfloat[]{{\includegraphics[width=0.4\linewidth]{./img/results/i1e20/2/traj_1.pdf}}}
	\sidesubfloat[]{{\includegraphics[width=0.4\linewidth]{./img/results/i1e20/2/traj_2.pdf}}}
	\caption{Two types of trajectories of randomly chosen electron samples for the case of simulations with the laser intensity I = $ 10^{20} $ W/cm$^2$ and with the beam waist \textbf{(a), (b)} $ w_0 = 0.5 \ \mu\mathrm{m} $ and \textbf{(c), (d)} $ w_0 = 2.0 \ \mu\mathrm{m} $. The trajectories are colored according to the Lorentz gamma factor of corresponding particles.}
	\label{fig:19}
\end{figure}

\floatsetup[figure]{style=plain, subcapbesideposition=top}
\begin{figure}[h!]
	\centering
	\sidesubfloat[]{{\includegraphics[width=0.45\linewidth]{./img/results/i1e20/05/abs_ex.pdf}}}
	\sidesubfloat[]{{\includegraphics[width=0.45\linewidth]{./img/results/i1e20/2/abs_ex.pdf}}}\\[2mm]
	\sidesubfloat[]{{\includegraphics[width=0.45\linewidth]{./img/results/i1e21/05/abs_ex.pdf}}}
	\sidesubfloat[]{{\includegraphics[width=0.45\linewidth]{./img/results/i1e21/2/abs_ex.pdf}}}
	\caption{Longitudinal electric field ($ E_{x} $) at the time  $ t = 120 \ \mathrm{fs} $ for the case of simulations with the laser intensity I = $ 10^{20} $ W/cm$^2$ and with the beam waist \textbf{(a)} $ w_0 = 0.5 \ \mu\mathrm{m} $, \textbf{(b)} $ w_0 = 2.0 \ \mu\mathrm{m} $ and for the case of simulations with the laser intensity I = $ 10^{21} $ W/cm$^2$ and with the beam waist \textbf{(c)} $ w_0 = 0.5 \ \mu\mathrm{m} $, \textbf{(d)} $ w_0 = 2.0 \ \mu\mathrm{m} $.}
	\label{fig:20}
\end{figure}

\floatsetup[figure]{style=plain, subcapbesideposition=top}
\begin{figure}[h!]
	\centering
	\sidesubfloat[]{{\includegraphics[width=0.45\linewidth]{./img/results/i1e20/05/fpx.pdf}}}
	\sidesubfloat[]{{\includegraphics[width=0.45\linewidth]{./img/results/i1e20/05/fpy.pdf}}}\\
	\sidesubfloat[]{{\includegraphics[width=0.45\linewidth]{./img/results/i1e20/2/fpx.pdf}}}
	\sidesubfloat[]{{\includegraphics[width=0.45\linewidth]{./img/results/i1e20/2/fpy.pdf}}}
	\caption{The x-component of ponderomotive force $ F_{p, x} $ \textbf{(a)} and the y-component of ponderomotive force $ F_{p, y} $ \textbf{(b)} for the case of the laser beam with intensity I = $ 10^{20} $ W/cm$^2$ and the beam waist $ w_0 = 0.5 \ \mu\mathrm{m} $ propagating in vacuum. The x-component of ponderomotive force $ F_{p, x} $ \textbf{(c)} and the y-component of ponderomotive force $ F_{p, y} $ \textbf{(d)} for the case of the laser beam with intensity I = $ 10^{20} $ W/cm$^2$ and the beam waist $ w_0 = 2.0 \ \mu\mathrm{m} $ propagating in vacuum. (max: \textbf{(a)} 5.1312e-06 \textbf{(b)} 1.5913e-06 ... pomer: x:y = 3.2:1 \textbf{(c)} 5.9528e-06 \textbf{(d)} 4.6193e-07 ... pomer: x:y = 12.8:1)}
	\label{fig:21}
\end{figure}

\section{Simulation model and initial conditions}
intro...
\section{Maxwell's equations}
The electromagnetic field is in general theory represented by two vectors, the intensity of the electric field $ \vec{E}\left( \vec{r}, t \right) $ and the magnetic induction $ \vec{B}\left( \vec{r}, t \right) $. These vectors are considered to be finite and continuous functions of position $ \vec{r} $ and time $ t $. The description of electromagnetic phenomena in classical electrodynamics is provided by the set of well-known Maxwell's equations. The microscopic variant for external sources in vacuum is formulated as follows,
\begin{equation}
\label{1.1}
\div{\vec{E}} = \frac{\rho}{\varepsilon_0},
\end{equation}
\begin{equation}
\label{1.2}
\div{\vec{B}} = 0,
\end{equation}
\begin{equation}
\label{1.3}
\rot{\vec{E}} + \diffp{\vec{B}}{t} = 0,
\end{equation}
\begin{equation}
\label{1.4}
\rot{\vec{B}} - \mu_0 \varepsilon_0 \diffp{\vec{E}}{t}= \mu_0 \vec{J},
\end{equation}
where $ \rho\left( \vec{r}, t \right) $ is total electric charge density and $ \vec{J}\left( \vec{r}, t \right) $ is total electric current density, which is constituted by the motion of a charge. These distributions may be continuous as well as discrete. As might be seen from Maxwell's equations (\ref{1.1} - \ref{1.4}), the charge density is the source of the electric field, whilst the magnetic field is produced by the current density. The lack of symmetry in Maxwell's equations (\ref{1.2}, \ref{1.3} are homogeneous) is caused by the experimental absence of magnetic charges and currents. The universal constants appearing in the Maxwell's equations (\ref{1.1}, \ref{1.4}) are the electric permittivity of vacuum $ \varepsilon_0 $ and the magnetic permeability of vacuum $ \mu_0 $.

The first equation, \ref{1.1}, is Gauss's law for electric field in the differential form. It states that the flux of the electric field through any closed surface is proportional to the total charge inside. The second equation, \ref{1.2}, is Gauss's law for magnetic field. It expresses the fact that there are no magnetic monopoles, so the flux of magnetic field through any closed surface is always zero. The third equation, \ref{1.3}, is Faraday's law describing how the electric field is associated with a time varying magnetic field. And the last equation, \ref{1.4}, is Amp\`ere's law with Maxwell's displacement current, which means that the time varying electric field causes the magnetic field. As a consequence, it predicts the existence of electromagnetic waves that can carry energy and momentum even in a free space.

To describe the effects of an electromagnetic field in the presence of macroscopic substances, the complicated distribution of charges and currents in matter over the atomic scale is not relevant. Thus one shall define a second set of auxiliary vectors that represent fields in which the material properties are already included in an average sense, the electric displacement $ \vec{D}\left( \vec{r}, t \right) $ and the magnetic vector $ \vec{H}\left( \vec{r}, t \right) $,
\begin{equation}
\label{1.5}
\vec{D} = \varepsilon_0 \vec{E} + \vec{P} = \varepsilon \vec{E},
\end{equation}
\begin{equation}
\label{1.6}
\vec{H} = \frac{\vec{B}}{\mu_0} - \vec{M} = \frac{\vec{B}}{\mu},
\end{equation}
where $ \vec{P}\left( \vec{r}, t \right) $ and $ \vec{M}\left( \vec{r}, t \right) $ are the vectors of polarization and magnetization, respectively. Note that the vectors of polarization and magnetization can be interpreted as a density of electric or magnetic dipole moment of the medium, therefore they are definitely associated with the state of a matter and vanish in vacuum. Similarly as in the case of free space, the factors $ \varepsilon $ and $ \mu $ are called electric permittivity of medium and magnetic permeability of medium. In general case, $ \varepsilon $ and $ \mu $ are tensors. The constitutive relations above (\ref{1.5}, \ref{1.6}) hold only if the medium is homogeneous and isotropic. For the sake of simplicity, only such materials will be considered in the following text.

The macroscopic variant of Maxwell's equations is formulated as follows,
\begin{equation}
\label{1.7}
\div{\vec{D}} = \rho,
\end{equation}
\begin{equation}
\label{1.8}
\div{\vec{B}} = 0,
\end{equation}
\begin{equation}
\label{1.9}
\rot{\vec{E}} + \diffp{\vec{B}}{t} = 0,
\end{equation}
\begin{equation}
\label{1.10}
\rot{\vec{H}} - \diffp{\vec{D}}{t} = \vec{J},
\end{equation}
where $ \rho\left(\vec{r}, t \right) $ and $ \vec{J}\left(\vec{r}, t \right) $ now stand for only external electric charge and current density, respectively.

By combining the time derivative of the equation \ref{1.7} with the divergence of the equation \ref{1.10}, one obtains the following relation between the electromagnetic field sources,
\begin{equation}
\label{1.11}
\div{\vec{J}} + \diffp{\rho}{t} = 0.
\end{equation}
The important result \ref{1.11}, which is frequently referred to as the equation of continuity, expresses nothing but the conservation of total electric charge in an isolated system. In other words, the time rate of change of the electric charge in any closed surface is balanced by the electric current flowing through the surface.

\section{Electrodynamic potentials}
The first-order partial differential Maxwell's equations can be effectively converted to a smaller number of second-order equations by introducing electrodynamic potentials. Hence, one can express the electric and magnetic field as follows,
\begin{equation}
\label{1.12}
\vec{E} = -\grad{\Phi} - \diffp{\vec{A}}{t},
\end{equation}
\begin{equation}
\label{1.13}
\vec{B} = \rot{\vec{A}},
\end{equation}
where $ \Phi\left(\vec{r}, t \right) $ is the scalar potential and $ \vec{A}\left(\vec{r}, t \right) $ is the vector potential of the corresponding fields. One can clearly see that using the definitions \ref{1.12}, \ref{1.13}, six vector components are replaced by only four potential functions and two Maxwell's homogeneous equations (\ref{1.8}, \ref{1.9}) are fulfilled identically. 

However, by definitions \ref{1.12}, \ref{1.13}, $ \Phi\left(\vec{r}, t \right) $ and $ \vec{A}\left(\vec{r}, t \right) $ are not defined uniquely, thus an infinite number of potentials which lead to the same fields may be constructed. To avoid that, one has to impose a supplementary condition, for example
\begin{equation}
\label{1.14}
\div{\vec{A}} + \mu \varepsilon \diffp{\Phi}{t} = 0.
\end{equation}
The condition \ref{1.14} is called the Lorenz gauge. Lorenz gauge is commonly used in electromagnetism because its independence of the coordinate system. Furthermore, it leads to the following uncoupled equations,
\begin{equation}
\label{1.15}
\laplace{\Phi} - \mu \varepsilon \diffp[2]{\Phi}{t} = -\frac{\rho}{\varepsilon},
\end{equation}
\begin{equation}
\label{1.16}
\laplace{\vec{A}} - \mu \varepsilon \diffp[2]{\vec{A}}{t} = -\mu \vec{J},
\end{equation}
that are in all respects equivalent to the Maxwell's equations and in many situations much simpler to solve.

Equations \ref{1.15}, \ref{1.16} correspond to the inhomogeneous wave equations for scalar potential $ \Phi\left(\vec{r}, t \right) $ and vector potential $ \vec{A}\left(\vec{r}, t \right) $. Their general solutions are given by the following expressions,
\begin{equation}
\label{1.17}
\Phi\left(\vec{r}, t \right) = \frac{1}{4 \pi \varepsilon} \int \frac{\rho\left(\vec{r^{\: \prime}}, t^{\: \prime} \right)}{\norm{\vec{r} - \vec{r^{\: \prime}}}} \mathrm{d} V,
\end{equation}
\begin{equation}
\label{1.18}
\vec{A}\left(\vec{r}, t \right) = \frac{\mu}{4 \pi} \int \frac{\vec{J}\left(\vec{r^{\: \prime}}, t^{\: \prime} \right)}{\norm{\vec{r} - \vec{r^{\: \prime}}}} \mathrm{d} V,
\end{equation}
where $ \mathrm{d} V $ is a volume element and $ \norm{.} $ stands for the standard Euclidean norm. Note that the solutions \ref{1.17}, \ref{1.18} are dependent only on charge and current densities at position $ \vec{r^{\: \prime}} $ at so-called retarded time $ t^{\: \prime} = t - \sqrt{\mu \epsilon} \norm{\vec{r} - \vec{r^{\: \prime}}} $ which takes into account the finite velocity of the wave. In other words, the fields at the observation point $ \vec{r} $ at the time $ t $ are proportional to the sum of all the electromagnetic waves that leave the source elements at point $ \vec{r^{\: \prime}} $ at the retarded time $ t^{\: \prime} $.

\section{Hertz vectors}
There exists also other possibilities how to express the electromagnetic field. Under ordinary conditions, an arbitrary electromagnetic field may be defined in terms of a single vector function. This may be helpful for solving of many problems of classical electromagnetic theory, particularly the wave propagation.

First, let us introduce the electric Hertz vector $ {\vec{\Pi_e}}\left(\vec{r}, t \right) $ in terms of the scalar and vector potentials,
\begin{equation}
\label{1.19}
\Phi = - \div{\vec{\Pi_e}},
\end{equation}
\begin{equation}
\label{1.20}
\vec{A} = \mu \varepsilon \diffp{\vec{\Pi_e}}{t}.
\end{equation}

Note that the definitions \ref{1.19}, \ref{1.20} are consistent with the Lorenz gauge condition \ref{1.14}. In the absence of magnetization, it might be easily shown that $ \vec{J} = \partial{\vec{P}}/\partial{t} $ and the electric Hertz vector $ {\vec{\Pi_e}}\left(\vec{r}, t \right) $ is governed by an inhomogeneous wave equation
\begin{equation}
\label{1.21}
\laplace{\vec{\Pi_e}} - \mu \varepsilon \diffp[2]{\vec{\Pi_e}}{t} = -\frac{\vec{P}}{\varepsilon}.
\end{equation}

The equation \ref{1.21} is of the same type as the equations \ref{1.15}, \ref{1.16} and has therefore the familiar general solution 
\begin{equation}
\label{1.22}
\vec{\Pi_e}\left(\vec{r}, t \right) = \frac{1}{4 \pi \varepsilon} \int \frac{\vec{P}\left(\vec{r^{\: \prime}}, t^{\: \prime} \right)}{\norm{\vec{r} - \vec{r^{\: \prime}}}} \mathrm{d} V.
\end{equation}

As might be seen form \ref{1.22}, the fields derived from the electric Hertz vector $ {\vec{\Pi_e}}\left(\vec{r}, t \right) $ can be interpreted as being due to a density distribution of electric dipoles. Every solution of \ref{1.22} then uniquely determines the electromagnetic field through
\begin{equation}
\label{1.23}
\vec{E} = \grad{\left(\div{\vec{\Pi_e}}\right)} - \mu \epsilon \diffp[2]{\vec{\Pi_e}}{t},
\end{equation}
\begin{equation}
\label{1.24}
\vec{B} = \mu \varepsilon \left(\rot{\diffp{\vec{\Pi_e}}{t}}\right).
\end{equation}

Second, one may introduce the magnetic Hertz vector $ {\vec{\Pi_m}}\left(\vec{r}, t \right) $ in terms of the scalar and vector potentials by the following expressions,
\begin{equation}
\label{1.25}
\Phi = 0,
\end{equation}
\begin{equation}
\label{1.26}
\vec{A} = \rot{\vec{\Pi_m}}.
\end{equation}
In the absence of polarization, $ \vec{J} = \rot{\vec{M}} $ and the magnetic Hertz vector $ {\vec{\Pi_m}}\left(\vec{r}, t \right) $ defined by \ref{1.25} and \ref{1.26} fulfills an inhomogeneous wave equation
\begin{equation}
\label{1.27}
\laplace{\vec{\Pi_m}} - \mu \varepsilon \diffp[2]{\vec{\Pi_m}}{t} = -\mu \vec{M}.
\end{equation}
As for the previous cases, one may easily find the solution of \ref{1.27},
\begin{equation}
\label{1.28}
\vec{\Pi_m}\left(\vec{r}, t \right) = \frac{\mu}{4 \pi} \int \frac{\vec{M}\left(\vec{r^{\: \prime}}, t^{\: \prime} \right)}{\norm{\vec{r} - \vec{r^{\: \prime}}}} \mathrm{d} V,
\end{equation}
thus the fields derived from the magnetic Hertz vector $ {\vec{\Pi_m}}\left(\vec{r}, t \right) $ may be imagined to be due to a density distribution of magnetic dipoles. Again, every solution of \ref{1.28} uniquely determines the electromagnetic field via
\begin{equation}
\label{1.29}
\vec{E} = \rot{\diffp{\vec{\Pi_m}}{t}},
\end{equation}
\begin{equation}
\label{1.30}
\vec{B} = \rot{\left(\rot{\vec{\Pi_m}}\right)}.
\end{equation}

Note that the above derivations considered electric and magnetic Hertz vectors as a separate quantities. It is also possible, however, to introduce them together in the form of one six-vector [source].

\section{Energy and momentum}
To be able to describe the interaction of electromagnetic field with matter, one has to know the energy distribution throughout the field as well as the momentum balance.

By scalar multiplications of \ref{1.9} by $ \vec{H}\left( \vec{r}, t \right) $, of \ref{1.10} by $ \vec{E}\left( \vec{r}, t \right) $, following subtraction of both obtained equations and using standard vector identities, one gets the expression 
\begin{equation}
\label{1.31}
\vec{E} \cdot \diffp{\vec{D}}{t} + \vec{H} \cdot \diffp{\vec{B}}{t} + \div{\left(\vec{E} \times \vec{H} \right)} = -\vec{E} \cdot \vec{J}.
\end{equation}
The equation \ref{1.31} can be rewritten in the form of conservation law,
\begin{equation}
\label{1.32}
\diffp{u}{t} + \div{\vec{S}} = - \vec{E} \cdot \vec{J},
\end{equation}
where
\begin{equation}
\label{1.33}
u = \frac{1}{2} \left(\vec{E} \cdot \vec{D} + \vec{H} \cdot \vec{B} \right), \quad \vec{S} = \vec{E} \times \vec{H}.
\end{equation}
The quantity $ u\left( \vec{r}, t \right) $ in \ref{1.33} describes the total energy density in the field and $ \vec{S}\left( \vec{r}, t \right) $ is so-called Poynting vector which represents the energy flow of the field.

The important statement \ref{1.32}, also referred to as the Poynting theorem, expresses the conservation of energy for the electromagnetic field. In other words, the time rate of change of the field energy within a certain region and the energy flowing out of that region is balanced by the conversion of the electromagnetic energy into mechanical or heat energy and vice-versa.

+momentum...

\noindent
Lorentz force:
\begin{equation}
\vec{F} = q \left(E + v \times B \right) 
\end{equation}
Ohm's law:
\begin{equation}
\vec{J} = \sigma \vec{E}
\end{equation}

\section{Electromagnetic waves and Gaussian beam}
In this section, the simplest mathematical description of a focused laser beam based on approximations to the wave equation is deduced. Since in numerical codes it is a common practice to prescribe the laser beams by their propagation in free space, the set of the microscopic Maxwell's equations \ref{1.1} - \ref{1.4} will be exploited.

In the absence of external sources, it might be easily shown that the equations \ref{1.1} - \ref{1.4} may be alternatively formulated as an uncoupled homogeneous wave equations for electric field $ \vec{E}\left( \vec{r}, t \right) $ and magnetic field $ \vec{B}\left( \vec{r}, t \right) $,
\begin{equation}
\label{1.34}
\laplace{\vec{E}} - \frac{1}{c^{2}} \diffp[2]{\vec{E}}{t} = 0,
\end{equation}
\begin{equation}
\label{1.35}
\laplace{\vec{B}} - \frac{1}{c^{2}} \diffp[2]{\vec{B}}{t} = 0,
\end{equation}
where the universal constant $ c = 1/\sqrt{\mu_0 \varepsilon_0} $ is the speed of light in vacuum, which leads to the essential fact, that the electromagnetic waves propagate in vacuum with the velocity of light $ c $.

Without any loss of generality, consider the laser beam as an electromagnetic wave propagating toward the positive direction of the z-axis with the electric field linearly polarized along the x-axis of the Cartesian coordinate system. A common way is to describe such a wave by the evolution of a single electric field component (although the more proper way would be to use the vector potential), therefore one has to look for the solution of the equation \ref{1.34}. 

According to the previous assumptions, the solution is expected to be in the form of the following plane wave,
\begin{equation}
\label{1.36}
\vec{E}\left(\vec{r_\bot}, z, t \right)  = E_0 \Psi \left(\vec{r_\bot}, z \right) \e^{\i \left(k_z z - \omega t \right)} \mathrm{\vec{\hat{e}_x}},
\end{equation}
where $ \vec{r_\bot} = (x, y)^{\mathrm{T}} $ is the vector of transverse Cartesian coordinates, $ E_0 $ is a constant amplitude, $ \Psi \left(\vec{r_\bot}, z \right) $ is the part of the wave function which is dependent only on the spatial coordinates, $ \omega $ denotes the angular frequency, $ k_z $ is the z-component of the wave vector $ \vec{k}\left(\omega \right) $ and $ \mathrm{\vec{\hat{e}_x}} $ is the unit vector pointing in the direction of the x-axis.

Direct substitution of expression \ref{1.36} into the equation \ref{1.34} yields the time-independent form of the scalar wave equation
\begin{equation}
\label{1.37}
\laplace{\Psi \left(\vec{r_\bot}, z \right)} + 2 \i k_z \diffp{\Psi \left(\vec{r_\bot}, z \right)}{z} = 0.
\end{equation}
The equation \ref{1.37} is called the Helmholtz equation. Note that it is sufficient to seek solutions to the equation \ref{1.37} since the wave \ref{1.36} is monochromatic.

It turned out, that the geometry of the focused laser beam can be expressed in terms of the laser wavelength $ \lambda $ and the following three parameters,
\begin{equation}
\label{1.38}
w_0, \qquad z_{\mathrm{R}} = \frac{k_z w_0^2}{2} = \frac{\pi w_0^2}{\lambda}, \qquad \Theta = \frac{w_0}{z_\mathrm{R}} = \frac{\lambda}{\pi w_0}.
\end{equation}
The parameter $ w_0 $ in \ref{1.38} is the beam waist, defined as a radius at which the laser intensity fall to $ 1/\e^2 $ of its axial value at the focal spot. The second parameter, $ z_\mathrm{R} $, is so-called Rayleigh range which is a distance in the longitudinal direction from the focal spot to the point where the beam radius is $ \sqrt{2} $ larger than the beam waist $ w_0 $. And the last parameter, $ \Theta $, is the divergence angle of the beam that represents the ratio of transverse and longitudinal extent.

Because of the symmetry about the longitudinal axis of the equation \ref{1.37}, the following calculations may be made simpler by introducing a dimensionless cylindrical coordinates that use the parameters \ref{1.38},
\begin{equation}
\label{1.39}
\rho = \frac{\norm{\vec{r_\bot}}}{w_0}, \qquad \zeta = \frac{z}{z_{\mathrm{R}}}.
\end{equation}
After performing a transformation of coordinates, the Helmholtz equation \ref{1.37} becomes 
\begin{equation}
\label{1.40}
\frac{1}{\rho} \diffp{}{\rho}\left(\rho \diffp{\Psi \left(\rho, \zeta \right)}{\rho} \right) + 4 \i \diffp{\Psi \left(\rho, \zeta \right)}{\zeta}  = - \Theta^2 \diffp[2]{\Psi \left(\rho, \zeta \right)}{\zeta}.
\end{equation}

In the following calculations, the beam divergence angle $ \Theta $ is assumed to be small ($ \Theta \ll 1 $), thus it can be used as an expansion parameter for $ \Psi $ and the solution of \ref{1.40} will always be consistent,
\begin{equation}
\label{1.41}
\Psi = \sum_{n = 0}^{+\infty} \Theta^{2n} \Psi_{2n}.
\end{equation}
Next, one shall insert \ref{1.41} into \ref{1.40} and collect the terms with the same power of $ \Theta $. Then the zeroth-order function $ \Psi_0 $ obeys the following equation,
\begin{equation}
\label{1.42}
\frac{1}{\rho} \diffp{}{\rho}\left(\rho \diffp{\Psi_0 \left(\rho, \zeta \right)}{\rho} \right) + 4 \i \diffp{\Psi_0\left(\rho, \zeta \right)}{\zeta} = 0.
\end{equation}

The equation \ref{1.42}, which is called the paraxial Helmholtz equation, is the starting point of traditional Gaussian beam theory. One can expect the solution of \ref{1.42} in the form of a Gaussian function with a width varying along the longitudinal direction, thus 
\begin{equation}
\label{1.43}
\Psi_0 \left(\rho, \zeta \right) = h\left(\zeta \right)\e^{-f\left(\zeta \right) \rho^2},
\end{equation}
where $ f\left(\zeta \right) $ and $ h\left(\zeta \right) $ are unknown complex functions that have to satisfy a condition $ f\left(0 \right) = h\left(0 \right) = 1 $. After plugging \ref{1.43} into \ref{1.42}, one gets the following equation,
\begin{equation}
\label{1.44}
-f\left(\zeta \right) h\left(\zeta \right) + \i \diff{h\left(\zeta \right)}{\zeta} + \rho^2 h\left(\zeta \right) \left(f\left(\zeta \right)^2 - \i \diff{f\left(\zeta \right)}{\zeta} \right) = 0.
\end{equation}
Since the equation \ref{1.44} has to hold for arbitrary value of $ \rho $, one may find two independent equations that are equivalent to \ref{1.44}
\begin{equation}
\label{1.45}
\frac{1}{f\left(\zeta \right)^2} \diff{f\left(\zeta \right)}{\zeta} + \i = 0, \qquad \frac{1}{f\left(\zeta \right) h\left(\zeta \right)} \diff{h\left(\zeta \right)}{\zeta} + \i = 0.
\end{equation}
It might be easily shown, that under specified conditions the solutions of equations \ref{1.45} have to be
\begin{equation}
\label{1.46}
h\left(\zeta \right) = f\left(\zeta \right), \qquad f\left(\zeta \right) = \frac{1}{\sqrt{1 + \zeta^2}} \e^{-\i \arctan{\zeta}},
\end{equation}
and therefore the complete expression for the zeroth-order wave function $ \Psi_0 \left(\rho, \zeta \right) $ is
\begin{equation}
\label{1.47}
\Psi_0 \left(\rho, \zeta \right) = \frac{1}{\sqrt{1 + \zeta^2}} \exp{\left[- \frac{\rho^2}{1 + \zeta^2} + \i \left(\frac{\rho^2 \zeta}{1 + \zeta^2} - \arctan{\zeta} \right) \right]}.
\end{equation}

In many situations, it is also useful to evaluate the expression \ref{1.47} in terms of Cartesian coordinates, in which the zeroth-order wave function $ \Psi_0 \left(\vec{r_\bot}, z \right) $ is
\begin{equation}
\label{1.48}
\Psi_0 \left(\vec{r_\bot}, z \right) = \frac{w_0}{w\left(z\right)} \exp{\left[- \frac{\vec{r_\bot}^2}{w\left(z \right)^2} + \i \left( k_z \frac{\vec{r_\bot}^2}{2 R\left(z \right)} - \varphi_\mathrm{G} \left( z\right) \right) \right]},
\end{equation}
where the parameters used to simplify the expression \ref{1.48} are defined as
\begin{equation}
\label{1.49}
w\left(z\right) = w_0 \sqrt{1 + \left(\frac{z}{z_\mathrm{R}}\right)^2}, \quad R\left(z \right) = z \left[1 + \left(\frac{z_\mathrm{R}}{z} \right)^2\right], \quad \varphi_\mathrm{G}\left(z\right) = \arctan{\left(\frac{z}{z_\mathrm{R}}\right)}.
\end{equation}
One shall discuss the physical meaning of the three parameters \ref{1.49}. The function $ w\left(z\right) $ represents the spot size parameter of the beam, that is the radius at which the laser intensity fall to $ 1/\e^2 $ of its axial value at any position $ z $ along the beam propagation. Note that the minimum of the spot size $ w(0) = w_0 $, consequently the focal spot is stationary and located at the origin of a Cartesian coordinate system. The second parameter, $ R\left(z \right) $ is known to be the radius of curvature of the beam's wavefront at any position $ z $ along the beam propagation. Note that $ \lim_{z \to 0^{\pm}} R(z) = \pm \infty $, therefore the beam behaves like a plane wave at focus as required. The last parameter, $ \varphi_\mathrm{G}\left(z\right) $, is the so-called Guoy phase of the beam at any position $ z $ along the beam propagation, which describes a phase shift in the wave as it passes through the focal spot.

Finally, by substituting \ref{1.48} for $ \Psi \left(\vec{r_\bot}, z \right) $ in \ref{1.36} and taking the real part of that complex quantity, one obtains the electric field of the so-called paraxial Gaussian beam,
\begin{equation}
\label{1.50}
\vec{E}\left(\vec{r_\bot}, z, t \right) = E_0 \frac{w_0}{w(z)} \exp\left(-\frac{\vec{r_\bot}^2}{w(z)^2}\right) \cos\left(\omega t - k_z \left(z + \frac{\vec{r_\bot}^2}{2 R(z)} \right) + \varphi_\mathrm{G}\left(z\right) \right) \mathrm{\vec{\hat{e}_x}}.
\end{equation}
Although given electric field \ref{1.50} describes the main features of the focused laser beam, it might be clearly seen that it does not satisfies the Maxwell equation \ref{1.1}. The correct electric field has to have at least two non-zero vector components. To fix that, one would have to solve the wave equation for the vector potential \ref{1.16} and afterwards exploit the solution to deduce all components of the electric and magnetic fields.   

In addition, since one assumed $ \Theta \ll 1 $, the solution \ref{1.50} is not accurate for strongly diverging beams. Since the divergence angle is inversely proportional to the beam waist, the previous condition yields $ w_0 \gg \lambda $. In other words, it means that \ref{1.50} is not valid for tightly focused laser beams and the need may arise for higher-order corrections.

\section{Simulation results}
wave propagation is limited to directions within a small angle of an axis.
Since the gaussian beam model uses the paraxial approximation, it fails when wavefronts are tilted by more than about 30 from the axis of the beam
Since this solution relies on the paraxial approximation, it is not accurate for very strongly diverging beams. In most practical cases the above form is valid
consequently $  w_0 \gg \lambda $
For many purposes the above form is a good enough approximation
Paraxial approximation seems to be sufficient as long as one is interested in the region close to the beam axis and the focusing is not too tight.

All components of the electric and magnetic fields can be deduced from a single scalar wave function.
In general, the forms of laser beams can be usefully deduced from a vector potential that has a single Cartesian coordinate.
In this section, the properties of a Gaussian beam will be presented.
Propagation along the z axis and a stationary focus at the origin of a Cartesian coordinate system will be assumed.
In this case E and B fields may be expressed by A alone. 
In general, the forms of laser beams can be usefully deduced from a vector potential that
has a single Cartesian coordinate.

Since the laser radiation is nothing but the electromagnetic wave...
Electromagnetic field because the time-varying magnetic field give rise to electric field and vice-versa. interconnection between e and b becomes clear in the framework of special relativity
The way in which charges and currents interact with the electromagnetic field is described by Lorentz force.
Electric and magnetic fields can be regarded as a forces produced by distribution of charge and currents
essential to electrodynamic is the speed of light in vacuum (universal constant), given by ...
Electromagnetic fields - forces produced by distribution of charge and currents - can exist in regions of space where there are no sources (they can carry energy, momentum , have existence totally independent of charge and currents) 

All components of the electric and magnetic fields of the laser beam that are consistent with Maxwell's equations are deduced. Afterwards, one may evaluate these fields at the boundary of simulation domain and prescribe the laser beam correctly. 

Taylor series:
\begin{equation}
k_z \left(\vec{k}_\bot, \omega \right) \approx \frac{\abs{\omega}}{c} - \frac{c}{2 \abs{\omega}} \left( k_x^2 + k_y^2 \right)  
\end{equation}
by subs. get back paraxial approximation:
\begin{equation}
\bar{\vec{E}}^{\pm}_{\bot} \left(\vec{k}_\bot, z, \omega \right) \approx \bar{\vec{E}}^{\pm}_{0, \bot} \left(\vec{k}_\bot, \omega \right) \e^{\pm \i \left[ \frac{\abs{\omega}}{c} - \frac{c}{2 \abs{\omega}} \left( k_x^2 + k_y^2 \right) \right] \left(z - z_0 \right)}
\end{equation}
\begin{equation}
\bar{E}^{\pm}_z \left(\vec{k}_\bot, z, \omega \right) \approx 0
\end{equation}
\begin{equation}
\bar{B}^{\pm}_x \left(\vec{k}_\bot, z, \omega \right) \approx \mp \frac{1}{c} \bar{E}^{\pm}_y \left(\vec{k}_\bot, z, \omega \right)
\end{equation}
\begin{equation}
\bar{B}^{\pm}_y \left(\vec{k}_\bot, z, \omega \right) \approx \pm \frac{1}{c} \bar{E}^{\pm}_x \left(\vec{k}_\bot, z, \omega \right)
\end{equation}
\begin{equation}
\bar{B}^{\pm}_z \left(\vec{k}_\bot, z, \omega \right) \approx 0
\end{equation}

where
\begingroup
\renewcommand*{\arraystretch}{2.0}
\begin{table}[h!]
	\begin{flushleft}
		\begin{tabular}{ l c r }
			$ w(z) = w_0 \sqrt{1 + \left(\frac{z}{z_R} \right)^2}  $ & \ldots & evolving beam width \\
			$ z_r = \frac{\pi w_0^2}{\lambda} $ & \ldots & Rayleigh range \\
			$ R(z) = z\left(1 + \left(\frac{z_R}{z} \right)^2 \right) $ & \ldots & radius of curvature \\
			$ \phi(z) = \tan^{-1}\left(\frac{z}{z_R} \right) $ & \ldots & Guoy phase 
		\end{tabular}
	\end{flushleft}
\end{table}
\endgroup

\begin{flalign*}
& w(z) = w_0 \sqrt{1 + \left(\frac{z}{z_R} \right)^2} \dots \mathrm{evolving \: beam \: width} & \\
& z_r = \frac{\pi w_0^2}{\lambda} \dots \mathrm{Rayleigh \: range} & \\
& R(z) = z\left(1 + \left(\frac{z_R}{z} \right)^2 \right) \dots \mathrm{evolving \: radius \: of \: curvature} & \\
& \phi(z) = \tan^{-1}\left(\frac{z}{z_R} \right) \dots \mathrm{Guoy \: phase} & \\
& \theta = \tan^{-1}\left(\frac{w(z)}{z} \right) \simeq \frac{\lambda}{\pi w_0} \dots \mathrm{beam \: divergence \: angle} &
\end{flalign*}

\noindent
Features:
\begin{itemize}
	\item 2D version of algorithm, Ey, Bx, Bz omitted (identically equal to 0) 
	\item Code written in C++, object oriented to be easily extended to 3D, compiled to static library
	\item Linked into EPOCH as a static library (in order not to disturb the code, for this reason also added support for CMake – machine independent)
	\item Parallelized using hybrid techniques (OpenMP + MPI – computation time in most cases negligible in comparison with the main simulation)
	\item Fourier transforms can be computed using Intel MKL library, FFTW library or without any library (compile time option)
	\item Computed fields dumped into shared files using binary coding (speed up output, save disk storage)
	\item Only transverse component of electric field (Ex) passed to the EPOCH at each time step (no significant slowdown or memory overhead), other fields computed by EPOCH
	\item All new parameters needed for tight-focusing (w0, focal length, etc.) may be specified via input file
	\item Implementation works generally regardless the number of lasers in the simulation or boundaries that they are attached to
\end{itemize}

\noindent
Laser:
\begin{itemize}
	\item wavelength: $ \lambda $ = 1.0 $ \mu m $
	\item amplitude: $ \vec{E}_0 $ = 1e15 V/m
	\item duration: t = 20 fs (in FWHM)
	\item beam waist in focus: $ w_0 $ = 0.7 $ \mu m $
	\item focus distance from boundary: $ x_\mathrm{B} - x_0 $ = 8 $ \mu m $
	\item polarization: P
	\item boundary: left 
\end{itemize}
Domain:
\begin{itemize}
	\item x min: -8 $ \mu m $
	\item x max: 8 $ \mu m $
	\item y min: -8 $ \mu m $
	\item y max: 8 $ \mu m $
	\item $ N_x $: 1600 cells ($ \delta x $ = $ \lambda/100 $ = 10 nm)
	\item $ N_y $: 1600 cells ($ \delta y $ = $ \lambda/100 $ = 10 nm)
	\item time step: $ \delta t $ = $ 1/(\sqrt{2} c) \lambda /100 \approx $ 0.05 fs 
	\item simulation time: $ \tau $ = 150 fs
\end{itemize}

\newpage
\noindent
The phase space distribution function  $ f_{s} \left(\vec{x}, \vec{v}, t\right) $ for a given species $ s $ is governed by the Vlasov equation:
\begin{equation*}
\diffp[]{f_{s}}{t} + \vec{v} \cdot \nabla f_s + \frac{q_{s}}{m_{s}}\left( \vec{E} + \vec{v} \times \vec{B} \right) \cdot \diffp[]{f_s}{\vec{v}} = 0.
\end{equation*}
The distribution function is approximated using finite-size quasi-particles with so-called shape functions $ S_{x} $ and $ S_{v} $, $ N_p $ is the number of physical particles: 
\begin{equation*}
f_{s} \left(\vec{x}, \vec{v}, t \right) =  \sum_{p} f_{p}\left(\vec{x}, \vec{v}, t \right), \quad f_{p}\left(\vec{x}, \vec{v}, t \right) = N_{p} S_{x}\left(\vec{x} - \vec{x}_{p}\left(t\right) \right)  S_{v}\left(\vec{v} - \vec{v}_{p}\left( t\right) \right).
\end{equation*}
Electromagnetic fields self-consistently evolved by Maxwell equations:
\begin{equation*}
\nabla \cdot \vec{E} = \frac{\rho}{\varepsilon_{0}}, \qquad \nabla \cdot \vec{B} = 0
\end{equation*}
\begin{equation*}
\nabla \times \vec{E} = - \diffp{\vec{B}}{t}, \qquad \nabla \times \vec{B} = \mu_{0} \vec{J} + \frac{1}{c^{2}} \diffp{\vec{E}}{t},
\end{equation*}
where charge density and current density are obtained from distribution functions:
\begin{equation*}
\rho\left(\vec{x}, t \right) = \sum_s q_s \int f_s \left(\vec{x}, \vec{v}, t \right) \mathrm{d} \vec{v}, \qquad \vec{J}\left(\vec{x}, t \right) = \sum_s q_s \int f_s \left(\vec{x}, \vec{v}, t \right) \vec{v} \, \mathrm{d} \vec{v}.
\end{equation*}



\begin{lstlisting}[style=CXX, caption=Function performing forward fast Fourier transform using MKL library]
std::vector<std::complex<double>> fft::mkl_fft_forward(std::vector<std::complex<double>> in) {
DFTI_DESCRIPTOR_HANDLE desc;
MKL_LONG status;
DftiCreateDescriptor(&desc, DFTI_DOUBLE, DFTI_COMPLEX, 1, static_cast<MKL_LONG>(in.size()));
DftiCommitDescriptor(desc);
status = DftiComputeForward(desc, in.data());
if(status != 0) {
std::cerr << DftiErrorMessage(status) << std::endl;
abort();
}
DftiFreeDescriptor(&desc);
return in;
}
\end{lstlisting}

\begin{lstlisting}[style=CXX, caption=Function performing backward fast Fourier transform using MKL library]
std::vector<std::complex<double>> fft::mkl_fft_backward(std::vector<std::complex<double>> in) {
DFTI_DESCRIPTOR_HANDLE desc;
MKL_LONG status;
DftiCreateDescriptor(&desc, DFTI_DOUBLE, DFTI_COMPLEX, 1, static_cast<MKL_LONG>(in.size()));
DftiCommitDescriptor(desc);
status = DftiComputeBackward(desc, in.data());
if(status != 0) {
std::cerr << DftiErrorMessage(status) << std::endl;
abort();
}
DftiFreeDescriptor(&desc);
return in;
}
\end{lstlisting}

\begin{lstlisting}[style=CXX, caption=Function performing forward fast Fourier transform using FFTW library]
std::vector<std::complex<double>> fft::fftw_fft_forward(std::vector<std::complex<double>> in) {
fftw_plan p = fftw_plan_dft_1d(in.size(), reinterpret_cast<fftw_complex*>(in.data()), reinterpret_cast<fftw_complex*>(in.data()), FFTW_FORWARD, FFTW_ESTIMATE);
fftw_execute(p);
fftw_destroy_plan(p);
return in;
}
\end{lstlisting}

\begin{lstlisting}[style=CXX, caption=Function performing backward fast Fourier transform using FFTW library]
std::vector<std::complex<double>> fft::fftw_fft_backward(std::vector<std::complex<double>> in) {
fftw_plan p = fftw_plan_dft_1d(in.size(), reinterpret_cast<fftw_complex*>(in.data()), reinterpret_cast<fftw_complex*>(in.data()), FFTW_BACKWARD, FFTW_ESTIMATE);
fftw_execute(p);
fftw_destroy_plan(p);
return in;
}
\end{lstlisting}

\begin{lstlisting}[style=CXX, caption=Function performing forward discrete Fourier transform without using any library]
std::vector<std::complex<double>> fft::fft_forward(std::vector<std::complex<double>> in) {
std::vector<std::complex<double>> out(in.size());
for(auto j = 0; j < out.size(); j++) {
for(auto l = 0; l < out.size(); l++) {
out.at(j) += in.at(l) * exp(-2.0 * constants::pi * I * l * j / in.size());
}
}
return out;
}
\end{lstlisting}

\begin{lstlisting}[style=CXX, caption=Function performing backward discrete Fourier transform without using any library]
std::vector<std::complex<double>> fft::fft_backward(std::vector<std::complex<double>> in) {
std::vector<std::complex<double>> out(in.size());
for(auto j = 0; j < out.size(); j++) {
for(auto l = 0; l < out.size(); l++) {
out.at(j) += in.at(l) * exp(+2.0 * constants::pi * I * l * j / in.size());
}
}
return out;
}
\end{lstlisting}

\begin{lstlisting}[style=CXX, caption=Method for performing discrete Fourier transform in time]
void laser_bcs::dft_time(field_2d<std::complex<double>>& field) const {
#ifdef OPENMP
#pragma omp parallel for schedule(static)
#endif
for(auto j = 0; j < this->domain->Nx; j++) {
#ifdef USE_MKL
field.add_col(fft::mkl_fft_backward(field.get_col(j)), j);
#elif USE_FFTW
field.add_col(fft::fftw_fft_backward(field.get_col(j)), j);
#else
field.add_col(fft::fft_backward(field.get_col(j)), j);
#endif
}
field.multiply(this->domain->dt / (2.0 * constants::pi));
return;
}
\end{lstlisting}

\begin{lstlisting}[style=CXX, caption=Method for performing inverse discrete Fourier transform in time]
void laser_bcs::idft_time(field_2d<std::complex<double>>& field) const {
#ifdef OPENMP
#pragma omp parallel for schedule(static)
#endif
for(auto j = 0; j < this->domain->Nx; j++) {
#ifdef USE_MKL
field.add_col(fft::mkl_fft_forward(field.get_col(j)), j);
#elif USE_FFTW
field.add_col(fft::fftw_fft_forward(field.get_col(j)), j);
#else
field.add_col(fft::fft_forward(field.get_col(j)), j);
#endif
}
field.multiply(2.0 * (2.0 * constants::pi) / (this->domain->Nt * this->domain->dt));
return;
}
\end{lstlisting}

\begin{lstlisting}[style=CXX, caption=Method for performing discrete Fourier transform in space]
void laser_bcs::dft_space(field_2d<std::complex<double>>& field) const {
std::vector<std::complex<double>> row_global(this->domain->Nx_global);
std::vector<std::complex<double>> row_local;
for(auto j = 0; j < this->domain->Nt; j++) {
row_local = field.get_row(j);
MPI_Gatherv(row_local.data(), this->domain->Nx, MPI_CXX_DOUBLE_COMPLEX, row_global.data(), this->domain->counts.data(), this->domain->displs.data(), MPI_CXX_DOUBLE_COMPLEX, 0, MPI_COMM_WORLD);
if(this->domain->rank == 0) {
#ifdef USE_MKL
row_global = fft::mkl_fft_forward(row_global);
#elif USE_FFTW
row_global = fft::fftw_fft_forward(row_global);
#else
row_global = fft::fft_forward(row_global);
#endif
}
MPI_Scatterv(row_global.data(), this->domain->counts.data(), this->domain->displs.data(), 	MPI_CXX_DOUBLE_COMPLEX, row_local.data(), this->domain->Nx, MPI_CXX_DOUBLE_COMPLEX, 0, MPI_COMM_WORLD);
field.add_row(row_local, j);
}
field.multiply(this->domain->dx / (2.0 * constants::pi));
return;
}
\end{lstlisting}

\begin{lstlisting}[style=CXX, caption=Method for performing inverse discrete Fourier transform in space]
void laser_bcs::idft_space(field_2d<std::complex<double>>& field) const {
std::vector<std::complex<double>> row_global(this->domain->Nx_global);
std::vector<std::complex<double>> row_local;
for(auto j = 0; j < this->domain->Nt; j++) {
row_local = field.get_row(j);
MPI_Gatherv(row_local.data(), this->domain->Nx, MPI_CXX_DOUBLE_COMPLEX, row_global.data(), this->domain->counts.data(), this->domain->displs.data(), MPI_CXX_DOUBLE_COMPLEX, 0, MPI_COMM_WORLD);
if(this->domain->rank == 0) {
#ifdef USE_MKL
row_global = fft::mkl_fft_backward(row_global);
#elif USE_FFTW
row_global = fft::fftw_fft_backward(row_global);
#else
row_global = fft::fft_backward(row_global);
#endif
}
MPI_Scatterv(row_global.data(), this->domain->counts.data(), this->domain->displs.data(), MPI_CXX_DOUBLE_COMPLEX, row_local.data(), this->domain->Nx, MPI_CXX_DOUBLE_COMPLEX, 0, MPI_COMM_WORLD);
field.add_row(row_local, j);
}
field.multiply((2.0 * constants::pi) / (this->domain->Nx_global * this->domain->dx));
return;
}
\end{lstlisting}

\begin{lstlisting}[style=CXX, caption=Method for dumping data into shared file]
template <typename T>
void field_2d<T>::dump_to_shared_file(std::string name, int row_first, int row_last, int row_size_local, int row_size_global, int col_start) const {
MPI_File file;
MPI_Offset offset = 0;
MPI_Status status;
MPI_Datatype local_array;
int col_size = row_last - row_first;
const int ndims = 2;
std::array<int, ndims> size_global = {col_size, row_size_global};
std::array<int, ndims> size_local = {col_size, row_size_local};
std::array<int, ndims> start_coords = {0, col_start};
MPI_Type_create_subarray(2, size_global.data(), size_local.data(), start_coords.data(), MPI_ORDER_C, MPI_DOUBLE, &local_array);
MPI_Type_commit(&local_array);
std::vector<double> real_part(col_size * row_size_local);
for(auto i = std::make_pair(row_first, 0); i.first < row_last; i.first++, i.second++) {
for(auto j = 0; j < row_size_local; j++) {
real_part[i.second * row_size_local + j] = std::real(this->data[i.first * row_size_local + j]);
}
}
MPI_File_open(MPI_COMM_WORLD, name.data(), MPI_MODE_CREATE|MPI_MODE_WRONLY, MPI_INFO_NULL, &file);
MPI_File_set_view(file, offset, MPI_DOUBLE, local_array, "native", MPI_INFO_NULL);
MPI_File_write_all(file, real_part.data(), col_size * row_size_local, MPI_DOUBLE, &status);
MPI_File_close(&file);
MPI_Type_free(&local_array);
return;
}
\end{lstlisting}

\begin{lstlisting}[style=CXX, caption=Extern C++ function to fill Fortran arrays with laser fields dumped in binary file]
void populate_laser_field_on_boundary(double* field, int* id, const char* data_dir, const char* name, int* timestep, int* size_global, int* first, int* last) {
double num = 0.0;
std::string laser_id = std::to_string(*id);
std::string output_path(data_dir);
std::string filename(name);
std::ifstream in;
in.open(output_path + "/" + filename + laser_id + ".dat", std::ios::binary);
if(in.is_open()) {
in.seekg(((*timestep) * (*size_global) + (*first) - 1) * sizeof(num));
for(auto i = 0; i < *last - *first + 1; i++) {
in.read(reinterpret_cast<char*>(&num), sizeof(num));
field[i] = num;
}
in.close();
} else {
std::cout << "error: cannot read file " << output_path + "/" + filename + laser_id + ".dat" << std::endl;
}
return;
}
\end{lstlisting}

\begin{lstlisting}[style=FORTRAN, caption=Fortran interfaces for C++ library functions]
INTERFACE

SUBROUTINE compute_laser_fields_on_boundary(rank, nproc, laser_start, laser_end, fwhm_time, t_0, omega, pos, amp, w_0, id, L_min, L_max, L_focus, T_min, T_max, T_ncells, cpml_thickness, t_end, T_cell_size, L_cell_size, dt, output_path) bind(c)
USE, INTRINSIC :: iso_c_binding
IMPLICIT NONE
INTEGER(c_int), INTENT(IN) :: rank, nproc, id, T_ncells, cpml_thickness
CHARACTER(kind=c_char), DIMENSION(*), INTENT(IN) :: output_path
REAL(c_double), INTENT(IN) :: laser_start, laser_end, fwhm_time, t_0, omega, pos,    &
amp, w_0, L_min, L_max, L_focus, T_min, T_max, t_end, T_cell_size, L_cell_size, dt
END SUBROUTINE compute_laser_fields_on_boundary

SUBROUTINE populate_laser_field_on_boundary(field, laser_id, output_path, field_name, timestep, size_global, first, last) bind(c)
USE, INTRINSIC :: iso_c_binding
IMPLICIT NONE
INTEGER(c_int), INTENT(IN) :: laser_id, timestep, size_global, first, last
CHARACTER(kind=c_char), DIMENSION(*), INTENT(IN) :: output_path, field_name
REAL(c_double), DIMENSION(*), INTENT(OUT) :: field
END SUBROUTINE populate_laser_field_on_boundary

END INTERFACE
\end{lstlisting}

\begin{lstlisting}[style=FORTRAN, caption=Fortran subroutines for Maxwell consistent computation of laser fields on boundaries]
SUBROUTINE Maxwell_consistent_computation_of_EM_fields

TYPE(laser_block), POINTER :: current

current => laser_x_min
DO WHILE(ASSOCIATED(current))
CALL compute_laser_fields_on_boundary(rank, nproc, current%t_start, current%t_end, current%fwhm_time, current%t_0, current%omega, current%pos, current%amp, current%w_0, current%id, x_min, x_max, current%focus, y_min, y_max, ny_global, cpml_thickness, t_end, dy, dx, dt, TRIM(data_dir)//C_NULL_CHAR)
current => current%next
ENDDO

current => laser_x_max
DO WHILE(ASSOCIATED(current))
CALL compute_laser_fields_on_boundary(rank, nproc, current%t_start, current%t_end, current%fwhm_time, current%t_0, current%omega, current%pos, current%amp, current%w_0, current%id, x_min, x_max, current%focus, y_min, y_max, ny_global, cpml_thickness, t_end, dy, dx, dt, TRIM(data_dir)//C_NULL_CHAR)
current => current%next
ENDDO

current => laser_y_min
DO WHILE(ASSOCIATED(current))
CALL compute_laser_fields_on_boundary(rank, nproc, current%t_start, current%t_end, current%fwhm_time, current%t_0, current%omega, current%pos, current%amp, current%w_0, current%id, y_min, y_max, current%focus, x_min, x_max, nx_global, cpml_thickness, t_end, dx, dy, dt, TRIM(data_dir)//C_NULL_CHAR)
current => current%next
ENDDO

current => laser_y_max
DO WHILE(ASSOCIATED(current))
CALL compute_laser_fields_on_boundary(rank, nproc, current%t_start, current%t_end, current%fwhm_time, current%t_0, current%omega, current%pos, current%amp, current%w_0, current%id, y_min, y_max, current%focus, x_min, x_max, nx_global, cpml_thickness, t_end, dx, dy, dt, TRIM(data_dir)//C_NULL_CHAR)
current => current%next
ENDDO

END SUBROUTINE Maxwell_consistent_computation_of_EM_fields
\end{lstlisting}

\begin{lstlisting}[style=FORTRAN, caption=Fortran subroutines for populating laser sources on boundaries]
SUBROUTINE get_source_x_boundary(source1, source2, laser_id) 
REAL(num), DIMENSION(:), INTENT(INOUT) :: source1, source2
REAL(num), DIMENSION(ny) :: laser_ex, laser_ey
INTEGER, INTENT(IN) :: laser_id
INTEGER :: i
CALL populate_laser_field_on_boundary(laser_ex, laser_id, TRIM(data_dir)//C_NULL_CHAR, "e_x"//C_NULL_CHAR, step, ny_global, ny_global_min, ny_global_max)
laser_ey = 0.0_num
DO i = 1, ny
source1(i) = source1(i) + laser_ex(i)
source2(i) = source2(i) + laser_ey(i)
ENDDO
END SUBROUTINE get_source_x_boundary

SUBROUTINE get_source_y_boundary(source1, source2, laser_id)
REAL(num), DIMENSION(:), INTENT(INOUT) :: source1, source2
REAL(num), DIMENSION(nx) :: laser_ex, laser_ey
INTEGER, INTENT(IN) :: laser_id
INTEGER :: i
CALL populate_laser_field_on_boundary(laser_ex, laser_id, TRIM(data_dir)//C_NULL_CHAR, "e_x"//C_NULL_CHAR, step, nx_global, nx_global_min, nx_global_max)
laser_ey = 0.0_num
DO i = 1, nx
source1(i) = source1(i) + laser_ey(i)
source2(i) = source2(i) + laser_ex(i)
ENDDO
END SUBROUTINE get_source_y_boundary
\end{lstlisting}

%-------------------------------------------------------------------------------

\chapter*{Conclusion\markboth{Conclusion}{Conclusion}}
\addcontentsline{toc}{chapter}{Conclusion}
The beginning of this work summarizes the knowledge required for the further understanding of the laser-plasma interaction. In the first chapter, the fundamental physical aspects of the classical electromagnetic field theory based on the Maxwell's equations as well as the description of Gaussian laser pulse using the paraxial approximation are provided. The second chapter is focused on basic physical processes which take place during the interaction of intense laser pulses with plasma. It includes approaches for plasma description, propagation of electromagnetic wave in plasma, laser absorption and plasma heating mechanisms or mechanisms of laser-driven ion acceleration. In the third chapter, one may find a mathematical derivation of the particle-in-cell method, description of individual steps of the algorithm as well as conditions of its stability. The last section of this chapter is dedicated to code EPOCH, which has been used for simulations within this work. 

Starting from the fourth chapter, the work is devoted mainly to tight-focusing. This chapter contains a description, implementation and evaluation of the algorithm for rigorous calculation of electromagnetic fields at boundaries of simulation domain as well as a brief overview of currently used experimental methods for tight-focusing. The fifth chapter then presents results of several large-scale simulations of tightly focused laser beams interacting with solid targets.

The main benefit of this work is a successful implementation of laser boundary conditions that allow simulate tightly focused laser beams using the two-dimensional version of the computational code EPOCH. The correctness of the algorithm as well as the proper implementation have been verified by plenty of simulations and numerical tests. It has been shown, that the laser beam initialized using the paraxial approximation can lead to unexpected field profiles in the case of tight-focusing - the focal spot is shifted, field profiles are distorted and asymmetric and the peak laser amplitude is lower. These deviations are far from negligible and have a strong impact on laser-matter or laser-plasma simulation results. On the other hand, the simulations of tight-focusing where the beam at the boundary has been prescribed using the Maxwell consistent approach fulfills specified requirement precisely.

The instrumented code has been further exploited for several two-dimensional large-scale simulations employing tightly focused laser beams interacting with solid targets. Obtained results have been processed and thoroughly analyzed while the emphasis has been placed mainly on identifying the effects of the laser beam focal spot size on the laser-matter interaction results. The results and observations may be summarized as follows:

\begin{itemize}
	\item[\tiny $\blacksquare$] in the case of tight-focusing, the laser energy absorption efficiency sharply increases
	\item[\tiny $\blacksquare$] in the case of tight-focusing, the plasma in the vicinity of the focal spot expands rapidly
	\item[\tiny $\blacksquare$] in the case of tight-focusing, there is a strong electric current along the target front surface
	\item[\tiny $\blacksquare$] as the focal spot size decreases, the transverse component of the ponderomotive force increases
	\item[\tiny $\blacksquare$] the direction of electrons moving forward is given by the ratio between the longitudinal and transverse component of ponderomotive force
	\item[\tiny $\blacksquare$] for the larger focal spot sizes, the electric field decays more slowly
	\item[\tiny $\blacksquare$] the number of non-relativistic hot electrons is higher for the larger focal spot size 
	\item[\tiny $\blacksquare$] there is a significant quantitative difference between the electron trajectories for different focal spot sizes
	\item[\tiny $\blacksquare$] in the case of tight-focusing, there is larger amount of hot electrons spreading in the transverse direction with respect to the direction of incoming laser pulse
	\item[\tiny $\blacksquare$] in the case of tight-focusing, one may observe a significant cloud of hot electrons in front of the target
	\item[\tiny $\blacksquare$] in the case of tight-focusing, the energy distribution of electrons is qualitatively different
	\item[\tiny $\blacksquare$] as the focal spot size increases, the laser energy transfer to ions is faster
	\item[\tiny $\blacksquare$] the maximum ion energies increase with the focal spot size
	\item[\tiny $\blacksquare$] ion acceleration efficiency is independent of the focal spot size
\end{itemize}



\newpage
\pagestyle{plain}
\null
\vfill
{\bf \noindent Acknowledgments} \\

I wish express my gratitude to both, my supervisor \klimo and consultant \weber for constant support and guidance, as well as for providing invaluable advice and direction.\\

Access to computing and storage facilities owned by parties and projects contributing to the National Grid Infrastructure MetaCentrum, provided under the programme "Projects of Large Infrastructure for Research, Development, and Innovations" (LM2010005), is greatly appreciated.\\

Access to the CERIT-SC computing and storage facilities provided under the programme Center CERIT Scientific Cloud, part of the Operational Program Research and Development for Innovations (reg. no. CZ.1.05/3.2.00/08.0144) is greatly  appreciated.\\

This work was supported by the project ELI: Extreme Light Infrastructure (reg. no. CZ.02.1.01/0.0/0.0/15\_008/0000162) from European Regional Development.\\

The development of the EPOCH code was funded in part by the UK EPSRC grants EP/G054950/1, EP/G056803/1, EP/G055165/1 and EP/M022463/1.\\
\begin{flushright}
\valenta
\end{flushright}

%-------------------------------------------------------------------------------

\newpage
\addcontentsline{toc}{chapter}{Bibliography}
\bibliographystyle{ieeetr}
\bibliography{bib/ref}

%-------------------------------------------------------------------------------

\part*{Appendices}
\addcontentsline{toc}{chapter}{Appendices}

\appendix

\chapter{Input files}
\begin{lstlisting}[style=CXX, caption=Function performing forward fast Fourier transform using MKL library]
std::vector<std::complex<double>> fft::mkl_fft_forward(std::vector<std::complex<double>> in) {
DFTI_DESCRIPTOR_HANDLE desc;
MKL_LONG status;
DftiCreateDescriptor(&desc, DFTI_DOUBLE, DFTI_COMPLEX, 1, static_cast<MKL_LONG>(in.size()));
DftiCommitDescriptor(desc);
status = DftiComputeForward(desc, in.data());
if(status != 0) {
std::cerr << DftiErrorMessage(status) << std::endl;
abort();
}
DftiFreeDescriptor(&desc);
return in;
}
\end{lstlisting}

\begin{lstlisting}[style=CXX, caption=Function performing backward fast Fourier transform using MKL library]
std::vector<std::complex<double>> fft::mkl_fft_backward(std::vector<std::complex<double>> in) {
DFTI_DESCRIPTOR_HANDLE desc;
MKL_LONG status;
DftiCreateDescriptor(&desc, DFTI_DOUBLE, DFTI_COMPLEX, 1, static_cast<MKL_LONG>(in.size()));
DftiCommitDescriptor(desc);
status = DftiComputeBackward(desc, in.data());
if(status != 0) {
std::cerr << DftiErrorMessage(status) << std::endl;
abort();
}
DftiFreeDescriptor(&desc);
return in;
}
\end{lstlisting}

\begin{lstlisting}[style=CXX, caption=Function performing forward fast Fourier transform using FFTW library]
std::vector<std::complex<double>> fft::fftw_fft_forward(std::vector<std::complex<double>> in) {
fftw_plan p = fftw_plan_dft_1d(in.size(), reinterpret_cast<fftw_complex*>(in.data()), reinterpret_cast<fftw_complex*>(in.data()), FFTW_FORWARD, FFTW_ESTIMATE);
fftw_execute(p);
fftw_destroy_plan(p);
return in;
}
\end{lstlisting}

\begin{lstlisting}[style=CXX, caption=Function performing backward fast Fourier transform using FFTW library]
std::vector<std::complex<double>> fft::fftw_fft_backward(std::vector<std::complex<double>> in) {
fftw_plan p = fftw_plan_dft_1d(in.size(), reinterpret_cast<fftw_complex*>(in.data()), reinterpret_cast<fftw_complex*>(in.data()), FFTW_BACKWARD, FFTW_ESTIMATE);
fftw_execute(p);
fftw_destroy_plan(p);
return in;
}
\end{lstlisting}

\begin{lstlisting}[style=CXX, caption=Function performing forward discrete Fourier transform without using any library]
std::vector<std::complex<double>> fft::fft_forward(std::vector<std::complex<double>> in) {
std::vector<std::complex<double>> out(in.size());
for(auto j = 0; j < out.size(); j++) {
for(auto l = 0; l < out.size(); l++) {
out.at(j) += in.at(l) * exp(-2.0 * constants::pi * I * l * j / in.size());
}
}
return out;
}
\end{lstlisting}

\begin{lstlisting}[style=CXX, caption=Function performing backward discrete Fourier transform without using any library]
std::vector<std::complex<double>> fft::fft_backward(std::vector<std::complex<double>> in) {
std::vector<std::complex<double>> out(in.size());
for(auto j = 0; j < out.size(); j++) {
for(auto l = 0; l < out.size(); l++) {
out.at(j) += in.at(l) * exp(+2.0 * constants::pi * I * l * j / in.size());
}
}
return out;
}
\end{lstlisting}

\begin{lstlisting}[style=CXX, caption=Method for performing discrete Fourier transform in time]
void laser_bcs::dft_time(field_2d<std::complex<double>>& field) const {
#ifdef OPENMP
#pragma omp parallel for schedule(static)
#endif
for(auto j = 0; j < this->domain->Nx; j++) {
#ifdef USE_MKL
field.add_col(fft::mkl_fft_backward(field.get_col(j)), j);
#elif USE_FFTW
field.add_col(fft::fftw_fft_backward(field.get_col(j)), j);
#else
field.add_col(fft::fft_backward(field.get_col(j)), j);
#endif
}
field.multiply(this->domain->dt / (2.0 * constants::pi));
return;
}
\end{lstlisting}

\begin{lstlisting}[style=CXX, caption=Method for performing inverse discrete Fourier transform in time]
void laser_bcs::idft_time(field_2d<std::complex<double>>& field) const {
#ifdef OPENMP
#pragma omp parallel for schedule(static)
#endif
for(auto j = 0; j < this->domain->Nx; j++) {
#ifdef USE_MKL
field.add_col(fft::mkl_fft_forward(field.get_col(j)), j);
#elif USE_FFTW
field.add_col(fft::fftw_fft_forward(field.get_col(j)), j);
#else
field.add_col(fft::fft_forward(field.get_col(j)), j);
#endif
}
field.multiply(2.0 * (2.0 * constants::pi) / (this->domain->Nt * this->domain->dt));
return;
}
\end{lstlisting}

\begin{lstlisting}[style=CXX, caption=Method for performing discrete Fourier transform in space]
void laser_bcs::dft_space(field_2d<std::complex<double>>& field) const {
std::vector<std::complex<double>> row_global(this->domain->Nx_global);
std::vector<std::complex<double>> row_local;
for(auto j = 0; j < this->domain->Nt; j++) {
row_local = field.get_row(j);
MPI_Gatherv(row_local.data(), this->domain->Nx, MPI_CXX_DOUBLE_COMPLEX, row_global.data(), this->domain->counts.data(), this->domain->displs.data(), MPI_CXX_DOUBLE_COMPLEX, 0, MPI_COMM_WORLD);
if(this->domain->rank == 0) {
#ifdef USE_MKL
row_global = fft::mkl_fft_forward(row_global);
#elif USE_FFTW
row_global = fft::fftw_fft_forward(row_global);
#else
row_global = fft::fft_forward(row_global);
#endif
}
MPI_Scatterv(row_global.data(), this->domain->counts.data(), this->domain->displs.data(), 	MPI_CXX_DOUBLE_COMPLEX, row_local.data(), this->domain->Nx, MPI_CXX_DOUBLE_COMPLEX, 0, MPI_COMM_WORLD);
field.add_row(row_local, j);
}
field.multiply(this->domain->dx / (2.0 * constants::pi));
return;
}
\end{lstlisting}

\begin{lstlisting}[style=CXX, caption=Method for performing inverse discrete Fourier transform in space]
void laser_bcs::idft_space(field_2d<std::complex<double>>& field) const {
std::vector<std::complex<double>> row_global(this->domain->Nx_global);
std::vector<std::complex<double>> row_local;
for(auto j = 0; j < this->domain->Nt; j++) {
row_local = field.get_row(j);
MPI_Gatherv(row_local.data(), this->domain->Nx, MPI_CXX_DOUBLE_COMPLEX, row_global.data(), this->domain->counts.data(), this->domain->displs.data(), MPI_CXX_DOUBLE_COMPLEX, 0, MPI_COMM_WORLD);
if(this->domain->rank == 0) {
#ifdef USE_MKL
row_global = fft::mkl_fft_backward(row_global);
#elif USE_FFTW
row_global = fft::fftw_fft_backward(row_global);
#else
row_global = fft::fft_backward(row_global);
#endif
}
MPI_Scatterv(row_global.data(), this->domain->counts.data(), this->domain->displs.data(), MPI_CXX_DOUBLE_COMPLEX, row_local.data(), this->domain->Nx, MPI_CXX_DOUBLE_COMPLEX, 0, MPI_COMM_WORLD);
field.add_row(row_local, j);
}
field.multiply((2.0 * constants::pi) / (this->domain->Nx_global * this->domain->dx));
return;
}
\end{lstlisting}

\begin{lstlisting}[style=CXX, caption=Method for dumping data into shared file]
template <typename T>
void field_2d<T>::dump_to_shared_file(std::string name, int row_first, int row_last, int row_size_local, int row_size_global, int col_start) const {
MPI_File file;
MPI_Offset offset = 0;
MPI_Status status;
MPI_Datatype local_array;
int col_size = row_last - row_first;
const int ndims = 2;
std::array<int, ndims> size_global = {col_size, row_size_global};
std::array<int, ndims> size_local = {col_size, row_size_local};
std::array<int, ndims> start_coords = {0, col_start};
MPI_Type_create_subarray(2, size_global.data(), size_local.data(), start_coords.data(), MPI_ORDER_C, MPI_DOUBLE, &local_array);
MPI_Type_commit(&local_array);
std::vector<double> real_part(col_size * row_size_local);
for(auto i = std::make_pair(row_first, 0); i.first < row_last; i.first++, i.second++) {
for(auto j = 0; j < row_size_local; j++) {
real_part[i.second * row_size_local + j] = std::real(this->data[i.first * row_size_local + j]);
}
}
MPI_File_open(MPI_COMM_WORLD, name.data(), MPI_MODE_CREATE|MPI_MODE_WRONLY, MPI_INFO_NULL, &file);
MPI_File_set_view(file, offset, MPI_DOUBLE, local_array, "native", MPI_INFO_NULL);
MPI_File_write_all(file, real_part.data(), col_size * row_size_local, MPI_DOUBLE, &status);
MPI_File_close(&file);
MPI_Type_free(&local_array);
return;
}
\end{lstlisting}

\begin{lstlisting}[style=CXX, caption=Extern C++ function to fill Fortran arrays with laser fields dumped in binary file]
void populate_laser_at_boundary(double* field, int* id, const char* data_dir, int* timestep, int* size_global, int* first, int* last) {
double num = 0.0;
std::string laser_id = std::to_string(*id);
std::string path(data_dir);
std::ifstream in;
in.open(path + "/laser_" + laser_id + ".dat", std::ios::binary);
if(in.is_open()) {
in.seekg(((*timestep) * (*size_global) + (*first) - 1) * sizeof(num));
for(auto i = 0; i < *last - *first + 1; i++) {
in.read(reinterpret_cast<char*>(&num), sizeof(num));
field[i] = num;
}
in.close();
} else {
std::cout << "error: cannot read file " << path + "/" + filename + laser_id + ".dat" << std::endl;
}
return;
}
\end{lstlisting}

\begin{lstlisting}[style=FORTRAN, caption=Fortran interfaces for C++ library functions]
INTERFACE

SUBROUTINE compute_laser_at_boundary(rank, nproc, laser_start, laser_end, &
fwhm_time, t_0, omega, pos, amp, w_0, id, L_min, L_max, L_focus, T_min, T_max, &
T_ncells, cpml_thickness, t_end, T_cell_size, L_cell_size, dt, output_path) bind(c)
USE, INTRINSIC :: iso_c_binding
IMPLICIT NONE
INTEGER(c_int), INTENT(IN) :: rank, nproc, id, T_ncells, cpml_thickness
CHARACTER(kind=c_char), DIMENSION(*), INTENT(IN) :: output_path
REAL(c_double), INTENT(IN) :: laser_start, laser_end, fwhm_time, t_0, omega, pos, &
amp, w_0, L_min, L_max, L_focus, T_min, T_max, t_end, T_cell_size, L_cell_size, dt
END SUBROUTINE compute_laser_at_boundary

SUBROUTINE populate_laser_at_boundary(field, laser_id, output_path, timestep, size_global, first, last) bind(c)
USE, INTRINSIC :: iso_c_binding
IMPLICIT NONE
INTEGER(c_int), INTENT(IN) :: laser_id, timestep, size_global, first, last
CHARACTER(kind=c_char), DIMENSION(*), INTENT(IN) :: output_path
REAL(c_double), DIMENSION(*), INTENT(OUT) :: field
END SUBROUTINE populate_laser_at_boundary

END INTERFACE
\end{lstlisting}

\begin{lstlisting}[style=FORTRAN, caption=Fortran subroutines for Maxwell consistent computation of laser fields at boundaries]
SUBROUTINE Maxwell_consistent_computation_of_EM_fields

TYPE(laser_block), POINTER :: current

current => laser_x_min
DO WHILE(ASSOCIATED(current))
CALL compute_laser_at_boundary(rank, nproc, current%t_start, current%t_end, &
current%fwhm_time, current%t_0, current%omega, current%pos, current%amp, current%w_0, &
current%id, x_min, x_max, current%focus, y_min, y_max, ny_global, cpml_thickness, t_end, &
dy, dx, dt, TRIM(data_dir)//C_NULL_CHAR)
current => current%next
ENDDO

current => laser_x_max
DO WHILE(ASSOCIATED(current))
CALL compute_laser_at_boundary(rank, nproc, current%t_start, current%t_end, &
current%fwhm_time, current%t_0, current%omega, current%pos, current%amp, current%w_0, &
current%id, x_min, x_max, current%focus, y_min, y_max, ny_global, cpml_thickness, t_end, &
dy, dx, dt, TRIM(data_dir)//C_NULL_CHAR)
current => current%next
ENDDO

current => laser_y_min
DO WHILE(ASSOCIATED(current))
CALL compute_laser_at_boundary(rank, nproc, current%t_start, current%t_end, &
current%fwhm_time, current%t_0, current%omega, current%pos, current%amp, current%w_0, &
current%id, y_min, y_max, current%focus, x_min, x_max, nx_global, cpml_thickness, t_end, &
dx, dy, dt, TRIM(data_dir)//C_NULL_CHAR)
current => current%next
ENDDO

current => laser_y_max
DO WHILE(ASSOCIATED(current))
CALL compute_laser_at_boundary(rank, nproc, current%t_start, current%t_end, &
current%fwhm_time, current%t_0, current%omega, current%pos, current%amp, current%w_0, &
current%id, y_min, y_max, current%focus, x_min, x_max, nx_global, cpml_thickness, t_end, &
dx, dy, dt, TRIM(data_dir)//C_NULL_CHAR)
current => current%next
ENDDO

END SUBROUTINE Maxwell_consistent_computation_of_EM_fields
\end{lstlisting}

\begin{lstlisting}[style=FORTRAN, caption=Fortran subroutines for populating laser sources at boundaries]
SUBROUTINE get_source_x_boundary(buffer, laser_id)
REAL(num), DIMENSION(:), INTENT(INOUT) :: buffer
INTEGER, INTENT(IN) :: laser_id
CALL populate_laser_at_boundary(buffer, laser_id, TRIM(data_dir)//C_NULL_CHAR, &
step, ny_global, ny_global_min, ny_global_max)
END SUBROUTINE get_source_x_boundary
  
SUBROUTINE get_source_y_boundary(buffer, laser_id)
REAL(num), DIMENSION(:), INTENT(INOUT) :: buffer
INTEGER, INTENT(IN) :: laser_id
CALL populate_laser_at_boundary(buffer, laser_id, TRIM(data_dir)//C_NULL_CHAR, &
step, nx_global, nx_global_min, nx_global_max)
END SUBROUTINE get_source_y_boundary
\end{lstlisting}

\begin{lstlisting}[style=FORTRAN, caption=CMakeLists]
cmake_minimum_required(VERSION 3.1)
project(EPOCH_2D)
enable_language(CXX Fortran)

set(CMAKE_MODULE_PATH ${CMAKE_SOURCE_DIR}/cmake)
set(CMAKE_Fortran_MODULE_DIRECTORY ${CMAKE_SOURCE_DIR}/obj)
set(CMAKE_ARCHIVE_OUTPUT_DIRECTORY ${CMAKE_SOURCE_DIR}/lib)
set(EXECUTABLE_OUTPUT_PATH ${CMAKE_SOURCE_DIR}/bin)

find_package(MPI REQUIRED)
find_package(SDF REQUIRED)

include_directories(${MPI_Fortran_INCLUDE_PATH})
include_directories(${SDF_Fortran_INCLUDE_PATH})
include_directories(src/include)

execute_process(COMMAND ./src/gen_commit_string.sh)
execute_process(COMMAND grep -oP "(?<=COMMIT=)[^ ]+" ./src/COMMIT OUTPUT_VARIABLE COMMIT)
execute_process(COMMAND date +%s OUTPUT_VARIABLE DATE)
execute_process(COMMAND uname -n OUTPUT_VARIABLE MACHINE)

add_definitions('-D_COMMIT="${COMMIT}"')
add_definitions('-D_DATE=${DATE}')
add_definitions('-D_MACHINE="${MACHINE}"')

if(NOT CMAKE_BUILD_TYPE AND NOT CMAKE_CONFIGURATION_TYPES)
message(STATUS "Setting build type to 'Release', Debug mode was not specified.")
set(CMAKE_BUILD_TYPE Release CACHE STRING "Choose the type of build." FORCE)
# Set the possible values of build type for cmake-gui
set_property(CACHE CMAKE_BUILD_TYPE PROPERTY STRINGS "Debug" "Release")
endif()

if(${CMAKE_Fortran_COMPILER_ID} MATCHES "Intel")
set(CMAKE_Fortran_FLAGS_RELEASE "-O3 -xHost -no-prec-div -fno-math-errno -unroll=3 -qopt-subscript-in-range -align all")
set(CMAKE_Fortran_FLAGS_DEBUG "-O0 -nothreads -traceback -fltconsistency -C -g -heap-arrays 64 -warn -fp-stack-check -check bounds -fpe0")
elseif(${CMAKE_Fortran_COMPILER_ID} MATCHES "GNU")
set(CMAKE_Fortran_FLAGS_RELEASE "-O2 -fimplicit-none -ffixed-line-length-132")
set(CMAKE_Fortran_FLAGS_DEBUG "-O0 -g -Wall -Wextra -pedantic -fbounds-check -ffpe-trap=invalid,zero,overflow -Wno-unused-parameter")
elseif(${CMAKE_Fortran_COMPILER_ID} MATCHES "PGI")
  set(CMAKE_Fortran_FLAGS_RELEASE "-r8 -fast -fastsse -O3 -Mipa=fast,inline -Minfo")
  set(CMAKE_Fortran_FLAGS_DEBUG "-Mbounds -g")
elseif(${CMAKE_Fortran_COMPILER_ID} MATCHES "G95")
  set(CMAKE_Fortran_FLAGS_RELEASE "-O3")
  set(CMAKE_Fortran_FLAGS_DEBUG "-O0 -g")
elseif(${CMAKE_Fortran_COMPILER_ID} MATCHES "XL")
  set(CMAKE_Fortran_FLAGS_RELEASE "-O5 -qhot -qipa")
  set(CMAKE_Fortran_FLAGS_DEBUG "-O0 -C -g -qfullpath -qinfo -qnosmp -qxflag=dvz -Q! -qnounwind -qnounroll")
else()
message(STATUS "No optimized Fortran compiler flags are known")
message(STATUS "Fortran compiler full path: " ${CMAKE_Fortran_COMPILER})
set(CMAKE_Fortran_FLAGS_RELEASE "-O2")
set(CMAKE_Fortran_FLAGS_DEBUG   "-O0 -g")
endif()

if(${CMAKE_CXX_COMPILER_ID} MATCHES "Intel")
set(CMAKE_CXX_FLAGS_RELEASE "-O3 -std=c++11 -no-prec-div -ansi-alias -qopt-prefetch=4 -unroll-aggressive -m64")
set(CMAKE_CXX_FLAGS_DEBUG "-O0 -std=c++11 -g -traceback -mp1 -fp-trap=common -fp-model strict")
elseif(${CMAKE_CXX_COMPILER_ID} MATCHES "GNU")
set(CMAKE_CXX_FLAGS_RELEASE "-O2 -std=c++11 -msse4 -mtune=native -march=native -funroll-loops -fno-math-errno -ffast-math")
set(CMAKE_CXX_FLAGS_DEBUG "-O0 -std=c++11 -g -pedantic -Wall -Wextra -Wno-unused")
elseif(${CMAKE_CXX_COMPILER_ID} MATCHES "PGI")
  set(CMAKE_CXX_FLAGS_RELEASE "-std=c++0x")
  set(CMAKE_CXX_FLAGS_DEBUG "-std=c++0x")
elseif(${CMAKE_CXX_COMPILER_ID} MATCHES "G95")
  set(CMAKE_CXX_FLAGS_RELEASE "-std=c++0x")
  set(CMAKE_CXX_FLAGS_DEBUG "-std=c++0x")
elseif(${CMAKE_CXX_COMPILER_ID} MATCHES "XL")
  set(CMAKE_CXX_FLAGS_RELEASE "-qlanglvl=extended0x")
  set(CMAKE_CXX_FLAGS_DEBUG "-qlanglvl=extended0x")
else()
message(STATUS "No optimized C++ compiler flags are known")
message(STATUS "C++ compiler full path: " ${CMAKE_CXX_COMPILER})
set(CMAKE_CXX_FLAGS_RELEASE "-O2 -std=c++11")
set(CMAKE_CXX_FLAGS_DEBUG "-O0 -std=c++11 -g")
endif()

set(SOURCES
${CMAKE_SOURCE_DIR}/src/epoch2d.F90
${CMAKE_SOURCE_DIR}/src/boundary.f90
${CMAKE_SOURCE_DIR}/src/fields.f90
${CMAKE_SOURCE_DIR}/src/laser.F90
${CMAKE_SOURCE_DIR}/src/particles.F90
${CMAKE_SOURCE_DIR}/src/shared_data.F90
)

set(FOLDERS deck housekeeping io parser physics_packages user_interaction)
foreach(FOLDER ${FOLDERS})
file(GLOB TMP ${CMAKE_SOURCE_DIR}/src/${FOLDER}/*)
list(APPEND SOURCES ${TMP})
endforeach()

option(OPENMP "Enable multithreading using OpenMP directives." OFF)
option(PER_SPECIES_WEIGHT "Set every pseudoparticle in a species to represent the same number of real particles." OFF)
option(NO_TRACER_PARTICLES "Don't enable support for tracer particles." OFF)
option(NO_PARTICLE_PROBES "Don't enable support for particle probes." OFF)
option(PARTICLE_SHAPE_TOPHAT "Use second order particle weighting." OFF)
option(PARTICLE_SHAPE_BSPLINE3 "Use fifth order particle weighting." OFF)
option(PARTICLE_ID "Include a unique global particle ID using an 8-byte integer." OFF)
option(PARTICLE_ID4 "Include a unique global particle ID using an 4-byte integer." OFF)
option(PARTICLE_COUNT_UPDATE "Keep global particle counts up to date." OFF)
option(PHOTONS "Include QED routines" OFF)
option(TRIDENT_PHOTONS "Use the Trident process for pair production." OFF)
option(PREFETCH "Use Intel-specific 'mm_prefetch' calls to load next particle in the list into cache ahead of time." OFF)
option(PARSER_DEBUG "Turn on debugging." OFF)
option(PARTICLE_DEBUG "Turn on debugging." OFF)
option(MPI_DEBUG "Turn on debugging." OFF)
option(SIMPLIFY_DEBUG "Turn on debugging." OFF)
option(NO_IO "Don't generate any output at all. Useful for benchmarking." OFF)
option(COLLISIONS_TEST "Bypass the main simulation and only perform collision tests." OFF)
option(PER_PARTICLE_CHARGE_MASS "specify charge and mass per particle rather than per species." OFF)
option(PARSER_CHECKING "Perform checks on evaluated deck expressions." OFF)
option(USE_INSITU "Link epoch with ParaView Catalyst." OFF)
option(INSITU_DOUBLE_PREC "Double precision for data exported insitu." OFF)
option(TIGHT_FOCUSING "Maxwell consistent computation of EM fields at boundary for tight-focusing." OFF)

if(OPENMP)
message(STATUS "Option 'OPENMP' enabled")
add_definitions("-DOPENMP")
if(${CMAKE_Fortran_COMPILER_ID} MATCHES "Intel")
set(CMAKE_Fortran_FLAGS_RELEASE "${CMAKE_Fortran_FLAGS_RELEASE} -qopenmp")
set(CMAKE_Fortran_FLAGS_DEBUG "${CMAKE_Fortran_FLAGS_DEBUG} -qopenmp")
elseif(${CMAKE_Fortran_COMPILER_ID} MATCHES "GNU")
set(CMAKE_Fortran_FLAGS_RELEASE "${CMAKE_Fortran_FLAGS_RELEASE} -fopenmp")
set(CMAKE_Fortran_FLAGS_DEBUG "${CMAKE_Fortran_FLAGS_DEBUG} -fopenmp")
else()
message(STATUS "Fortran OpenMP compiler flag not known.")
endif()
if(${CMAKE_CXX_COMPILER_ID} MATCHES "Intel")
set(CMAKE_CXX_FLAGS_RELEASE "${CMAKE_CXX_FLAGS_RELEASE} -qopenmp")
set(CMAKE_CXX_FLAGS_DEBUG "${CMAKE_CXX_FLAGS_DEBUG} -qopenmp")
elseif(${CMAKE_CXX_COMPILER_ID} MATCHES "GNU")
set(CMAKE_CXX_FLAGS_RELEASE "${CMAKE_CXX_FLAGS_RELEASE} -fopenmp")
set(CMAKE_CXX_FLAGS_DEBUG "${CMAKE_CXX_FLAGS_DEBUG} -fopenmp")
else()
message(STATUS "C++ OpenMP compiler flag not known.")
endif()
endif()

if(TIGHT_FOCUSING AND NOT FFT_LIBRARY)
message(STATUS "No FFT library specified. Setting FFT library to 'none'.")
set(FFT_LIBRARY none CACHE STRING "Choose the FFT library." FORCE)
set_property(CACHE FFT_LIBRARY PROPERTY STRINGS "FFTW" "MKL" "None")
endif()

if(PER_SPECIES_WEIGHT)
message(STATUS "Option 'PER_SPECIES_WEIGHT' enabled")
add_definitions("-DPER_SPECIES_WEIGHT")
endif()

if(NO_TRACER_PARTICLES)
message(STATUS "Option 'NO_TRACER_PARTICLES' enabled")
add_definitions("-DNO_TRACER_PARTICLES")
endif()

if(NO_PARTICLE_PROBES)
message(STATUS "Option 'NO_PARTICLE_PROBES' enabled")
add_definitions("-DNO_PARTICLE_PROBES")
endif()

if(PARTICLE_SHAPE_TOPHAT)
message(STATUS "Option 'PARTICLE_SHAPE_TOPHAT' enabled")
add_definitions("-DPARTICLE_SHAPE_TOPHAT")
endif()

if(PARTICLE_SHAPE_BSPLINE3)
message(STATUS "Option 'PARTICLE_SHAPE_BSPLINE3' enabled")
add_definitions("-DPARTICLE_SHAPE_BSPLINE3")
endif()

if(PARTICLE_ID)
message(STATUS "Option 'PARTICLE_ID' enabled")
add_definitions("-DPARTICLE_ID")
endif()

if(PARTICLE_ID4)
message(STATUS "Option 'PARTICLE_ID4' enabled")
add_definitions("-DPARTICLE_ID4")
endif()

if(PARTICLE_COUNT_UPDATE)
message(STATUS "Option 'PARTICLE_COUNT_UPDATE' enabled")
add_definitions("-DPARTICLE_COUNT_UPDATE")
endif()

if(PHOTONS)
message(STATUS "Option 'PHOTONS' enabled")
add_definitions("-DPHOTONS")
endif()

if(TRIDENT_PHOTONS)
message(STATUS "Option 'TRIDENT_PHOTONS' enabled")
add_definitions("-DTRIDENT_PHOTONS")
endif()

if(PREFETCH)
message(STATUS "Option 'PREFETCH' enabled")
add_definitions("-DPREFETCH")
endif()

if(PARSER_DEBUG)
message(STATUS "Option 'PARSER_DEBUG' enabled")
add_definitions("-DPARSER_DEBUG")
endif()

if(PARTICLE_DEBUG)
message(STATUS "Option 'PARTICLE_DEBUG' enabled")
add_definitions("-DPARTICLE_DEBUG")
endif()

if(MPI_DEBUG)
message(STATUS "Option 'MPI_DEBUG' enabled")
add_definitions("-DMPI_DEBUG")
endif()

if(SIMPLIFY_DEBUG)
message(STATUS "Option 'SIMPLIFY_DEBUG' enabled")
add_definitions("-DSIMPLIFY_DEBUG")
endif()

if(NO_IO)
message(STATUS "Option 'NO_IO' enabled")
add_definitions("-DNO_IO")
endif()

if(COLLISIONS_TEST)
message(STATUS "Option 'COLLISIONS_TEST' enabled")
add_definitions("-DCOLLISIONS_TEST")
endif()

if(PER_PARTICLE_CHARGE_MASS)
message(STATUS "Option 'PER_PARTICLE_CHARGE_MASS' enabled")
add_definitions("-DPER_PARTICLE_CHARGE_MASS")
endif()

if(PARSER_CHECKING)
message(STATUS "Option 'PARSER_CHECKING' enabled")
add_definitions("-DPARSER_CHECKING")
endif()

if(USE_INSITU)
message(STATUS "Option 'USE_INSITU' enabled")
find_package(ParaView 5.2 REQUIRED COMPONENTS vtkPVPythonCatalyst)
include(${PARAVIEW_USE_FILE})
file(GLOB ADAPTOR ${CMAKE_SOURCE_DIR}/src/adaptor/*)
list(APPEND SOURCES ${ADAPTOR})
add_definitions("-DINSITU")
if(NOT PARAVIEW_USE_MPI)
message(SEND_ERROR "ParaView must be built with MPI enabled")
endif()
if(INSITU_DOUBLE_PREC)
message(STATUS "Option 'INSITU_DOUBLE_PREC' enabled")
add_definitions("-DINSITU_DOUBLE_PREC")
endif()
endif()

if(TIGHT_FOCUSING)
message(STATUS "Option 'TIGHT_FOCUSING' enabled")
file(GLOB FOCUSING ${CMAKE_SOURCE_DIR}/src/focusing/interface.f90)
list(APPEND SOURCES ${FOCUSING})
include_directories(${MPI_CXX_INCLUDE_PATH})
add_definitions("-DTIGHT_FOCUSING")
set(FOCUS_SRC
${CMAKE_SOURCE_DIR}/src/focusing/main.cpp
${CMAKE_SOURCE_DIR}/src/focusing/domain_param.cpp
${CMAKE_SOURCE_DIR}/src/focusing/laser_param.cpp
${CMAKE_SOURCE_DIR}/src/focusing/global.cpp
${CMAKE_SOURCE_DIR}/src/focusing/laser_bcs.cpp
)
include_directories(${CMAKE_SOURCE_DIR}/src/focusing/inc)
if(${FFT_LIBRARY} MATCHES "FFTW")
if(OPENMP)
message(SEND_ERROR "Multi-threaded FFTW routines are not supported. Disable OpenMP.")
endif()
message(STATUS "FFT library: FFTW")
message(STATUS "Option 'USE_FFTW' enabled")
add_definitions("-DUSE_FFTW")
find_package(FFTW REQUIRED)
include_directories(${FFTW_INCLUDES})
endif()
if(${FFT_LIBRARY} MATCHES "MKL")
if(NOT ${CMAKE_CXX_COMPILER_ID} MATCHES "Intel")
message(SEND_ERROR "MKL library can be used only with Intel compilers")
endif()
message(STATUS "FFT library: MKL")
message(STATUS "Option 'USE_MKL' enabled")
add_definitions("-DUSE_MKL")
set(CMAKE_CXX_FLAGS_RELEASE "${CMAKE_CXX_FLAGS_RELEASE} -mkl")
set(CMAKE_CXX_FLAGS_DEBUG "${CMAKE_CXX_FLAGS_DEBUG} -mkl")
set(CMAKE_Fortran_FLAGS_RELEASE "${CMAKE_Fortran_FLAGS_RELEASE} -mkl")
set(CMAKE_Fortran_FLAGS_DEBUG "${CMAKE_Fortran_FLAGS_DEBUG} -mkl")
endif()
if(${FFT_LIBRARY} MATCHES "None")
message(STATUS "FFT library: None")
endif()
add_library(focus ${FOCUS_SRC})
if(${FFT_LIBRARY} MATCHES "FFTW")
#if(OPENMP)
#  target_link_libraries(focus LINK_PUBLIC ${FFTW_LIBRARIES} ${FFTW_LIBRARIES_OMP})
#else()
target_link_libraries(focus LINK_PUBLIC ${FFTW_LIBRARIES})
#endif()
endif()
set_target_properties(focus PROPERTIES LINKER_LANGUAGE CXX)
endif()

add_executable(epoch2d ${SOURCES})
target_link_libraries(epoch2d LINK_PRIVATE ${MPI_Fortran_LIBRARIES} ${SDF_Fortran_LIBRARIES})
if(USE_INSITU)
target_link_libraries(epoch2d LINK_PRIVATE vtkPVPythonCatalyst vtkParallelMPI)
endif()
if(TIGHT_FOCUSING)
target_link_libraries(epoch2d LINK_PRIVATE ${MPI_CXX_LIBRARIES} focus)
endif()
set_target_properties(epoch2d PROPERTIES LINKER_LANGUAGE Fortran)
\end{lstlisting}

\chapter{Code listings}
Below, one can find the most important functions and methods that has been created within this work to provide some new functionalities and features into the code EPOCH. These serve mainly for the computations of laser fields at boundaries that are consistent with the Maxwell's equations using discrete Fourier transforms. Also, one can find the methods for the data manipulation and routines that interface corresponding C++ functions into the FORTRAN simulation code. Finally, the C++ and FORTRAN adaptors for ParaView Catalyst that enable in-situ visualization and diagnostics with a sample visualization Python script pipeline are all attached.

The following part is provided as it is, only with a short captions. It is mainly intended for those who are interested in the way of implementation and do not want to browse in the full source code, which can be found on the attached CD. Closer details are discussed in the third and fourth chapter of this work.

\begin{lstlisting}[style=CXX, caption=Function performing forward fast Fourier transform using Intel$ ^{\scriptsize \textregistered} $ MKL library]
std::vector<std::complex<double>> fft::mkl_fft_forward(std::vector<std::complex<double>> in) {
DFTI_DESCRIPTOR_HANDLE desc;
MKL_LONG status;
DftiCreateDescriptor(&desc, DFTI_DOUBLE, DFTI_COMPLEX, 1, static_cast<MKL_LONG>(in.size()));
DftiCommitDescriptor(desc);
status = DftiComputeForward(desc, in.data());
if(status != 0) {
std::cerr << DftiErrorMessage(status) << std::endl;
abort();
}
DftiFreeDescriptor(&desc);
return in;
}
\end{lstlisting}

\begin{lstlisting}[style=CXX, caption=Function performing backward fast Fourier transform using Intel$ ^{\scriptsize \textregistered} $ MKL library]
std::vector<std::complex<double>> fft::mkl_fft_backward(std::vector<std::complex<double>> in) {
DFTI_DESCRIPTOR_HANDLE desc;
MKL_LONG status;
DftiCreateDescriptor(&desc, DFTI_DOUBLE, DFTI_COMPLEX, 1, static_cast<MKL_LONG>(in.size()));
DftiCommitDescriptor(desc);
status = DftiComputeBackward(desc, in.data());
if(status != 0) {
std::cerr << DftiErrorMessage(status) << std::endl;
abort();
}
DftiFreeDescriptor(&desc);
return in;
}
\end{lstlisting}

\begin{lstlisting}[style=CXX, caption=Function performing forward fast Fourier transform using FFTW library]
std::vector<std::complex<double>> fft::fftw_fft_forward(std::vector<std::complex<double>> in) {
fftw_plan p = fftw_plan_dft_1d(in.size(), reinterpret_cast<fftw_complex*>(in.data()), reinterpret_cast<fftw_complex*>(in.data()), FFTW_FORWARD, FFTW_ESTIMATE);
fftw_execute(p);
fftw_destroy_plan(p);
return in;
}
\end{lstlisting}

\begin{lstlisting}[style=CXX, caption=Function performing backward fast Fourier transform using FFTW library]
std::vector<std::complex<double>> fft::fftw_fft_backward(std::vector<std::complex<double>> in) {
fftw_plan p = fftw_plan_dft_1d(in.size(), reinterpret_cast<fftw_complex*>(in.data()), reinterpret_cast<fftw_complex*>(in.data()), FFTW_BACKWARD, FFTW_ESTIMATE);
fftw_execute(p);
fftw_destroy_plan(p);
return in;
}
\end{lstlisting}

\begin{lstlisting}[style=CXX, caption=Function performing forward discrete Fourier transform without using any library]
std::vector<std::complex<double>> fft::fft_forward(std::vector<std::complex<double>> in) {
std::vector<std::complex<double>> out(in.size());
for(auto j = 0; j < out.size(); j++) {
for(auto l = 0; l < out.size(); l++) {
out.at(j) += in.at(l) * exp(-2.0 * constants::pi * I * l * j / in.size());
}
}
return out;
}
\end{lstlisting}

\begin{lstlisting}[style=CXX, caption=Function performing backward discrete Fourier transform without using any library]
std::vector<std::complex<double>> fft::fft_backward(std::vector<std::complex<double>> in) {
std::vector<std::complex<double>> out(in.size());
for(auto j = 0; j < out.size(); j++) {
for(auto l = 0; l < out.size(); l++) {
out.at(j) += in.at(l) * exp(+2.0 * constants::pi * I * l * j / in.size());
}
}
return out;
}
\end{lstlisting}

\begin{lstlisting}[style=CXX, caption=Method for performing discrete Fourier transform in time]
void laser_bcs::dft_time(field_2d<std::complex<double>>& field) const {
#ifdef OPENMP
#pragma omp parallel for schedule(static)
#endif
for(auto j = 0; j < this->domain->Nx; j++) {
#ifdef USE_MKL
field.add_col(fft::mkl_fft_backward(field.get_col(j)), j);
#elif USE_FFTW
field.add_col(fft::fftw_fft_backward(field.get_col(j)), j);
#else
field.add_col(fft::fft_backward(field.get_col(j)), j);
#endif
}
field.multiply(this->domain->dt / (2.0 * constants::pi));
return;
}
\end{lstlisting}

\begin{lstlisting}[style=CXX, caption=Method for performing inverse discrete Fourier transform in time]
void laser_bcs::idft_time(field_2d<std::complex<double>>& field) const {
#ifdef OPENMP
#pragma omp parallel for schedule(static)
#endif
for(auto j = 0; j < this->domain->Nx; j++) {
#ifdef USE_MKL
field.add_col(fft::mkl_fft_forward(field.get_col(j)), j);
#elif USE_FFTW
field.add_col(fft::fftw_fft_forward(field.get_col(j)), j);
#else
field.add_col(fft::fft_forward(field.get_col(j)), j);
#endif
}
field.multiply(2.0 * (2.0 * constants::pi) / (this->domain->Nt * this->domain->dt));
return;
}
\end{lstlisting}

\begin{lstlisting}[style=CXX, caption=Method for performing discrete Fourier transform in space]
void laser_bcs::dft_space(field_2d<std::complex<double>>& field) const {
std::vector<std::complex<double>> row_global(this->domain->Nx_global);
std::vector<std::complex<double>> row_local;
for(auto j = 0; j < this->domain->Nt; j++) {
row_local = field.get_row(j);
MPI_Gatherv(row_local.data(), this->domain->Nx, MPI_CXX_DOUBLE_COMPLEX, row_global.data(), this->domain->counts.data(), this->domain->displs.data(), MPI_CXX_DOUBLE_COMPLEX, 0, MPI_COMM_WORLD);
if(this->domain->rank == 0) {
#ifdef USE_MKL
row_global = fft::mkl_fft_forward(row_global);
#elif USE_FFTW
row_global = fft::fftw_fft_forward(row_global);
#else
row_global = fft::fft_forward(row_global);
#endif
}
MPI_Scatterv(row_global.data(), this->domain->counts.data(), this->domain->displs.data(), 	MPI_CXX_DOUBLE_COMPLEX, row_local.data(), this->domain->Nx, MPI_CXX_DOUBLE_COMPLEX, 0, MPI_COMM_WORLD);
field.add_row(row_local, j);
}
field.multiply(this->domain->dx / (2.0 * constants::pi));
return;
}
\end{lstlisting}

\begin{lstlisting}[style=CXX, caption=Method for performing inverse discrete Fourier transform in space]
void laser_bcs::idft_space(field_2d<std::complex<double>>& field) const {
std::vector<std::complex<double>> row_global(this->domain->Nx_global);
std::vector<std::complex<double>> row_local;
for(auto j = 0; j < this->domain->Nt; j++) {
row_local = field.get_row(j);
MPI_Gatherv(row_local.data(), this->domain->Nx, MPI_CXX_DOUBLE_COMPLEX, row_global.data(), this->domain->counts.data(), this->domain->displs.data(), MPI_CXX_DOUBLE_COMPLEX, 0, MPI_COMM_WORLD);
if(this->domain->rank == 0) {
#ifdef USE_MKL
row_global = fft::mkl_fft_backward(row_global);
#elif USE_FFTW
row_global = fft::fftw_fft_backward(row_global);
#else
row_global = fft::fft_backward(row_global);
#endif
}
MPI_Scatterv(row_global.data(), this->domain->counts.data(), this->domain->displs.data(), MPI_CXX_DOUBLE_COMPLEX, row_local.data(), this->domain->Nx, MPI_CXX_DOUBLE_COMPLEX, 0, MPI_COMM_WORLD);
field.add_row(row_local, j);
}
field.multiply((2.0 * constants::pi) / (this->domain->Nx_global * this->domain->dx));
return;
}
\end{lstlisting}

\begin{lstlisting}[style=CXX, caption=Method for dumping data into shared file]
template <typename T>
void field_2d<T>::dump_to_shared_file(std::string name, int row_first, int row_last, int row_size_local, int row_size_global, int col_start) const {
MPI_File file;
MPI_Offset offset = 0;
MPI_Status status;
MPI_Datatype local_array;
int col_size = row_last - row_first;
const int ndims = 2;
std::array<int, ndims> size_global = {col_size, row_size_global};
std::array<int, ndims> size_local = {col_size, row_size_local};
std::array<int, ndims> start_coords = {0, col_start};
MPI_Type_create_subarray(2, size_global.data(), size_local.data(), start_coords.data(), MPI_ORDER_C, MPI_DOUBLE, &local_array);
MPI_Type_commit(&local_array);
std::vector<double> real_part(col_size * row_size_local);
for(auto i = std::make_pair(row_first, 0); i.first < row_last; i.first++, i.second++) {
for(auto j = 0; j < row_size_local; j++) {
real_part[i.second * row_size_local + j] = std::real(this->data[i.first * row_size_local + j]);
}
}
MPI_File_open(MPI_COMM_WORLD, name.data(), MPI_MODE_CREATE|MPI_MODE_WRONLY, MPI_INFO_NULL, &file);
MPI_File_set_view(file, offset, MPI_DOUBLE, local_array, "native", MPI_INFO_NULL);
MPI_File_write_all(file, real_part.data(), col_size * row_size_local, MPI_DOUBLE, &status);
MPI_File_close(&file);
MPI_Type_free(&local_array);
return;
}
\end{lstlisting}

\hspace{1cm}

\begin{lstlisting}[style=CXX, caption=Extern C++ function to fill Fortran arrays with laser fields dumped in binary file]
void populate_laser_at_boundary(double* field, int* id, const char* data_dir, int* timestep, int* size_global, int* first, int* last) {
double num = 0.0;
std::string laser_id = std::to_string(*id);
std::string path(data_dir);
std::ifstream in;
in.open(path + "/laser_" + laser_id + ".dat", std::ios::binary);
if(in.is_open()) {
in.seekg(((*timestep) * (*size_global) + (*first) - 1) * sizeof(num));
for(auto i = 0; i < *last - *first + 1; i++) {
in.read(reinterpret_cast<char*>(&num), sizeof(num));
field[i] = num;
}
in.close();
} else {
std::cout << "error: cannot read file " << path + "/" + filename + laser_id + ".dat" << std::endl;
}
return;
}
\end{lstlisting}

\begin{lstlisting}[style=FORTRAN, caption=Fortran interfaces for C++ library functions]
INTERFACE

SUBROUTINE compute_laser_at_boundary(rank, nproc, laser_start, laser_end, &
fwhm_time, t_0, omega, pos, amp, w_0, id, L_min, L_max, L_focus, T_min, T_max, &
T_ncells, cpml_thickness, t_end, T_cell_size, L_cell_size, dt, output_path) bind(c)
USE, INTRINSIC :: iso_c_binding
IMPLICIT NONE
INTEGER(c_int), INTENT(IN) :: rank, nproc, id, T_ncells, cpml_thickness
CHARACTER(kind=c_char), DIMENSION(*), INTENT(IN) :: output_path
REAL(c_double), INTENT(IN) :: laser_start, laser_end, fwhm_time, t_0, omega, pos, &
amp, w_0, L_min, L_max, L_focus, T_min, T_max, t_end, T_cell_size, L_cell_size, dt
END SUBROUTINE compute_laser_at_boundary

SUBROUTINE populate_laser_at_boundary(field, laser_id, output_path, timestep, size_global, first, last) bind(c)
USE, INTRINSIC :: iso_c_binding
IMPLICIT NONE
INTEGER(c_int), INTENT(IN) :: laser_id, timestep, size_global, first, last
CHARACTER(kind=c_char), DIMENSION(*), INTENT(IN) :: output_path
REAL(c_double), DIMENSION(*), INTENT(OUT) :: field
END SUBROUTINE populate_laser_at_boundary

END INTERFACE
\end{lstlisting}

\begin{lstlisting}[style=FORTRAN, caption=Fortran subroutines for Maxwell consistent computation of laser fields at boundaries]
SUBROUTINE Maxwell_consistent_computation_of_EM_fields

TYPE(laser_block), POINTER :: current

current => laser_x_min
DO WHILE(ASSOCIATED(current))
CALL compute_laser_at_boundary(rank, nproc, current%t_start, current%t_end, &
current%fwhm_time, current%t_0, current%omega, current%pos, current%amp, current%w_0, &
current%id, x_min, x_max, current%focus, y_min, y_max, ny_global, cpml_thickness, t_end, &
dy, dx, dt, TRIM(data_dir)//C_NULL_CHAR)
current => current%next
ENDDO

current => laser_x_max
DO WHILE(ASSOCIATED(current))
CALL compute_laser_at_boundary(rank, nproc, current%t_start, current%t_end, &
current%fwhm_time, current%t_0, current%omega, current%pos, current%amp, current%w_0, &
current%id, x_min, x_max, current%focus, y_min, y_max, ny_global, cpml_thickness, t_end, &
dy, dx, dt, TRIM(data_dir)//C_NULL_CHAR)
current => current%next
ENDDO

current => laser_y_min
DO WHILE(ASSOCIATED(current))
CALL compute_laser_at_boundary(rank, nproc, current%t_start, current%t_end, &
current%fwhm_time, current%t_0, current%omega, current%pos, current%amp, current%w_0, &
current%id, y_min, y_max, current%focus, x_min, x_max, nx_global, cpml_thickness, t_end, &
dx, dy, dt, TRIM(data_dir)//C_NULL_CHAR)
current => current%next
ENDDO

current => laser_y_max
DO WHILE(ASSOCIATED(current))
CALL compute_laser_at_boundary(rank, nproc, current%t_start, current%t_end, &
current%fwhm_time, current%t_0, current%omega, current%pos, current%amp, current%w_0, &
current%id, y_min, y_max, current%focus, x_min, x_max, nx_global, cpml_thickness, t_end, &
dx, dy, dt, TRIM(data_dir)//C_NULL_CHAR)
current => current%next
ENDDO

END SUBROUTINE Maxwell_consistent_computation_of_EM_fields
\end{lstlisting}

\begin{lstlisting}[style=FORTRAN, caption=Fortran subroutines for populating laser sources at boundaries]
SUBROUTINE get_source_x_boundary(buffer, laser_id)
REAL(num), DIMENSION(:), INTENT(INOUT) :: buffer
INTEGER, INTENT(IN) :: laser_id
CALL populate_laser_at_boundary(buffer, laser_id, TRIM(data_dir)//C_NULL_CHAR, &
step, ny_global, ny_global_min, ny_global_max)
END SUBROUTINE get_source_x_boundary
  
SUBROUTINE get_source_y_boundary(buffer, laser_id)
REAL(num), DIMENSION(:), INTENT(INOUT) :: buffer
INTEGER, INTENT(IN) :: laser_id
CALL populate_laser_at_boundary(buffer, laser_id, TRIM(data_dir)//C_NULL_CHAR, &
step, nx_global, nx_global_min, nx_global_max)
END SUBROUTINE get_source_y_boundary
\end{lstlisting}

\begin{lstlisting}[style=FORTRAN, caption=Fortran adaptor for ParaView Catalyst]
MODULE coprocessor

USE, INTRINSIC :: iso_c_binding
USE fields

IMPLICIT NONE

CONTAINS

SUBROUTINE init_coproc(step, time)
INTEGER, INTENT(in) :: step
REAL(num), INTENT(in) :: time
INTEGER :: ilen, i
CHARACTER(len=200) :: arg
CALL coprocessorinitialize()
DO i = 1, iargc()
CALL getarg(i, arg)
ilen = len_trim(arg)
arg(ilen+1:) = C_NULL_CHAR
CALL coprocessoraddpythonscript(arg, ilen)
ENDDO
CALL createinputdatadescription(step, time, "essential")
END SUBROUTINE init_coproc

SUBROUTINE run_coproc(step, time)
INTEGER, INTENT(in) :: step
REAL(num), INTENT(in) :: time
INTEGER :: flag, mytid, ntids, n, i, j
INTEGER, DIMENSION(6) :: local_extent, global_extent
INTEGER, DIMENSION(4) :: lim
REAL(num), DIMENSION(3*nx*ny) :: e_field, b_field
#ifdef OPENMP
    INTEGER :: omp_get_thread_num, omp_get_num_threads, omp_get_max_threads
    EXTERNAL :: omp_get_thread_num, omp_get_num_threads, omp_get_max_threads
#endif

local_extent = (/ nx_global_min, nx_global_max, ny_global_min, ny_global_max, 0, 0 /)
global_extent = (/ 1, nx_global, 1, ny_global, 0, 0 /)
lim = (/ 1 + ng, nx + ng, 1 + ng, ny + ng /)

CALL requestdatadescription(step, time, flag)
IF (flag /= 0) THEN
CALL buildgrid(rank, nproc, step, time, local_extent, global_extent, &
x(lim(1):lim(2)), y(lim(3):lim(4)), "essential"//C_NULL_CHAR)

!$omp parallel default(none) private(mytid,i,j,n) &
shared(ntids,lim,e_field,b_field,ex,ey,ez,bx,by,bz)
#ifdef OPENMP
     mytid = OMP_GET_THREAD_NUM()
     ntids = OMP_GET_NUM_THREADS()
#endif
DO j = 0, lim(4) - lim(3), ntids
n = (j+mytid)*(1 + lim(2) - lim(1))*3 + 1
IF(j + mytid <= lim(4)) THEN
DO i = 0, lim(2) - lim(1)
e_field(n + 0) = ex(lim(1) + i, lim(3) + j + mytid)
e_field(n + 1) = ey(lim(1) + i, lim(3) + j + mytid)
e_field(n + 2) = ez(lim(1) + i, lim(3) + j + mytid)
b_field(n + 0) = bx(lim(1) + i, lim(3) + j + mytid)
b_field(n + 1) = by(lim(1) + i, lim(3) + j + mytid)
b_field(n + 2) = bz(lim(1) + i, lim(3) + j + mytid)
n = n + 3
ENDDO
ENDIF
ENDDO
!$omp end parallel

CALL addfield(rank, e_field, "E (V/m)"//C_NULL_CHAR, 3, "essential"//C_NULL_CHAR)
CALL addfield(rank, b_field, "B (T)"//C_NULL_CHAR, 3, "essential"//C_NULL_CHAR)
CALL coprocess()
ENDIF
END SUBROUTINE run_coproc

SUBROUTINE finalise_coproc()
CALL coprocessorfinalize()
END SUBROUTINE finalise_coproc

END MODULE coprocessor
\end{lstlisting}

\begin{lstlisting}[style=CXX, caption=C++ adaptor for ParaView Catalyst]
#include "vtkCPDataDescription.h"
#include "vtkCPInputDataDescription.h"
#include "vtkCPProcessor.h"
#include "vtkCPPythonScriptPipeline.h"
#include "vtkCPPythonAdaptorAPI.h"

#ifdef INSITU_DOUBLE_PREC
#include "vtkDoubleArray.h"
#else
#include "vtkFloatArray.h"
#endif

#include "vtkSmartPointer.h"
#include "vtkRectilinearGrid.h"
#include "vtkPointData.h"
#include "vtkImageData.h"
#include "vtkMultiBlockDataSet.h"
#include "vtkMultiPieceDataSet.h"

#include <iostream>
#include <fstream>
#include <vector>
#include <array>

#ifdef __cplusplus
extern "C" {
#endif
void createinputdatadescription_(int* step, double* time, const char* grid_name);
void buildgrid_(int* rank, int* size, int* step, double* time, int* local_extent, int* global_extent, double* x_coords, double* y_coords, const char* grid_name);
void addfield_(int* rank, double* input_field, char* name, int* components, const char* grid_name);
#ifdef __cplusplus
}
#endif

void createinputdatadescription_(int* step, double* time, const char* grid_name) {
if (!vtkCPPythonAdaptorAPI::GetCoProcessorData()) {
vtkGenericWarningMacro("unable to access coprocessor data");
return;
}
vtkCPPythonAdaptorAPI::GetCoProcessorData()->AddInput(grid_name);
vtkCPPythonAdaptorAPI::GetCoProcessorData()->SetTimeData(*time, static_cast<vtkIdType>(*step));
return;
}

void buildgrid_(int* rank, int* size, int* step, double* time, int* local_extent, int* global_extent, double* x_coords, double* y_coords, const char* grid_name) {
if (!vtkCPPythonAdaptorAPI::GetCoProcessorData()) {
vtkGenericWarningMacro("unable to access coprocessor data");
return;
}
vtkCPPythonAdaptorAPI::GetCoProcessorData()->SetTimeData(*time, static_cast<vtkIdType>(*step));
if(!vtkCPPythonAdaptorAPI::GetCoProcessorData()->GetInputDescriptionByName(grid_name)->GetGrid()) {
vtkCPInputDataDescription* idd = vtkCPPythonAdaptorAPI::GetCoProcessorData()->GetInputDescriptionByName(grid_name);
if (!idd) {
vtkGenericWarningMacro("cannot access data description to attach grid to");
return;
}

vtkSmartPointer<vtkRectilinearGrid> rectilinear_grid =
vtkSmartPointer<vtkRectilinearGrid>::New();
rectilinear_grid->SetExtent(local_extent);

int* ext = rectilinear_grid->GetExtent();
int dim[3] = {ext[1] - ext[0] + 1, ext[3] - ext[2] + 1, ext[5] - ext[4] + 1};

#ifdef INSITU_DOUBLE_PREC
vtkSmartPointer<vtkDoubleArray> x_array = vtkSmartPointer<vtkDoubleArray>::New();
vtkSmartPointer<vtkDoubleArray> y_array = vtkSmartPointer<vtkDoubleArray>::New();
double* x_c = x_coords;
double* y_c = y_coords;
#else
vtkSmartPointer<vtkFloatArray> x_array = vtkSmartPointer<vtkFloatArray>::New();
vtkSmartPointer<vtkFloatArray> y_array = vtkSmartPointer<vtkFloatArray>::New();
float* x_c = new float[dim[0]];
float* y_c = new float[dim[1]];
for(std::size_t i = 0; i < dim[0]; i++) {
x_c[i] = static_cast<float>(x_coords[i]);
}
for(std::size_t i = 0; i < dim[1]; i++) {
y_c[i] = static_cast<float>(y_coords[i]);
}
#endif
x_array->SetNumberOfComponents(1);
y_array->SetNumberOfComponents(1);
x_array->SetArray(x_c, static_cast<vtkIdType>(dim[0]), 1);
y_array->SetArray(y_c, static_cast<vtkIdType>(dim[1]), 1);
rectilinear_grid->SetXCoordinates(x_array);
rectilinear_grid->SetYCoordinates(y_array);

vtkSmartPointer<vtkMultiPieceDataSet> multi_piece = vtkSmartPointer<vtkMultiPieceDataSet>::New();
multi_piece->SetNumberOfPieces(*size);
multi_piece->SetPiece(*rank, rectilinear_grid);

vtkSmartPointer<vtkMultiBlockDataSet> grid = vtkSmartPointer<vtkMultiBlockDataSet>::New();
grid->SetNumberOfBlocks(1);
grid->SetBlock(0, multi_piece);

idd->SetWholeExtent(global_extent);
idd->SetGrid(grid);
}
return;
}

void addfield_(int* rank, double* input_field, char* name, int* components, const char* grid_name) {
if (!vtkCPPythonAdaptorAPI::GetCoProcessorData()) {
vtkGenericWarningMacro("unable to access coprocessor data");
return;
}
vtkCPInputDataDescription* idd = vtkCPPythonAdaptorAPI::GetCoProcessorData()->GetInputDescriptionByName(grid_name);
vtkMultiBlockDataSet* multi_block = vtkMultiBlockDataSet::SafeDownCast(idd->GetGrid());
vtkMultiPieceDataSet* multi_piece = vtkMultiPieceDataSet::SafeDownCast(multi_block->GetBlock(0));
vtkRectilinearGrid* type = vtkRectilinearGrid::SafeDownCast(multi_piece->GetPiece(*rank));
if (!type) {
vtkGenericWarningMacro("no adaptor grid to attach field data to");
return;
}
if (idd->IsFieldNeeded(name)) {
int size = (*components)*type->GetNumberOfPoints();
#ifdef INSITU_DOUBLE_PREC
double* array = input_field;
vtkSmartPointer<vtkDoubleArray> field = vtkSmartPointer<vtkDoubleArray>::New();
#else
float* array = new float[size];
for(std::size_t i = 0; i < size; i++) {
array[i] = static_cast<float>(input_field[i]);
}
vtkSmartPointer<vtkFloatArray> field = vtkSmartPointer<vtkFloatArray>::New();
#endif
field->SetName(name);
field->SetNumberOfComponents(*components);
field->SetArray(array, static_cast<vtkIdType>(size), 1);
type->GetPointData()->AddArray(field);
}
return;
}
\end{lstlisting}

\begin{lstlisting}[style=CXX, caption=Sample visualization pipeline using Python script]
try: paraview.simple
except: from paraview.simple import *

from paraview import coprocessing

inputs = ['essential']
update_freq = 1
output_freq = 10000

def CreateCoProcessor():
def _CreatePipeline(coprocessor, datadescription):
class Pipeline:

essential = coprocessor.CreateProducer(datadescription, "essential")

multi_block_binary_writer = servermanager.writers.XMLMultiBlockDataWriter(Input=essential, DataMode='Appended', EncodeAppendedData=0, HeaderType='UInt32', CompressorType='ZLib')
coprocessor.RegisterWriter(multi_block_binary_writer, filename='full_grid_%t.vtm', freq=output_freq)

class CoProcessor(coprocessing.CoProcessor):
def CreatePipeline(self, datadescription):
self.Pipeline = _CreatePipeline(self, datadescription)

coprocessor = CoProcessor()
freqs = {}
for name in inputs:
freqs[name] = [update_freq]

coprocessor.SetUpdateFrequencies(freqs)
return coprocessor

coprocessor = CreateCoProcessor()
coprocessor.EnableLiveVisualization(True)

def RequestDataDescription(datadescription):
"Callback to populate the request for current timestep"
global coprocessor

if datadescription.GetForceOutput() == True:
for i in range(datadescription.GetNumberOfInputDescriptions()):
datadescription.GetInputDescription(i).AllFieldsOn()
datadescription.GetInputDescription(i).GenerateMeshOn()
return

coprocessor.LoadRequestedData(datadescription)

def DoCoProcessing(datadescription):
"Callback to do co-processing for current timestep"
global coprocessor

coprocessor.UpdateProducers(datadescription)
coprocessor.WriteData(datadescription);
coprocessor.WriteImages(datadescription, rescale_lookuptable=False)
coprocessor.DoLiveVisualization(datadescription, "visualization_node", 11111)
\end{lstlisting}

\begin{lstlisting}[style=FORTRAN, caption=EPOCH CMakeLists file to generate platform-specific build scripts]
cmake_minimum_required(VERSION 3.1)
project(EPOCH_2D)
enable_language(CXX Fortran)

set(CMAKE_MODULE_PATH ${CMAKE_SOURCE_DIR}/cmake)
set(CMAKE_Fortran_MODULE_DIRECTORY ${CMAKE_SOURCE_DIR}/obj)
set(CMAKE_ARCHIVE_OUTPUT_DIRECTORY ${CMAKE_SOURCE_DIR}/lib)
set(EXECUTABLE_OUTPUT_PATH ${CMAKE_SOURCE_DIR}/bin)

find_package(MPI REQUIRED)
find_package(SDF REQUIRED)

include_directories(${MPI_Fortran_INCLUDE_PATH})
include_directories(${SDF_Fortran_INCLUDE_PATH})
include_directories(src/include)

execute_process(COMMAND ./src/gen_commit_string.sh)
execute_process(COMMAND grep -oP "(?<=COMMIT=)[^ ]+" ./src/COMMIT OUTPUT_VARIABLE COMMIT)
execute_process(COMMAND date +%s OUTPUT_VARIABLE DATE)
execute_process(COMMAND uname -n OUTPUT_VARIABLE MACHINE)

add_definitions('-D_COMMIT="${COMMIT}"')
add_definitions('-D_DATE=${DATE}')
add_definitions('-D_MACHINE="${MACHINE}"')

if(NOT CMAKE_BUILD_TYPE AND NOT CMAKE_CONFIGURATION_TYPES)
message(STATUS "Setting build type to 'Release', Debug mode was not specified.")
set(CMAKE_BUILD_TYPE Release CACHE STRING "Choose the type of build." FORCE)
# Set the possible values of build type for cmake-gui
set_property(CACHE CMAKE_BUILD_TYPE PROPERTY STRINGS "Debug" "Release")
endif()

if(${CMAKE_Fortran_COMPILER_ID} MATCHES "Intel")
set(CMAKE_Fortran_FLAGS_RELEASE "-O3 -xHost -no-prec-div -fno-math-errno -unroll=3 -qopt-subscript-in-range -align all")
set(CMAKE_Fortran_FLAGS_DEBUG "-O0 -nothreads -traceback -fltconsistency -C -g -heap-arrays 64 -warn -fp-stack-check -check bounds -fpe0")
elseif(${CMAKE_Fortran_COMPILER_ID} MATCHES "GNU")
set(CMAKE_Fortran_FLAGS_RELEASE "-O2 -fimplicit-none -ffixed-line-length-132")
set(CMAKE_Fortran_FLAGS_DEBUG "-O0 -g -Wall -Wextra -pedantic -fbounds-check -ffpe-trap=invalid,zero,overflow -Wno-unused-parameter")
elseif(${CMAKE_Fortran_COMPILER_ID} MATCHES "PGI")
  set(CMAKE_Fortran_FLAGS_RELEASE "-r8 -fast -fastsse -O3 -Mipa=fast,inline -Minfo")
  set(CMAKE_Fortran_FLAGS_DEBUG "-Mbounds -g")
elseif(${CMAKE_Fortran_COMPILER_ID} MATCHES "G95")
  set(CMAKE_Fortran_FLAGS_RELEASE "-O3")
  set(CMAKE_Fortran_FLAGS_DEBUG "-O0 -g")
elseif(${CMAKE_Fortran_COMPILER_ID} MATCHES "XL")
  set(CMAKE_Fortran_FLAGS_RELEASE "-O5 -qhot -qipa")
  set(CMAKE_Fortran_FLAGS_DEBUG "-O0 -C -g -qfullpath -qinfo -qnosmp -qxflag=dvz -Q! -qnounwind -qnounroll")
else()
message(STATUS "No optimized Fortran compiler flags are known")
message(STATUS "Fortran compiler full path: " ${CMAKE_Fortran_COMPILER})
set(CMAKE_Fortran_FLAGS_RELEASE "-O2")
set(CMAKE_Fortran_FLAGS_DEBUG   "-O0 -g")
endif()

if(${CMAKE_CXX_COMPILER_ID} MATCHES "Intel")
set(CMAKE_CXX_FLAGS_RELEASE "-O3 -std=c++11 -no-prec-div -ansi-alias -qopt-prefetch=4 -unroll-aggressive -m64")
set(CMAKE_CXX_FLAGS_DEBUG "-O0 -std=c++11 -g -traceback -mp1 -fp-trap=common -fp-model strict")
elseif(${CMAKE_CXX_COMPILER_ID} MATCHES "GNU")
set(CMAKE_CXX_FLAGS_RELEASE "-O2 -std=c++11 -msse4 -mtune=native -march=native -funroll-loops -fno-math-errno -ffast-math")
set(CMAKE_CXX_FLAGS_DEBUG "-O0 -std=c++11 -g -pedantic -Wall -Wextra -Wno-unused")
elseif(${CMAKE_CXX_COMPILER_ID} MATCHES "PGI")
  set(CMAKE_CXX_FLAGS_RELEASE "-std=c++0x")
  set(CMAKE_CXX_FLAGS_DEBUG "-std=c++0x")
elseif(${CMAKE_CXX_COMPILER_ID} MATCHES "G95")
  set(CMAKE_CXX_FLAGS_RELEASE "-std=c++0x")
  set(CMAKE_CXX_FLAGS_DEBUG "-std=c++0x")
elseif(${CMAKE_CXX_COMPILER_ID} MATCHES "XL")
  set(CMAKE_CXX_FLAGS_RELEASE "-qlanglvl=extended0x")
  set(CMAKE_CXX_FLAGS_DEBUG "-qlanglvl=extended0x")
else()
message(STATUS "No optimized C++ compiler flags are known")
message(STATUS "C++ compiler full path: " ${CMAKE_CXX_COMPILER})
set(CMAKE_CXX_FLAGS_RELEASE "-O2 -std=c++11")
set(CMAKE_CXX_FLAGS_DEBUG "-O0 -std=c++11 -g")
endif()

set(SOURCES
${CMAKE_SOURCE_DIR}/src/epoch2d.F90
${CMAKE_SOURCE_DIR}/src/boundary.f90
${CMAKE_SOURCE_DIR}/src/fields.f90
${CMAKE_SOURCE_DIR}/src/laser.F90
${CMAKE_SOURCE_DIR}/src/particles.F90
${CMAKE_SOURCE_DIR}/src/shared_data.F90
)

set(FOLDERS deck housekeeping io parser physics_packages user_interaction)
foreach(FOLDER ${FOLDERS})
file(GLOB TMP ${CMAKE_SOURCE_DIR}/src/${FOLDER}/*)
list(APPEND SOURCES ${TMP})
endforeach()

option(OPENMP "Enable multithreading using OpenMP directives." OFF)
option(PER_SPECIES_WEIGHT "Set every pseudoparticle in a species to represent the same number of real particles." OFF)
option(NO_TRACER_PARTICLES "Don't enable support for tracer particles." OFF)
option(NO_PARTICLE_PROBES "Don't enable support for particle probes." OFF)
option(PARTICLE_SHAPE_TOPHAT "Use second order particle weighting." OFF)
option(PARTICLE_SHAPE_BSPLINE3 "Use fifth order particle weighting." OFF)
option(PARTICLE_ID "Include a unique global particle ID using an 8-byte integer." OFF)
option(PARTICLE_ID4 "Include a unique global particle ID using an 4-byte integer." OFF)
option(PARTICLE_COUNT_UPDATE "Keep global particle counts up to date." OFF)
option(PHOTONS "Include QED routines" OFF)
option(TRIDENT_PHOTONS "Use the Trident process for pair production." OFF)
option(PREFETCH "Use Intel-specific 'mm_prefetch' calls to load next particle in the list into cache ahead of time." OFF)
option(PARSER_DEBUG "Turn on debugging." OFF)
option(PARTICLE_DEBUG "Turn on debugging." OFF)
option(MPI_DEBUG "Turn on debugging." OFF)
option(SIMPLIFY_DEBUG "Turn on debugging." OFF)
option(NO_IO "Don't generate any output at all. Useful for benchmarking." OFF)
option(COLLISIONS_TEST "Bypass the main simulation and only perform collision tests." OFF)
option(PER_PARTICLE_CHARGE_MASS "specify charge and mass per particle rather than per species." OFF)
option(PARSER_CHECKING "Perform checks on evaluated deck expressions." OFF)
option(USE_INSITU "Link epoch with ParaView Catalyst." OFF)
option(INSITU_DOUBLE_PREC "Double precision for data exported insitu." OFF)
option(TIGHT_FOCUSING "Maxwell consistent computation of EM fields at boundary for tight-focusing." OFF)

if(OPENMP)
message(STATUS "Option 'OPENMP' enabled")
add_definitions("-DOPENMP")
if(${CMAKE_Fortran_COMPILER_ID} MATCHES "Intel")
set(CMAKE_Fortran_FLAGS_RELEASE "${CMAKE_Fortran_FLAGS_RELEASE} -qopenmp")
set(CMAKE_Fortran_FLAGS_DEBUG "${CMAKE_Fortran_FLAGS_DEBUG} -qopenmp")
elseif(${CMAKE_Fortran_COMPILER_ID} MATCHES "GNU")
set(CMAKE_Fortran_FLAGS_RELEASE "${CMAKE_Fortran_FLAGS_RELEASE} -fopenmp")
set(CMAKE_Fortran_FLAGS_DEBUG "${CMAKE_Fortran_FLAGS_DEBUG} -fopenmp")
else()
message(STATUS "Fortran OpenMP compiler flag not known.")
endif()
if(${CMAKE_CXX_COMPILER_ID} MATCHES "Intel")
set(CMAKE_CXX_FLAGS_RELEASE "${CMAKE_CXX_FLAGS_RELEASE} -qopenmp")
set(CMAKE_CXX_FLAGS_DEBUG "${CMAKE_CXX_FLAGS_DEBUG} -qopenmp")
elseif(${CMAKE_CXX_COMPILER_ID} MATCHES "GNU")
set(CMAKE_CXX_FLAGS_RELEASE "${CMAKE_CXX_FLAGS_RELEASE} -fopenmp")
set(CMAKE_CXX_FLAGS_DEBUG "${CMAKE_CXX_FLAGS_DEBUG} -fopenmp")
else()
message(STATUS "C++ OpenMP compiler flag not known.")
endif()
endif()

if(TIGHT_FOCUSING AND NOT FFT_LIBRARY)
message(STATUS "No FFT library specified. Setting FFT library to 'none'.")
set(FFT_LIBRARY none CACHE STRING "Choose the FFT library." FORCE)
set_property(CACHE FFT_LIBRARY PROPERTY STRINGS "FFTW" "MKL" "None")
endif()

if(PER_SPECIES_WEIGHT)
message(STATUS "Option 'PER_SPECIES_WEIGHT' enabled")
add_definitions("-DPER_SPECIES_WEIGHT")
endif()

if(NO_TRACER_PARTICLES)
message(STATUS "Option 'NO_TRACER_PARTICLES' enabled")
add_definitions("-DNO_TRACER_PARTICLES")
endif()

if(NO_PARTICLE_PROBES)
message(STATUS "Option 'NO_PARTICLE_PROBES' enabled")
add_definitions("-DNO_PARTICLE_PROBES")
endif()

if(PARTICLE_SHAPE_TOPHAT)
message(STATUS "Option 'PARTICLE_SHAPE_TOPHAT' enabled")
add_definitions("-DPARTICLE_SHAPE_TOPHAT")
endif()

if(PARTICLE_SHAPE_BSPLINE3)
message(STATUS "Option 'PARTICLE_SHAPE_BSPLINE3' enabled")
add_definitions("-DPARTICLE_SHAPE_BSPLINE3")
endif()

if(PARTICLE_ID)
message(STATUS "Option 'PARTICLE_ID' enabled")
add_definitions("-DPARTICLE_ID")
endif()

if(PARTICLE_ID4)
message(STATUS "Option 'PARTICLE_ID4' enabled")
add_definitions("-DPARTICLE_ID4")
endif()

if(PARTICLE_COUNT_UPDATE)
message(STATUS "Option 'PARTICLE_COUNT_UPDATE' enabled")
add_definitions("-DPARTICLE_COUNT_UPDATE")
endif()

if(PHOTONS)
message(STATUS "Option 'PHOTONS' enabled")
add_definitions("-DPHOTONS")
endif()

if(TRIDENT_PHOTONS)
message(STATUS "Option 'TRIDENT_PHOTONS' enabled")
add_definitions("-DTRIDENT_PHOTONS")
endif()

if(PREFETCH)
message(STATUS "Option 'PREFETCH' enabled")
add_definitions("-DPREFETCH")
endif()

if(PARSER_DEBUG)
message(STATUS "Option 'PARSER_DEBUG' enabled")
add_definitions("-DPARSER_DEBUG")
endif()

if(PARTICLE_DEBUG)
message(STATUS "Option 'PARTICLE_DEBUG' enabled")
add_definitions("-DPARTICLE_DEBUG")
endif()

if(MPI_DEBUG)
message(STATUS "Option 'MPI_DEBUG' enabled")
add_definitions("-DMPI_DEBUG")
endif()

if(SIMPLIFY_DEBUG)
message(STATUS "Option 'SIMPLIFY_DEBUG' enabled")
add_definitions("-DSIMPLIFY_DEBUG")
endif()

if(NO_IO)
message(STATUS "Option 'NO_IO' enabled")
add_definitions("-DNO_IO")
endif()

if(COLLISIONS_TEST)
message(STATUS "Option 'COLLISIONS_TEST' enabled")
add_definitions("-DCOLLISIONS_TEST")
endif()

if(PER_PARTICLE_CHARGE_MASS)
message(STATUS "Option 'PER_PARTICLE_CHARGE_MASS' enabled")
add_definitions("-DPER_PARTICLE_CHARGE_MASS")
endif()

if(PARSER_CHECKING)
message(STATUS "Option 'PARSER_CHECKING' enabled")
add_definitions("-DPARSER_CHECKING")
endif()

if(USE_INSITU)
message(STATUS "Option 'USE_INSITU' enabled")
find_package(ParaView 5.2 REQUIRED COMPONENTS vtkPVPythonCatalyst)
include(${PARAVIEW_USE_FILE})
file(GLOB ADAPTOR ${CMAKE_SOURCE_DIR}/src/adaptor/*)
list(APPEND SOURCES ${ADAPTOR})
add_definitions("-DINSITU")
if(NOT PARAVIEW_USE_MPI)
message(SEND_ERROR "ParaView must be built with MPI enabled")
endif()
if(INSITU_DOUBLE_PREC)
message(STATUS "Option 'INSITU_DOUBLE_PREC' enabled")
add_definitions("-DINSITU_DOUBLE_PREC")
endif()
endif()

if(TIGHT_FOCUSING)
message(STATUS "Option 'TIGHT_FOCUSING' enabled")
file(GLOB FOCUSING ${CMAKE_SOURCE_DIR}/src/focusing/interface.f90)
list(APPEND SOURCES ${FOCUSING})
include_directories(${MPI_CXX_INCLUDE_PATH})
add_definitions("-DTIGHT_FOCUSING")
set(FOCUS_SRC
${CMAKE_SOURCE_DIR}/src/focusing/main.cpp
${CMAKE_SOURCE_DIR}/src/focusing/domain_param.cpp
${CMAKE_SOURCE_DIR}/src/focusing/laser_param.cpp
${CMAKE_SOURCE_DIR}/src/focusing/global.cpp
${CMAKE_SOURCE_DIR}/src/focusing/laser_bcs.cpp
)
include_directories(${CMAKE_SOURCE_DIR}/src/focusing/inc)
if(${FFT_LIBRARY} MATCHES "FFTW")
if(OPENMP)
message(SEND_ERROR "Multi-threaded FFTW routines are not supported. Disable OpenMP.")
endif()
message(STATUS "FFT library: FFTW")
message(STATUS "Option 'USE_FFTW' enabled")
add_definitions("-DUSE_FFTW")
find_package(FFTW REQUIRED)
include_directories(${FFTW_INCLUDES})
endif()
if(${FFT_LIBRARY} MATCHES "MKL")
if(NOT ${CMAKE_CXX_COMPILER_ID} MATCHES "Intel")
message(SEND_ERROR "MKL library can be used only with Intel compilers")
endif()
message(STATUS "FFT library: MKL")
message(STATUS "Option 'USE_MKL' enabled")
add_definitions("-DUSE_MKL")
set(CMAKE_CXX_FLAGS_RELEASE "${CMAKE_CXX_FLAGS_RELEASE} -mkl")
set(CMAKE_CXX_FLAGS_DEBUG "${CMAKE_CXX_FLAGS_DEBUG} -mkl")
set(CMAKE_Fortran_FLAGS_RELEASE "${CMAKE_Fortran_FLAGS_RELEASE} -mkl")
set(CMAKE_Fortran_FLAGS_DEBUG "${CMAKE_Fortran_FLAGS_DEBUG} -mkl")
endif()
if(${FFT_LIBRARY} MATCHES "None")
message(STATUS "FFT library: None")
endif()
add_library(focus ${FOCUS_SRC})
if(${FFT_LIBRARY} MATCHES "FFTW")
#if(OPENMP)
#  target_link_libraries(focus LINK_PUBLIC ${FFTW_LIBRARIES} ${FFTW_LIBRARIES_OMP})
#else()
target_link_libraries(focus LINK_PUBLIC ${FFTW_LIBRARIES})
#endif()
endif()
set_target_properties(focus PROPERTIES LINKER_LANGUAGE CXX)
endif()

add_executable(epoch2d ${SOURCES})
target_link_libraries(epoch2d LINK_PRIVATE ${MPI_Fortran_LIBRARIES} ${SDF_Fortran_LIBRARIES})
if(USE_INSITU)
target_link_libraries(epoch2d LINK_PRIVATE vtkPVPythonCatalyst vtkParallelMPI)
endif()
if(TIGHT_FOCUSING)
target_link_libraries(epoch2d LINK_PRIVATE ${MPI_CXX_LIBRARIES} focus)
endif()
set_target_properties(epoch2d PROPERTIES LINKER_LANGUAGE Fortran)
\end{lstlisting}

\chapter{CD content}
All the stuff created during the work. Nothing less, nothing more. Enjoy.\\

\centering
{\renewcommand{\arraystretch}{1.5}
	\begin{tabular}{rl}
		\hline \textbf{directory/file} & \textbf{specification} \\
		\hline
		\hline
		\texttt{master\_thesis/} & \parbox[t]{8cm}{directory containing this document and all related sources} \\
		\texttt{epoch/} & \parbox[t]{8cm}{directory containing EPOCH source code (cloned on 15-04-2017)} \\
		\texttt{scripts/} & \parbox[t]{8cm}{directory containing scripts intended for post-processing of simulation data} \\
		\hline
	\end{tabular}
}

\end{document}