The collective behavior of particles and fields in laser-plasma interactions represent a complex and strongly non-linear problem. The investigation of such systems cannot be carried out only through the application of two traditional techniques, namely, theoretical and experimental work. Since there is a large number of a very complex simultaneous interactions with many degrees of freedom in plasma, analytical modeling seems to be impractical. On the other hand, many of the significant details of laser-plasma interaction are extremely difficult or even impossible to obtain experimentally. Therefore, for the further understanding in this field of research other tools and techniques are required \cite{jaroszynsky}.

Numerical simulation is now an integrated part of science and technology. With the advent of powerful computational systems, numerical simulations now play an important role in physics as an essential tool in developing theoretical models and understanding experimental results. Numerical simulation is now rightfully considered as a separate discipline from theory and experiment \cite{pang}.

Numerical simulations help researchers to develop models covering a wide range of physical scenarios and to investigate their properties. For example, numerical schemes for Newton's equation can be implemented in the study of the molecular dynamic, the techniques used to solve hydrodynamic equations are needed in weather prediction and algorithms for solving the diffusion equation can be applied to air pollution control problems. Only one numerical code can solve a variety of physical problems by modification of the initial and boundary conditions. These so-called computer experiments are often faster and much cheaper than a single real experiment in laboratory \cite{gould}.
 
Nowadays, it is clear that a detailed understanding of the physical mechanisms in laser-plasma interaction can only be achieved through the combination of theory, experiment and simulation. Development of parallel algorithms that lead to a stable and sufficiently exact solution, however, belong to the most challenging fields of modern science.

The most part of this chapter is focused on the description of the particle-in-cell method, which is the very popular numerical algorithm used for plasma simulations. There can be found the mathematical background of this method, description of the steps of the simulation loop and stability conditions. The last section provides a brief overview of the particle-in-cell code EPOCH, which has been used for simulations within this work.