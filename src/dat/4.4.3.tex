Alternative to a plasma mirror could be a plasma lens. This concept is based on creating a short plasma channel which is able to guide an intense laser pulse. The plasma channel can be formed by variations in either the laser intensity across the beam regions or a plasma density. In the higher laser intensity region, plasma is pushed aside in the radial direction of the beam due to the ponderomotive force (see chapter 2). This effect reduces the plasma density locally and consequently increases the index of refraction of the plasma. The resulting index of refraction is seen by the laser pulse as a focusing lens, thus prevents it from further spreading. Note that apart from ponderomotive force, there is a variety of other mechanisms that lead to a change of the refractive index of plasma. These include collisions, a thermal, or relativistic effects.

Similarly as for the plasma mirrors, plasma lenses would tolerate energy densities above the damage threshold for conventional solid state optics and allow to manipulate with the laser beam in close proximity to the interaction region. The focal length of the plasma lens is expected to be independent of the laser intensity as long as the interaction regime is non-relativistic. The plasma lens may, in principle, be tuned by controlling the plasma density. Consequently, it would be possible to change the position of the focal point without physically moving any optical element [source].