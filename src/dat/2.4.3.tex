Similarly as for the resonance absorption, Brunel's vaccum heating takes place when a linearly p-polarized laser pulse is obliquely incident on a plasma (note that the incident angle has to be relatively large). In this case, however, the plasma density profile has to be steep, so the amplitude of the oscillating electrons driven by the electric laser field is larger than the density scale length. Consequently, the energy of laser pulse carried by electrons is transfered into mechanical or heat energy in the overdense plasma region.

The energy absorbed via Brunel's vacuum heating is carried by hot electrons in the bunches
ejected once per laser period. To give a physical picture, in the first half of laser cycle, the electrons are pushed inside the plasma gaining only a small amount of energies because the electric laser field is shielded. On the other hand, in the second half of laser cycle, the electrons gain very high energies while they are ejected into vacuum.

The trajectory of charged particles is strongly influenced by the time of their expulsion. Furthermore, a self-consistent electric field may be generated if a large amount of charged particles is ejected at the same time. However, the majority of the charged particles returns into the plasma without restoring forces behind the skin layer due to the oscillating component of the laser field and self-consistent electric field. Note that since the charged particles are accelerated by different phases of the laser field, the distribution of such particles can be in most cases considered as a Maxwellian [source].