Similarly as for the resonance absorption, Brunel's vaccum heating takes place when a linearly p-polarized laser pulse is obliquely incident on a plasma. In this case, however, the plasma density profile has to be steep, so the amplitude of the oscillating electrons driven by the electric laser field is larger than the density scale length. Consequently, the energy of laser pulse carried by electrons is transfered into mechanical or heat energy in the overdense plasma region.

The energy absorbed via Brunel's vacuum heating is carried by hot electrons in the bunches
ejected once per laser period. To give a physical picture, in the first half of laser cycle, the electrons are pushed inside the plasma gaining only a small amount of energies. 