Landau damping is a very efficient damping mechanism of longitudinal plasma waves, which may occur even in the absence of collisions. Consider ions as a stationary neutralizing background. Using a linearized version of the Vlasov equation (\ref{2.2.1}), one can find dispersion relation for electron plasma waves,
\begin{equation}
\label{2.4.3.1}
\omega^2 \approx \omega_{pe}^2 + 3 \frac{\omega_{pe}^2}{\omega^2} v_{th}^2 k^2 + \mathrm{i} \pi \frac{\omega_{pe}^2 \omega^2}{k^2} \diff{\hat{f}_{0}}{v_x}\bigg|_{v_x = v_{ph}},
\end{equation}
where $ v_{th} $ and $ v_{ph} $ are, respectively, electron thermal velocity and phase velocity of the wave, and $ \hat{f}_{0}(v_{x}) $ is the one-dimensional normalized distribution function in the zeroth-order approximation. The imaginary part in (\ref{2.4.3.1}) expresses Landau damping.

Only if the velocity distribution is not constant, the collisionless energy exchange between the wave and the particles occurs. The particles with a velocity almost equal to the phase velocity of the wave $ v_{ph} $ can become resonant with the field and therefore can be either slowed down or accelerated.

Let $ \hat{f}_{0}(v_{x}) $ be Maxwellian, then one can calculate the Landau damping rate $ \gamma_\mathrm{L} $, which is given by
\begin{equation}
\label{2.4.3.3}
\gamma_\mathrm{L} \approx -\sqrt{\frac{\pi}{2}} \frac{\omega_{pe}}{\left(k \lambda_D\right)^{3}} \exp \left(-\frac{1}{2 \left(k \lambda_D\right)^{2}}\right).
\end{equation}
Since $ \gamma_\mathrm{L} $ is negative, there is without doubt a collisionless damping of plasma waves. As is evident from (\ref{2.4.3.3}) this damping becomes important when the wavelength is comparable to the Debye length.

However, the main feature of collisionless heating mechanisms is that only a minority fraction of the plasma electrons acquires most of the absorbed energy. This means that, as a side effect, a population of hot electrons is created, which significantly preheats the plasma.

