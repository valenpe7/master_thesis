The stability and accuracy of the standard PIC method is directly dependent on the size of the spatial and temporal simulation steps. In order to find correct parameters, one has to know the absolute accuracy and corresponding stability conditions.

The effect of the spatial grid is to smooth the interaction forces and to couple plasma perturbations to perturbations at other wavelengths, called aliases. It may lead to non-physical instabilities and numerical heating. To avoid these effects, the spatial step needs to resolve the Debye length (see \ref{2.1.3}). Thus, it is desirable to fulfill the following condition,
\begin{equation}
\Delta x, \Delta y, \Delta z \leq \lambda_{D}.
\end{equation}
In the general electromagnetic case, the time step has to satisfy the Courant--Fridrichs--Levy (CFL) condition \cite{jaroszynsky},
\begin{equation}
\label{3.1.4.1}
C = c^{2} \Delta t^{2} \left(\frac{1}{\Delta x^{2}} + \frac{1}{\Delta y^{2}} + \frac{1}{\Delta z^{2}}\right),
\end{equation}
where the dimensionless number $ C \leq 1 $ is called the CFL number. This condition limits the range of motion of all objects in the simulation during one time step. It ensures, that these particles would not cross more than one cell in one simulation time step. When this condition is violated, the growth of non-physical effects can be very rapid. 

The leap-frog scheme, used to solve the field equations and equations of motion, is second-order accurate in both, time and space. In addition, this scheme is explicit and time-reversible. A thorough study of PIC method can be found in [source].