Recently, it has also been shown by PIC simulations that a tightly focused laser beams could be potentially obtained by using cone-shaped targets \cite{Levy2011, Yu2010, Zeng2012, Cao2008}. More specifically, the inner surface of the hollow cone that is open at both ends interacts with the outer parts of incoming collimated laser beam leading to the surface ionization and generation of dense plasmas. The laser beam structure is in turn modified and its energy is gradually squeezed due to the multiple reflections. In spite of the laser energy absorption by the target walls, the laser pulse is nonlinearly guided to the cone tip that can result in a highly localized spot of around wavelength radius with the peak intensity amplified by an order of magnitude \cite{Zeng2012}.

The PIC simulations have also shown, that after leaving the cone-tip opening, the tightly focused laser beam is still fairly coherent and can propagate over a distance of several wavelengths before it starts to diffract \cite{Yu2010}. This fact indicates that the hollow cone might act as an optical device that focuses down the laser pulse to a wavelength level.

The cone-focusing effect is mainly characterized by the material, the opening angle and the cone tip size of the target. By controlling these parameters, a laser pulse can be focused efficiently and the quality of the focal spot can be significantly enhanced. However, the tip of the cone channel has to be of a dimension comparable to the laser wavelength, making the manufacturing process difficult at present. Such a small conical channel should be realizable in the near future with the rapid advances in nanofabrication \cite{Cao2008}.