\begin{lstlisting}[style=CXX, caption=Function performing forward fast Fourier transform using MKL library]
std::vector<std::complex<double>> fft::mkl_fft_forward(std::vector<std::complex<double>> in) {
DFTI_DESCRIPTOR_HANDLE desc;
MKL_LONG status;
DftiCreateDescriptor(&desc, DFTI_DOUBLE, DFTI_COMPLEX, 1, static_cast<MKL_LONG>(in.size()));
DftiCommitDescriptor(desc);
status = DftiComputeForward(desc, in.data());
if(status != 0) {
std::cerr << DftiErrorMessage(status) << std::endl;
abort();
}
DftiFreeDescriptor(&desc);
return in;
}
\end{lstlisting}

\begin{lstlisting}[style=CXX, caption=Function performing backward fast Fourier transform using MKL library]
std::vector<std::complex<double>> fft::mkl_fft_backward(std::vector<std::complex<double>> in) {
DFTI_DESCRIPTOR_HANDLE desc;
MKL_LONG status;
DftiCreateDescriptor(&desc, DFTI_DOUBLE, DFTI_COMPLEX, 1, static_cast<MKL_LONG>(in.size()));
DftiCommitDescriptor(desc);
status = DftiComputeBackward(desc, in.data());
if(status != 0) {
std::cerr << DftiErrorMessage(status) << std::endl;
abort();
}
DftiFreeDescriptor(&desc);
return in;
}
\end{lstlisting}

\begin{lstlisting}[style=CXX, caption=Function performing forward fast Fourier transform using FFTW library]
std::vector<std::complex<double>> fft::fftw_fft_forward(std::vector<std::complex<double>> in) {
fftw_plan p = fftw_plan_dft_1d(in.size(), reinterpret_cast<fftw_complex*>(in.data()), reinterpret_cast<fftw_complex*>(in.data()), FFTW_FORWARD, FFTW_ESTIMATE);
fftw_execute(p);
fftw_destroy_plan(p);
return in;
}
\end{lstlisting}

\begin{lstlisting}[style=CXX, caption=Function performing backward fast Fourier transform using FFTW library]
std::vector<std::complex<double>> fft::fftw_fft_backward(std::vector<std::complex<double>> in) {
fftw_plan p = fftw_plan_dft_1d(in.size(), reinterpret_cast<fftw_complex*>(in.data()), reinterpret_cast<fftw_complex*>(in.data()), FFTW_BACKWARD, FFTW_ESTIMATE);
fftw_execute(p);
fftw_destroy_plan(p);
return in;
}
\end{lstlisting}

\begin{lstlisting}[style=CXX, caption=Function performing forward discrete Fourier transform without using any library]
std::vector<std::complex<double>> fft::fft_forward(std::vector<std::complex<double>> in) {
std::vector<std::complex<double>> out(in.size());
for(auto j = 0; j < out.size(); j++) {
for(auto l = 0; l < out.size(); l++) {
out.at(j) += in.at(l) * exp(-2.0 * constants::pi * I * l * j / in.size());
}
}
return out;
}
\end{lstlisting}

\begin{lstlisting}[style=CXX, caption=Function performing backward discrete Fourier transform without using any library]
std::vector<std::complex<double>> fft::fft_backward(std::vector<std::complex<double>> in) {
std::vector<std::complex<double>> out(in.size());
for(auto j = 0; j < out.size(); j++) {
for(auto l = 0; l < out.size(); l++) {
out.at(j) += in.at(l) * exp(+2.0 * constants::pi * I * l * j / in.size());
}
}
return out;
}
\end{lstlisting}

\begin{lstlisting}[style=CXX, caption=Method for performing discrete Fourier transform in time]
void laser_bcs::dft_time(field_2d<std::complex<double>>& field) const {
#ifdef OPENMP
#pragma omp parallel for schedule(static)
#endif
for(auto j = 0; j < this->domain->Nx; j++) {
#ifdef USE_MKL
field.add_col(fft::mkl_fft_backward(field.get_col(j)), j);
#elif USE_FFTW
field.add_col(fft::fftw_fft_backward(field.get_col(j)), j);
#else
field.add_col(fft::fft_backward(field.get_col(j)), j);
#endif
}
field.multiply(this->domain->dt / (2.0 * constants::pi));
return;
}
\end{lstlisting}

\begin{lstlisting}[style=CXX, caption=Method for performing inverse discrete Fourier transform in time]
void laser_bcs::idft_time(field_2d<std::complex<double>>& field) const {
#ifdef OPENMP
#pragma omp parallel for schedule(static)
#endif
for(auto j = 0; j < this->domain->Nx; j++) {
#ifdef USE_MKL
field.add_col(fft::mkl_fft_forward(field.get_col(j)), j);
#elif USE_FFTW
field.add_col(fft::fftw_fft_forward(field.get_col(j)), j);
#else
field.add_col(fft::fft_forward(field.get_col(j)), j);
#endif
}
field.multiply(2.0 * (2.0 * constants::pi) / (this->domain->Nt * this->domain->dt));
return;
}
\end{lstlisting}

\begin{lstlisting}[style=CXX, caption=Method for performing discrete Fourier transform in space]
void laser_bcs::dft_space(field_2d<std::complex<double>>& field) const {
std::vector<std::complex<double>> row_global(this->domain->Nx_global);
std::vector<std::complex<double>> row_local;
for(auto j = 0; j < this->domain->Nt; j++) {
row_local = field.get_row(j);
MPI_Gatherv(row_local.data(), this->domain->Nx, MPI_CXX_DOUBLE_COMPLEX, row_global.data(), this->domain->counts.data(), this->domain->displs.data(), MPI_CXX_DOUBLE_COMPLEX, 0, MPI_COMM_WORLD);
if(this->domain->rank == 0) {
#ifdef USE_MKL
row_global = fft::mkl_fft_forward(row_global);
#elif USE_FFTW
row_global = fft::fftw_fft_forward(row_global);
#else
row_global = fft::fft_forward(row_global);
#endif
}
MPI_Scatterv(row_global.data(), this->domain->counts.data(), this->domain->displs.data(), 	MPI_CXX_DOUBLE_COMPLEX, row_local.data(), this->domain->Nx, MPI_CXX_DOUBLE_COMPLEX, 0, MPI_COMM_WORLD);
field.add_row(row_local, j);
}
field.multiply(this->domain->dx / (2.0 * constants::pi));
return;
}
\end{lstlisting}

\begin{lstlisting}[style=CXX, caption=Method for performing inverse discrete Fourier transform in space]
void laser_bcs::idft_space(field_2d<std::complex<double>>& field) const {
std::vector<std::complex<double>> row_global(this->domain->Nx_global);
std::vector<std::complex<double>> row_local;
for(auto j = 0; j < this->domain->Nt; j++) {
row_local = field.get_row(j);
MPI_Gatherv(row_local.data(), this->domain->Nx, MPI_CXX_DOUBLE_COMPLEX, row_global.data(), this->domain->counts.data(), this->domain->displs.data(), MPI_CXX_DOUBLE_COMPLEX, 0, MPI_COMM_WORLD);
if(this->domain->rank == 0) {
#ifdef USE_MKL
row_global = fft::mkl_fft_backward(row_global);
#elif USE_FFTW
row_global = fft::fftw_fft_backward(row_global);
#else
row_global = fft::fft_backward(row_global);
#endif
}
MPI_Scatterv(row_global.data(), this->domain->counts.data(), this->domain->displs.data(), MPI_CXX_DOUBLE_COMPLEX, row_local.data(), this->domain->Nx, MPI_CXX_DOUBLE_COMPLEX, 0, MPI_COMM_WORLD);
field.add_row(row_local, j);
}
field.multiply((2.0 * constants::pi) / (this->domain->Nx_global * this->domain->dx));
return;
}
\end{lstlisting}

\begin{lstlisting}[style=CXX, caption=Method for dumping data into shared file]
template <typename T>
void field_2d<T>::dump_to_shared_file(std::string name, int row_first, int row_last, int row_size_local, int row_size_global, int col_start) const {
MPI_File file;
MPI_Offset offset = 0;
MPI_Status status;
MPI_Datatype local_array;
int col_size = row_last - row_first;
const int ndims = 2;
std::array<int, ndims> size_global = {col_size, row_size_global};
std::array<int, ndims> size_local = {col_size, row_size_local};
std::array<int, ndims> start_coords = {0, col_start};
MPI_Type_create_subarray(2, size_global.data(), size_local.data(), start_coords.data(), MPI_ORDER_C, MPI_DOUBLE, &local_array);
MPI_Type_commit(&local_array);
std::vector<double> real_part(col_size * row_size_local);
for(auto i = std::make_pair(row_first, 0); i.first < row_last; i.first++, i.second++) {
for(auto j = 0; j < row_size_local; j++) {
real_part[i.second * row_size_local + j] = std::real(this->data[i.first * row_size_local + j]);
}
}
MPI_File_open(MPI_COMM_WORLD, name.data(), MPI_MODE_CREATE|MPI_MODE_WRONLY, MPI_INFO_NULL, &file);
MPI_File_set_view(file, offset, MPI_DOUBLE, local_array, "native", MPI_INFO_NULL);
MPI_File_write_all(file, real_part.data(), col_size * row_size_local, MPI_DOUBLE, &status);
MPI_File_close(&file);
MPI_Type_free(&local_array);
return;
}
\end{lstlisting}

\begin{lstlisting}[style=CXX, caption=Extern C++ function to fill Fortran arrays with laser fields dumped in binary file]
void populate_laser_at_boundary(double* field, int* id, const char* data_dir, int* timestep, int* size_global, int* first, int* last) {
double num = 0.0;
std::string laser_id = std::to_string(*id);
std::string path(data_dir);
std::ifstream in;
in.open(path + "/laser_" + laser_id + ".dat", std::ios::binary);
if(in.is_open()) {
in.seekg(((*timestep) * (*size_global) + (*first) - 1) * sizeof(num));
for(auto i = 0; i < *last - *first + 1; i++) {
in.read(reinterpret_cast<char*>(&num), sizeof(num));
field[i] = num;
}
in.close();
} else {
std::cout << "error: cannot read file " << path + "/" + filename + laser_id + ".dat" << std::endl;
}
return;
}
\end{lstlisting}

\begin{lstlisting}[style=FORTRAN, caption=Fortran interfaces for C++ library functions]
INTERFACE

SUBROUTINE compute_laser_at_boundary(rank, nproc, laser_start, laser_end, &
fwhm_time, t_0, omega, pos, amp, w_0, id, L_min, L_max, L_focus, T_min, T_max, &
T_ncells, cpml_thickness, t_end, T_cell_size, L_cell_size, dt, output_path) bind(c)
USE, INTRINSIC :: iso_c_binding
IMPLICIT NONE
INTEGER(c_int), INTENT(IN) :: rank, nproc, id, T_ncells, cpml_thickness
CHARACTER(kind=c_char), DIMENSION(*), INTENT(IN) :: output_path
REAL(c_double), INTENT(IN) :: laser_start, laser_end, fwhm_time, t_0, omega, pos, &
amp, w_0, L_min, L_max, L_focus, T_min, T_max, t_end, T_cell_size, L_cell_size, dt
END SUBROUTINE compute_laser_at_boundary

SUBROUTINE populate_laser_at_boundary(field, laser_id, output_path, timestep, size_global, first, last) bind(c)
USE, INTRINSIC :: iso_c_binding
IMPLICIT NONE
INTEGER(c_int), INTENT(IN) :: laser_id, timestep, size_global, first, last
CHARACTER(kind=c_char), DIMENSION(*), INTENT(IN) :: output_path
REAL(c_double), DIMENSION(*), INTENT(OUT) :: field
END SUBROUTINE populate_laser_at_boundary

END INTERFACE
\end{lstlisting}

\begin{lstlisting}[style=FORTRAN, caption=Fortran subroutines for Maxwell consistent computation of laser fields at boundaries]
SUBROUTINE Maxwell_consistent_computation_of_EM_fields

TYPE(laser_block), POINTER :: current

current => laser_x_min
DO WHILE(ASSOCIATED(current))
CALL compute_laser_at_boundary(rank, nproc, current%t_start, current%t_end, &
current%fwhm_time, current%t_0, current%omega, current%pos, current%amp, current%w_0, &
current%id, x_min, x_max, current%focus, y_min, y_max, ny_global, cpml_thickness, t_end, &
dy, dx, dt, TRIM(data_dir)//C_NULL_CHAR)
current => current%next
ENDDO

current => laser_x_max
DO WHILE(ASSOCIATED(current))
CALL compute_laser_at_boundary(rank, nproc, current%t_start, current%t_end, &
current%fwhm_time, current%t_0, current%omega, current%pos, current%amp, current%w_0, &
current%id, x_min, x_max, current%focus, y_min, y_max, ny_global, cpml_thickness, t_end, &
dy, dx, dt, TRIM(data_dir)//C_NULL_CHAR)
current => current%next
ENDDO

current => laser_y_min
DO WHILE(ASSOCIATED(current))
CALL compute_laser_at_boundary(rank, nproc, current%t_start, current%t_end, &
current%fwhm_time, current%t_0, current%omega, current%pos, current%amp, current%w_0, &
current%id, y_min, y_max, current%focus, x_min, x_max, nx_global, cpml_thickness, t_end, &
dx, dy, dt, TRIM(data_dir)//C_NULL_CHAR)
current => current%next
ENDDO

current => laser_y_max
DO WHILE(ASSOCIATED(current))
CALL compute_laser_at_boundary(rank, nproc, current%t_start, current%t_end, &
current%fwhm_time, current%t_0, current%omega, current%pos, current%amp, current%w_0, &
current%id, y_min, y_max, current%focus, x_min, x_max, nx_global, cpml_thickness, t_end, &
dx, dy, dt, TRIM(data_dir)//C_NULL_CHAR)
current => current%next
ENDDO

END SUBROUTINE Maxwell_consistent_computation_of_EM_fields
\end{lstlisting}

\begin{lstlisting}[style=FORTRAN, caption=Fortran subroutines for populating laser sources at boundaries]
SUBROUTINE get_source_x_boundary(buffer, laser_id)
REAL(num), DIMENSION(:), INTENT(INOUT) :: buffer
INTEGER, INTENT(IN) :: laser_id
CALL populate_laser_at_boundary(buffer, laser_id, TRIM(data_dir)//C_NULL_CHAR, &
step, ny_global, ny_global_min, ny_global_max)
END SUBROUTINE get_source_x_boundary
  
SUBROUTINE get_source_y_boundary(buffer, laser_id)
REAL(num), DIMENSION(:), INTENT(INOUT) :: buffer
INTEGER, INTENT(IN) :: laser_id
CALL populate_laser_at_boundary(buffer, laser_id, TRIM(data_dir)//C_NULL_CHAR, &
step, nx_global, nx_global_min, nx_global_max)
END SUBROUTINE get_source_y_boundary
\end{lstlisting}

\begin{lstlisting}[style=FORTRAN, caption=CMakeLists]
cmake_minimum_required(VERSION 3.1)
project(EPOCH_2D)
enable_language(CXX Fortran)

set(CMAKE_MODULE_PATH ${CMAKE_SOURCE_DIR}/cmake)
set(CMAKE_Fortran_MODULE_DIRECTORY ${CMAKE_SOURCE_DIR}/obj)
set(CMAKE_ARCHIVE_OUTPUT_DIRECTORY ${CMAKE_SOURCE_DIR}/lib)
set(EXECUTABLE_OUTPUT_PATH ${CMAKE_SOURCE_DIR}/bin)

find_package(MPI REQUIRED)
find_package(SDF REQUIRED)

include_directories(${MPI_Fortran_INCLUDE_PATH})
include_directories(${SDF_Fortran_INCLUDE_PATH})
include_directories(src/include)

execute_process(COMMAND ./src/gen_commit_string.sh)
execute_process(COMMAND grep -oP "(?<=COMMIT=)[^ ]+" ./src/COMMIT OUTPUT_VARIABLE COMMIT)
execute_process(COMMAND date +%s OUTPUT_VARIABLE DATE)
execute_process(COMMAND uname -n OUTPUT_VARIABLE MACHINE)

add_definitions('-D_COMMIT="${COMMIT}"')
add_definitions('-D_DATE=${DATE}')
add_definitions('-D_MACHINE="${MACHINE}"')

if(NOT CMAKE_BUILD_TYPE AND NOT CMAKE_CONFIGURATION_TYPES)
message(STATUS "Setting build type to 'Release', Debug mode was not specified.")
set(CMAKE_BUILD_TYPE Release CACHE STRING "Choose the type of build." FORCE)
# Set the possible values of build type for cmake-gui
set_property(CACHE CMAKE_BUILD_TYPE PROPERTY STRINGS "Debug" "Release")
endif()

if(${CMAKE_Fortran_COMPILER_ID} MATCHES "Intel")
set(CMAKE_Fortran_FLAGS_RELEASE "-O3 -xHost -no-prec-div -fno-math-errno -unroll=3 -qopt-subscript-in-range -align all")
set(CMAKE_Fortran_FLAGS_DEBUG "-O0 -nothreads -traceback -fltconsistency -C -g -heap-arrays 64 -warn -fp-stack-check -check bounds -fpe0")
elseif(${CMAKE_Fortran_COMPILER_ID} MATCHES "GNU")
set(CMAKE_Fortran_FLAGS_RELEASE "-O2 -fimplicit-none -ffixed-line-length-132")
set(CMAKE_Fortran_FLAGS_DEBUG "-O0 -g -Wall -Wextra -pedantic -fbounds-check -ffpe-trap=invalid,zero,overflow -Wno-unused-parameter")
elseif(${CMAKE_Fortran_COMPILER_ID} MATCHES "PGI")
  set(CMAKE_Fortran_FLAGS_RELEASE "-r8 -fast -fastsse -O3 -Mipa=fast,inline -Minfo")
  set(CMAKE_Fortran_FLAGS_DEBUG "-Mbounds -g")
elseif(${CMAKE_Fortran_COMPILER_ID} MATCHES "G95")
  set(CMAKE_Fortran_FLAGS_RELEASE "-O3")
  set(CMAKE_Fortran_FLAGS_DEBUG "-O0 -g")
elseif(${CMAKE_Fortran_COMPILER_ID} MATCHES "XL")
  set(CMAKE_Fortran_FLAGS_RELEASE "-O5 -qhot -qipa")
  set(CMAKE_Fortran_FLAGS_DEBUG "-O0 -C -g -qfullpath -qinfo -qnosmp -qxflag=dvz -Q! -qnounwind -qnounroll")
else()
message(STATUS "No optimized Fortran compiler flags are known")
message(STATUS "Fortran compiler full path: " ${CMAKE_Fortran_COMPILER})
set(CMAKE_Fortran_FLAGS_RELEASE "-O2")
set(CMAKE_Fortran_FLAGS_DEBUG   "-O0 -g")
endif()

if(${CMAKE_CXX_COMPILER_ID} MATCHES "Intel")
set(CMAKE_CXX_FLAGS_RELEASE "-O3 -std=c++11 -no-prec-div -ansi-alias -qopt-prefetch=4 -unroll-aggressive -m64")
set(CMAKE_CXX_FLAGS_DEBUG "-O0 -std=c++11 -g -traceback -mp1 -fp-trap=common -fp-model strict")
elseif(${CMAKE_CXX_COMPILER_ID} MATCHES "GNU")
set(CMAKE_CXX_FLAGS_RELEASE "-O2 -std=c++11 -msse4 -mtune=native -march=native -funroll-loops -fno-math-errno -ffast-math")
set(CMAKE_CXX_FLAGS_DEBUG "-O0 -std=c++11 -g -pedantic -Wall -Wextra -Wno-unused")
elseif(${CMAKE_CXX_COMPILER_ID} MATCHES "PGI")
  set(CMAKE_CXX_FLAGS_RELEASE "-std=c++0x")
  set(CMAKE_CXX_FLAGS_DEBUG "-std=c++0x")
elseif(${CMAKE_CXX_COMPILER_ID} MATCHES "G95")
  set(CMAKE_CXX_FLAGS_RELEASE "-std=c++0x")
  set(CMAKE_CXX_FLAGS_DEBUG "-std=c++0x")
elseif(${CMAKE_CXX_COMPILER_ID} MATCHES "XL")
  set(CMAKE_CXX_FLAGS_RELEASE "-qlanglvl=extended0x")
  set(CMAKE_CXX_FLAGS_DEBUG "-qlanglvl=extended0x")
else()
message(STATUS "No optimized C++ compiler flags are known")
message(STATUS "C++ compiler full path: " ${CMAKE_CXX_COMPILER})
set(CMAKE_CXX_FLAGS_RELEASE "-O2 -std=c++11")
set(CMAKE_CXX_FLAGS_DEBUG "-O0 -std=c++11 -g")
endif()

set(SOURCES
${CMAKE_SOURCE_DIR}/src/epoch2d.F90
${CMAKE_SOURCE_DIR}/src/boundary.f90
${CMAKE_SOURCE_DIR}/src/fields.f90
${CMAKE_SOURCE_DIR}/src/laser.F90
${CMAKE_SOURCE_DIR}/src/particles.F90
${CMAKE_SOURCE_DIR}/src/shared_data.F90
)

set(FOLDERS deck housekeeping io parser physics_packages user_interaction)
foreach(FOLDER ${FOLDERS})
file(GLOB TMP ${CMAKE_SOURCE_DIR}/src/${FOLDER}/*)
list(APPEND SOURCES ${TMP})
endforeach()

option(OPENMP "Enable multithreading using OpenMP directives." OFF)
option(PER_SPECIES_WEIGHT "Set every pseudoparticle in a species to represent the same number of real particles." OFF)
option(NO_TRACER_PARTICLES "Don't enable support for tracer particles." OFF)
option(NO_PARTICLE_PROBES "Don't enable support for particle probes." OFF)
option(PARTICLE_SHAPE_TOPHAT "Use second order particle weighting." OFF)
option(PARTICLE_SHAPE_BSPLINE3 "Use fifth order particle weighting." OFF)
option(PARTICLE_ID "Include a unique global particle ID using an 8-byte integer." OFF)
option(PARTICLE_ID4 "Include a unique global particle ID using an 4-byte integer." OFF)
option(PARTICLE_COUNT_UPDATE "Keep global particle counts up to date." OFF)
option(PHOTONS "Include QED routines" OFF)
option(TRIDENT_PHOTONS "Use the Trident process for pair production." OFF)
option(PREFETCH "Use Intel-specific 'mm_prefetch' calls to load next particle in the list into cache ahead of time." OFF)
option(PARSER_DEBUG "Turn on debugging." OFF)
option(PARTICLE_DEBUG "Turn on debugging." OFF)
option(MPI_DEBUG "Turn on debugging." OFF)
option(SIMPLIFY_DEBUG "Turn on debugging." OFF)
option(NO_IO "Don't generate any output at all. Useful for benchmarking." OFF)
option(COLLISIONS_TEST "Bypass the main simulation and only perform collision tests." OFF)
option(PER_PARTICLE_CHARGE_MASS "specify charge and mass per particle rather than per species." OFF)
option(PARSER_CHECKING "Perform checks on evaluated deck expressions." OFF)
option(USE_INSITU "Link epoch with ParaView Catalyst." OFF)
option(INSITU_DOUBLE_PREC "Double precision for data exported insitu." OFF)
option(TIGHT_FOCUSING "Maxwell consistent computation of EM fields at boundary for tight-focusing." OFF)

if(OPENMP)
message(STATUS "Option 'OPENMP' enabled")
add_definitions("-DOPENMP")
if(${CMAKE_Fortran_COMPILER_ID} MATCHES "Intel")
set(CMAKE_Fortran_FLAGS_RELEASE "${CMAKE_Fortran_FLAGS_RELEASE} -qopenmp")
set(CMAKE_Fortran_FLAGS_DEBUG "${CMAKE_Fortran_FLAGS_DEBUG} -qopenmp")
elseif(${CMAKE_Fortran_COMPILER_ID} MATCHES "GNU")
set(CMAKE_Fortran_FLAGS_RELEASE "${CMAKE_Fortran_FLAGS_RELEASE} -fopenmp")
set(CMAKE_Fortran_FLAGS_DEBUG "${CMAKE_Fortran_FLAGS_DEBUG} -fopenmp")
else()
message(STATUS "Fortran OpenMP compiler flag not known.")
endif()
if(${CMAKE_CXX_COMPILER_ID} MATCHES "Intel")
set(CMAKE_CXX_FLAGS_RELEASE "${CMAKE_CXX_FLAGS_RELEASE} -qopenmp")
set(CMAKE_CXX_FLAGS_DEBUG "${CMAKE_CXX_FLAGS_DEBUG} -qopenmp")
elseif(${CMAKE_CXX_COMPILER_ID} MATCHES "GNU")
set(CMAKE_CXX_FLAGS_RELEASE "${CMAKE_CXX_FLAGS_RELEASE} -fopenmp")
set(CMAKE_CXX_FLAGS_DEBUG "${CMAKE_CXX_FLAGS_DEBUG} -fopenmp")
else()
message(STATUS "C++ OpenMP compiler flag not known.")
endif()
endif()

if(TIGHT_FOCUSING AND NOT FFT_LIBRARY)
message(STATUS "No FFT library specified. Setting FFT library to 'none'.")
set(FFT_LIBRARY none CACHE STRING "Choose the FFT library." FORCE)
set_property(CACHE FFT_LIBRARY PROPERTY STRINGS "FFTW" "MKL" "None")
endif()

if(PER_SPECIES_WEIGHT)
message(STATUS "Option 'PER_SPECIES_WEIGHT' enabled")
add_definitions("-DPER_SPECIES_WEIGHT")
endif()

if(NO_TRACER_PARTICLES)
message(STATUS "Option 'NO_TRACER_PARTICLES' enabled")
add_definitions("-DNO_TRACER_PARTICLES")
endif()

if(NO_PARTICLE_PROBES)
message(STATUS "Option 'NO_PARTICLE_PROBES' enabled")
add_definitions("-DNO_PARTICLE_PROBES")
endif()

if(PARTICLE_SHAPE_TOPHAT)
message(STATUS "Option 'PARTICLE_SHAPE_TOPHAT' enabled")
add_definitions("-DPARTICLE_SHAPE_TOPHAT")
endif()

if(PARTICLE_SHAPE_BSPLINE3)
message(STATUS "Option 'PARTICLE_SHAPE_BSPLINE3' enabled")
add_definitions("-DPARTICLE_SHAPE_BSPLINE3")
endif()

if(PARTICLE_ID)
message(STATUS "Option 'PARTICLE_ID' enabled")
add_definitions("-DPARTICLE_ID")
endif()

if(PARTICLE_ID4)
message(STATUS "Option 'PARTICLE_ID4' enabled")
add_definitions("-DPARTICLE_ID4")
endif()

if(PARTICLE_COUNT_UPDATE)
message(STATUS "Option 'PARTICLE_COUNT_UPDATE' enabled")
add_definitions("-DPARTICLE_COUNT_UPDATE")
endif()

if(PHOTONS)
message(STATUS "Option 'PHOTONS' enabled")
add_definitions("-DPHOTONS")
endif()

if(TRIDENT_PHOTONS)
message(STATUS "Option 'TRIDENT_PHOTONS' enabled")
add_definitions("-DTRIDENT_PHOTONS")
endif()

if(PREFETCH)
message(STATUS "Option 'PREFETCH' enabled")
add_definitions("-DPREFETCH")
endif()

if(PARSER_DEBUG)
message(STATUS "Option 'PARSER_DEBUG' enabled")
add_definitions("-DPARSER_DEBUG")
endif()

if(PARTICLE_DEBUG)
message(STATUS "Option 'PARTICLE_DEBUG' enabled")
add_definitions("-DPARTICLE_DEBUG")
endif()

if(MPI_DEBUG)
message(STATUS "Option 'MPI_DEBUG' enabled")
add_definitions("-DMPI_DEBUG")
endif()

if(SIMPLIFY_DEBUG)
message(STATUS "Option 'SIMPLIFY_DEBUG' enabled")
add_definitions("-DSIMPLIFY_DEBUG")
endif()

if(NO_IO)
message(STATUS "Option 'NO_IO' enabled")
add_definitions("-DNO_IO")
endif()

if(COLLISIONS_TEST)
message(STATUS "Option 'COLLISIONS_TEST' enabled")
add_definitions("-DCOLLISIONS_TEST")
endif()

if(PER_PARTICLE_CHARGE_MASS)
message(STATUS "Option 'PER_PARTICLE_CHARGE_MASS' enabled")
add_definitions("-DPER_PARTICLE_CHARGE_MASS")
endif()

if(PARSER_CHECKING)
message(STATUS "Option 'PARSER_CHECKING' enabled")
add_definitions("-DPARSER_CHECKING")
endif()

if(USE_INSITU)
message(STATUS "Option 'USE_INSITU' enabled")
find_package(ParaView 5.2 REQUIRED COMPONENTS vtkPVPythonCatalyst)
include(${PARAVIEW_USE_FILE})
file(GLOB ADAPTOR ${CMAKE_SOURCE_DIR}/src/adaptor/*)
list(APPEND SOURCES ${ADAPTOR})
add_definitions("-DINSITU")
if(NOT PARAVIEW_USE_MPI)
message(SEND_ERROR "ParaView must be built with MPI enabled")
endif()
if(INSITU_DOUBLE_PREC)
message(STATUS "Option 'INSITU_DOUBLE_PREC' enabled")
add_definitions("-DINSITU_DOUBLE_PREC")
endif()
endif()

if(TIGHT_FOCUSING)
message(STATUS "Option 'TIGHT_FOCUSING' enabled")
file(GLOB FOCUSING ${CMAKE_SOURCE_DIR}/src/focusing/interface.f90)
list(APPEND SOURCES ${FOCUSING})
include_directories(${MPI_CXX_INCLUDE_PATH})
add_definitions("-DTIGHT_FOCUSING")
set(FOCUS_SRC
${CMAKE_SOURCE_DIR}/src/focusing/main.cpp
${CMAKE_SOURCE_DIR}/src/focusing/domain_param.cpp
${CMAKE_SOURCE_DIR}/src/focusing/laser_param.cpp
${CMAKE_SOURCE_DIR}/src/focusing/global.cpp
${CMAKE_SOURCE_DIR}/src/focusing/laser_bcs.cpp
)
include_directories(${CMAKE_SOURCE_DIR}/src/focusing/inc)
if(${FFT_LIBRARY} MATCHES "FFTW")
if(OPENMP)
message(SEND_ERROR "Multi-threaded FFTW routines are not supported. Disable OpenMP.")
endif()
message(STATUS "FFT library: FFTW")
message(STATUS "Option 'USE_FFTW' enabled")
add_definitions("-DUSE_FFTW")
find_package(FFTW REQUIRED)
include_directories(${FFTW_INCLUDES})
endif()
if(${FFT_LIBRARY} MATCHES "MKL")
if(NOT ${CMAKE_CXX_COMPILER_ID} MATCHES "Intel")
message(SEND_ERROR "MKL library can be used only with Intel compilers")
endif()
message(STATUS "FFT library: MKL")
message(STATUS "Option 'USE_MKL' enabled")
add_definitions("-DUSE_MKL")
set(CMAKE_CXX_FLAGS_RELEASE "${CMAKE_CXX_FLAGS_RELEASE} -mkl")
set(CMAKE_CXX_FLAGS_DEBUG "${CMAKE_CXX_FLAGS_DEBUG} -mkl")
set(CMAKE_Fortran_FLAGS_RELEASE "${CMAKE_Fortran_FLAGS_RELEASE} -mkl")
set(CMAKE_Fortran_FLAGS_DEBUG "${CMAKE_Fortran_FLAGS_DEBUG} -mkl")
endif()
if(${FFT_LIBRARY} MATCHES "None")
message(STATUS "FFT library: None")
endif()
add_library(focus ${FOCUS_SRC})
if(${FFT_LIBRARY} MATCHES "FFTW")
#if(OPENMP)
#  target_link_libraries(focus LINK_PUBLIC ${FFTW_LIBRARIES} ${FFTW_LIBRARIES_OMP})
#else()
target_link_libraries(focus LINK_PUBLIC ${FFTW_LIBRARIES})
#endif()
endif()
set_target_properties(focus PROPERTIES LINKER_LANGUAGE CXX)
endif()

add_executable(epoch2d ${SOURCES})
target_link_libraries(epoch2d LINK_PRIVATE ${MPI_Fortran_LIBRARIES} ${SDF_Fortran_LIBRARIES})
if(USE_INSITU)
target_link_libraries(epoch2d LINK_PRIVATE vtkPVPythonCatalyst vtkParallelMPI)
endif()
if(TIGHT_FOCUSING)
target_link_libraries(epoch2d LINK_PRIVATE ${MPI_CXX_LIBRARIES} focus)
endif()
set_target_properties(epoch2d PROPERTIES LINKER_LANGUAGE Fortran)
\end{lstlisting}