This part contains input files of several most important simulations that have been performed for the purposes within this work. All the files below are accepted by the PIC simulation code EPOCH (see chapter 3). Due to the limited space, provided input files shall be taken as a templates where several parameters have been frequently changing (such as beam waist and the laser intensity).

\setlength{\columnsep}{30pt}
\begin{multicols}{2}
\begin{lstlisting}[style=CXX, caption={Input file for evaluation of the implemented algorithm for tight-focusing (chapter 4). In this case, the laser beam propagates according to the Maxwell consistent approach.\vspace{1mm}}]
begin:control
	nx = 1600
	ny = 4800
	x_min = -8 * micron
	x_max = 8 * micron
	y_min = -24 * micron
	y_max = 24 * micron
	t_end = 150 * femto
	field_order = 2
	stdout_frequency = 1
	smooth_currents = T
	dt_multiplier = 0.95
end:control

begin:boundaries
	cpml_thickness = 16
	cpml_kappa_max = 20
	cpml_a_max = 0.2
	cpml_sigma_max = 0.7
	bc_x_min_field = cpml_laser
	bc_x_max_field = cpml_outflow
	bc_x_min_particle = thermal
	bc_x_max_particle = thermal
	bc_y_min_field = cpml_outflow
	bc_y_max_field = cpml_outflow
	bc_y_min_particle = thermal
	bc_y_max_particle = thermal
end:boundaries

begin:laser
	boundary = x_min
	id = 1
	fwhm_time = 20 * femto
	t_0 = 50 * femto
	w_0 = 0.7 * micron # 5.0 * micron
	pos = 0.0 * micron
	focus = 0.0 * micron
	amp = 1.0e15
	lambda = 1.0 * micron
	pol_angle = 0.0 * pi
	t_start = 0.0 * femto
	t_end = 300 * femto
end:laser

begin:output
	name = fields
	file_prefix = field
	dump_at_times = 50.00 * femto, 76.685127616 * femto, 103.370255232 * femto
	grid = always + single
	ex = always + single
	ey = always + single
	bz = always + single
	dump_first = F
	dump_last = F
	restartable = F
end:output
\end{lstlisting}

\begin{lstlisting}[style=CXX, caption={Input file for identifying the conditions for which the propagation of the laser beam under the paraxial approximation is valid (chapter 4).\vspace{1mm}}]
begin:control
	nx = 1600
	ny = 4800
	x_min = -8 * micron
	x_max = 8 * micron
	y_min = -24 * micron
	y_max = 24 * micron
	t_end = 150 * femto
	field_order = 2
	stdout_frequency = 1
	smooth_currents = T
	dt_multiplier = 0.95
end:control

begin:boundaries
	cpml_thickness = 16
	cpml_kappa_max = 20
	cpml_a_max = 0.2
	cpml_sigma_max = 0.7
	bc_x_min_field = cpml_laser
	bc_x_max_field = cpml_outflow
	bc_x_min_particle = thermal
	bc_x_max_particle = thermal
	bc_y_min_field = cpml_outflow
	bc_y_max_field = cpml_outflow
	bc_y_min_particle = thermal
	bc_y_max_particle = thermal
end:boundaries

begin:constant
	cell_xsize = (x_max - x_min) / nx
	cell_ysize = (y_max - y_min) / ny
	las_lambda0 = 1.0 * micron                            
	las_omega = 2.0 * pi * c / las_lambda0                    
	las_time = 2.0 * pi / las_omega                          
	theta = 0                                                                              
	w0 = 0.7 * micron # 5.0 * micron                                     
	fwhm_time = 20 * femto                                         
	w_time = fwhm_time / (2.0 * (sqrt(loge(2.0))))
	xfocus = 8 * micron                                         
	yc = -xfocus * tan(theta)                                      
	zR = pi * w0^2 / las_lambda0                                  
	lfocus = xfocus / cos(theta)                                   
	wl = w0 * sqrt(1.0+(lfocus^2)/(zR^2))                        
	w_y = wl / cos(theta)                                         
	r_curv = lfocus * (1.0+(zR^2)/(lfocus^2))                
	intensity_fac2d = 1.0 / sqrt(1.0+lfocus^2/zR^2) * cos(theta)   
	n_crit = critical(las_omega)                                   
end:constant

begin:laser
	boundary = x_min
	amp = 1.8895615869e14
	lambda = 1.0 * micron
	pol_angle = 0.0 * pi
	phase = -2.0*pi*(y-yc)*sin(theta)/las_lambda0 + pi/(las_lambda0)*((y-yc)*cos(theta))^2*1.0/r_curv - atan(((yc-y)*sin(theta)+lfocus)/zR)
	profile = gauss(y, yc, sqrt(2)*w_y)
	t_profile = gauss(time, 50*femto, w_time)
	t_start = 0.0 * femto
	t_end = 300 * femto
end:laser

begin:output
	name = fields
	file_prefix = field
	dump_at_times = 50.00 * femto, 76.685127616 * femto, 103.370255232 * femto
	grid = always + single
	ex = always + single
	ey = always + single
	bz = always + single
	dump_first = F
	dump_last = F
	restartable = F
end:output
\end{lstlisting}

\begin{lstlisting}[style=CXX, caption={Input file for the large-scale simulations of tightly focused laser beams interacting with solid targets (chapter 5).\vspace{1mm}}]
begin:control
	nx = 1875
	ny = 5000
	x_min = 0 * micron
	x_max = 15 * micron
	y_min = -20 * micron
	y_max = 20 * micron
	t_end = 200 * femto
	field_order = 2
	stdout_frequency = 1
	smooth_currents = T
	dt_multiplier = 0.95
end:control

begin:boundaries
	cpml_thickness = 16
	cpml_kappa_max = 20
	cpml_a_max = 0.2
	cpml_sigma_max = 0.7
	bc_x_min_field = cpml_laser
	bc_x_max_field = cpml_outflow
	bc_x_min_particle = thermal
	bc_x_max_particle = thermal
	bc_y_min_field = cpml_outflow
	bc_y_max_field = cpml_outflow
	bc_y_min_particle = thermal
	bc_y_max_particle = thermal
end:boundaries

begin:species
	name = electron
	charge = -1.0
	mass = 1.0
	npart_per_cell = 2000
	density = if((x gt 8 * micron) and (x lt 10 * micron) and (abs(y) lt 15 * micron), 100.0 * n_crit, 0.0)
	temp_ev = 100
end:species

begin:species
	name = proton
	charge = 1.0
	mass = 1836.2
	npart_per_cell = 100
	density = density(electron)
	temp_ev = 100
end:species

begin:laser
	boundary = x_min
	id = 1
	fwhm_time = 30 * femto
	t_0 = 60 * femto
	w_0 = 0.5 * micron
	pos = 0.0 * micron
	focus = 8.0 * micron
	intensity_w_cm2 = 1.0e20
	lambda = 0.8 * micron
	pol_angle = 0.0 * pi
	t_start = 0.0 * femto
	t_end = 300 * femto
end:laser

begin:subset
	name = tenth
	random_fraction = 0.1
	include_species:electron
	include_species:proton
end:subset

begin:output
	name = particles
	file_prefix = part
	dt_snapshot = 10 * femto
	particle_grid = tenth + single
	px = tenth + single
	py = tenth + single
	pz = tenth + single
	particle_weight = tenth + single
	dump_first = F
	dump_last = F
	restartable = F
end:output

begin:output
	name = fields
	file_prefix = field
	dt_snapshot = 10 * femto
	grid = always + single
	ex = always + single
	ey = always + single
	bz = always + single
	dump_first = F
	dump_last = F
	restartable = F
end:output

begin:output
	name = density
	file_prefix = dens
	dt_snapshot = 10 * femto
	mass_density = always + single + species
	charge_density = always + single + species
	number_density = always + single + species
	dump_first = F
	dump_last = F
	restartable = F
end:output

begin:output
	name = diag
	file_prefix = diag
	dt_snapshot = 10 * femto
	ekbar = always + single + species
	ekflux = always + single + species
	poynt_flux = always + single
	temperature = always + single + species
	total_energy_sum = always
	dump_first = F
	dump_last = F
	restartable = F
end:output

begin:output
	name = restart
	file_prefix = r
	dt_snapshot = 50.0 * femto
	dump_first = F
	dump_last = F
	restartable = T
end:output
\end{lstlisting}
\end{multicols}