The intensive research and development in the field of laser technology have seen a tremendous progress over the last five decades. New technologies has improved the quality of lasers and offered the prospect of building ultra-short pulsed laser facilities of an extremely high peak output power. Thanks to that, this source of coherent light allows many experiments in various fields of human research to be performed, which otherwise would not be possible. Laser is a unique device that provides an unprecedented capability for the laboratory study of a matter at high energy densities, its interaction with intense radiation and for meeting the challenge of controlled thermonuclear fusion. Therefore, it is no coincidence that a new, more sophisticated methods and techniques of its ignition have been discovered recently.

In the early days of the fusion research, it was thought that the whole volume of the fuel has to be heated and compressed to the fusion conditions. It soon turned out, however, that this would require an unrealistically high amount of driver energy. The solution could be the ignition only of a small fraction of the fuel assembly \cite{pfalzner}. At the moment of stagnation, the plasma at the center of the fuel reaches the highest pressure and temperature as the result of a spherical implosion. If the drive symmetry requirements are achieved, the fusion reactions should start in this region. Once the core of the capsule starts to burn, the energy produced by the fusion reactions is sufficient to heat up the rest of the capsule. A thermonuclear burn wave is then spreading outwards, igniting the whole fuel, which expands rapidly. In this approach, the cold fuel is compressed at minimum energy, and the total energy to be invested into the fuel is therefore significantly reduced. This is known as a central hot-spot ignition scenario \cite{ribeyre}.

Although the hot-spot approach has a potential for a success, there is also considerable interest in a complementary scheme called fast ignition \cite{atzeni}, in which compression phase is separated from the ignition. In this concept, a target is compressed without the requirement to form the central hot-spot, thus it relaxes symmetry requirements significantly. The hot-spot will be then located asymmetrically at the periphery of the compressed fuel core, since the idea is to provide the ignition by an external trigger, which could be an additional high-energy laser pulse.

The crucial question for this concept is how to efficiently transport such a pulse into a dense plasma. The problem is that the compressed plasma is surrounded by an extended corona which is prone to parametric instabilities. One way to bring the laser beam closer to the fuel core is offered by the intense laser beam itself, which tends to bore a hole into the overdense plasma. Another straightforward experimental approach is to guide the beam by a gold cone.

A third, relatively new scenario, which is called shock ignition, might also offer another route to achieving ignition in a non-isobaric target. The cryogenic shell is initially imploded at low velocity using a laser drive of modest peak power and low total energy. The assembled fuel is then separately ignited from a central hotspot heated by a strong, spherically convergent shock wave driven by a high intensity spike at the end of the laser pulse. The launching of the ignition shock is timed to reach the center just as the main fuel is stagnating and starting to rebound \cite{theobald}.

As it involves low velocity implosions, this scheme is relatively robust with regards to hydrodynamic instabilities during the shell acceleration. Crucially, because the implosion velocity is less than that required for conventional compression ignition, considerably more fuel mass may be assembled for the same kinetic energy in the shell, offering significantly higher fusion yields for the same laser energy \cite{batani}.

On the other hand, the laser spike intensity in the coronal plasma is above the threshold of parametric instabilities, thus the collisional absorption of laser light becomes inefficient and strong non-linear effects play an important role. These manifest themselves in enhanced scattering, accompanied by the generation of a population of hot electrons \cite{klimo3}. An understanding of the parametric instabilities and developing ways to control it are of great importance to the ultimate goal of achieving inertial confinement fusion in the laboratory.