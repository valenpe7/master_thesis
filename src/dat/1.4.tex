To be able to describe the interaction of electromagnetic field with matter, one has to know the energy distribution throughout the field as well as the momentum balance.

By scalar multiplications of \ref{1.9} by $ \vec{H}\left( \vec{r}, t \right) $, of \ref{1.10} by $ \vec{E}\left( \vec{r}, t \right) $, following subtraction of both obtained equations and using standard vector identities, one gets the expression 
\begin{equation}
\label{1.31}
\vec{E} \cdot \diffp{\vec{D}}{t} + \vec{H} \cdot \diffp{\vec{B}}{t} + \div{\left(\vec{E} \times \vec{H} \right)} = -\vec{E} \cdot \vec{J}.
\end{equation}
The equation \ref{1.31} can be rewritten in the form of conservation law,
\begin{equation}
\label{1.32}
\diffp{u}{t} + \div{\vec{S}} = - \vec{E} \cdot \vec{J},
\end{equation}
where
\begin{equation}
\label{1.33}
u = \frac{1}{2} \left(\vec{E} \cdot \vec{D} + \vec{H} \cdot \vec{B} \right), \quad \vec{S} = \vec{E} \times \vec{H}.
\end{equation}
The quantity $ u\left( \vec{r}, t \right) $ in \ref{1.33} describes the total energy density in the field and $ \vec{S}\left( \vec{r}, t \right) $ is so-called Poynting vector which represents the energy flow of the field.

The important statement \ref{1.32}, also referred to as the Poynting theorem, expresses the conservation of energy for the electromagnetic field. In other words, the time rate of change of the field energy within a certain region and the energy flowing out of that region is balanced by the conversion of the electromagnetic energy into mechanical or heat energy and vice-versa.

+momentum...

\noindent
Lorentz force:
\begin{equation}
\vec{F} = q \left(E + v \times B \right) 
\end{equation}
Ohm's law:
\begin{equation}
\vec{J} = \sigma \vec{E}
\end{equation}