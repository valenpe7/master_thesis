An off-axis parabolic mirror is a frequently used tool to focus an incoming collimated beam. It is made by cutting out a small section from a full parabolic mirror and thus it allows to deviate the beam path off the optical axis. Therefore, the focal point is at more accessible location and the target is not blocking the incoming collimated laser beam as in the case of a complete parabolic mirror. Obviously, the off-axis parabolic mirror is able to work reversibly, so it can take the light coming from a point source and produce a collimated beam. These physical properties make the off-axis parabolic mirror a valuable tool for many different optical purposes.

As the laser intensities exceed $ 10^{13} \ \mathrm{W/cm}^{2} $, the surface of any material becomes strongly ionized. Therefore, in order to keep the energy density on the optical components below the damage threshold, the beam diameter has to be increased. The beam diameter is usually constrained by the cost of the optical components.

The focal spot can be decreased by implementing a small f-number focusing optic. However, since their focal length is inevitably very short and therefore they must be placed in close proximity to the interaction region, additional care has to be taken to protect the optical components because they can be easily damaged from debris induced by the exploded target flow. Note that such optics are typically very expensive.