Plasma is strongly non-linear medium. This means that it may be accompanied by very strong electric fields and its non-linearities can be excited easily. When laser pulse with intensity above certain threshold irradiates the plasma, a number of collective non-linear processes may occur which can either enhance or reduce the energy absorption. As discussed before, the knowledge of the penetration depth and absorbed fraction of the incident energy is of vital importance for inertial confinement fusion research. 

The non-linear interaction is conveniently described in terms of light pressure and the ponderomotive force, which is introduced first. In the next section, the parametric instabilities are discussed as a consequence of non-linear coupling of electromagnetic laser waves to the plasma. The last section describes the laser beam filamentation and self-focusing.