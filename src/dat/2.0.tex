When a high-power laser pulse is focused onto the surface of a solid target, a high density plasma layer is produced almost immediately. At the higher irradiances, the electric field of the laser light is sufficient to directly ionize the atoms. At the lower irradiances, the process is more complicated but nonetheless a plasma is produced rapidly. The whole region dominated by laser-plasma interactions, which is called corona, expands radially outwards at sonic velocities.

In the inertial fusion experiments, it is essential to deliver the laser energy into the capsule of fusion fuel as much as possible. However, the efficient transport of radiation within the target strongly depends on non-linear processes that take place during the propagation of the laser light in the coronal plasma. Therefore, laser-plasma interactions provide a key gateway to the study of inertial confinement fusion. A brief introduction to this exciting field, which is rich both in physics and in applications is provided in the following chapter. 