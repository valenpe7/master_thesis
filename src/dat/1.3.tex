The release of the nuclear energy by the inertial confinement fusion was demonstrated for the first time using a thermonuclear bomb shortly after the World War II. The devastating effects of explosion are, however, indisputable in that range. It is desirable to obtain the energy in a controlled and meaningful manner. These considerations lead to a typical values of the fusion energy which can be produced by burning only a few milligrams of a deuterium and tritium mixture. In a cryogenic state, such amount of the fuel fits into the volume of a sphere with a diameter of one millimeter. There appeared, however, to be no source powerful enough to deposit sufficient amount of energy into such a small area to induce thermonuclear fusion. Therefore, the inertial confinement was no further taken into account.

Things have changed when the first working laser was demonstrated by T. H. Maiman in the beginning of the 1960s. The idea of achieving nuclear fusion by laser irradiation is based on the conservation of momentum. Multiple intensive beams of laser pulses can be focused directly on a small spherical capsule containing mixture of nuclear fuel. The beam energy deposition leads to the rapid heating and ablation of the outer surface of the capsule shell, forming a coronal plasma that expands into a surrounding vacuum. This process generates a high pressure which accelerates the inner part of the shell inwards. Then the enormous implosion is driven via the rocket effect. It should be understood that it is not beam pressure acting on the target surface, but the ablation pressure generated by the recoil of the expanding material. Under certain conditions, it is possible to achieve the required level of compression for the ignition of the nuclear fuel. This scheme is known as direct drive \cite{basko}.

However, the direct drive entails some serious physics issues and constraints. Primarily, the target implosion is inherently susceptible to hydrodynamic instabilities that lead immediately to the ignition failure. These arise by the very small irradiation non-uniformities due to the finite number of naturally coherent driver beams and due to the macroscopic imperfections on the target surface. In addition, the interaction of high-power laser beams with coronal plasma can generate the population of hot electrons. These subsequently move towards the target core causing premature excessive fuel preheat that makes the compression more difficult. Ideally, the core of the capsule should reach the ignition temperature at the time of the maximum compression to achieve a minimum cost of driver energy \cite{atzeni}.

To minimize the growth of the instabilities, it was later proposed to insert the fusion fuel capsule into a cylindrical cavity made of a largely opaque material. This cavity is today known as a hohlraum \cite{murakami}. The laser beams are pointed through holes onto the interior surface of the hohlraum, which in turn absorbs and emits approximately black body radiation in the form of X-rays. The significant advantage with this approach is that the energy is re-radiated in a much more symmetric fashion than would be possible in the direct drive approach, resulting in a more uniform implosion. This configuration, which is very similar to the original hydrogen bomb design, is known as indirect drive \cite{atzeni}.

The downside to using this approach is obviously lower overall efficiency of energy transfer. The partial loss of the radiation source energy takes place during conversion into thermal X-rays, another significant radiation loss occurs mainly through the beam entrance windows of the hohlraum to the outside. Consequently, only a small fraction of laser energy is absorbed by the capsule. 

In addition, the laser intensity in the plasma corona is above the threshold that gives rise to parametric instabilities, mainly stimulated Raman scattering, Brillouin scattering and two-plasmon decay. The study of these parametric instabilities and the interplay between them is provided in the second chapter of this work.