There exists also other possibilities how to express the electromagnetic field. Under ordinary conditions, an arbitrary electromagnetic field may be defined in terms of a single vector function \cite{Essex1977}. This may be helpful for solving of many problems of classical electromagnetic theory, particularly the wave propagation.

First, let us introduce the electric Hertz vector $ {\vec{\Pi_e}}\left(\vec{r}, t \right) $ in terms of the scalar and vector potentials \cite{Stratton2007},
\begin{equation}
\label{1.19}
\Phi = - \div{\vec{\Pi_e}},
\end{equation}
\begin{equation}
\label{1.20}
\vec{A} = \mu \varepsilon \diffp{\vec{\Pi_e}}{t}.
\end{equation}

Note that the definitions \ref{1.19}, \ref{1.20} are consistent with the Lorenz gauge condition \ref{1.14}. In the absence of magnetization, it might be easily shown that $ \vec{J} = \partial{\vec{P}}/\partial{t} $ and the electric Hertz vector $ {\vec{\Pi_e}}\left(\vec{r}, t \right) $ is governed by an inhomogeneous wave equation
\begin{equation}
\label{1.21}
\laplace{\vec{\Pi_e}} - \mu \varepsilon \diffp[2]{\vec{\Pi_e}}{t} = -\frac{\vec{P}}{\varepsilon}.
\end{equation}

The equation \ref{1.21} is of the same type as the equations \ref{1.15}, \ref{1.16} and has therefore the familiar general solution 
\begin{equation}
\label{1.22}
\vec{\Pi_e}\left(\vec{r}, t \right) = \frac{1}{4 \pi \varepsilon} \int \frac{\vec{P}\left(\vec{r^{\: \prime}}, t^{\: \prime} \right)}{\norm{\vec{r} - \vec{r^{\: \prime}}}} \mathrm{d} V.
\end{equation}

As might be seen form \ref{1.22}, the fields derived from the electric Hertz vector $ {\vec{\Pi_e}}\left(\vec{r}, t \right) $ can be interpreted as being due to a density distribution of electric dipoles \cite{Essex1977}. Every solution of \ref{1.22} then uniquely determines the electromagnetic field through
\begin{equation}
\label{1.23}
\vec{E} = \grad{\left(\div{\vec{\Pi_e}}\right)} - \mu \epsilon \diffp[2]{\vec{\Pi_e}}{t},
\end{equation}
\begin{equation}
\label{1.24}
\vec{B} = \mu \varepsilon \left(\rot{\diffp{\vec{\Pi_e}}{t}}\right).
\end{equation}

Second, one may introduce the magnetic Hertz vector $ {\vec{\Pi_m}}\left(\vec{r}, t \right) $ in terms of the scalar and vector potentials by the following expressions \cite{Stratton2007},
\begin{equation}
\label{1.25}
\Phi = 0,
\end{equation}
\begin{equation}
\label{1.26}
\vec{A} = \rot{\vec{\Pi_m}}.
\end{equation}
In the absence of polarization, $ \vec{J} = \rot{\vec{M}} $ and the magnetic Hertz vector $ {\vec{\Pi_m}}\left(\vec{r}, t \right) $ defined by \ref{1.25} and \ref{1.26} fulfills an inhomogeneous wave equation
\begin{equation}
\label{1.27}
\laplace{\vec{\Pi_m}} - \mu \varepsilon \diffp[2]{\vec{\Pi_m}}{t} = -\mu \vec{M}.
\end{equation}
As for the previous cases, one may easily find the solution of \ref{1.27},
\begin{equation}
\label{1.28}
\vec{\Pi_m}\left(\vec{r}, t \right) = \frac{\mu}{4 \pi} \int \frac{\vec{M}\left(\vec{r^{\: \prime}}, t^{\: \prime} \right)}{\norm{\vec{r} - \vec{r^{\: \prime}}}} \mathrm{d} V,
\end{equation}
thus the fields derived from the magnetic Hertz vector $ {\vec{\Pi_m}}\left(\vec{r}, t \right) $ may be imagined to be due to a density distribution of magnetic dipoles \cite{Essex1977}. Again, every solution of \ref{1.28} uniquely determines the electromagnetic field via
\begin{equation}
\label{1.29}
\vec{E} = \rot{\diffp{\vec{\Pi_m}}{t}},
\end{equation}
\begin{equation}
\label{1.30}
\vec{B} = \rot{\left(\rot{\vec{\Pi_m}}\right)}.
\end{equation}

Note that the above derivations considered electric and magnetic Hertz vectors as a separate quantities. It is also possible, however, to introduce them together in the form of one six-vector \cite{Nisbet1955}.