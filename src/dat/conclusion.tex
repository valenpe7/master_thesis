The work briefly presents the introduction to the inertial fusion research as well as basic physical processes which take place during the interaction of intense laser pulse with plasma. Particularly, for better interpretation and understanding of ongoing experiments that study the possibilities of nuclear fusion ignition by shock wave, the conditions of interaction have been set accordingly to them.

The main benefits of this work are successful implementation of boundary conditions for the effective absorption of hot electrons in the two-dimensional version of the computational code EPOCH. Its correct functionality has been later verified by plenty of numerical tests. Afterwards, two large scale simulations of laser system PALS in two-dimensional geometry on its fundamental wavelength 1,315 $ \mathrm{\mu m} $ with intensity $ 1 \cdot 10^{20} \: \mathrm{W/m^2} $ and with initial electron temperatures $ \mathrm{T_e} = 0.5 \: \mathrm{keV} $ and $ \mathrm{T_e} = 2.5 \: \mathrm{keV} $ have been performed. Both simulations capture the time period of 20 ps. Simulations have been performed using the particle-in-cell code EPOCH \cite{bennett}. Initial profiles of plasma density and temperature have been approximated from hydrodynamic simulations, which have been performed previously.

The total absorption of incident laser energy in plasma for the case of the simulation with $ \mathrm{T_e} = 0.5 \: \mathrm{keV} $ was estimated to 42.4 \%, for the case of the simulation with $ \mathrm{T_e} = 2.5 \: \mathrm{keV} $, the total absorption was significantly lower, about 33.1 \%. The temperature of the hot electrons in the case of the simulation with $ \mathrm{T_e} = 0.5 \: \mathrm{keV} $ was estimated to 36 keV, in the case of the simulation with $ \mathrm{T_e} = 2.5 \: \mathrm{keV} $ the temperature was about 25 keV. In both cases, the number of hot electrons is relatively low and their temperatures are not too high to prevent the fuel target compression in the later phase. However, it is necessary to further investigate their effect performing more accurate simulations.

