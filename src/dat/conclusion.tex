Increasing the peak laser intensity enables new routes of research in diverse fields (e.g. coherent diffractive imaging, X-ray diffraction and spectroscopy, pulse radiolysis, production of compact sources for radiotherapy \cite{Zewail2010, Bulanov2004, Malka2004}). One of the easiest way of increasing the laser intensity is reducing the focal spot size. Tight-focusing to the spot size of the order of laser wavelength can be achieved using curved plasma mirrors, which allow to reach laser intensities an order of magnitude higher than the focusing with a typical $ f/3 $ off-axis parabolic mirror. The structure of the electromagnetic field of a tightly focused intense beam includes a strong longitudinal component of the electric field and the transverse component of the ponderomotive force. The interaction of such pulses with solid targets is studied in this work.

The first part of this work summarizes the knowledge required for further understanding of the laser-plasma interaction. In the first chapter, the fundamental physical aspects of the classical electromagnetic field theory \cite{Stratton2007, Jackson2005, Feynman1963, Thide2011} based on the Maxwell's equations as well as the description of Gaussian laser pulse using the paraxial approximation \cite{Born2013} are summarized. The second chapter is focused on basic physical processes which take place during the interaction of intense laser pulses with plasma. It includes approaches for plasma description, propagation of electromagnetic wave in plasma, laser absorption and plasma heating mechanisms and mechanisms of laser-driven ion acceleration. In the third chapter, one may find a mathematical derivation of the particle-in-cell method, description of individual steps of the algorithm as well as conditions of its stability. The last section of this chapter is dedicated to the code EPOCH \cite{bennett}, which has been used for simulations within this work. Starting from the fourth chapter, the work is devoted mainly to tight-focusing. This chapter contains a description, implementation and evaluation of the algorithm for rigorous calculation of electromagnetic fields at boundaries of simulation domain \cite{Thiele2016} as well as a brief overview of currently used experimental methods for tight-focusing. The fifth chapter presents results of several large-scale simulations of tightly focused laser beams interacting with solid targets.

The main contribution of this work is a successful implementation of laser boundary conditions that allows simulate tightly focused laser beams using the two-dimensional version of the computational code EPOCH \cite{bennett}. The correctness of the algorithm as well as the proper implementation have been verified by plenty of simulations and numerical tests. It has been shown, that the laser beam initialized using the paraxial approximation can lead to unexpected field profiles in the case of tight-focusing (the focal spot is shifted, field profiles are distorted and asymmetric, the peak laser amplitude is lower). These deviations are far from negligible and have a strong impact on laser-matter or laser-plasma simulation results. On the other hand, the simulations of tight-focusing where the beam at the boundary has been prescribed using the Maxwell consistent approach \cite{Thiele2016} fulfills specified requirements precisely.

The instrumented code has been further exploited for several two-dimensional large-scale simulations employing tightly focused laser beams interacting with solid targets. Obtained results have been processed and thoroughly analyzed while the emphasis has been placed mainly on identifying the effects of the laser beam focal spot size on the laser-matter interaction. The results and observations may be summarized as follows: for a given laser pulse with constant energy, focusing to a smaller spot size leads to higher energy absorption, hot electron temperature and maximum energy of accelerated ions. However, the results do not clearly demonstrate the effects of the structure of the electromagnetic field of tightly focused beams on the interaction because the results strongly depend on the peak laser intensity and probably also on the duration of the pulse which has not been investigated within this work. 

We identified several characteristics of the laser-mater interaction that we consider to be due to the focal spot size in the simulations, where the laser intensity was constant. We performed series of simulations for four different sizes of the laser focus. The main differences have been observed between the cases of the focal spot larger than the center laser wavelength and the focus of sub-wavelength level (tight-focusing). It has been shown that the laser energy absorption efficiency sharply increases in the case of tight-focusing. The direction of hot electrons moving forward is given by the ratio between the transverse and the longitudinal component of ponderomotive force, which increases as the focal spot size decreases. Consequently, there is a larger amount of hot electrons spreading in the transverse direction with respect to the direction of incoming laser beam in the case of tight-focusing. Also, a significant cloud of hot electrons in front of the target has been observed in this case. The space charge induced by hot electrons ejected into vacuum causes rapid expansion of plasma in the vicinity of the focal spot and a strong electric current along the front surface of the target which compensates the charge unbalance. This affects the electron trajectories and absorption processes during the interaction. The energy distribution functions of electrons are quantitatively different and the temperature of hot electrons is significantly higher in the case of tight-focusing. The laser energy transfer to ions and the maximum ion energies increase with the focal spot size due to the higher total energy in the laser beam and in hot electrons. However, the ion acceleration efficiency is almost independent of the focus.

