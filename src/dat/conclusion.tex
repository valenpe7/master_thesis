Increasing the peak laser intensity enables a new avenues of research in diverse fields \cite{Zewail2010, Bulanov2004, Malka2004}. A typical approach for intensity amplification involves increasing the pulse energy. However, this is costly and it requires higher level of complexity for the laser chain \cite{Fuchs2014}. A more effective way to increase the laser intensity would be to reduce the focal spot size. Tight-focusing may also enhance the spatial and temporal contrast ratio of the laser pulse which is important for many applications \cite{Fuchs2014}. However, the conventional solid state optics are inappropriate in this case (expensive, susceptible to damage from solid target debris, sensitive to small misalignments). Nevertheless, it seems that many drawbacks could be in future overcome by using a plasma-based focusing optics.

The first part of this work summarizes the knowledge required for further understanding of the laser-plasma interaction. In the first chapter, the fundamental physical aspects of the classical electromagnetic field theory \cite{Stratton2007, Jackson2005, Feynman1963, Thide2011} based on the Maxwell's equations as well as the description of Gaussian laser pulse using the paraxial approximation \cite{Born2013} are described. The second chapter is focused on basic physical processes which take place during the interaction of intense laser pulses with plasma. It includes approaches for plasma description, propagation of electromagnetic wave in plasma, laser absorption and plasma heating mechanisms or mechanisms of laser-driven ion acceleration. In the third chapter, one may find a mathematical derivation of the particle-in-cell method, description of individual steps of the algorithm as well as conditions of its stability. The last section of this chapter is dedicated to code EPOCH \cite{bennett}, which has been used for simulations within this work. Starting from the fourth chapter, the work is devoted mainly to tight-focusing. This chapter contains a description, implementation and evaluation of the algorithm for rigorous calculation of electromagnetic fields at boundaries of simulation domain \cite{Thiele2016} as well as a brief overview of currently used experimental methods for tight-focusing. The fifth chapter presents results of several large-scale simulations of tightly focused laser beams interacting with solid targets.

The main contribution of this work is a successful implementation of laser boundary conditions that allows simulate tightly focused laser beams using the two-dimensional version of the computational code EPOCH \cite{bennett}. The correctness of the algorithm as well as the proper implementation have been verified by plenty of simulations and numerical tests. It has been shown, that the laser beam initialized using the paraxial approximation can lead to unexpected field profiles in the case of tight-focusing - the focal spot is shifted, field profiles are distorted and asymmetric and the peak laser amplitude is lower. These deviations are far from negligible and have a strong impact on laser-matter or laser-plasma simulation results. On the other hand, the simulations of tight-focusing where the beam at the boundary has been prescribed using the Maxwell consistent approach \cite{Thiele2016} fulfills specified requirement precisely.

The instrumented code has been further exploited for several two-dimensional large-scale simulations employing tightly focused laser beams interacting with solid targets. Obtained results have been processed and thoroughly analyzed while the emphasis has been placed mainly on identifying the effects of the laser beam focal spot size on the laser-matter interaction results. The results and observations may be summarized as follows:

\begin{itemize}
	\item[\tiny $\blacksquare$] in the case of tight-focusing, the laser energy absorption efficiency sharply increases
	\item[\tiny $\blacksquare$] in the case of tight-focusing, the plasma in the vicinity of the focal spot expands rapidly
	\item[\tiny $\blacksquare$] in the case of tight-focusing, there is a strong electric current along the target front surface
	\item[\tiny $\blacksquare$] as the focal spot size decreases, the transverse component of the ponderomotive force increases
	\item[\tiny $\blacksquare$] the direction of electrons moving forward is given by the ratio between the longitudinal and transverse component of ponderomotive force
	\item[\tiny $\blacksquare$] for the larger focal spot sizes, the electric field decays more slowly
	\item[\tiny $\blacksquare$] the number of non-relativistic hot electrons is higher for the larger focal spot size 
	\item[\tiny $\blacksquare$] there is a significant quantitative difference between the electron trajectories for different focal spot sizes
	\item[\tiny $\blacksquare$] in the case of tight-focusing, there is larger amount of hot electrons spreading in the transverse direction with respect to the direction of incoming laser pulse
	\item[\tiny $\blacksquare$] in the case of tight-focusing, one may observe a significant cloud of hot electrons in front of the target
	\item[\tiny $\blacksquare$] in the case of tight-focusing, the energy distribution of electrons is qualitatively different
	\item[\tiny $\blacksquare$] as the focal spot size increases, the laser energy transfer to ions is faster
	\item[\tiny $\blacksquare$] the maximum ion energies increase with the focal spot size
	\item[\tiny $\blacksquare$] ion acceleration efficiency is independent of the focal spot size
\end{itemize}

