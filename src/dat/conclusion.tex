The beginning of this work summarizes the knowledge required for the further understanding of the laser-plasma interaction. In the first chapter, the fundamental physical aspects of the classical electromagnetic field theory based on the Maxwell's equations as well as the description of Gaussian laser pulse using the paraxial approximation are provided. The second chapter is focused on basic physical processes which take place during the interaction of intense laser pulse with plasma. It includes approaches for plasma description, propagation of electromagnetic wave in plasma, laser absorption and plasma heating mechanisms or mechanisms of laser-driven ion acceleration. In the third chapter, one may find a mathematical derivation of the particle-in-cell method, description of individual steps of the algorithm as well as conditions of its stability. The last section of this chapter is dedicated to code EPOCH, which has been used for simulations within this work. 

Starting from the fourth chapter, the work is devoted mainly to tight-focusing. This chapter contains a description, implementation and evaluation of the algorithm for rigorous calculation of electromagnetic fields at boundaries of simulation domain as well as a brief overview of currently used experimental methods for tight-focusing. The fifth chapter then presents results of several large-scale simulations employing tightly focused laser beams interacting with solid targets.

The main benefit of this work is a successful implementation of laser boundary conditions that allow simulate tightly focused laser beams using the two-dimensional version of the computational code EPOCH. The correctness of the algorithm as well as the proper implementation have been verified by plenty of simulations and numerical tests. It has been shown, that the laser beam initialized using the paraxial approximation can lead to unexpected field profiles in the case of tight-focusing - the focal spot is shifted, field profiles are distorted and asymmetric and the peak laser amplitude is lower. These deviations are far from negligible and have a strong impact on laser-matter or laser-plasma simulation results. On the other hand, the simulations of tight-focusing where the beam at the boundary has been prescribed using the Maxwell consistent approach fulfills specified requirement precisely.

The instrumented code has been further exploited for several two-dimensional large-scale simulations employing tightly focused laser beams interacting with solid targets. Obtained results have been processed and thoroughly analyzed while the emphasis has been placed mainly on identifying the effects of the laser beam focal spot size on the laser-matter interaction results. The results and observations may be summarized as follows:

