Since the laser radiation is nothing but the electromagnetic wave...
Electromagnetic field because the time-varying magnetic field give rise to electric field and vice-versa. interconnection between e and b becomes clear in the framework of special relativity

The electromagnetic field is in general theory represented by two vectors, the intensity of the electric field $ \vec{E}\left( \vec{r}, t \right) $ and the magnetic induction $ \vec{B}\left( \vec{r}, t \right) $. These vectors are considered to be finite and continuous functions of position $ \vec{r} $ and time $ t $. The description of electromagnetic phenomena in classical electrodynamics is provided by the set of well-known Maxwell's equations. The microscopic variant for external sources in vacuum is formulated as follows,
\begin{equation}
\label{1.1}
\div{\vec{E}} = \frac{\rho}{\varepsilon_0},
\end{equation}
\begin{equation}
\label{1.2}
\div{\vec{B}} = 0,
\end{equation}
\begin{equation}
\label{1.3}
\rot{\vec{E}} + \diffp{\vec{B}}{t} = 0,
\end{equation}
\begin{equation}
\label{1.4}
\rot{\vec{B}} - \mu_0 \varepsilon_0 \diffp{\vec{E}}{t}= \mu_0 \vec{J},
\end{equation}
where $ \rho\left( \vec{r}, t \right) $ is total electric charge density and $ \vec{J}\left( \vec{r}, t \right) $ is total electric current density, which is constituted by the motion of a charge. These distributions may be continuous as well as discrete. As might be seen from Maxwell's equations (\ref{1.1} - \ref{1.4}), the charge density is the source of the electric field, whilst the magnetic field is produced by the current density. The lack of symmetry in Maxwell's equations (\ref{1.2}, \ref{1.3} are homogeneous) is caused by the experimental absence of magnetic charges and currents. The universal constants appearing in the Maxwell's equations (\ref{1.1}, \ref{1.4}) are the electric permittivity of vacuum $ \varepsilon_0 $ and the magnetic permeability of vacuum $ \mu_0 $.

The first equation, \ref{1.1}, is Gauss's law for electric field. It states that the flux of electric field through any closed surface is proportional to the total charge inside. The second equation, \ref{1.2}, is Gauss's law for magnetic field. It expresses the fact that there are no magnetic monopoles, so the flux of magnetic field through any closed surface is always zero. The third equation, \ref{1.3}, is Faraday's law describing how the electric field is associated with a time varying magnetic field. And the last equation, \ref{1.4}, is Ampere's law with Maxwell's displacement current, which means that the time varying electric field causes the magnetic field. As a consequence, it predicts the existence of electromagnetic waves that can carry energy and momentum even in a free space.

To describe the effects of an electromagnetic field in the presence of macroscopic substances, the complicated distribution of charges and currents in matter over the atomic scale is not relevant. Thus one shall define a second set of auxiliary vectors that represent fields in which the material properties are already included in an average sense, the electric displacement $ \vec{D}\left( \vec{r}, t \right) $ and the magnetic vector $ \vec{H}\left( \vec{r}, t \right) $,
\begin{equation}
\label{1.5}
\vec{D} = \varepsilon_0 \vec{E} + \vec{P} = \varepsilon \vec{E},
\end{equation}
\begin{equation}
\label{1.6}
\vec{H} = \frac{\vec{B}}{\mu_0} - \vec{M} = \frac{\vec{B}}{\mu},
\end{equation}
where $ \vec{P}\left( \vec{r}, t \right) $ and $ \vec{M}\left( \vec{r}, t \right) $ are the vectors of electric and magnetic polarization, respectively. Note, that the vectors of electric and magnetic polarization are definitely associated with the state of a matter and vanish in a vacuum. Similarly as in the case of a free space, the factors $ \varepsilon $ and $ \mu $ are called electric permittivity of medium and magnetic permeability of medium. In general case, $ \varepsilon $ and $ \mu $ are tensors. The constitutive relations above (\ref{1.5}, \ref{1.6}) hold only if the relations between derived and primary fields are linear.

The macroscopic variant of Maxwell's equations is formulated as follows,
\begin{equation}
\label{1.7}
\div{\vec{D}} = \rho,
\end{equation}
\begin{equation}
\label{1.8}
\div{\vec{B}} = 0,
\end{equation}
\begin{equation}
\label{1.9}
\rot{\vec{E}} + \diffp{\vec{B}}{t} = 0,
\end{equation}
\begin{equation}
\label{1.10}
\rot{\vec{H}} - \diffp{\vec{D}}{t} = \vec{J},
\end{equation}
where $ \rho\left(\vec{r}, t \right) $ and $ \vec{J}\left(\vec{r}, t \right) $ are now macroscopic electric charge and current density, respectively.

By combining the time derivative of the equation \ref{1.7} with the divergence of the equation \ref{1.10}, one obtains the following relation between the electromagnetic field sources,
\begin{equation}
\label{1.11}
\div{\vec{J}} + \diffp{\rho}{t} = 0.
\end{equation}
The equation \ref{1.11}, which is frequently referred to as the equation of continuity, expresses nothing but the conservation of total electric charge in an isolated system. In other words, the time rate of change of the electric charge in any closed surface is balanced by the electric current flowing through the surface.

The first-order partial differential Maxwell's equations can be effectively converted to a smaller number of second-order equations by introducing electrodynamic potentials. Hence, one can express the electric and magnetic field as follows,
\begin{equation}
\label{1.12}
\vec{E} = -\grad{\Phi} - \diffp{\vec{A}}{t},
\end{equation}
\begin{equation}
\label{1.13}
\vec{B} = \rot{\vec{A}},
\end{equation}
where $ \Phi\left(\vec{r}, t \right) $ is the scalar potential and $ \vec{A}\left(\vec{r}, t \right) $ is the vector potential of the corresponding fields. One can clearly see that using the definitions \ref{1.12}, \ref{1.13}, six vector components are replaced by only four potential functions and two Maxwell's homogeneous equations (\ref{1.8}, \ref{1.9}) are fulfilled identically. 

However, by definitions \ref{1.12}, \ref{1.13}, $ \Phi\left(\vec{r}, t \right) $ and $ \vec{A}\left(\vec{r}, t \right) $ are not defined uniquely, thus an infinite number of potentials which lead to the same fields may be constructed. To avoid that, one has to impose an supplementary condition, for example
\begin{equation}
\label{1.14}
\div{\vec{A}} + \mu \varepsilon \diffp{\Phi}{t} = 0.
\end{equation}
The condition
\begin{equation}
\laplace{\Phi} - \mu \varepsilon \diffp[2]{\Phi}{t} = -\frac{\rho}{\varepsilon}
\end{equation}
\begin{equation}
\laplace{\vec{A}} - \mu \varepsilon \diffp[2]{\vec{A}}{t} = -\mu \vec{J}
\end{equation}
solutions:
\begin{equation}
\Phi\left(\vec{r}, t \right) = \frac{1}{4 \pi \varepsilon} \int\limits_{V} \frac{\rho\left(\vec{\tilde{r}}, t \right)}{\abs{\vec{r} - \vec{\tilde{r}}}} \mathrm{d} \tilde{V}
\end{equation}
\begin{equation}
\vec{A}\left(\vec{r}, t \right) = \frac{\mu}{4 \pi} \int\limits_{V} \frac{\vec{J}\left(\vec{\tilde{r}}, t \right)}{\abs{\vec{r} - \vec{\tilde{r}}}} \mathrm{d} \tilde{V}
\end{equation}
Hertz vectors (under ordinary conditions):
\begin{equation}
\Phi = - \div{\vec{\Pi_e}}
\end{equation}
\begin{equation}
\vec{A} = \mu \varepsilon \diffp{\vec{\Pi_e}}{t}
\end{equation}
\begin{equation}
\Phi = 0
\end{equation}
\begin{equation}
\vec{A} = \rot{\vec{\Pi_m}}
\end{equation}
E and B in terms of electric Hertz vector:
\begin{equation}
\vec{E} = \grad{\left(\div{\vec{\Pi_e}}\right)} - \mu \epsilon \diffp[2]{\vec{\Pi_e}}{t}
\end{equation}
\begin{equation}
\vec{B} = \mu \varepsilon \left(\rot{\diffp{\vec{\Pi_e}}{t}}\right)
\end{equation}
E and B in terms of magnetic Hertz vector:
\begin{equation}
\vec{E} = \rot{\diffp{\vec{\Pi_m}}{t}}
\end{equation}
\begin{equation}
\vec{B} = \rot{\left(\rot{\vec{\Pi_m}}\right)}
\end{equation}
equations for Hertz vectors:
\begin{equation}
\laplace{\vec{\Pi_e}} - \mu \varepsilon \diffp[2]{\vec{\Pi_e}}{t} = -\frac{\vec{P}\left(\vec{r} \right)}{\varepsilon}
\end{equation}
\begin{equation}
\laplace{\vec{\Pi_m}} - \mu \varepsilon \diffp[2]{\vec{\Pi_m}}{t} = -\mu \vec{M}\left(\vec{r} \right)
\end{equation}
solutions:
\begin{equation}
\vec{\Pi_e}\left(\vec{r}, t \right) = \frac{1}{4 \pi \varepsilon} \int\limits_{V} \frac{\vec{P}\left(\vec{\tilde{r}} \right)}{\abs{\vec{r} - \vec{\tilde{r}}}} \mathrm{d} \tilde{V}
\end{equation}
\begin{equation}
\vec{\Pi_m}\left(\vec{r}, t \right) = \frac{\mu}{4 \pi} \int\limits_{V} \frac{\vec{M}\left(\vec{\tilde{r}} \right)}{\abs{\vec{r} - \vec{\tilde{r}}}} \mathrm{d} \tilde{V}
\end{equation}
Lorentz force:
\begin{equation}
\vec{F} = q \left(E + v \times B \right) 
\end{equation}
Ohm's law:
\begin{equation}
\vec{J} = \sigma \vec{E}
\end{equation}
Energy (Poynting) theorem:
\begin{equation}
\diffp{u}{t} + \div{\vec{S}} = - \vec{E} \cdot \vec{J}
\end{equation}
\begin{equation}
u = \frac{1}{2} \left(\vec{E} \cdot \vec{D} + \vec{H} \cdot \vec{B} \right)
\end{equation}
\begin{equation}
\int\limits_{S} \left(\vec{E} \times \vec{H} \right) \cdot \vec{n} \: \d \tilde{S} + \int\limits_{V} \vec{E} \cdot \vec{J} \: \d \tilde{V} = - \diffp{}{t} \int\limits_{V} \frac{1}{2} \left(\vec{E} \cdot \vec{D} + \vec{H} \cdot \vec{B} \right) \: \d \tilde{V}
\end{equation}
\begin{equation}
\vec{S} = \vec{E} \times \vec{H}
\end{equation}
\begin{equation}
\diffp{W}{t} = \int\limits_{V} \left(\vec{E} \cdot \diffp{\vec{D}}{t} + \vec{H} \cdot \diffp{\vec{B}}{t}\right) \: \d \tilde{V}
\end{equation}
wave equations:
\begin{equation}
\laplace{\vec{E}} - \frac{1}{c^{2}} \diffp[2]{\vec{E}}{t} = 0
\end{equation}
\begin{equation}
\laplace{\vec{B}} - \frac{1}{c^{2}} \diffp[2]{\vec{B}}{t} = 0
\end{equation}

\section{Gaussian Beam}
Laser beams are represented by Gaussian beams.
The simplest mathematical description of the essential features of a focused optical beam is provided by Gaussian beams.
All components of the electric and magnetic fields can be deduced from a single scalar wave function.
In general, the forms of laser beams can be usefully deduced from a vector potential that has a single Cartesian coordinate.
vector potential:
\begin{equation}
\vec{E}\left(\vec{r_\bot}, z, t \right)  = E_0 \Psi \left(\vec{r_\bot}, z \right) \e^{i \left(k_z z - \omega t \right)} \mathrm{\hat{e}_x}
\end{equation}
angular frequency $ \omega $, wave number $ k_z = \omega/c $, 
Helmholtz equation:
\begin{equation}
\laplace{\Psi \left(\vec{r_\bot}, z \right)} + 2 \i k_z \diffp{\Psi \left(\vec{r_\bot}, z \right)}{z} = 0
\end{equation}
parameters (w0 radius of the beam waist):
\begin{equation}
w_0, \quad z_r = \frac{k_z w_0^2}{2} = \frac{\pi w_0^2}{\lambda}, \quad \Theta = \frac{w_0}{z_R} = \frac{\lambda}{\pi w_0}
\end{equation}
dimensionless coordinates:
\begin{equation}
\rho = \frac{\norm{\vec{r_\bot}}}{w_0}, \quad \zeta = \frac{z}{z_r}
\end{equation}
after transformation:
\begin{equation}
\frac{1}{\rho} \diffp{}{\rho}\left(\rho \diffp{\Psi \left(\rho, \zeta \right)}{\rho} \right) + 4 \i \diffp{\Psi \left(\rho, \zeta \right)}{\zeta}  = - \Theta^2 \diffp[2]{\Psi \left(\rho, \zeta \right)}{\zeta}
\end{equation}
expansion (assume $ \Theta \ll 1 $ and consequently $  w_0 \gg \lambda $):
\begin{equation}
\Psi = \sum_{n = 0}^{+\infty} \Theta^{2n} \Psi_{2n}
\end{equation}
zero-th order with transverse laplacian in cylindrical coords:
\begin{equation}
\frac{1}{\rho} \diffp{}{\rho}\left(\rho \diffp{\Psi_0 \left(\rho, \zeta \right)}{\rho} \right) + 4 \i \diffp{\Psi_0\left(\rho, \zeta \right)}{\zeta} = 0
\end{equation}
solution in dimensionless coordinates:
\begin{equation}
\Psi_0 \left(\rho, \zeta \right) = \frac{1}{\sqrt{1 + \zeta^2}} \exp{\left[- \frac{\rho^2}{1 + \zeta^2} + \i \left(\frac{\rho^2 \zeta}{1 + \zeta^2} - \arctan{\zeta} \right) \right]} 
\end{equation}
in Cartesian coordinates:
\begin{equation}
\Psi_0 \left(\vec{r_\bot}, z \right) = \frac{w_0}{w\left(z\right)} \exp{\left[- \frac{\vec{r_\bot}^2}{w\left(z \right)^2} + \i \left( k_z \frac{\vec{r_\bot}^2}{2 R\left(z \right)} - \varphi_G \left( z\right) \right) \right]}
\end{equation}
where:
\begin{equation}
w\left(z\right) = w_0 \sqrt{1 + \left(\frac{z}{z_R}\right)^2}, \quad R\left(z \right) = z \left[1 + \left(\frac{z_R}{z} \right)^2\right], \quad \varphi_G = \arctan{\left(\frac{z}{z_R}\right)}
\end{equation}
Solution - Gaussian beam:
\begin{equation}
\vec{E}\left(\vec{r_\bot}, z, t \right) = E_0 \frac{w_0}{w(z)} \exp\left(-\frac{\vec{r_\bot}^2}{w(z)^2}\right) \cos\left(\omega t - k_z \left(z + \frac{\vec{r_\bot}^2}{2 R(z)} \right) + \varphi_G(z) \right) \mathrm{\hat{e}_x}
\end{equation}
where
\begingroup
\renewcommand*{\arraystretch}{2.0}
\begin{table}[h!]
\begin{flushleft}
\begin{tabular}{ l c r }
	$ w(z) = w_0 \sqrt{1 + \left(\frac{z}{z_R} \right)^2}  $ & \ldots & evolving beam width \\
	$ z_r = \frac{\pi w_0^2}{\lambda} $ & \ldots & Rayleigh range \\
	$ R(z) = z\left(1 + \left(\frac{z_R}{z} \right)^2 \right) $ & \ldots & radius of curvature \\
	$ \phi(z) = \tan^{-1}\left(\frac{z}{z_R} \right) $ & \ldots & Guoy phase 
	\end{tabular}
\end{flushleft}
\end{table}
\endgroup

\begin{flalign*}
& w(z) = w_0 \sqrt{1 + \left(\frac{z}{z_R} \right)^2} \dots \mathrm{evolving \: beam \: width} & \\
& z_r = \frac{\pi w_0^2}{\lambda} \dots \mathrm{Rayleigh \: range} & \\
& R(z) = z\left(1 + \left(\frac{z_R}{z} \right)^2 \right) \dots \mathrm{evolving \: radius \: of \: curvature} & \\
& \phi(z) = \tan^{-1}\left(\frac{z}{z_R} \right) \dots \mathrm{Guoy \: phase} & \\
& \theta = \tan^{-1}\left(\frac{w(z)}{z} \right) \simeq \frac{\lambda}{\pi w_0} \dots \mathrm{beam \: divergence \: angle} &
\end{flalign*}

\noindent
Features:
\begin{itemize}
	\item 2D version of algorithm, Ey, Bx, Bz omitted (identically equal to 0) 
	\item Code written in C++, object oriented to be easily extended to 3D, compiled to static library
	\item Linked into EPOCH as a static library (in order not to disturb the code, for this reason also added support for CMake – machine independent)
	\item Parallelized using hybrid techniques (OpenMP + MPI – computation time in most cases negligible in comparison with the main simulation)
	\item Fourier transforms can be computed using Intel MKL library, FFTW library or without any library (compile time option)
	\item Computed fields dumped into shared files using binary coding (speed up output, save disk storage)
	\item Only transverse component of electric field (Ex) passed to the EPOCH at each time step (no significant slowdown or memory overhead), other fields computed by EPOCH
	\item All new parameters needed for tight-focusing (w0, focal length, etc.) may be specified via input file
	\item Implementation works generally regardless the number of lasers in the simulation or boundaries that they are attached to
\end{itemize}

\noindent
Laser:
\begin{itemize}
\item wavelength: $ \lambda $ = 1.0 $ \mu m $
\item amplitude: $ \vec{E}_0 $ = 1e15 V/m
\item duration: t = 20 fs (in FWHM)
\item beam waist in focus: $ w_0 $ = 0.7 $ \mu m $
\item focus distance from boundary: $ x_\mathrm{B} - x_0 $ = 8 $ \mu m $
\item polarization: P
\item boundary: left 
\end{itemize}
Domain:
\begin{itemize}
\item x min: -8 $ \mu m $
\item x max: 8 $ \mu m $
\item y min: -8 $ \mu m $
\item y max: 8 $ \mu m $
\item $ N_x $: 1600 cells ($ \delta x $ = $ \lambda/100 $ = 10 nm)
\item $ N_y $: 1600 cells ($ \delta y $ = $ \lambda/100 $ = 10 nm)
\item time step: $ \delta t $ = $ 1/(\sqrt{2} c) \lambda /100 \approx $ 0.05 fs 
\item simulation time: $ \tau $ = 150 fs
\end{itemize}

\newpage
\noindent
The phase space distribution function  $ f_{s} \left(\vec{x}, \vec{v}, t\right) $ for a given species $ s $ is governed by the Vlasov equation:
\begin{equation*}
\diffp[]{f_{s}}{t} + \vec{v} \cdot \nabla f_s + \frac{q_{s}}{m_{s}}\left( \vec{E} + \vec{v} \times \vec{B} \right) \cdot \diffp[]{f_s}{\vec{v}} = 0.
\end{equation*}
The distribution function is approximated using finite-size quasi-particles with so-called shape functions $ S_{x} $ and $ S_{v} $, $ N_p $ is the number of physical particles: 
\begin{equation*}
f_{s} \left(\vec{x}, \vec{v}, t \right) =  \sum_{p} f_{p}\left(\vec{x}, \vec{v}, t \right), \quad f_{p}\left(\vec{x}, \vec{v}, t \right) = N_{p} S_{x}\left(\vec{x} - \vec{x}_{p}\left(t\right) \right)  S_{v}\left(\vec{v} - \vec{v}_{p}\left( t\right) \right).
\end{equation*}
Electromagnetic fields self-consistently evolved by Maxwell equations:
\begin{equation*}
\nabla \cdot \vec{E} = \frac{\rho}{\varepsilon_{0}}, \qquad \nabla \cdot \vec{B} = 0
\end{equation*}
\begin{equation*}
\nabla \times \vec{E} = - \diffp{\vec{B}}{t}, \qquad \nabla \times \vec{B} = \mu_{0} \vec{J} + \frac{1}{c^{2}} \diffp{\vec{E}}{t},
\end{equation*}
where charge density and current density are obtained from distribution functions:
\begin{equation*}
\rho\left(\vec{x}, t \right) = \sum_s q_s \int f_s \left(\vec{x}, \vec{v}, t \right) \mathrm{d} \vec{v}, \qquad \vec{J}\left(\vec{x}, t \right) = \sum_s q_s \int f_s \left(\vec{x}, \vec{v}, t \right) \vec{v} \, \mathrm{d} \vec{v}.
\end{equation*}



\begin{lstlisting}[style=CXX, caption=Function performing forward fast Fourier transform using MKL library]
std::vector<std::complex<double>> fft::mkl_fft_forward(std::vector<std::complex<double>> in) {
DFTI_DESCRIPTOR_HANDLE desc;
MKL_LONG status;
DftiCreateDescriptor(&desc, DFTI_DOUBLE, DFTI_COMPLEX, 1, static_cast<MKL_LONG>(in.size()));
DftiCommitDescriptor(desc);
status = DftiComputeForward(desc, in.data());
if(status != 0) {
std::cerr << DftiErrorMessage(status) << std::endl;
abort();
}
DftiFreeDescriptor(&desc);
return in;
}
\end{lstlisting}

\begin{lstlisting}[style=CXX, caption=Function performing backward fast Fourier transform using MKL library]
std::vector<std::complex<double>> fft::mkl_fft_backward(std::vector<std::complex<double>> in) {
DFTI_DESCRIPTOR_HANDLE desc;
MKL_LONG status;
DftiCreateDescriptor(&desc, DFTI_DOUBLE, DFTI_COMPLEX, 1, static_cast<MKL_LONG>(in.size()));
DftiCommitDescriptor(desc);
status = DftiComputeBackward(desc, in.data());
if(status != 0) {
std::cerr << DftiErrorMessage(status) << std::endl;
abort();
}
DftiFreeDescriptor(&desc);
return in;
}
\end{lstlisting}

\begin{lstlisting}[style=CXX, caption=Function performing forward fast Fourier transform using FFTW library]
std::vector<std::complex<double>> fft::fftw_fft_forward(std::vector<std::complex<double>> in) {
fftw_plan p = fftw_plan_dft_1d(in.size(), reinterpret_cast<fftw_complex*>(in.data()), reinterpret_cast<fftw_complex*>(in.data()), FFTW_FORWARD, FFTW_ESTIMATE);
fftw_execute(p);
fftw_destroy_plan(p);
return in;
}
\end{lstlisting}

\begin{lstlisting}[style=CXX, caption=Function performing backward fast Fourier transform using FFTW library]
std::vector<std::complex<double>> fft::fftw_fft_backward(std::vector<std::complex<double>> in) {
fftw_plan p = fftw_plan_dft_1d(in.size(), reinterpret_cast<fftw_complex*>(in.data()), reinterpret_cast<fftw_complex*>(in.data()), FFTW_BACKWARD, FFTW_ESTIMATE);
fftw_execute(p);
fftw_destroy_plan(p);
return in;
}
\end{lstlisting}

\begin{lstlisting}[style=CXX, caption=Function performing forward discrete Fourier transform without using any library]
std::vector<std::complex<double>> fft::fft_forward(std::vector<std::complex<double>> in) {
std::vector<std::complex<double>> out(in.size());
for(auto j = 0; j < out.size(); j++) {
for(auto l = 0; l < out.size(); l++) {
out.at(j) += in.at(l) * exp(-2.0 * constants::pi * I * l * j / in.size());
}
}
return out;
}
\end{lstlisting}

\begin{lstlisting}[style=CXX, caption=Function performing backward discrete Fourier transform without using any library]
std::vector<std::complex<double>> fft::fft_backward(std::vector<std::complex<double>> in) {
std::vector<std::complex<double>> out(in.size());
for(auto j = 0; j < out.size(); j++) {
for(auto l = 0; l < out.size(); l++) {
out.at(j) += in.at(l) * exp(+2.0 * constants::pi * I * l * j / in.size());
}
}
return out;
}
\end{lstlisting}

\begin{lstlisting}[style=CXX, caption=Method for performing discrete Fourier transform in time]
void laser_bcs::dft_time(field_2d<std::complex<double>>& field) const {
#ifdef OPENMP
#pragma omp parallel for schedule(static)
#endif
for(auto j = 0; j < this->domain->Nx; j++) {
#ifdef USE_MKL
field.add_col(fft::mkl_fft_backward(field.get_col(j)), j);
#elif USE_FFTW
field.add_col(fft::fftw_fft_backward(field.get_col(j)), j);
#else
field.add_col(fft::fft_backward(field.get_col(j)), j);
#endif
}
field.multiply(this->domain->dt / (2.0 * constants::pi));
return;
}
\end{lstlisting}

\begin{lstlisting}[style=CXX, caption=Method for performing inverse discrete Fourier transform in time]
void laser_bcs::idft_time(field_2d<std::complex<double>>& field) const {
#ifdef OPENMP
#pragma omp parallel for schedule(static)
#endif
for(auto j = 0; j < this->domain->Nx; j++) {
#ifdef USE_MKL
field.add_col(fft::mkl_fft_forward(field.get_col(j)), j);
#elif USE_FFTW
field.add_col(fft::fftw_fft_forward(field.get_col(j)), j);
#else
field.add_col(fft::fft_forward(field.get_col(j)), j);
#endif
}
field.multiply(2.0 * (2.0 * constants::pi) / (this->domain->Nt * this->domain->dt));
return;
}
\end{lstlisting}

\begin{lstlisting}[style=CXX, caption=Method for performing discrete Fourier transform in space]
void laser_bcs::dft_space(field_2d<std::complex<double>>& field) const {
std::vector<std::complex<double>> row_global(this->domain->Nx_global);
std::vector<std::complex<double>> row_local;
for(auto j = 0; j < this->domain->Nt; j++) {
row_local = field.get_row(j);
MPI_Gatherv(row_local.data(), this->domain->Nx, MPI_CXX_DOUBLE_COMPLEX, row_global.data(), this->domain->counts.data(), this->domain->displs.data(), MPI_CXX_DOUBLE_COMPLEX, 0, MPI_COMM_WORLD);
if(this->domain->rank == 0) {
#ifdef USE_MKL
row_global = fft::mkl_fft_forward(row_global);
#elif USE_FFTW
row_global = fft::fftw_fft_forward(row_global);
#else
row_global = fft::fft_forward(row_global);
#endif
}
MPI_Scatterv(row_global.data(), this->domain->counts.data(), this->domain->displs.data(), 	MPI_CXX_DOUBLE_COMPLEX, row_local.data(), this->domain->Nx, MPI_CXX_DOUBLE_COMPLEX, 0, MPI_COMM_WORLD);
field.add_row(row_local, j);
}
field.multiply(this->domain->dx / (2.0 * constants::pi));
return;
}
\end{lstlisting}

\begin{lstlisting}[style=CXX, caption=Method for performing inverse discrete Fourier transform in space]
void laser_bcs::idft_space(field_2d<std::complex<double>>& field) const {
std::vector<std::complex<double>> row_global(this->domain->Nx_global);
std::vector<std::complex<double>> row_local;
for(auto j = 0; j < this->domain->Nt; j++) {
row_local = field.get_row(j);
MPI_Gatherv(row_local.data(), this->domain->Nx, MPI_CXX_DOUBLE_COMPLEX, row_global.data(), this->domain->counts.data(), this->domain->displs.data(), MPI_CXX_DOUBLE_COMPLEX, 0, MPI_COMM_WORLD);
if(this->domain->rank == 0) {
#ifdef USE_MKL
row_global = fft::mkl_fft_backward(row_global);
#elif USE_FFTW
row_global = fft::fftw_fft_backward(row_global);
#else
row_global = fft::fft_backward(row_global);
#endif
}
MPI_Scatterv(row_global.data(), this->domain->counts.data(), this->domain->displs.data(), MPI_CXX_DOUBLE_COMPLEX, row_local.data(), this->domain->Nx, MPI_CXX_DOUBLE_COMPLEX, 0, MPI_COMM_WORLD);
field.add_row(row_local, j);
}
field.multiply((2.0 * constants::pi) / (this->domain->Nx_global * this->domain->dx));
return;
}
\end{lstlisting}

\begin{lstlisting}[style=CXX, caption=Method for dumping data into shared file]
template <typename T>
void field_2d<T>::dump_to_shared_file(std::string name, int row_first, int row_last, int row_size_local, int row_size_global, int col_start) const {
MPI_File file;
MPI_Offset offset = 0;
MPI_Status status;
MPI_Datatype local_array;
int col_size = row_last - row_first;
const int ndims = 2;
std::array<int, ndims> size_global = {col_size, row_size_global};
std::array<int, ndims> size_local = {col_size, row_size_local};
std::array<int, ndims> start_coords = {0, col_start};
MPI_Type_create_subarray(2, size_global.data(), size_local.data(), start_coords.data(), MPI_ORDER_C, MPI_DOUBLE, &local_array);
MPI_Type_commit(&local_array);
std::vector<double> real_part(col_size * row_size_local);
for(auto i = std::make_pair(row_first, 0); i.first < row_last; i.first++, i.second++) {
for(auto j = 0; j < row_size_local; j++) {
real_part[i.second * row_size_local + j] = std::real(this->data[i.first * row_size_local + j]);
}
}
MPI_File_open(MPI_COMM_WORLD, name.data(), MPI_MODE_CREATE|MPI_MODE_WRONLY, MPI_INFO_NULL, &file);
MPI_File_set_view(file, offset, MPI_DOUBLE, local_array, "native", MPI_INFO_NULL);
MPI_File_write_all(file, real_part.data(), col_size * row_size_local, MPI_DOUBLE, &status);
MPI_File_close(&file);
MPI_Type_free(&local_array);
return;
}
\end{lstlisting}

\begin{lstlisting}[style=CXX, caption=Extern C++ function to fill Fortran arrays with laser fields dumped in binary file]
void populate_laser_field_on_boundary(double* field, int* id, const char* data_dir, const char* name, int* timestep, int* size_global, int* first, int* last) {
double num = 0.0;
std::string laser_id = std::to_string(*id);
std::string output_path(data_dir);
std::string filename(name);
std::ifstream in;
in.open(output_path + "/" + filename + laser_id + ".dat", std::ios::binary);
if(in.is_open()) {
in.seekg(((*timestep) * (*size_global) + (*first) - 1) * sizeof(num));
for(auto i = 0; i < *last - *first + 1; i++) {
in.read(reinterpret_cast<char*>(&num), sizeof(num));
field[i] = num;
}
in.close();
} else {
std::cout << "error: cannot read file " << output_path + "/" + filename + laser_id + ".dat" << std::endl;
}
return;
}
\end{lstlisting}

\begin{lstlisting}[style=FORTRAN, caption=Fortran interfaces for C++ library functions]
INTERFACE

SUBROUTINE compute_laser_fields_on_boundary(rank, nproc, laser_start, laser_end, fwhm_time, t_0, omega, pos, amp, w_0, id, L_min, L_max, L_focus, T_min, T_max, T_ncells, cpml_thickness, t_end, T_cell_size, L_cell_size, dt, output_path) bind(c)
USE, INTRINSIC :: iso_c_binding
IMPLICIT NONE
INTEGER(c_int), INTENT(IN) :: rank, nproc, id, T_ncells, cpml_thickness
CHARACTER(kind=c_char), DIMENSION(*), INTENT(IN) :: output_path
REAL(c_double), INTENT(IN) :: laser_start, laser_end, fwhm_time, t_0, omega, pos,    &
amp, w_0, L_min, L_max, L_focus, T_min, T_max, t_end, T_cell_size, L_cell_size, dt
END SUBROUTINE compute_laser_fields_on_boundary

SUBROUTINE populate_laser_field_on_boundary(field, laser_id, output_path, field_name, timestep, size_global, first, last) bind(c)
USE, INTRINSIC :: iso_c_binding
IMPLICIT NONE
INTEGER(c_int), INTENT(IN) :: laser_id, timestep, size_global, first, last
CHARACTER(kind=c_char), DIMENSION(*), INTENT(IN) :: output_path, field_name
REAL(c_double), DIMENSION(*), INTENT(OUT) :: field
END SUBROUTINE populate_laser_field_on_boundary

END INTERFACE
\end{lstlisting}

\begin{lstlisting}[style=FORTRAN, caption=Fortran subroutines for Maxwell consistent computation of laser fields on boundaries]
SUBROUTINE Maxwell_consistent_computation_of_EM_fields

TYPE(laser_block), POINTER :: current

current => laser_x_min
DO WHILE(ASSOCIATED(current))
CALL compute_laser_fields_on_boundary(rank, nproc, current%t_start, current%t_end, current%fwhm_time, current%t_0, current%omega, current%pos, current%amp, current%w_0, current%id, x_min, x_max, current%focus, y_min, y_max, ny_global, cpml_thickness, t_end, dy, dx, dt, TRIM(data_dir)//C_NULL_CHAR)
current => current%next
ENDDO

current => laser_x_max
DO WHILE(ASSOCIATED(current))
CALL compute_laser_fields_on_boundary(rank, nproc, current%t_start, current%t_end, current%fwhm_time, current%t_0, current%omega, current%pos, current%amp, current%w_0, current%id, x_min, x_max, current%focus, y_min, y_max, ny_global, cpml_thickness, t_end, dy, dx, dt, TRIM(data_dir)//C_NULL_CHAR)
current => current%next
ENDDO

current => laser_y_min
DO WHILE(ASSOCIATED(current))
CALL compute_laser_fields_on_boundary(rank, nproc, current%t_start, current%t_end, current%fwhm_time, current%t_0, current%omega, current%pos, current%amp, current%w_0, current%id, y_min, y_max, current%focus, x_min, x_max, nx_global, cpml_thickness, t_end, dx, dy, dt, TRIM(data_dir)//C_NULL_CHAR)
current => current%next
ENDDO

current => laser_y_max
DO WHILE(ASSOCIATED(current))
CALL compute_laser_fields_on_boundary(rank, nproc, current%t_start, current%t_end, current%fwhm_time, current%t_0, current%omega, current%pos, current%amp, current%w_0, current%id, y_min, y_max, current%focus, x_min, x_max, nx_global, cpml_thickness, t_end, dx, dy, dt, TRIM(data_dir)//C_NULL_CHAR)
current => current%next
ENDDO

END SUBROUTINE Maxwell_consistent_computation_of_EM_fields
\end{lstlisting}

\begin{lstlisting}[style=FORTRAN, caption=Fortran subroutines for populating laser sources on boundaries]
SUBROUTINE get_source_x_boundary(source1, source2, laser_id) 
REAL(num), DIMENSION(:), INTENT(INOUT) :: source1, source2
REAL(num), DIMENSION(ny) :: laser_ex, laser_ey
INTEGER, INTENT(IN) :: laser_id
INTEGER :: i
CALL populate_laser_field_on_boundary(laser_ex, laser_id, TRIM(data_dir)//C_NULL_CHAR, "e_x"//C_NULL_CHAR, step, ny_global, ny_global_min, ny_global_max)
laser_ey = 0.0_num
DO i = 1, ny
source1(i) = source1(i) + laser_ex(i)
source2(i) = source2(i) + laser_ey(i)
ENDDO
END SUBROUTINE get_source_x_boundary

SUBROUTINE get_source_y_boundary(source1, source2, laser_id)
REAL(num), DIMENSION(:), INTENT(INOUT) :: source1, source2
REAL(num), DIMENSION(nx) :: laser_ex, laser_ey
INTEGER, INTENT(IN) :: laser_id
INTEGER :: i
CALL populate_laser_field_on_boundary(laser_ex, laser_id, TRIM(data_dir)//C_NULL_CHAR, "e_x"//C_NULL_CHAR, step, nx_global, nx_global_min, nx_global_max)
laser_ey = 0.0_num
DO i = 1, nx
source1(i) = source1(i) + laser_ey(i)
source2(i) = source2(i) + laser_ex(i)
ENDDO
END SUBROUTINE get_source_y_boundary
\end{lstlisting}