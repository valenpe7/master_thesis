intro...
\section{Maxwell's equations}
The electromagnetic field is in general theory represented by two vectors, the intensity of the electric field $ \vec{E}\left( \vec{r}, t \right) $ and the magnetic induction $ \vec{B}\left( \vec{r}, t \right) $. These vectors are considered to be finite and continuous functions of position $ \vec{r} $ and time $ t $. The description of electromagnetic phenomena in classical electrodynamics is provided by the set of well-known Maxwell's equations. The microscopic variant for external sources in vacuum is formulated as follows,
\begin{equation}
\label{1.1}
\div{\vec{E}} = \frac{\rho}{\varepsilon_0},
\end{equation}
\begin{equation}
\label{1.2}
\div{\vec{B}} = 0,
\end{equation}
\begin{equation}
\label{1.3}
\rot{\vec{E}} + \diffp{\vec{B}}{t} = 0,
\end{equation}
\begin{equation}
\label{1.4}
\rot{\vec{B}} - \mu_0 \varepsilon_0 \diffp{\vec{E}}{t}= \mu_0 \vec{J},
\end{equation}
where $ \rho\left( \vec{r}, t \right) $ is total electric charge density and $ \vec{J}\left( \vec{r}, t \right) $ is total electric current density, which is constituted by the motion of a charge. These distributions may be continuous as well as discrete. As might be seen from Maxwell's equations (\ref{1.1} - \ref{1.4}), the charge density is the source of the electric field, whilst the magnetic field is produced by the current density. The lack of symmetry in Maxwell's equations (\ref{1.2}, \ref{1.3} are homogeneous) is caused by the experimental absence of magnetic charges and currents. The universal constants appearing in the Maxwell's equations (\ref{1.1}, \ref{1.4}) are the electric permittivity of vacuum $ \varepsilon_0 $ and the magnetic permeability of vacuum $ \mu_0 $.

The first equation, \ref{1.1}, is Gauss's law for electric field in the differential form. It states that the flux of electric field through any closed surface is proportional to the total charge inside. The second equation, \ref{1.2}, is Gauss's law for magnetic field. It expresses the fact that there are no magnetic monopoles, so the flux of magnetic field through any closed surface is always zero. The third equation, \ref{1.3}, is Faraday's law describing how the electric field is associated with a time varying magnetic field. And the last equation, \ref{1.4}, is Amp\`ere's law with Maxwell's displacement current, which means that the time varying electric field causes the magnetic field. As a consequence, it predicts the existence of electromagnetic waves that can carry energy and momentum even in a free space.

To describe the effects of an electromagnetic field in the presence of macroscopic substances, the complicated distribution of charges and currents in matter over the atomic scale is not relevant. Thus one shall define a second set of auxiliary vectors that represent fields in which the material properties are already included in an average sense, the electric displacement $ \vec{D}\left( \vec{r}, t \right) $ and the magnetic vector $ \vec{H}\left( \vec{r}, t \right) $,
\begin{equation}
\label{1.5}
\vec{D} = \varepsilon_0 \vec{E} + \vec{P} = \varepsilon \vec{E},
\end{equation}
\begin{equation}
\label{1.6}
\vec{H} = \frac{\vec{B}}{\mu_0} - \vec{M} = \frac{\vec{B}}{\mu},
\end{equation}
where $ \vec{P}\left( \vec{r}, t \right) $ and $ \vec{M}\left( \vec{r}, t \right) $ are the vectors of polarization and magnetization, respectively. Note that the vectors of polarization and magnetization can be interpreted as electric or magnetic dipole moment per unit volume of the medium, therefore they are definitely associated with the state of a matter and vanish in vacuum. Similarly as in the case of a free space, the factors $ \varepsilon $ and $ \mu $ are called electric permittivity of medium and magnetic permeability of medium. In general case, $ \varepsilon $ and $ \mu $ are tensors. The constitutive relations above (\ref{1.5}, \ref{1.6}) hold only if the relations between derived and primary fields are linear.

The macroscopic variant of Maxwell's equations is formulated as follows,
\begin{equation}
\label{1.7}
\div{\vec{D}} = \rho,
\end{equation}
\begin{equation}
\label{1.8}
\div{\vec{B}} = 0,
\end{equation}
\begin{equation}
\label{1.9}
\rot{\vec{E}} + \diffp{\vec{B}}{t} = 0,
\end{equation}
\begin{equation}
\label{1.10}
\rot{\vec{H}} - \diffp{\vec{D}}{t} = \vec{J},
\end{equation}
where $ \rho\left(\vec{r}, t \right) $ and $ \vec{J}\left(\vec{r}, t \right) $ now stand for only external electric charge and current density, respectively.

By combining the time derivative of the equation \ref{1.7} with the divergence of the equation \ref{1.10}, one obtains the following relation between the electromagnetic field sources,
\begin{equation}
\label{1.11}
\div{\vec{J}} + \diffp{\rho}{t} = 0.
\end{equation}
The important result \ref{1.11}, which is frequently referred to as the equation of continuity, expresses nothing but the conservation of total electric charge in an isolated system. In other words, the time rate of change of the electric charge in any closed surface is balanced by the electric current flowing through the surface.

\section{Electrodynamic potentials}
The first-order partial differential Maxwell's equations can be effectively converted to a smaller number of second-order equations by introducing electrodynamic potentials. Hence, one can express the electric and magnetic field as follows,
\begin{equation}
\label{1.12}
\vec{E} = -\grad{\Phi} - \diffp{\vec{A}}{t},
\end{equation}
\begin{equation}
\label{1.13}
\vec{B} = \rot{\vec{A}},
\end{equation}
where $ \Phi\left(\vec{r}, t \right) $ is the scalar potential and $ \vec{A}\left(\vec{r}, t \right) $ is the vector potential of the corresponding fields. One can clearly see that using the definitions \ref{1.12}, \ref{1.13}, six vector components are replaced by only four potential functions and two Maxwell's homogeneous equations (\ref{1.8}, \ref{1.9}) are fulfilled identically. 

However, by definitions \ref{1.12}, \ref{1.13}, $ \Phi\left(\vec{r}, t \right) $ and $ \vec{A}\left(\vec{r}, t \right) $ are not defined uniquely, thus an infinite number of potentials which lead to the same fields may be constructed. To avoid that, one has to impose an supplementary condition, for example
\begin{equation}
\label{1.14}
\div{\vec{A}} + \mu \varepsilon \diffp{\Phi}{t} = 0.
\end{equation}
The condition \ref{1.14} is called the Lorenz gauge. Lorenz gauge is commonly used in electromagnetism because its independence of the coordinate system. Furthermore, it leads to the following uncoupled equations,
\begin{equation}
\label{1.15}
\laplace{\Phi} - \mu \varepsilon \diffp[2]{\Phi}{t} = -\frac{\rho}{\varepsilon},
\end{equation}
\begin{equation}
\label{1.16}
\laplace{\vec{A}} - \mu \varepsilon \diffp[2]{\vec{A}}{t} = -\mu \vec{J},
\end{equation}
that are in all respects equivalent to the Maxwell's equations and in many situations much simpler to solve.

Equations \ref{1.15}, \ref{1.16} correspond to the inhomogeneous wave equations for scalar potential $ \Phi\left(\vec{r}, t \right) $ and vector potential $ \vec{A}\left(\vec{r}, t \right) $. Their general solutions are given by the following expressions,
\begin{equation}
\label{1.17}
\Phi\left(\vec{r}, t \right) = \frac{1}{4 \pi \varepsilon} \int \frac{\rho\left(\vec{r^{\: \prime}}, t^{\: \prime} \right)}{\norm{\vec{r} - \vec{r^{\: \prime}}}} \mathrm{d} V,
\end{equation}
\begin{equation}
\label{1.18}
\vec{A}\left(\vec{r}, t \right) = \frac{\mu}{4 \pi} \int \frac{\vec{J}\left(\vec{r^{\: \prime}}, t^{\: \prime} \right)}{\norm{\vec{r} - \vec{r^{\: \prime}}}} \mathrm{d} V,
\end{equation}
where $ \mathrm{d} V $ is a volume element and $ \norm{.} $ stands for the standard Euclidean norm. Note that the solutions \ref{1.17}, \ref{1.18} are dependent only on charge and current densities at position $ \vec{r^{\: \prime}} $ at so-called retarded time $ t^{\: \prime} = t - \sqrt{\mu \epsilon} \norm{\vec{r} - \vec{r^{\: \prime}}} $ which takes into account the finite velocity of the wave. In other words, the fields at the observation point $ \vec{r} $ at the time $ t $ are proportional to the sum of all the electromagnetic waves that leave the source elements at point $ \vec{r^{\: \prime}} $ at the retarded time $ t^{\: \prime} $.

\section{Hertz vectors}
There exists also other possibilities how to express the electromagnetic field. Under ordinary conditions, an arbitrary electromagnetic field may be defined in terms of a single vector function. This may be helpful for solving of many problems of classical electromagnetic theory, particularly the wave propagation.

First, let us introduce the electric Hertz vector $ {\vec{\Pi_e}}\left(\vec{r}, t \right) $ in terms of the scalar and vector potentials,
\begin{equation}
\label{1.19}
\Phi = - \div{\vec{\Pi_e}},
\end{equation}
\begin{equation}
\label{1.20}
\vec{A} = \mu \varepsilon \diffp{\vec{\Pi_e}}{t}.
\end{equation}

Note that the definitions \ref{1.19}, \ref{1.20} are consistent with the Lorenz gauge condition \ref{1.14}. In the absence of magnetization, it might be easily shown that the electric Hertz vector $ {\vec{\Pi_e}}\left(\vec{r}, t \right) $ is governed by inhomogeneous wave equation
\begin{equation}
\label{1.21}
\laplace{\vec{\Pi_e}} - \mu \varepsilon \diffp[2]{\vec{\Pi_e}}{t} = -\frac{\vec{P}}{\varepsilon}.
\end{equation}

The equation \ref{1.21} is of the same type as the equations \ref{1.15}, \ref{1.16} and has therefore the familiar general solution 
\begin{equation}
\label{1.22}
\vec{\Pi_e}\left(\vec{r}, t \right) = \frac{1}{4 \pi \varepsilon} \int \frac{\vec{P}\left(\vec{r^{\: \prime}}, t^{\: \prime} \right)}{\norm{\vec{r} - \vec{r^{\: \prime}}}} \mathrm{d} V.
\end{equation}

As might be seen form \ref{1.22}, the fields derived from the electric Hertz vector $ {\vec{\Pi_e}}\left(\vec{r}, t \right) $ can be interpreted as being due to a density distribution of electric dipoles. Every solution of \ref{1.22} then uniquely determines the electromagnetic field through
\begin{equation}
\label{1.23}
\vec{E} = \grad{\left(\div{\vec{\Pi_e}}\right)} - \mu \epsilon \diffp[2]{\vec{\Pi_e}}{t},
\end{equation}
\begin{equation}
\label{1.24}
\vec{B} = \mu \varepsilon \left(\rot{\diffp{\vec{\Pi_e}}{t}}\right).
\end{equation}

Second, one may introduce the magnetic Hertz vector $ {\vec{\Pi_m}}\left(\vec{r}, t \right) $ in terms of the scalar and vector potentials by the following expressions,
\begin{equation}
\label{1.25}
\Phi = 0,
\end{equation}
\begin{equation}
\label{1.26}
\vec{A} = \rot{\vec{\Pi_m}}.
\end{equation}
In the absence of polarization, the magnetic Hertz vector $ {\vec{\Pi_m}}\left(\vec{r}, t \right) $ defined by \ref{1.25} and \ref{1.26} fulfills an inhomogeneous wave equation
\begin{equation}
\label{1.27}
\laplace{\vec{\Pi_m}} - \mu \varepsilon \diffp[2]{\vec{\Pi_m}}{t} = -\mu \vec{M}.
\end{equation}
As for the previous cases, one may find the solution of \ref{1.27},
\begin{equation}
\label{1.28}
\vec{\Pi_m}\left(\vec{r}, t \right) = \frac{\mu}{4 \pi} \int \frac{\vec{M}\left(\vec{r^{\: \prime}}, t^{\: \prime} \right)}{\norm{\vec{r} - \vec{r^{\: \prime}}}} \mathrm{d} V,
\end{equation}
thus the fields derived from the magnetic Hertz vector $ {\vec{\Pi_m}}\left(\vec{r}, t \right) $ are now due to a density distribution of magnetic dipoles. Again, every solution of \ref{1.28} uniquely determines the electromagnetic field via
\begin{equation}
\label{1.29}
\vec{E} = \rot{\diffp{\vec{\Pi_m}}{t}},
\end{equation}
\begin{equation}
\label{1.30}
\vec{B} = \rot{\left(\rot{\vec{\Pi_m}}\right)}.
\end{equation}

The above derivations considered electric and magnetic Hertz vectors as a separate quantities. It is also possible, however, to introduce them together in the form of one six-vector [source].

\section{Energy and momentum}
To be able to describe the interaction of electromagnetic field with matter, one has to know the energy distribution throughout the field as well as the momentum balance.

By scalar multiplications of \ref{1.9} by $ \vec{H}\left( \vec{r}, t \right) $, of \ref{1.10} by $ \vec{E}\left( \vec{r}, t \right) $, following subtraction of both obtained equations and using standard vector identities, one gets the expression 
\begin{equation}
\label{1.31}
\vec{E} \cdot \diffp{\vec{D}}{t} + \vec{H} \cdot \diffp{\vec{B}}{t} + \div{\left(\vec{E} \times \vec{H} \right)} = -\vec{E} \cdot \vec{J}.
\end{equation}
The equation \ref{1.31} can be rewritten in the form of conservation law,
\begin{equation}
\label{1.32}
\diffp{u}{t} + \div{\vec{S}} = - \vec{E} \cdot \vec{J},
\end{equation}
where
\begin{equation}
\label{1.33}
u = \frac{1}{2} \left(\vec{E} \cdot \vec{D} + \vec{H} \cdot \vec{B} \right), \quad \vec{S} = \vec{E} \times \vec{H}.
\end{equation}
The quantity $ u\left( \vec{r}, t \right) $ in \ref{1.33} describes the total energy density in the field and $ \vec{S}\left( \vec{r}, t \right) $ is so-called Poynting vector which represents the energy flow of the field.

The important statement \ref{1.32}, also referred to as the Poynting theorem, expresses the conservation of energy for the electromagnetic field. In other words, the time rate of change of the field energy within a certain region and the energy flowing out of that region is balanced by the conversion of the electromagnetic energy into mechanical or heat energy and vice-versa.

Lorentz force:
\begin{equation}
\vec{F} = q \left(E + v \times B \right) 
\end{equation}
Ohm's law:
\begin{equation}
\vec{J} = \sigma \vec{E}
\end{equation}

\section{Electromagnetic waves and Gaussian beam}
wave equations:
\begin{equation}
\laplace{\vec{E}} - \frac{1}{c^{2}} \diffp[2]{\vec{E}}{t} = 0
\end{equation}
\begin{equation}
\laplace{\vec{B}} - \frac{1}{c^{2}} \diffp[2]{\vec{B}}{t} = 0
\end{equation}
Laser beams are represented by Gaussian beams.
The simplest mathematical description of the essential features of a focused optical beam is provided by Gaussian beams.
All components of the electric and magnetic fields can be deduced from a single scalar wave function.
In general, the forms of laser beams can be usefully deduced from a vector potential that has a single Cartesian coordinate.
vector potential:
\begin{equation}
\vec{E}\left(\vec{r_\bot}, z, t \right)  = E_0 \Psi \left(\vec{r_\bot}, z \right) \e^{i \left(k_z z - \omega t \right)} \mathrm{\hat{e}_x}
\end{equation}
angular frequency $ \omega $, wave number $ k_z = \omega/c $, 
Helmholtz equation:
\begin{equation}
\laplace{\Psi \left(\vec{r_\bot}, z \right)} + 2 \i k_z \diffp{\Psi \left(\vec{r_\bot}, z \right)}{z} = 0
\end{equation}
parameters (w0 radius of the beam waist):
\begin{equation}
w_0, \quad z_r = \frac{k_z w_0^2}{2} = \frac{\pi w_0^2}{\lambda}, \quad \Theta = \frac{w_0}{z_R} = \frac{\lambda}{\pi w_0}
\end{equation}
dimensionless coordinates:
\begin{equation}
\rho = \frac{\norm{\vec{r_\bot}}}{w_0}, \quad \zeta = \frac{z}{z_r}
\end{equation}
after transformation:
\begin{equation}
\frac{1}{\rho} \diffp{}{\rho}\left(\rho \diffp{\Psi \left(\rho, \zeta \right)}{\rho} \right) + 4 \i \diffp{\Psi \left(\rho, \zeta \right)}{\zeta}  = - \Theta^2 \diffp[2]{\Psi \left(\rho, \zeta \right)}{\zeta}
\end{equation}
expansion (assume $ \Theta \ll 1 $ and consequently $  w_0 \gg \lambda $):
\begin{equation}
\Psi = \sum_{n = 0}^{+\infty} \Theta^{2n} \Psi_{2n}
\end{equation}
zero-th order with transverse laplacian in cylindrical coords:
\begin{equation}
\frac{1}{\rho} \diffp{}{\rho}\left(\rho \diffp{\Psi_0 \left(\rho, \zeta \right)}{\rho} \right) + 4 \i \diffp{\Psi_0\left(\rho, \zeta \right)}{\zeta} = 0
\end{equation}
solution in dimensionless coordinates:
\begin{equation}
\Psi_0 \left(\rho, \zeta \right) = \frac{1}{\sqrt{1 + \zeta^2}} \exp{\left[- \frac{\rho^2}{1 + \zeta^2} + \i \left(\frac{\rho^2 \zeta}{1 + \zeta^2} - \arctan{\zeta} \right) \right]} 
\end{equation}
in Cartesian coordinates:
\begin{equation}
\Psi_0 \left(\vec{r_\bot}, z \right) = \frac{w_0}{w\left(z\right)} \exp{\left[- \frac{\vec{r_\bot}^2}{w\left(z \right)^2} + \i \left( k_z \frac{\vec{r_\bot}^2}{2 R\left(z \right)} - \varphi_G \left( z\right) \right) \right]}
\end{equation}
where:
\begin{equation}
w\left(z\right) = w_0 \sqrt{1 + \left(\frac{z}{z_R}\right)^2}, \quad R\left(z \right) = z \left[1 + \left(\frac{z_R}{z} \right)^2\right], \quad \varphi_G = \arctan{\left(\frac{z}{z_R}\right)}
\end{equation}
Solution - Gaussian beam:
\begin{equation}
\vec{E}\left(\vec{r_\bot}, z, t \right) = E_0 \frac{w_0}{w(z)} \exp\left(-\frac{\vec{r_\bot}^2}{w(z)^2}\right) \cos\left(\omega t - k_z \left(z + \frac{\vec{r_\bot}^2}{2 R(z)} \right) + \varphi_G(z) \right) \mathrm{\hat{e}_x}
\end{equation}