intro...
\section{Maxwell's equations}
The electromagnetic field is in general theory represented by two vectors, the intensity of the electric field $ \vec{E}\left( \vec{r}, t \right) $ and the magnetic induction $ \vec{B}\left( \vec{r}, t \right) $. These vectors are considered to be finite and continuous functions of position $ \vec{r} $ and time $ t $. The description of electromagnetic phenomena in classical electrodynamics is provided by the set of well-known Maxwell's equations. The microscopic variant for external sources in vacuum is formulated as follows,
\begin{equation}
\label{1.1}
\div{\vec{E}} = \frac{\rho}{\varepsilon_0},
\end{equation}
\begin{equation}
\label{1.2}
\div{\vec{B}} = 0,
\end{equation}
\begin{equation}
\label{1.3}
\rot{\vec{E}} + \diffp{\vec{B}}{t} = 0,
\end{equation}
\begin{equation}
\label{1.4}
\rot{\vec{B}} - \mu_0 \varepsilon_0 \diffp{\vec{E}}{t}= \mu_0 \vec{J},
\end{equation}
where $ \rho\left( \vec{r}, t \right) $ is total electric charge density and $ \vec{J}\left( \vec{r}, t \right) $ is total electric current density, which is constituted by the motion of a charge. These distributions may be continuous as well as discrete. As might be seen from Maxwell's equations (\ref{1.1} - \ref{1.4}), the charge density is the source of the electric field, whilst the magnetic field is produced by the current density. The lack of symmetry in Maxwell's equations (\ref{1.2}, \ref{1.3} are homogeneous) is caused by the experimental absence of magnetic charges and currents. The universal constants appearing in the Maxwell's equations (\ref{1.1}, \ref{1.4}) are the electric permittivity of vacuum $ \varepsilon_0 $ and the magnetic permeability of vacuum $ \mu_0 $.

The first equation, \ref{1.1}, is Gauss's law for electric field in the differential form. It states that the flux of the electric field through any closed surface is proportional to the total charge inside. The second equation, \ref{1.2}, is Gauss's law for magnetic field. It expresses the fact that there are no magnetic monopoles, so the flux of magnetic field through any closed surface is always zero. The third equation, \ref{1.3}, is Faraday's law describing how the electric field is associated with a time varying magnetic field. And the last equation, \ref{1.4}, is Amp\`ere's law with Maxwell's displacement current, which means that the time varying electric field causes the magnetic field. As a consequence, it predicts the existence of electromagnetic waves that can carry energy and momentum even in a free space.

To describe the effects of an electromagnetic field in the presence of macroscopic substances, the complicated distribution of charges and currents in matter over the atomic scale is not relevant. Thus one shall define a second set of auxiliary vectors that represent fields in which the material properties are already included in an average sense, the electric displacement $ \vec{D}\left( \vec{r}, t \right) $ and the magnetic vector $ \vec{H}\left( \vec{r}, t \right) $,
\begin{equation}
\label{1.5}
\vec{D} = \varepsilon_0 \vec{E} + \vec{P} = \varepsilon \vec{E},
\end{equation}
\begin{equation}
\label{1.6}
\vec{H} = \frac{\vec{B}}{\mu_0} - \vec{M} = \frac{\vec{B}}{\mu},
\end{equation}
where $ \vec{P}\left( \vec{r}, t \right) $ and $ \vec{M}\left( \vec{r}, t \right) $ are the vectors of polarization and magnetization, respectively. Note that the vectors of polarization and magnetization can be interpreted as a density of electric or magnetic dipole moment of the medium, therefore they are definitely associated with the state of a matter and vanish in vacuum. Similarly as in the case of free space, the factors $ \varepsilon $ and $ \mu $ are called electric permittivity of medium and magnetic permeability of medium. In general case, $ \varepsilon $ and $ \mu $ are tensors. The constitutive relations above (\ref{1.5}, \ref{1.6}) hold only if the medium is homogeneous and isotropic. For the sake of simplicity, only such materials will be considered in the following text.

The macroscopic variant of Maxwell's equations is formulated as follows,
\begin{equation}
\label{1.7}
\div{\vec{D}} = \rho,
\end{equation}
\begin{equation}
\label{1.8}
\div{\vec{B}} = 0,
\end{equation}
\begin{equation}
\label{1.9}
\rot{\vec{E}} + \diffp{\vec{B}}{t} = 0,
\end{equation}
\begin{equation}
\label{1.10}
\rot{\vec{H}} - \diffp{\vec{D}}{t} = \vec{J},
\end{equation}
where $ \rho\left(\vec{r}, t \right) $ and $ \vec{J}\left(\vec{r}, t \right) $ now stand for only external electric charge and current density, respectively.

By combining the time derivative of the equation \ref{1.7} with the divergence of the equation \ref{1.10}, one obtains the following relation between the electromagnetic field sources,
\begin{equation}
\label{1.11}
\div{\vec{J}} + \diffp{\rho}{t} = 0.
\end{equation}
The important result \ref{1.11}, which is frequently referred to as the equation of continuity, expresses nothing but the conservation of total electric charge in an isolated system. In other words, the time rate of change of the electric charge in any closed surface is balanced by the electric current flowing through the surface.

\section{Electrodynamic potentials}
The first-order partial differential Maxwell's equations can be effectively converted to a smaller number of second-order equations by introducing electrodynamic potentials. Hence, one can express the electric and magnetic field as follows,
\begin{equation}
\label{1.12}
\vec{E} = -\grad{\Phi} - \diffp{\vec{A}}{t},
\end{equation}
\begin{equation}
\label{1.13}
\vec{B} = \rot{\vec{A}},
\end{equation}
where $ \Phi\left(\vec{r}, t \right) $ is the scalar potential and $ \vec{A}\left(\vec{r}, t \right) $ is the vector potential of the corresponding fields. One can clearly see that using the definitions \ref{1.12}, \ref{1.13}, six vector components are replaced by only four potential functions and two Maxwell's homogeneous equations (\ref{1.8}, \ref{1.9}) are fulfilled identically. 

However, by definitions \ref{1.12}, \ref{1.13}, $ \Phi\left(\vec{r}, t \right) $ and $ \vec{A}\left(\vec{r}, t \right) $ are not defined uniquely, thus an infinite number of potentials which lead to the same fields may be constructed. To avoid that, one has to impose a supplementary condition, for example
\begin{equation}
\label{1.14}
\div{\vec{A}} + \mu \varepsilon \diffp{\Phi}{t} = 0.
\end{equation}
The condition \ref{1.14} is called the Lorenz gauge. Lorenz gauge is commonly used in electromagnetism because its independence of the coordinate system. Furthermore, it leads to the following uncoupled equations,
\begin{equation}
\label{1.15}
\laplace{\Phi} - \mu \varepsilon \diffp[2]{\Phi}{t} = -\frac{\rho}{\varepsilon},
\end{equation}
\begin{equation}
\label{1.16}
\laplace{\vec{A}} - \mu \varepsilon \diffp[2]{\vec{A}}{t} = -\mu \vec{J},
\end{equation}
that are in all respects equivalent to the Maxwell's equations and in many situations much simpler to solve.

Equations \ref{1.15}, \ref{1.16} correspond to the inhomogeneous wave equations for scalar potential $ \Phi\left(\vec{r}, t \right) $ and vector potential $ \vec{A}\left(\vec{r}, t \right) $. Their general solutions are given by the following expressions,
\begin{equation}
\label{1.17}
\Phi\left(\vec{r}, t \right) = \frac{1}{4 \pi \varepsilon} \int \frac{\rho\left(\vec{r^{\: \prime}}, t^{\: \prime} \right)}{\norm{\vec{r} - \vec{r^{\: \prime}}}} \mathrm{d} V,
\end{equation}
\begin{equation}
\label{1.18}
\vec{A}\left(\vec{r}, t \right) = \frac{\mu}{4 \pi} \int \frac{\vec{J}\left(\vec{r^{\: \prime}}, t^{\: \prime} \right)}{\norm{\vec{r} - \vec{r^{\: \prime}}}} \mathrm{d} V,
\end{equation}
where $ \mathrm{d} V $ is a volume element and $ \norm{.} $ stands for the standard Euclidean norm. Note that the solutions \ref{1.17}, \ref{1.18} are dependent only on charge and current densities at position $ \vec{r^{\: \prime}} $ at so-called retarded time $ t^{\: \prime} = t - \sqrt{\mu \epsilon} \norm{\vec{r} - \vec{r^{\: \prime}}} $ which takes into account the finite velocity of the wave. In other words, the fields at the observation point $ \vec{r} $ at the time $ t $ are proportional to the sum of all the electromagnetic waves that leave the source elements at point $ \vec{r^{\: \prime}} $ at the retarded time $ t^{\: \prime} $.

\section{Hertz vectors}
There exists also other possibilities how to express the electromagnetic field. Under ordinary conditions, an arbitrary electromagnetic field may be defined in terms of a single vector function. This may be helpful for solving of many problems of classical electromagnetic theory, particularly the wave propagation.

First, let us introduce the electric Hertz vector $ {\vec{\Pi_e}}\left(\vec{r}, t \right) $ in terms of the scalar and vector potentials,
\begin{equation}
\label{1.19}
\Phi = - \div{\vec{\Pi_e}},
\end{equation}
\begin{equation}
\label{1.20}
\vec{A} = \mu \varepsilon \diffp{\vec{\Pi_e}}{t}.
\end{equation}

Note that the definitions \ref{1.19}, \ref{1.20} are consistent with the Lorenz gauge condition \ref{1.14}. In the absence of magnetization, it might be easily shown that $ \vec{J} = \partial{\vec{P}}/\partial{t} $ and the electric Hertz vector $ {\vec{\Pi_e}}\left(\vec{r}, t \right) $ is governed by an inhomogeneous wave equation
\begin{equation}
\label{1.21}
\laplace{\vec{\Pi_e}} - \mu \varepsilon \diffp[2]{\vec{\Pi_e}}{t} = -\frac{\vec{P}}{\varepsilon}.
\end{equation}

The equation \ref{1.21} is of the same type as the equations \ref{1.15}, \ref{1.16} and has therefore the familiar general solution 
\begin{equation}
\label{1.22}
\vec{\Pi_e}\left(\vec{r}, t \right) = \frac{1}{4 \pi \varepsilon} \int \frac{\vec{P}\left(\vec{r^{\: \prime}}, t^{\: \prime} \right)}{\norm{\vec{r} - \vec{r^{\: \prime}}}} \mathrm{d} V.
\end{equation}

As might be seen form \ref{1.22}, the fields derived from the electric Hertz vector $ {\vec{\Pi_e}}\left(\vec{r}, t \right) $ can be interpreted as being due to a density distribution of electric dipoles. Every solution of \ref{1.22} then uniquely determines the electromagnetic field through
\begin{equation}
\label{1.23}
\vec{E} = \grad{\left(\div{\vec{\Pi_e}}\right)} - \mu \epsilon \diffp[2]{\vec{\Pi_e}}{t},
\end{equation}
\begin{equation}
\label{1.24}
\vec{B} = \mu \varepsilon \left(\rot{\diffp{\vec{\Pi_e}}{t}}\right).
\end{equation}

Second, one may introduce the magnetic Hertz vector $ {\vec{\Pi_m}}\left(\vec{r}, t \right) $ in terms of the scalar and vector potentials by the following expressions,
\begin{equation}
\label{1.25}
\Phi = 0,
\end{equation}
\begin{equation}
\label{1.26}
\vec{A} = \rot{\vec{\Pi_m}}.
\end{equation}
In the absence of polarization, $ \vec{J} = \rot{\vec{M}} $ and the magnetic Hertz vector $ {\vec{\Pi_m}}\left(\vec{r}, t \right) $ defined by \ref{1.25} and \ref{1.26} fulfills an inhomogeneous wave equation
\begin{equation}
\label{1.27}
\laplace{\vec{\Pi_m}} - \mu \varepsilon \diffp[2]{\vec{\Pi_m}}{t} = -\mu \vec{M}.
\end{equation}
As for the previous cases, one may easily find the solution of \ref{1.27},
\begin{equation}
\label{1.28}
\vec{\Pi_m}\left(\vec{r}, t \right) = \frac{\mu}{4 \pi} \int \frac{\vec{M}\left(\vec{r^{\: \prime}}, t^{\: \prime} \right)}{\norm{\vec{r} - \vec{r^{\: \prime}}}} \mathrm{d} V,
\end{equation}
thus the fields derived from the magnetic Hertz vector $ {\vec{\Pi_m}}\left(\vec{r}, t \right) $ may be imagined to be due to a density distribution of magnetic dipoles. Again, every solution of \ref{1.28} uniquely determines the electromagnetic field via
\begin{equation}
\label{1.29}
\vec{E} = \rot{\diffp{\vec{\Pi_m}}{t}},
\end{equation}
\begin{equation}
\label{1.30}
\vec{B} = \rot{\left(\rot{\vec{\Pi_m}}\right)}.
\end{equation}

Note that the above derivations considered electric and magnetic Hertz vectors as a separate quantities. It is also possible, however, to introduce them together in the form of one six-vector [source].

\section{Energy and momentum}
To be able to describe the interaction of electromagnetic field with matter, one has to know the energy distribution throughout the field as well as the momentum balance.

By scalar multiplications of \ref{1.9} by $ \vec{H}\left( \vec{r}, t \right) $, of \ref{1.10} by $ \vec{E}\left( \vec{r}, t \right) $, following subtraction of both obtained equations and using standard vector identities, one gets the expression 
\begin{equation}
\label{1.31}
\vec{E} \cdot \diffp{\vec{D}}{t} + \vec{H} \cdot \diffp{\vec{B}}{t} + \div{\left(\vec{E} \times \vec{H} \right)} = -\vec{E} \cdot \vec{J}.
\end{equation}
The equation \ref{1.31} can be rewritten in the form of conservation law,
\begin{equation}
\label{1.32}
\diffp{u}{t} + \div{\vec{S}} = - \vec{E} \cdot \vec{J},
\end{equation}
where
\begin{equation}
\label{1.33}
u = \frac{1}{2} \left(\vec{E} \cdot \vec{D} + \vec{H} \cdot \vec{B} \right), \quad \vec{S} = \vec{E} \times \vec{H}.
\end{equation}
The quantity $ u\left( \vec{r}, t \right) $ in \ref{1.33} describes the total energy density in the field and $ \vec{S}\left( \vec{r}, t \right) $ is so-called Poynting vector which represents the energy flow of the field.

The important statement \ref{1.32}, also referred to as the Poynting theorem, expresses the conservation of energy for the electromagnetic field. In other words, the time rate of change of the field energy within a certain region and the energy flowing out of that region is balanced by the conversion of the electromagnetic energy into mechanical or heat energy and vice-versa.

+momentum...

\noindent
Lorentz force:
\begin{equation}
\vec{F} = q \left(E + v \times B \right) 
\end{equation}
Ohm's law:
\begin{equation}
\vec{J} = \sigma \vec{E}
\end{equation}

\section{Electromagnetic waves and Gaussian beam}
In this section, the simplest mathematical description of a focused laser beam based on approximations to the wave equation is deduced. Since in numerical codes it is a common practice to prescribe the laser beams by their propagation in free space, the set of the microscopic Maxwell's equations \ref{1.1} - \ref{1.4} will be exploited.

In the absence of external sources, it might be easily shown that the equations \ref{1.1} - \ref{1.4} may be alternatively formulated as an uncoupled homogeneous wave equations for electric field $ \vec{E}\left( \vec{r}, t \right) $ and magnetic field $ \vec{B}\left( \vec{r}, t \right) $,
\begin{equation}
\label{1.34}
\laplace{\vec{E}} - \frac{1}{c^{2}} \diffp[2]{\vec{E}}{t} = 0,
\end{equation}
\begin{equation}
\label{1.35}
\laplace{\vec{B}} - \frac{1}{c^{2}} \diffp[2]{\vec{B}}{t} = 0,
\end{equation}
where the universal constant $ c = 1/\sqrt{\mu_0 \varepsilon_0} $ is the speed of light in vacuum, which leads to the essential fact, that the electromagnetic waves propagate in vacuum with the velocity of light $ c $.

Without any loss of generality, consider the laser beam as an electromagnetic wave propagating toward the positive direction of the z-axis with the electric field linearly polarized along the x-axis of the Cartesian coordinate system. A common way is to describe such a wave by the evolution of a single electric field component (although the more proper way would be to use the vector potential), therefore one has to look for the solution of the equation \ref{1.34}. 

According to the previous assumptions, the solution is expected to be in the form of the following plane wave,
\begin{equation}
\label{1.36}
\vec{E}\left(\vec{r_\bot}, z, t \right)  = E_0 \Psi \left(\vec{r_\bot}, z \right) \e^{\i \left(k_z z - \omega t \right)} \mathrm{\vec{\hat{e}_x}},
\end{equation}
where $ \vec{r_\bot} = (x, y)^{\mathrm{T}} $ is the vector of transverse Cartesian coordinates, $ E_0 $ is a constant amplitude, $ \Psi \left(\vec{r_\bot}, z \right) $ is the part of the wave function which is dependent only on the spatial coordinates, $ \omega $ denotes the angular frequency, $ k_z $ is the z-component of the wave vector $ \vec{k}\left(\omega \right) $ and $ \mathrm{\vec{\hat{e}_x}} $ is the unit vector pointing in the direction of the x-axis.

Direct substitution of expression \ref{1.36} into the equation \ref{1.34} yields the time-independent form of the scalar wave equation
\begin{equation}
\label{1.37}
\laplace{\Psi \left(\vec{r_\bot}, z \right)} + 2 \i k_z \diffp{\Psi \left(\vec{r_\bot}, z \right)}{z} = 0.
\end{equation}
The equation \ref{1.37} is called the Helmholtz equation. Note that it is sufficient to seek solutions to the equation \ref{1.37} since the wave \ref{1.36} is monochromatic.

It turned out, that the geometry of the focused laser beam can be expressed in terms of the laser wavelength $ \lambda $ and the following three parameters,
\begin{equation}
\label{1.38}
w_0, \qquad z_{\mathrm{R}} = \frac{k_z w_0^2}{2} = \frac{\pi w_0^2}{\lambda}, \qquad \Theta = \frac{w_0}{z_\mathrm{R}} = \frac{\lambda}{\pi w_0}.
\end{equation}
The parameter $ w_0 $ in \ref{1.38} is the beam waist, defined as a radius at which the laser intensity fall to $ 1/\e^2 $ of its axial value at the focal spot. The second parameter, $ z_\mathrm{R} $, is so-called Rayleigh range which is a distance in the longitudinal direction from the focal spot to the point where the beam radius is $ \sqrt{2} $ larger than the beam waist $ w_0 $. And the last parameter, $ \Theta $, is the divergence angle of the beam that represents the ratio of transverse and longitudinal extent.

Because of the symmetry about the longitudinal axis of the equation \ref{1.37}, the following calculations may be made simpler by introducing a dimensionless cylindrical coordinates that use the parameters \ref{1.38},
\begin{equation}
\label{1.39}
\rho = \frac{\norm{\vec{r_\bot}}}{w_0}, \qquad \zeta = \frac{z}{z_{\mathrm{R}}}.
\end{equation}
After performing a transformation of coordinates, the Helmholtz equation \ref{1.37} becomes 
\begin{equation}
\label{1.40}
\frac{1}{\rho} \diffp{}{\rho}\left(\rho \diffp{\Psi \left(\rho, \zeta \right)}{\rho} \right) + 4 \i \diffp{\Psi \left(\rho, \zeta \right)}{\zeta}  = - \Theta^2 \diffp[2]{\Psi \left(\rho, \zeta \right)}{\zeta}.
\end{equation}

In the following calculations, the beam divergence angle $ \Theta $ is assumed to be small ($ \Theta \ll 1 $), thus it can be used as an expansion parameter for $ \Psi $ and the solution of \ref{1.40} will always be consistent,
\begin{equation}
\label{1.41}
\Psi = \sum_{n = 0}^{+\infty} \Theta^{2n} \Psi_{2n}.
\end{equation}
Next, one shall insert \ref{1.41} into \ref{1.40} and collect the terms with the same power of $ \Theta $. Then the zeroth-order function $ \Psi_0 $ obeys the following equation,
\begin{equation}
\label{1.42}
\frac{1}{\rho} \diffp{}{\rho}\left(\rho \diffp{\Psi_0 \left(\rho, \zeta \right)}{\rho} \right) + 4 \i \diffp{\Psi_0\left(\rho, \zeta \right)}{\zeta} = 0.
\end{equation}

The equation \ref{1.42}, which is called the paraxial Helmholtz equation, is the starting point of traditional Gaussian beam theory. One can expect the solution of \ref{1.42} in the form of a Gaussian function with a width varying along the longitudinal direction, thus 
\begin{equation}
\label{1.43}
\Psi_0 \left(\rho, \zeta \right) = h\left(\zeta \right)\e^{-f\left(\zeta \right) \rho^2},
\end{equation}
where $ f\left(\zeta \right) $ and $ h\left(\zeta \right) $ are unknown complex functions that have to satisfy a condition $ f\left(0 \right) = h\left(0 \right) = 1 $. After plugging \ref{1.43} into \ref{1.42}, one gets the following equation,
\begin{equation}
\label{1.44}
-f\left(\zeta \right) h\left(\zeta \right) + \i \diff{h\left(\zeta \right)}{\zeta} + \rho^2 h\left(\zeta \right) \left(f\left(\zeta \right)^2 - \i \diff{f\left(\zeta \right)}{\zeta} \right) = 0.
\end{equation}
Since the equation \ref{1.44} has to hold for arbitrary value of $ \rho $, one may find two independent equations that are equivalent to \ref{1.44}
\begin{equation}
\label{1.45}
\frac{1}{f\left(\zeta \right)^2} \diff{f\left(\zeta \right)}{\zeta} + \i = 0, \qquad \frac{1}{f\left(\zeta \right) h\left(\zeta \right)} \diff{h\left(\zeta \right)}{\zeta} + \i = 0.
\end{equation}
It might be easily shown, that under specified conditions the solutions of equations \ref{1.45} have to be
\begin{equation}
\label{1.46}
h\left(\zeta \right) = f\left(\zeta \right), \qquad f\left(\zeta \right) = \frac{1}{\sqrt{1 + \zeta^2}} \e^{-\i \arctan{\zeta}},
\end{equation}
and therefore the complete expression for the zeroth-order wave function $ \Psi_0 \left(\rho, \zeta \right) $ is
\begin{equation}
\label{1.47}
\Psi_0 \left(\rho, \zeta \right) = \frac{1}{\sqrt{1 + \zeta^2}} \exp{\left[- \frac{\rho^2}{1 + \zeta^2} + \i \left(\frac{\rho^2 \zeta}{1 + \zeta^2} - \arctan{\zeta} \right) \right]}.
\end{equation}

In many situations, it is also useful to evaluate the expression \ref{1.47} in terms of Cartesian coordinates, in which the zeroth-order wave function $ \Psi_0 \left(\vec{r_\bot}, z \right) $ is
\begin{equation}
\label{1.48}
\Psi_0 \left(\vec{r_\bot}, z \right) = \frac{w_0}{w\left(z\right)} \exp{\left[- \frac{\vec{r_\bot}^2}{w\left(z \right)^2} + \i \left( k_z \frac{\vec{r_\bot}^2}{2 R\left(z \right)} - \varphi_\mathrm{G} \left( z\right) \right) \right]},
\end{equation}
where the parameters used to simplify the expression \ref{1.48} are defined as
\begin{equation}
\label{1.49}
w\left(z\right) = w_0 \sqrt{1 + \left(\frac{z}{z_\mathrm{R}}\right)^2}, \quad R\left(z \right) = z \left[1 + \left(\frac{z_\mathrm{R}}{z} \right)^2\right], \quad \varphi_\mathrm{G}\left(z\right) = \arctan{\left(\frac{z}{z_\mathrm{R}}\right)}.
\end{equation}
One shall discuss the physical meaning of the three parameters \ref{1.49}. The function $ w\left(z\right) $ represents the spot size parameter of the beam, that is the radius at which the laser intensity fall to $ 1/\e^2 $ of its axial value at any position $ z $ along the beam propagation. Note that the minimum of the spot size $ w(0) = w_0 $, consequently the focal spot is stationary and located at the origin of a Cartesian coordinate system. The second parameter, $ R\left(z \right) $ is known to be the radius of curvature of the beam's wavefront at any position $ z $ along the beam propagation. Note that $ \lim_{z \to 0^{\pm}} R(z) = \pm \infty $, therefore the beam behaves like a plane wave at focus as required. The last parameter, $ \varphi_\mathrm{G}\left(z\right) $, is the so-called Guoy phase of the beam at any position $ z $ along the beam propagation, which describes a phase shift in the wave as it passes through the focal spot.

Finally, by substituting \ref{1.48} for $ \Psi \left(\vec{r_\bot}, z \right) $ in \ref{1.36} and taking the real part of that complex quantity, one obtains the electric field of the so-called paraxial Gaussian beam,
\begin{equation}
\label{1.50}
\vec{E}\left(\vec{r_\bot}, z, t \right) = E_0 \frac{w_0}{w(z)} \exp\left(-\frac{\vec{r_\bot}^2}{w(z)^2}\right) \cos\left(\omega t - k_z \left(z + \frac{\vec{r_\bot}^2}{2 R(z)} \right) + \varphi_\mathrm{G}\left(z\right) \right) \mathrm{\vec{\hat{e}_x}}.
\end{equation}
Although given electric field \ref{1.50} describes the main features of the focused laser beam, it might be clearly seen that it does not satisfies the Maxwell equation \ref{1.1}. The correct electric field has to have at least two non-zero vector components. To fix that, one would have to solve the wave equation for the vector potential \ref{1.16} and afterwards exploit the solution to deduce all components of the electric and magnetic fields.   

In addition, since one assumed $ \Theta \ll 1 $, the solution \ref{1.50} is not accurate for strongly diverging beams. Since the divergence angle is inversely proportional to the beam waist, the previous condition yields $ w_0 \gg \lambda $. In other words, it means that \ref{1.50} is not valid for tightly focused laser beams and the need may arise for higher-order corrections.