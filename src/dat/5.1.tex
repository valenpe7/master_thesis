In total, three sets of simulations have been conducted. The first set contains simulations of laser pulses with constant energy $ E \approx 30 \ \mathrm{mJ} $ and four different sizes of the focal spot characterized by $ w_0 = 0.5, 1.0, 2.0 \ \mathrm{and} \ 4.0 \ \mathrm{\mu m} $. Two more sets for the same beam waists as before and constant peak intensities $ I = 10^{20} \ \mathrm{W/cm^2} $ and $ I = 10^{21} \ \mathrm{W/cm^2} $ have been calculated to be able to better distinguish between the intensity-dependent effects and the effects of the focal spot size.

Apart from the laser intensity $ I $ and the beam waist $ w_0 $, all the input parameters remained identical for all simulations. The p-polarized Gaussian laser pulse with the wavelength $ \lambda = 0.8 \ \mathrm{\mu m} $ and the duration $ \tau = 30 \ \mathrm{fs} $ in FWHM propagates from the left hand side boundary to the right. The focal spot is located at the surface of a solid target at the distance $ x = 8 \ \mathrm{\mu m} $ from the boundary that the laser is attached to. The laser pulse is incident normally.

The size of the simulation domain is $ 15 \ \mathrm{\mu m} \times 40 \ \mathrm{\mu m} $, with 1875 cells in the x-direction and 5000 cells in the y-direction. This domain set-up corresponds to 100 cells per $ \lambda $ in both directions, thus the simulation cell sizes $ \Delta x = \Delta y = \lambda/100 = 8 \ \mathrm{nm} $ are similar to the minimum Debye length in the simulation box. The simulation time step is chosen to fulfill the CFL condition \cite{CFL1967} $ \Delta t = C \sqrt{2} \lambda/ 100 c $, where the CFL constant is $ C = 0.95 $, thus $ \Delta t \approx 0.035 \ \mathrm{fs} $. The peak laser intensity hits the target at $ t \approx 100 \ \mathrm{fs} $, the whole simulation time has been then specified to $ t = 150 \ \mathrm{fs} $.

The distance between the left hand side boundary and the front side of the solid target of density $ 100 \ n_c \left(\omega \right) $ is $ 8 \ \mathrm{\mu m} $, the width of the target is $ 30 \ \mathrm{\mu m} $ and the thickness $ 2 \ \mathrm{\mu m} $. The target is initially at rest and is made of electrons and protons that are represented by weighted computational particles. The number of particles per one simulation cell is $ 2000 $ for electrons and $ 100 $ for protons. Therefore, each electron represents $ 5 \ \% $ of $ n_c \left(\omega \right) $ and each proton $ 100 \ \% $ of $ n_c \left(\omega \right) $. The initial temperature of both particle species is constant throughout the target and equal to $ 100 \ \mathrm{eV} $.

The boundary conditions are thermalizing to the initial temperature for particles and CPML \cite{Roden2000} with thickness of 16 cells for fields on all sides of the simulation box. The collisions of plasma particles are not accounted for since their contribution to laser energy absorption can be neglected under considered conditions. The quantum electrodynamics effects are not included as well.

In total, the simulation domain contains $ 9.6 \cdot 10^{6} $ cells and $ 2 \cdot 10^{9} $ computational particles ($ 1.9 \cdot 10^{9} $ electrons, $ 9.4 \cdot 10^{7} $ protons). The whole simulation requires approximately $ 4125 $ time steps. The template for all the simulation input files can be found in the appendix A.