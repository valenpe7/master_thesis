Radiation pressure acceleration (RPA) stands for the mechanism in which the ions are accelerated from the target front side in the vicinity of the laser focal spot. The acceleration is driven by the ponderomotive force (see the section 2.4) which expels the electrons into the regions of a lower laser intensities and consequently generates a strong electrostatic fields as a result of a charge separation. In the case of intense laser beams, the radiation pressure is strong enough to push an overdense target inwards whilst changing the shape of its surface and correspondingly the density profile. This process is commonly named hole boring.

For a plane, monochromatic wave at a normal incidence onto a target at rest, the balance between the electrostatic pressure and the radiation pressure at the target surface can be expressed as follows,
\begin{equation}
\label{2.7.1.1}
\frac{1}{2} \varepsilon_0 E_{\mathrm{es}}^2 = \frac{\left( 1 + R - T \right)}{c} I,
\end{equation}
where $ \vec{E}_{\mathrm{es}} \left(\vec{r}, t \right) $ is the electrostatic field generated by the charge separation, $ R $ and $ T $ are the reflection and transmission coefficients of the target, respectively, and $ I $ is the intensity of incident laser pulse.

The formula \ref{2.7.1.1} determines the extension of a charge depletion layer, which is established at the front side of the target. Ions in the depletion layer are accelerated by the electrostatic field $ \vec{E}_{\mathrm{es}} \left(\vec{r}, t \right) $, which amplitude can be obtained by solving the Poisson equation. The maximum energy, that the ions may gain, is then estimated as follows [source],
\begin{equation}
E_{max} = \frac{Z m_e c^2 a_0^2}{m_i \gamma}, \quad \gamma = \sqrt{1 + \frac{a_0^2}{2}}.
\end{equation}
 
The RPA regime starts to dominate for a laser beams with peak intensity higher than $ 10^{21} \  \mathrm{W/cm}^2 $. Also, RPA could be more efficient for circularly polarized laser beams normally incident onto a target, because of their ability to suppress the effects of electron heating mechanisms mentioned in the previous section. Note that the RPA mechanism typically produce an ion beam of a large divergence because the critical density interface where the charge separation occurs is curved by the shape of the laser beam.

