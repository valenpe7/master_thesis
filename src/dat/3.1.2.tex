The particle weighting refers to the part of the code in which the charge and current densities are assigned to the discrete grid points from the continuous particle positions. After the fields are obtained, they are assigned back at the particles to calculate the Lorentz force. This step is called field weighting. It implies some form of interpolation.

First, have a look at the particle weighting. Charge density $ \rho\left(\vec{x}, t \right) $ and current density $ \vec{J}\left(\vec{x}, t \right) $ are given by the following integrals over the velocity space,
\begin{equation}
\label{3.1.2.1}
\rho\left(\vec{x}, t \right) = \sum_s q_s \int f_s \left(\vec{x}, \vec{v}, t \right) \mathrm{d} \vec{v}, \qquad \vec{J}\left(\vec{x}, t \right) = \sum_s q_s \int f_s \left(\vec{x}, \vec{v}, t \right) \vec{v} \, \mathrm{d} \vec{v}.
\end{equation}
The interpolation function is defined as
\begin{equation}
\label{3.1.2.2}
W_{i, j, k}\left(\vec{x}_{p}^{n}\right) =  \frac{1}{\Delta V}\int\limits_{\Omega} S_{x}\left(\vec{x} - \vec{x}_{p}^{n} \right) \mathrm{d} \vec{x}, \qquad \Delta V = \Delta x \Delta y \Delta z,
\end{equation}
where $ \Omega = \left\lbrace \vec{x} \in \mathbb{R}^{3} : \abs{x_{1} - x_{i, j, k}} \leq \Delta x/2 \wedge \abs{x_{2} - x_{i, j, k}} \leq \Delta y/2 \wedge \abs{x_{3} - x_{i, j, k}} \leq \Delta z/2 \right\rbrace $. Then the charge density $ \rho_{i, j, k}^{n} $ in the arbitrary grid point $ x_{i, j, k} $ and time level $ t^{n} $ is constructed as
\begin{equation}
\label{3.1.2.3}
\rho_{i, j, k}^{n} = \sum_{p} q_{p} W_{i, j, k}\left(\vec{x}_{p}^{n}\right), \qquad q_{p} = q_{s} N_{p},
\end{equation}
where the sum is taken over all computational particles.

By analogy, one could assign the current densities to the grid points using this interpolation function, but the discrete continuity equation for charge (\ref{3.1.1.13}) would not be satisfied exactly. In this case one have to solve Poisson's equation for correction of electric field or use one of the several numerical schemes for computing the current density satisfying the continuity equation, which are called charge conservation methods. In the next paragraph, the charge density decomposition method proposed by Esirkepov \cite{esirkepov} is described.

Due to linearity of the continuity equation (\ref{3.1.1.13}), it is sufficient to define the current density associated with the motion of a single computational particle,
\begin{equation}
\label{3.1.2.4}
\begin{split}
& \left(J_{x}\right)_{i+1,\: j,\: k}^{n} = \left(J_{x}\right)_{i,\: j,\: k}^{n} - q_{p} \left(v_{x}\right)_{p}^{n} \left(\Pi_{x}\right)_{i,\: j,\: k}^{n}, \\
& \left(J_{y}\right)_{i,\: j+1,\: k}^{n} = \left(J_{y}\right)_{i,\: j,\: k}^{n} - q_{p} \left(v_{y}\right)_{p}^{n} \left(\Pi_{y}\right)_{i,\: j,\: k}^{n}, \\
& \left(J_{z}\right)_{i,\: j,\: k+1}^{n} = \left(J_{z}\right)_{i,\: j,\: k}^{n} - q_{p} \left(v_{z}\right)_{p}^{n} \left(\Pi_{z}\right)_{i,\: j,\: k}^{n}.
\end{split}
\end{equation}
In accordance with continuity equation one can write
\begin{equation}
\label{3.1.2.7}
\left(\Pi_{x}\right)_{i,\: j,\: k}^{n} + \left(\Pi_{y}\right)_{i,\: j,\: k}^{n} + \left(\Pi_{z}\right)_{i,\: j,\: k}^{n} = W_{i, j, k}\left(\vec{x}_{p}^{n} + \vec{\tilde{x}}\right) - W_{i, j, k}\left(\vec{x}_{p}^{n}\right),
\end{equation}
where $ \vec{\tilde{x}} \in \mathbb{R}^{3} $ is the motion induced position shift of the computational particle in one simulation time step. Shift of the computational particle generates eight variants of interpolation functions,
\begin{equation}
\label{3.1.2.8}
\begin{split}
& W_{i, j, k}\left(\vec{x}_{p}^{n}\right), \quad W_{i, j, k}\left(\vec{x}_{p}^{n} + \tilde{x}_{1}\right), \quad W_{i, j, k}\left(\vec{x}_{p}^{n} + \tilde{x}_{2}\right), \quad W_{i, j, k}\left(\vec{x}_{p}^{n} + \tilde{x}_{3}\right), \\
& W_{i, j, k}\left(\vec{x}_{p}^{n} + \tilde{x}_{1} + \tilde{x}_{2}\right), \quad W_{i, j, k}\left(\vec{x}_{p}^{n} + \tilde{x}_{1} + \tilde{x}_{3}\right), \quad W_{i, j, k}\left(\vec{x}_{p}^{n} + \tilde{x}_{2} + \tilde{x}_{3}\right), \\
& W_{i, j, k}\left(\vec{x}_{p}^{n} + \tilde{x}_{1} + \tilde{x}_{2} + \tilde{x}_{3}\right).
\end{split}
\end{equation}
We assume that the vector $ \Pi_{i,\: j,\: k}^{\:n} $ is linearly dependent on the functions (\ref{3.1.2.8}). It turned out that only one linear combination exists.

Now, have a look at the field weighting. After calculating Maxwell equations at the grid points, it is necessary to assign electric and magnetic fields back at the particle positions. Recall the definition of interpolation function (\ref{3.1.2.2}). By analogy to the particle weighting, one can assign the components of electric field,
\begin{equation}
\begin{split}
& \left(E_{x}\right)_{p}^{n}  = \sum_{i, j, k} \left(E_{x}\right)^{n}_{i,\: j + 1/2,\: k + 1/2} W_{i,\: j + 1/2,\: k + 1/2} \left(\vec{x}_{p}^{n} \right), \\
& \left(E_{y}\right)_{p}^{n}  = \sum_{i, j, k} \left(E_{y}\right)^{n}_{i + 1/2,\: j,\: k + 1/2} W_{i + 1/2,\: j,\: k + 1/2} \left(\vec{x}_{p}^{n} \right), \\
& \left(E_{z}\right)_{p}^{n}  = \sum_{i, j, k} \left(E_{z}\right)^{n}_{i + 1/2,\: j + 1/2,\: k} W_{i + 1/2,\: j + 1/2,\: k} \left(\vec{x}_{p}^{n} \right),
\end{split}
\end{equation}
and the components of magnetic field,
\begin{equation}
\begin{split}
& \left(B_{x}\right)_{p}^{n}  = \sum_{i, j, k} \left(B_{x}\right)^{n}_{i + 1/2,\: j,\: k} W_{i + 1/2,\: j,\: k} \left(\vec{x}_{p}^{n} \right), \\
& \left(B_{y}\right)_{p}^{n}  = \sum_{i, j, k} \left(B_{y}\right)^{n}_{i,\: j + 1/2,\: k} W_{i,\: j + 1/2,\: k} \left(\vec{x}_{p}^{n} \right), \\
& \left(B_{z}\right)_{p}^{n}  = \sum_{i, j, k} \left(B_{z}\right)^{n}_{i,\: j,\: k + 1/2} W_{i,\: j,\: k + 1/2} \left(\vec{x}_{p}^{n} \right).
\end{split}
\end{equation} 
It is desirable to use the same interpolation function for both, density and force calculations, in order to eliminate a self-force and ensure conservation of momentum \cite{fehske}.
