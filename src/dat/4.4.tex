In this section, several possible approaches for focusing a laser beam in experiments are discussed. Before providing a brief overview of currently used methods, it is necessary to introduce a few most fundamental parameters that describe any optical system.

First, one defines a numerical aperture of an optical system $ N\!A $, which is a dimensionless number characterizing the range of angles over which the system can accept or emit light \cite{Greivenkamp2004}. Second characteristic of any optical system is a focal length $ f $ describing a distance between the center of the aperture and the point in space over which collimated light rays are brought to a focus \cite{Greivenkamp2004}. Finally, one defines a f-number $ f/\# $ as a ratio of the focal length $ f $ to the size of the aperture $ D $. The f-number is thus dimensionless and stands for a quantitative measure of a speed of the optical system \cite{Smith2007}.

In experiments, focusing of laser beams is usually achieved by means of an optical system, such as a lens or a curved mirror. However, when dealing with short pulses, lenses (in general transmissive optics) are not preferred because they may affect the beam and consequently the quality of the focal spot (Strehl ratio, encircled energy) by frequency-dependent effects, such as chromatic aberration and other nonlinear optical distortions. On the other hand, reflective optics are generally able to produce a diffraction limited focus without chromatic effects \cite{Marimont1994}.