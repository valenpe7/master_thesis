Since the electromagnetic wave is able to carry momentum (\ref{1.54}), there is ... It leads to a wide range of non-linear phenomena. The ponderomotive force is probably the most important quantity that describes the interaction of high intensity laser pulses with plasma.

The ponderomotive force represents the gradient of the laser electric field that pushes charged particles into the regions of a lower field amplitude. It may be interpreted as a result of the Lorentz force that works on a charged particles in the electromagnetic wave.

Since the mass of the ions is much higher than the electron mass, the ponderomotive force acting on ions is in most cases negligible. However, the ponderomotive force exerted on the electrons may be consequently transmitted to the ions by the electric field which is created due to the charge separation in plasma.