The ponderomotive force is a most important quantity in the interaction of high intensity laser pulses with plasma. It leads to a wide range of non-linear phenomena. For normal laser light incidence on plasma, the ponderomotive force $ \vec{f}_{p} $ per unit volume is given by
\begin{equation}
\label{2.5.1.1}
\vec{f}_{p} = - \frac{\omega_{pe}^{2}}{\omega^{2}} \nabla \frac{\varepsilon_0 \left\langle E^{2} \right\rangle}{2},
\end{equation}
where the $ \left\langle \: \right\rangle $ symbol denotes the time average over one laser oscillating period. Notice that in a homogeneous field this time-averaged force vanishes.

The ponderomotive force represents the gradient of the laser electric field acting in a way to push charged particles into regions of lower field amplitude. It is the result of the Lorentz force that works on a charged particles in the electromagnetic wave.

Since the mass of the ions is much higher than electrons, the ponderomotive force acting on the ions is negligible. However, the ponderomotive force exerted on the electrons is transmitted to the ions by the electric field, which is created due to charge separation in the plasma.