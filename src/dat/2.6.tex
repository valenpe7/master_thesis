As mentioned in the section 2.3, the electromagnetic wave cannot propagate through plasma with densities greater than the critical density $ n_c $. However, in the case of intense laser beam, one has to perform a relativistic generalization of the dispersion relation, thus the corresponding cut-off frequency is given by
\begin{equation}
\omega = \sqrt{\frac{\omega_{pe}^2 + \omega_{pi}^2}{\gamma}}, \quad \gamma = \left( 1 - \frac{v^{2}}{c} \right)^{-\frac{1}{2}}.
\end{equation}
Therefore, one may immediately see that in the relativistic case, the laser beam actually propagates up to densities $ n_c^{\: \prime} = n_c \gamma > n_c $. This increase of the effective critical density is called relativistic self-induced transparency [source].

In the physical picture, relativistic transparency occurs when the increased mass of the electrons slows their motion such that they can no longer shield the plasma from the incident laser. As a result of relativistic transparency dynamics, both the reflected and transmitted laser intensity profiles evidence temporal chopping of the original laser intensity profile.
