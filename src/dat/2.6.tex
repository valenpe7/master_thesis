As already mentioned in the section 2.3, the electromagnetic wave with frequency $ \omega $ cannot propagate through plasma with densities greater than the critical density $ n_c \left( \omega \right) $. However, in the case of intense laser beam, one has to perform a relativistic generalization of the dispersion relation, thus the corresponding cut-off frequency is given by the following formula,
\begin{equation}
\label{2.6.1}
\omega = \sqrt{\frac{\omega_{pe}^2 + \omega_{pi}^2}{\gamma}}, \quad \gamma = \left( 1 - \frac{v^{2}}{c} \right)^{-\frac{1}{2}}.
\end{equation}
Therefore, one may immediately see that in the relativistic case, the laser beam actually propagates up to densities $ n_c^{\: \prime} \left( \omega \right) = n_c \left( \omega \right) \gamma > n_c \left( \omega \right) $. This increase of the effective critical density is called relativistic self-induced transparency (SIT) [source].

Note that in the case of real laser pulse, the ability to penetrate into the overdense plasma depends on the local amplitude. Therefore, there are logically some parts of laser beam that may propagate, while the others cannot. Consequently, both the reflected and transmitted laser intensity profiles reveal temporal chopping that leads to the significant modifications of the pulse shape.

Furthermore, one has to be aware that the plasma does not remain homogeneous during the laser-plasma interaction. As mentioned in the section 2.4, the ponderomotive force pushes charged particles in plasma into the regions of a lower field amplitude creating a charge separation layer. Consequently, the plasma density rapidly increases in these regions. Therefore, it is necessary to resolve the propagation of the electromagnetic wave self-consistently with the modification of plasma density profile.

Note that the electromagnetic wave propagating in a relativistically transparent target may be a subject of strong instabilities that lead to an additional plasma heating. Thus this way, the energy of electromagnetic wave may be also absorbed. Furthermore, in the SIT regime, the acceleration of charged particles due to the ponderomotive force may take place in the whole volume of the target.

At the end of this section, an approach to SIT based on a steady state solutions that follows [source] is presented. Consider a laser pulse to be a monochromatic plane wave at normal incidence and with circular polarization. Then the thresholds for SIT in the case of two limit cases, a semi-infinite plasma and an ultrathin plasma slab, are given by following two formulas,
\begin{equation}
\label{2.6.2}
a_0 \cong \frac{3^{3/2}}{2^3} \left( \frac{n_0}{n_c} \right)^2 \quad \mathrm{for} \quad n = n_0 H\left(x \right),
\end{equation}
\begin{equation}
\label{2.6.3}
a_0 \cong \frac{\pi l}{\lambda} \frac{n_0}{n_c} \quad \mathrm{for} \quad  n = n_0 l \delta\left(x \right),
\end{equation}
where $ H\left(x \right) $ stands for a Heaviside step function and $ \delta\left(x \right) $ is Dirac delta function. The threshold values are expressed in terms of laser dimensionless potential $ a_0 = e E_0/m_e \omega c $. If the conditions \ref{2.6.2} and \ref{2.6.3} are fulfilled, laser beam penetrates plasma as a non-linear wave.
 
Note that the model presented above is based on several simplifications (e.g. the pulse profile or electron oscillations and heating are not included). In general case, the SIT dynamics is obviously much more complicated.
