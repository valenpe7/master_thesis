The abbreviation EPOCH refers to an Extendable PIC Open Collaboration project \cite{bennett}. EPOCH is a multi-dimensional, relativistic, electromagnetic code for plasma physics simulations based on the PIC method. The code, which is developed in University of Warwick, is written in FORTRAN and parallelized using MPI library. EPOCH is explicit and is able to achieve second-order accuracy. The entire core of the code uses SI units.

The main features include dynamic load balancing option for making optimal use of all processors when run in parallel, allowing restart on an arbitrary number of processors. The setup of EPOCH is controlled through a customizable input deck. An input deck is text file which can be used to set simulation parameters for EPOCH without needing to edit or recompile the source code. Most aspects of a simulation can be controlled, such as the number of grid points in the simulation domain, the initial distribution of particles and the initial electromagnetic field configuration. In addition, EPOCH was written to add more modern features and to structure the code in such a way that the future expansion of the code may be made as easily as possible.

By default, EPOCH uses triangular particle shape functions with the peak located at the position of computational particle and a width of two cells, which provides relatively clean and fast solution. However, user can select higher order particle shape functions based on spline interpolation by enabling compile time option in the makefile.

The electromagnetic field solver uses a FDTD scheme with second order of accuracy. The field components are spatially staggered on a standard Cartesian Yee cell. The solver is directly based on the scheme derived by Hartmut Ruhl \cite{ruhl}. The particle pusher is relativistic, Birdsall and Landon type \cite{birdsall} and uses Villasenor and Buneman current weighting \cite{villasenor}.  

