The ponderomotive force can be derived from the motion of single particle in prescribed electromagnetic field. Thus one may exploit the Newton's equation of motion with the Lorentz force (\ref{2.2.5}). As the laser propagates through plasma, one may expect the oscillation of charged particles in a high-frequency electromagnetic field. For the mathematical description, one may exploit the so-called oscillation-center approximation in this case. It consist of splitting the radius vector $ \vec{r} $ and velocity $ \vec{v} $ of an arbitrary charged particle into two parts,
\begin{equation}
\label{2.5.1.2}
\vec{r} = \vec{r}_0 + \delta \vec{r}, \quad \vec{v} = \vec{v}_0 + \delta \vec{v},
\end{equation}
where the index $ 0 $ denotes the quantities at the center of oscillation and $ \delta \vec{x} = \vec{x} - \vec{x}_0 $ stands for the oscillating vector.

Next, one may exploit the Taylor series to  expand both, the electric field $ \vec{E}\left(\vec{r}, t\right) $ and magnetic field $ \vec{B}\left(\vec{r}, t\right) $ around the oscillation center $ \vec{r}_0 $,
\begin{equation}
\label{2.5.1.3}
\vec{E}\left(\vec{r}_0 + \delta \vec{r}, t\right) \cong \vec{E}\left(\vec{r}_0, t\right) + \left(\delta \vec{r} \cdot \nabla \right) \vec{E}\left(\vec{r}_0, t\right),
\end{equation}
\begin{equation}
\label{2.5.1.4}
\vec{B}\left(\vec{r}_0 + \delta \vec{r}, t\right) \cong \vec{B}\left(\vec{r}_0, t\right) + \left(\delta \vec{r} \cdot \nabla \right) \vec{B}\left(\vec{r}_0, t\right).
\end{equation}
Note that taking only the zeroth and first-order terms of the fields is sufficiently accurate approximation for the description of ponderomotive force.

By substituting \ref{2.5.1.2}, \ref{2.5.1.3} and \ref{2.5.1.4} into \ref{2.5.1.1} and collecting the terms of the same order, one obtains following two equations, 
\begin{equation}
\label{2.5.1.5}
m_s \diff[2]{\delta \vec{r}}{t} = q_s \left[ \vec{E}\left(\vec{r}_0, t\right) + \vec{v}_0 \times \vec{B}\left(\vec{r}_0, t\right)\right],
\end{equation}
\begin{equation}
\label{2.5.1.6}
m_s \diff[2]{\vec{r}_0}{t} = q_s \left[\left(\delta \vec{r} \cdot \nabla \right) \vec{E}\left(\vec{r}_0, t\right) + \delta \vec{v} \times \vec{B}\left(\vec{r}_0, t\right) \right]. 
\end{equation}

To solve equations \ref{2.5.1.5} and \ref{2.5.1.6}, assume further the laser fields in the form of standing monochromatic wave, 
\begin{equation}
\label{2.5.1.7}
\vec{E}\left(\vec{r}_0, t\right) = \vec{E}_0 \left(\vec{r}_0 \right) \cos \omega t, \quad  \vec{B}\left(\vec{r}_0, t\right) = - \frac{1}{\omega} \rot{\vec{E}_0 \left(\vec{r}_0 \right)} \sin \omega t.
\end{equation}
Note that the magnetic field in \ref{2.5.1.7} has been obtained by the integration of Faraday's law (\ref{1.3}). For the non-relativistic case, the term $ \vec{v}_0 \times \vec{B}\left(\vec{r}_0, t\right) $ is small in comparison with the term $ \vec{E}\left(\vec{r}_0, t\right) $ in the equation \ref{2.5.1.5}, thus it can be neglected. Taking into account the previous considerations, the solution of \ref{2.5.1.5} yields the oscillating quantities,
\begin{equation}
\label{2.5.1.8}
\delta \vec{r} = -\frac{q_s}{m_s \omega^2} \vec{E}_0 \left(\vec{r}_0 \right) \cos \omega t, \quad \delta \vec{v} = \frac{q_s}{m_s \omega} \vec{E}_0 \left(\vec{r}_0 \right) \sin \omega t.
\end{equation}
Since the integration constants can be considered as a slowly varying time functions, they have been included in $ \vec{r}_0 $.

By inserting \ref{2.5.1.8} into \ref{2.5.1.6} one gets immediately the following formula,
\begin{equation}
\label{2.5.1.9}
m_s \diff[2]{\vec{r}_0}{t} = - \frac{q_s^{2}}{m_s \omega^2} \left[\left(\vec{E}_0 \left(\vec{r}_0 \right) \cdot \nabla \right) \vec{E}_0 \left(\vec{r}_0 \right) \cos^2 \omega t + \vec{E}_0 \left(\vec{r}_0 \right) \times \left( \nabla \times \vec{E}_0 \left(\vec{r}_0 \right) \right) \sin^2 \omega t \right].
\end{equation}

To get the quasi-stationary ponderomotive force, one shall compute the time average values over the laser period $ T = 2 \pi/ \omega $. Thus by performing the time average of \ref{2.5.1.9} (note that $ \left\langle \sin^2 \omega t \right\rangle_T = \left\langle \cos^2 \omega t \right\rangle_T = 1/2$ ) one gets
\begin{equation}
\label{2.5.1.10}
m_s \left\langle \diff[2]{\vec{r}_0}{t} \right\rangle_{T} = - \frac{q_s^{2}}{2 m_s \omega^2} \left[\left(\vec{E}_0 \left(\vec{r}_0 \right) \cdot \nabla \right) \vec{E}_0 \left(\vec{r}_0 \right) + \vec{E}_0 \left(\vec{r}_0 \right) \times \left( \nabla \times \vec{E}_0 \left(\vec{r}_0 \right) \right)\right].
\end{equation}
The first term on the right-hand side of \ref{2.5.1.10} contributes to the transverse component of the ponderomotive force, whilst its longitudinal component originates from the second term. Utilizing standard identities of vector calculus, one finally gets the resulting quasi-stationary ponderomotive force $ \vec{F}_{\mathrm{p}} \left( \vec{r} \right) $ in the non-relativistic case,
\begin{equation}
\label{2.5.1.11}
\vec{F}_{\mathrm{p}} = - \grad{\Phi_{\mathrm{p}}}, \quad \Phi_{\mathrm{p}} = \frac{q_s^{2}}{4 m_s \omega^2} \vec{E}_0^{2},
\end{equation}
where $ \Phi_{\mathrm{p}} \left( \vec{r} \right) $ is the corresponding ponderomotive potential, which may be interpreted as a cycle-averaged oscillation energy.

Sometimes, it may be yet useful to have the expression for the ponderomotive force per unit volume $ \vec{f}_{p} \left( \vec{r} \right) $. To obtain it, one has to multiply the formula \ref{2.5.1.10} by the particle density. In the case of the electron fluid and normal laser light incidence on plasma, one gets the following easy-to-remember form,
\begin{equation}
\vec{f}_{p} = - \frac{\varepsilon_0}{4} \frac{\omega_{pe}^{2}}{\omega^{2}} \nabla \vec{E}_0^{2}.
\end{equation}