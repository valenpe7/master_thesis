A parametric instability is defined in general by the non-linear phenomenon where a periodic variation in the medium supports the excitation of waves at a different frequency. In the following section we describe these wave excitation processes in more detail.

The resonance conditions in the three-wave interactions, under which parametric instabilities may grow, can be written in the form of the energy and momentum conservation laws,
\begin{equation}
\label{2.5.2.1}
\omega_0 = \omega_1 + \omega_2, \qquad \vec{k}_0 = \vec{k}_1 + \vec{k}_2,
\end{equation}
where index $ 0 $ denotes parameters of the incident so-called pumping laser wave, which provides the driving force to excite other wave modes in the plasma. Indices $ 1, 2 $ stand for the parameters of the two daughter waves.

These conditions imply that parametric instabilities take place only if a certain relations between the incident pump wave and the plasma frequency are fulfilled. A pumping laser wave can induce the following parametric instabilities that directly correspond to a plasma domain, because the plasma frequency is determined by the density of the plasma. Depending on which types of waves are excited, either the reflectivity or the absorption can be enhanced:
\begin{enumerate}[nolistsep, topsep=5pt]
\item A laser pump with frequency $ \omega_0 \approx \omega_{pe} $ may decay into an electron wave and an ion wave, leading to laser absorption. This phenomena occurs near the critical density $ n_c $ of plasma and is known as parametric decay instability.
\item A laser pump with frequency $ \omega_0 > \omega_{pe} $ may decay into an ion wave and another electromagnetic wave, leading to laser scattering, including backscattering. This phenomena occurs in the whole domain of underdense plasma and is known as stimulated Brillouin scattering.
\item A laser pump with frequency $ \omega_0 = 2 \omega_{pe} $ may decay into two electron waves, leading to laser absorption. This phenomena occurs at $ n_c/4 $ and is known as two-plasmon decay instability.
\item A laser pump with frequency $ \omega_0 \geq 2 \omega_{pe} $ may decay into an electron wave and another electromagnetic wave, leading to laser scattering, including backscattering. This phenomena occurs in a plasma with density lower than $ n_c/4 $ and is known as stimulated Raman scattering.
\end{enumerate}

Even if these conditions are fulfilled, however, these instabilities do not need to be present in a plasma. The intensity of the incoming laser wave has to exceed a certain threshold in order for the parametric instability to occur, because all waves in plasma are damped either by collision and collisionless processes. If the laser intensity is higher than this limit, the amplitude of the parametric decay mode increases.