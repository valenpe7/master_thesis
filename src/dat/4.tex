The investigation of laser-matter interactions also involves exploring of specific themes of the ultra-relativistic regime, which requires extremely high intensities of the external field. Tight-focusing using the plasma-based optics could potentially increase the laser pulse intensity by an order of magnitude in comparison with the conventional focusing approaches and thus the laser intensities of the order $ 10^{22} \ \mathrm{W/cm^2} $ could be reached by Petawatt laser systems. Such intensities would then allow a broad spectrum of many new physics discoveries and applications.

As mentioned in the previous chapter, various aspects of the electromagnetic interaction are usually studied using sophisticated numerical simulation codes. Vast majority of these codes, however, use a paraxial approximation (closer described in chapter 1) to prescribe the laser fields at the boundaries, and afterwards, a field solver propagates the beam across the simulation domain. As already mentioned, the paraxial approximation is valid only if the angular spectrum of the laser pulse is sufficiently narrow. Therefore, it is not possible to simulate tightly focused laser beams using this approach. As will be seen later, the paraxial approximation in this case leads to a distorted field profiles which would have strong impact on the laser-matter interaction results.

Several interesting solutions, how to simulate strongly focused beams, have been already proposed \cite{Sepke2006, Fedorov2016, Wang2002, Sepke22006, Agrawal1979, Hua2004, Hora1990, Couairon2015, Moloney2012, Sheppard1999}. Within this work, a simple and efficient algorithm for a Maxwell consistent calculation of the electromagnetic fields at the boundaries of the computational domain \cite{Thiele2016} (also called laser boundary conditions) has been used and implemented into the PIC code EPOCH \cite{bennett}. Note that this algorithm is able to describe laser beams with an arbitrary shape in the focal spot.