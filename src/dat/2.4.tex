The ponderomotive force is one of the most important quantities in the the interaction of high intensity laser pulses with plasma. It represents the gradient of the laser intensity that pushes charged particles in plasma into the regions of a lower field amplitude and it consequently leads to a wide range of non-linear phenomena that make the plasma inherently unstable. The ponderomotive force is involved e. g. in filamentation and self-focusing of laser beam, in the formation of cavities and solitons in the plasma profile or various parametric instabilities.

Note that since the mass of the ions is much higher than the electron mass, the ponderomotive force acting on ions is in most cases negligible. However, the ponderomotive force exerted on the electrons may be consequently transmitted to the ions by the electric field which is created due to the charge separation in plasma.

In the following two subsections, one can find the derivation of the ponderomotive force for the non-relativistic as well as the relativistic case. The derivation of the non-relativistic formula for ponderomotive force is easy to understand and clearly depicts its main characteristics that are valid even for the relativistic case. However, one has to be aware that the non-relativistic approximation breaks down if the laser intensity exceeds the values around $ 10^{18} \ \mathrm{W/cm^2} $. This may be the case of short or tightly focused high-power laser beams. For this reason, the relativistic case is briefly discussed as well in the second subsection.