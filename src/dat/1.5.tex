Na podzim roku 2013 se údajně podařilo vědcům z americké Lawrence Livermore National Laboratory dosáhnout významného milníku, poprvé získali pomocí inerciálně držené termojaderné fúze více energie, než do ní vložili na vstupu. Tyto zprávy byly potvrzeny a publikovány v únoru následujícího roku. V experimentu bylo využito 192 svazků laserového zařízení National Ignition Facility o celkové energii 1{,}8 MJ \cite{hurricane}. 

Fúzní zisk je ovšem měřen v poměru k energii, které byla absorbována palivovým terčem, přičemž současná elektrická účinnost laserových systémů dosahuje pouze jednotek procent. Je to tedy energetická bilance z hlediska fyzikálního mechanismu na té nejnižší úrovni. Pro skutečnou produkci elektrické energie v hypotetické fúzní elektrárně bude potřeba, aby hodnota tohoto zisku byla ještě zhruba o dva řády vyšší \cite{nif}.

Ač se tedy o žádnou senzaci nejedná, jsou i tyto výsledky významným průlomem v celém termojaderném výzkumu. Pro další kroky k cíli je ale zásadní kvalitnější informovanost politických představitelů široké laické veřejnosti. Výstavba velkých laserových zařízení je značně nákladná a nebýt jejich významného vojenského využití, pravděpodobně by nebyla zafinancována.

Nicméně velké laserové systémy jsou široce využitelné také v řadě společensky významných mezioborových aplikací, mohou sloužit například pro vývoj protonových zdrojů určených k léčbě nádorových onemocnění \cite{bin}. I kdyby se tedy nakonec nedokázalo využít energii z inerciální fúze, jsou takto vynaložené prostředky účelné a smysluplné.