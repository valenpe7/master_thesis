In this section, the simplest mathematical description of a focused laser beam based on approximations to the wave equation is deduced. Since in numerical codes it is a common practice to prescribe the laser beams by their propagation in free space, the set of the microscopic Maxwell's equations \ref{1.1} - \ref{1.4} will be exploited.

In the absence of external sources, it might be easily shown that the equations \ref{1.1} - \ref{1.4} may be alternatively formulated as an uncoupled homogeneous wave equations for electric field $ \vec{E}\left( \vec{r}, t \right) $ and magnetic field $ \vec{B}\left( \vec{r}, t \right) $,
\begin{equation}
\label{1.34}
\laplace{\vec{E}} - \frac{1}{c^{2}} \diffp[2]{\vec{E}}{t} = 0,
\end{equation}
\begin{equation}
\label{1.35}
\laplace{\vec{B}} - \frac{1}{c^{2}} \diffp[2]{\vec{B}}{t} = 0,
\end{equation}
where the universal constant $ c = 1/\sqrt{\mu_0 \varepsilon_0} $ is the speed of light in vacuum, which leads to the essential fact, that the electromagnetic waves propagate in vacuum with the velocity of light $ c $. However, the wave equations \ref{1.34}, \ref{1.35} do not provide all the information about the electric and magnetic field of the wave. There are further constraints due to Maxwell's equations restricting the orientation and proportional magnitudes of the fields. From the set \ref{1.1} - \ref{1.4}, it might be clearly seen that $ \vec{E}\left( \vec{r}, t \right) $ and $ \vec{B}\left( \vec{r}, t \right) $ must be mutually perpendicular to each other as well as to the direction of the wave propagation. 

Without any loss of generality, consider the laser beam as a monochromatic electromagnetic wave propagating toward the positive direction of the z-axis. In many standard references [source], the description of such a wave is given by the evolution of a single electric field component linearly polarized along the x-axis of the Cartesian coordinate system (although the more proper way would be to use the vector potential [source]), therefore one has to look for the solution of the equation \ref{1.34}. 

According to the previous assumptions, the solution is expected to be in the form of the following plane wave,
\begin{equation}
\label{1.36}
\vec{E}\left(\vec{r_\bot}, z, t \right)  = E_0 \Psi \left(\vec{r_\bot}, z \right) \e^{\i \left(k_z z - \omega t \right)} \mathrm{\vec{\hat{e}_x}},
\end{equation}
where $ \vec{r_\bot} = (x, y)^{\mathrm{T}} $ is the vector of transverse Cartesian coordinates, $ E_0 $ is a constant amplitude, $ \Psi \left(\vec{r_\bot}, z \right) $ is the part of the wave function which is dependent only on the spatial coordinates, $ \omega $ denotes the angular frequency, $ k_z $ is the z-component of the wave vector $ \vec{k}\left(\omega \right) $, $ \mathrm{i} $ denotes the imaginary unit and $ \mathrm{\vec{\hat{e}_x}} $ is the unit vector pointing in the direction of the x-axis.

Direct substitution of expression \ref{1.36} into the equation \ref{1.34} yields the time-independent form of the scalar wave equation
\begin{equation}
\label{1.37}
\laplace{\Psi \left(\vec{r_\bot}, z \right)} + 2 \i k_z \diffp{\Psi \left(\vec{r_\bot}, z \right)}{z} = 0.
\end{equation}
The equation \ref{1.37} is called the Helmholtz equation. Note that it is sufficient to seek solutions to the equation \ref{1.37} since the wave \ref{1.36} is monochromatic.

It turned out, that the geometry of the focused laser beam can be expressed in terms of the laser wavelength $ \lambda $ and the following three parameters,
\begin{equation}
\label{1.38}
w_0, \qquad z_{\mathrm{R}} = \frac{k_z w_0^2}{2} = \frac{\pi w_0^2}{\lambda}, \qquad \Theta = \frac{w_0}{z_\mathrm{R}} = \frac{\lambda}{\pi w_0}.
\end{equation}
The parameter $ w_0 $ in \ref{1.38} is the beam waist, defined as a radius at which the laser intensity fall to $ 1/\e^2 $ of its axial value at the focal spot. The second parameter, $ z_\mathrm{R} $, is so-called Rayleigh range which is a distance in the longitudinal direction from the focal spot to the point where the beam radius is $ \sqrt{2} $ larger than the beam waist $ w_0 $. And the last parameter, $ \Theta $, is the divergence angle of the beam that represents the ratio of transverse and longitudinal extent.

Because of the symmetry about the longitudinal axis of the equation \ref{1.37}, the following calculations may be made simpler by introducing a dimensionless cylindrical coordinates that use the parameters \ref{1.38},
\begin{equation}
\label{1.39}
\rho = \frac{\norm{\vec{r_\bot}}}{w_0}, \qquad \zeta = \frac{z}{z_{\mathrm{R}}}.
\end{equation}
After performing a transformation of coordinates, the Helmholtz equation \ref{1.37} becomes 
\begin{equation}
\label{1.40}
\frac{1}{\rho} \diffp{}{\rho}\left(\rho \diffp{\Psi \left(\rho, \zeta \right)}{\rho} \right) + 4 \i \diffp{\Psi \left(\rho, \zeta \right)}{\zeta}  = - \Theta^2 \diffp[2]{\Psi \left(\rho, \zeta \right)}{\zeta}.
\end{equation}

In the following calculations, the beam divergence angle $ \Theta $ is assumed to be small ($ \Theta \ll 1 $), thus it can be used as an expansion parameter for $ \Psi $ and the solution of \ref{1.40} will always be consistent,
\begin{equation}
\label{1.41}
\Psi = \sum_{n = 0}^{+\infty} \Theta^{2n} \Psi_{2n}.
\end{equation}
Next, one shall insert \ref{1.41} into \ref{1.40} and collect the terms with the same power of $ \Theta $. Then the zeroth-order function $ \Psi_0 $ obeys the following equation,
\begin{equation}
\label{1.42}
\frac{1}{\rho} \diffp{}{\rho}\left(\rho \diffp{\Psi_0 \left(\rho, \zeta \right)}{\rho} \right) + 4 \i \diffp{\Psi_0\left(\rho, \zeta \right)}{\zeta} = 0.
\end{equation}

The equation \ref{1.42}, which is called the paraxial Helmholtz equation, is the starting point of traditional Gaussian beam theory. One can expect the solution of \ref{1.42} in the form of a Gaussian function with a width varying along the longitudinal direction, thus 
\begin{equation}
\label{1.43}
\Psi_0 \left(\rho, \zeta \right) = h\left(\zeta \right)\e^{-f\left(\zeta \right) \rho^2},
\end{equation}
where $ f\left(\zeta \right) $ and $ h\left(\zeta \right) $ are unknown complex functions that have to satisfy a condition $ f\left(0 \right) = h\left(0 \right) = 1 $. After plugging \ref{1.43} into \ref{1.42}, one gets the following equation,
\begin{equation}
\label{1.44}
-f\left(\zeta \right) h\left(\zeta \right) + \i \diff{h\left(\zeta \right)}{\zeta} + \rho^2 h\left(\zeta \right) \left(f\left(\zeta \right)^2 - \i \diff{f\left(\zeta \right)}{\zeta} \right) = 0.
\end{equation}
Since the equation \ref{1.44} has to hold for arbitrary value of $ \rho $, one may find two independent equations that are equivalent to \ref{1.44}
\begin{equation}
\label{1.45}
\frac{1}{f\left(\zeta \right)^2} \diff{f\left(\zeta \right)}{\zeta} + \i = 0, \qquad \frac{1}{f\left(\zeta \right) h\left(\zeta \right)} \diff{h\left(\zeta \right)}{\zeta} + \i = 0.
\end{equation}
It might be easily shown, that under specified conditions the solutions of equations \ref{1.45} have to be
\begin{equation}
\label{1.46}
h\left(\zeta \right) = f\left(\zeta \right), \qquad f\left(\zeta \right) = \frac{1}{\sqrt{1 + \zeta^2}} \e^{-\i \arctan{\zeta}},
\end{equation}
and therefore the complete expression for the zeroth-order wave function $ \Psi_0 \left(\rho, \zeta \right) $ is
\begin{equation}
\label{1.47}
\Psi_0 \left(\rho, \zeta \right) = \frac{1}{\sqrt{1 + \zeta^2}} \exp{\left[- \frac{\rho^2}{1 + \zeta^2} + \i \left(\frac{\rho^2 \zeta}{1 + \zeta^2} - \arctan{\zeta} \right) \right]}.
\end{equation}

In many situations, it is also useful to evaluate the expression \ref{1.47} in terms of Cartesian coordinates, in which the zeroth-order wave function $ \Psi_0 \left(\vec{r_\bot}, z \right) $ is
\begin{equation}
\label{1.48}
\Psi_0 \left(\vec{r_\bot}, z \right) = \frac{w_0}{w\left(z\right)} \exp{\left[- \frac{\vec{r_\bot}^2}{w\left(z \right)^2} + \i \left( k_z \frac{\vec{r_\bot}^2}{2 R\left(z \right)} - \varphi_\mathrm{G} \left( z\right) \right) \right]},
\end{equation}
where the parameters used to simplify the expression \ref{1.48} are defined as
\begin{equation}
\label{1.49}
w\left(z\right) = w_0 \sqrt{1 + \left(\frac{z}{z_\mathrm{R}}\right)^2}, \quad R\left(z \right) = z \left[1 + \left(\frac{z_\mathrm{R}}{z} \right)^2\right], \quad \varphi_\mathrm{G}\left(z\right) = \arctan{\left(\frac{z}{z_\mathrm{R}}\right)}.
\end{equation}
One shall discuss the physical meaning of the three parameters \ref{1.49}. The function $ w\left(z\right) $ represents the spot size parameter of the beam, that is the radius at which the laser intensity fall to $ 1/\e^2 $ of its axial value at any position $ z $ along the beam propagation. Note that the minimum of the spot size $ w(0) = w_0 $, consequently the focal spot is stationary and located at the origin of a Cartesian coordinate system. The second parameter, $ R\left(z \right) $ is known to be the radius of curvature of the beam's wavefront at any position $ z $ along the beam propagation. Note that $ \lim_{z \to 0^{\pm}} R(z) = \pm \infty $, therefore the beam behaves like a plane wave at focus as required. The last parameter, $ \varphi_\mathrm{G}\left(z\right) $, is the so-called Guoy phase of the beam at any position $ z $ along the beam propagation, which describes a phase shift in the wave as it passes through the focal spot.

Finally, by substituting \ref{1.48} for $ \Psi \left(\vec{r_\bot}, z \right) $ in \ref{1.36} and taking the real part of that complex quantity, one obtains the electric field of the so-called paraxial Gaussian beam,
\begin{equation}
\label{1.50}
\vec{E}\left(\vec{r_\bot}, z, t \right) = E_0 \frac{w_0}{w(z)} \exp\left(-\frac{\vec{r_\bot}^2}{w(z)^2}\right) \cos\left(\omega t - k_z \left(z + \frac{\vec{r_\bot}^2}{2 R(z)} \right) + \varphi_\mathrm{G}\left(z\right) \right) \mathrm{\vec{\hat{e}_x}}.
\end{equation}
Although given electric field \ref{1.50} describes the main features of the focused laser beam, it might be clearly seen that it does not satisfy Gauss's law (\ref{1.1}). The correct electric field cannot vary with the direction of its polarization or has to have at least two non-zero vector components. To fix that, one would have to solve the wave equation for the vector potential \ref{1.16} and afterwards exploit the solution to deduce all components of the electric and magnetic fields.   

In addition, since one assumed $ \Theta \ll 1 $, the solution \ref{1.50} is not accurate for strongly diverging beams. Since the divergence angle is inversely proportional to the beam waist, the previous condition yields $ w_0 \gg \lambda $. In other words, it means that \ref{1.50} is not valid for tightly focused laser beams and the need may arise for higher-order corrections. 