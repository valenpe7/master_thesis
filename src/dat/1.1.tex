The electromagnetic field is in theory represented by two vector quantities, the electric field intensity $ \vec{E}\left( \vec{r}, t \right) $ and the magnetic induction $ \vec{B}\left( \vec{r}, t \right) $. These vectors are finite and continuous functions of position $ \vec{r} $ and time $ t $. The description of electromagnetic phenomena in classical electrodynamics is provided by the set of Maxwell's equations. The microscopic variant with external sources in vacuum is formulated as follows,
\begin{equation}
\label{1.1}
\div{\vec{E}} = \frac{\rho}{\varepsilon_0},
\end{equation}
\begin{equation}
\label{1.2}
\div{\vec{B}} = 0,
\end{equation}
\begin{equation}
\label{1.3}
\rot{\vec{E}} + \diffp{\vec{B}}{t} = 0,
\end{equation}
\begin{equation}
\label{1.4}
\rot{\vec{B}} - \mu_0 \varepsilon_0 \diffp{\vec{E}}{t}= \mu_0 \vec{J},
\end{equation}
where $ \rho\left( \vec{r}, t \right) $ is total electric charge density and $ \vec{J}\left( \vec{r}, t \right) $ is total electric current density due to the motion of charged particles. These quantities may be continuous as well as discrete. As might be seen from Maxwell's equations (\ref{1.1} - \ref{1.4}), the charge density is the source of the electric field, whilst the magnetic field is produced by the current density. The lack of symmetry in Maxwell's equations (\ref{1.2}, \ref{1.3} are homogeneous) is caused by the experimental absence of magnetic charges and currents. The universal constants appearing in the Maxwell's equations (\ref{1.1}, \ref{1.4}) are the electric permittivity of vacuum $ \varepsilon_0 $ and the magnetic permeability of vacuum $ \mu_0 $.

The first equation, \ref{1.1}, is Gauss's law for electric field in the differential form. It states that the flux of the electric field through any closed surface is proportional to the total charge inside. The second equation, \ref{1.2}, is Gauss's law for magnetic field. It expresses the fact that there are no magnetic monopoles, so the flux of magnetic field through any closed surface is always zero. The third equation, \ref{1.3}, is Faraday's law describing how the electric field is associated with a time varying magnetic field. And the last equation, \ref{1.4}, is Amp\`ere's law with Maxwell's displacement current, which means that the time varying electric field causes the magnetic field. As a consequence, it predicts the existence of electromagnetic waves that can carry energy and momentum even in a free space.

To describe the effects of an electromagnetic field in the presence of macroscopic substances, the complicated distribution of charges and currents in matter over the atomic scale is not relevant. Thus one shall define a second set of auxiliary vectors that represent fields in which the material properties are already included in an average sense, the electric displacement $ \vec{D}\left( \vec{r}, t \right) $ and the magnetic vector $ \vec{H}\left( \vec{r}, t \right) $,
\begin{equation}
\label{1.5}
\vec{D} = \varepsilon_0 \vec{E} + \vec{P} = \varepsilon \vec{E},
\end{equation}
\begin{equation}
\label{1.6}
\vec{H} = \frac{\vec{B}}{\mu_0} - \vec{M} = \frac{\vec{B}}{\mu},
\end{equation}
where $ \vec{P}\left( \vec{r}, t \right) $ and $ \vec{M}\left( \vec{r}, t \right) $ are the vectors of polarization and magnetization, respectively. Note that the vectors of polarization and magnetization can be interpreted as a density of electric or magnetic dipole moment of the medium, therefore they are definitely associated with the state of a matter and vanish in vacuum. Similarly as in the case of free space, the factors $ \varepsilon $ and $ \mu $ are called electric permittivity of medium and magnetic permeability of medium. In general case, $ \varepsilon $ and $ \mu $ are tensors. The constitutive relations above (\ref{1.5}, \ref{1.6}) hold only if the medium is homogeneous and isotropic. For the sake of simplicity, only such materials will be considered in the following text. The macroscopic variant of Maxwell's equations is formulated as follows,
\begin{equation}
\label{1.7}
\div{\vec{D}} = \rho,
\end{equation}
\begin{equation}
\label{1.8}
\div{\vec{B}} = 0,
\end{equation}
\begin{equation}
\label{1.9}
\rot{\vec{E}} + \diffp{\vec{B}}{t} = 0,
\end{equation}
\begin{equation}
\label{1.10}
\rot{\vec{H}} - \diffp{\vec{D}}{t} = \vec{J},
\end{equation}
where $ \rho\left(\vec{r}, t \right) $ and $ \vec{J}\left(\vec{r}, t \right) $ now stand for only external electric charge and current density, respectively.

By combining the time derivative of the equation \ref{1.7} with the divergence of the equation \ref{1.10}, one obtains the following relation between the electromagnetic field sources,
\begin{equation}
\label{1.11}
\diffp{\rho}{t} + \div{\vec{J}} = 0.
\end{equation}
The important result \ref{1.11}, which is frequently referred to as the equation of continuity, expresses nothing but the conservation of total electric charge in an isolated system. In other words, the time rate of change of the electric charge in any closed surface is balanced by the electric current flowing through the surface.