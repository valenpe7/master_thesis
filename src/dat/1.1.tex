In order to achieve fusion, the reacting nuclei have to get very close to each other to activate the strong nuclear force. However, there is a large electrostatic repulsion between them as they come together because the protons in nuclei are positively charged. There are several possibilities of overcoming this barrier. One can for example utilize the initial kinetic energy of the chaotic thermal motion of the particles. In thermal equilibrium, the Coulomb collisions redistribute the kinetic energy among the plasma particles, and fusion reactions will eventually occur after a sequence of collisions. This approach is called thermonuclear fusion.

Reactions which could be usable in the case of the energy production on Earth must have high cross-section values at relatively low required incident energies. Furthermore, the economic viability and competitiveness is crucial for commercial use of fusion power. In other words, the reaction must be able to cover obviously no less than the energetic costs incurred for its ignition. The word ignition refers to the moment when a controlled fusion reaction generates as much or more energy than is needed to spark the reaction. In regards to each of these attributes, the best reaction candidate appears to be the reaction of two hydrogen isotopes - deuterium and tritium,
\begin{equation}
\nucl{2}{1}{H} + \nucl{3}{1}{H} \rightarrow \nucl{4}{2}{He} \ \mathrm{(3{,}5 \ MeV)} + \nucl{1}{0}{n} \ \mathrm{(14{,}1 \ MeV)}.
\end{equation}

Deuterium has quite rich natural abundance in Earth's oceans, thus it is commonly available in sufficient quantity. The efficient extraction from seawater is supposed to be technically ready \cite{bradshaw}. On the contrary, naturally occurring tritium is extremely rare due to its short half-life, therefore the fuel cycle requires the breeding of tritium. The production is possible with irradiation of lithium by fusion neutrons using the following reaction,
\begin{equation}
\nucl{6}{3}{Li} + \nucl{1}{0}{n} \rightarrow \nucl{4}{2}{He} \ \mathrm{(2{,}1 \ MeV)} + \nucl{3}{1}{H} \ \mathrm{(2{,}7 \ MeV)}.
\end{equation}

Lithium is widely distributed in the Earth's crust and seawater. Consequently, both fusion fuel resources are relatively easily accessible, uniformly geographically distributed and have near unlimited availability. The product of their reaction is a neutron and a helium nucleus, of which the latter product is not radioactive. These attributes and the potential to produce large amounts of carbon-free energy with almost no environmental impact predetermine nuclear fusion to become a global energy source.