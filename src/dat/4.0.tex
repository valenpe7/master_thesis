The investigation of laser-matter interaction also involves exploring of specific themes of the ultra-relativistic regime, which requires extremely high intensities of the external field. These intensities, that are experimentally inaccessible at present, could be potentially achieved by tight-focusing and that would allow a broad spectrum of many multidisciplinary applications.

As mentioned in the previous chapter, various aspects of electromagnetic interaction are usually studied using sophisticated numerical simulation codes. Vast majority of these codes, however, use a paraxial approximation (closer described in chapter 1) to prescribe the laser fields at the boundaries, and afterwards, a field solver guides the beam across the simulation domain. As already mentioned, the paraxial approximation is valid only if the angular spectrum of laser pulse is sufficiently narrow, therefore it is not possible to simulate tightly focused laser beams using this approach. As might be seen later, paraxial approximation in this case leads to a distorted field profiles which have strong impact on the results of laser-matter interaction.

Several interesting solutions, how to simulate strongly focused beams, have been already proposed [source]. Within this work, a simple and efficient algorithm for a Maxwell consistent calculation of the electromagnetic fields at the boundaries of the computational domain [source] (also called laser boundary conditions) has been used and implemented into the PIC code EPOCH [source]. Note, that this algorithm is able to describe laser beams with arbitrary shape.