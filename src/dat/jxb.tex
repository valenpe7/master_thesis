For the relativistic laser beams, the high-frequency $ \vec{v} \times \vec{B} $ component of the Lorentz force becomes important and may contribute to plasma heating. Similarly to the Brunel's vacuum heating, discussed in the previous subsection, $ \vec{J} \times \vec{B} $ heating requires very steep plasma density gradients. On the contrary, this heating scenario works for any polarization apart from circular and is most efficient for the laser pulses normally incident onto the target surface, so there is no oscillating component of the electric field perpendicular to plasma.

In the case of $ \vec{J} \times \vec{B} $ heating, the absorbed energy is carried by bunches of hot electrons that are ejected twice per laser period. This fact may help to distinguish between the effects of Brunel's and $ \vec{J} \times \vec{B} $ heating. Again, the ejected electrons are pushed back into the plasma by self-consistent electric field generated by charged particles without restoring forces behind the skin layer.