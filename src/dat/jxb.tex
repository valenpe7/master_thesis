For the relativistic laser beams, the term $ \vec{v} \times \vec{B} $ in Lorentz force becomes important and may contribute to plasma heating. On the contrary to Brunel's vacuum heating, discussed in the previous subsection, this heating scenario takes place at the normal incidence of laser pulse onto the target surface, where the oscillating component of the electric field perpendicular to plasma is zero.

In the case of $ \vec{J} \times \vec{B} $ heating, the absorbed energy is carried by bunches of hot electrons that are ejected once per laser period. This fact may help to distinguish between the effects of Brunel's and $ \vec{J} \times \vec{B} $ heating. Again, the ejected electrons are pushed back into the plasma by self-consistent electric field generated by charged particles without restoring forces behind the skin layer.