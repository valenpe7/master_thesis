The non-relativistic case is based on neglecting the term $ \vec{v} \times \vec{B} $ in the Newton's equation of motion (\ref{2.2.5}). However, this approximation is not valid for particles moving with velocities close to the velocity of light $ c $. The derivation of the relativistic ponderomotive force presented in this subsection follows \cite{Bauer1995, Mulser2010}.

A system containing a charged particle in an arbitrary electromagnetic field can be rigorously described using Lagrangian mechanics. The relativistic Lagrangian $ L \left( \vec{r}, \vec{v}, t \right)  $ for this case is given by the following expression,
\begin{equation}
\label{2.5.3.1}
L = -\frac{m_s c^2}{\gamma} - q_s \left(\Phi - \vec{v} \cdot \vec{A} \right), \quad \gamma = \left( 1 - \frac{v^{2}}{c^{2}} \right)^{-\frac{1}{2}}.
\end{equation}
For the traveling monochromatic wave, the Lagrangian $ L \left( \vec{r}, \vec{v}, t \right)  $ in \ref{2.5.3.1} can be transformed to $ L \left( \varphi \right)  $ using the wave phase $ \varphi = \vec{k} \cdot \vec{r} - \omega t $. The reason of doing this is that the transformed Lagrangian can be used to obtain the average over the wave period of motion,
\begin{equation}
\label{2.5.3.2}
\mathcal{L}_0 \left( \varphi \right) = \frac{1}{2 \pi} \int \limits_{\varphi}^{\varphi + 2\pi} \mathcal{L} \left( \varphi^{\: \prime} \right) \mathrm{d} \varphi^{\: \prime}, \quad \mathcal{L} \left( \varphi \right) = L \left( \varphi \right) \left( \diff{\varphi}{t} \right)^{-1}.
\end{equation}
Note that the Langrangian density \ref{2.5.3.2} depends only on the quantities at the center of oscillation $ \vec{r}_0 $ and $ \vec{v}_0 $ defined through $ \varphi $. Then the motion of the oscillation center is governed by the Lagrange equations,
\begin{equation}
\label{2.5.3.3}
\diff{}{t} \left( \diffp[]{L_0}{\vec{v}_0} \right) -  \diffp[]{L_0}{\vec{r}_0} = 0, \quad L_0 \left( \varphi \right) = \mathcal{L}_0 \left( \varphi \right) \left( \diff{\varphi}{t} \right).
\end{equation}
The complete proof of assertions above can be found in \cite{Bauer1995}.

The relativistic ponderomotive force in the oscillation center system can be easily found from the Lagrange equations \ref{2.5.3.3},
\begin{equation}
\vec{F}_{\mathrm{p}} \equiv \diff{}{t} \left( \diffp[]{L_0}{\vec{v}_0} \right) = -c^2 \grad{m_{\mathrm{eff}}},
\end{equation}
where the so-called effective mass $ m_{\mathrm{eff}} $ is the space and time dependent quantity that has been introduced as follows,
\begin{equation}
\label{2.5.3.4}
m_{\mathrm{eff}} = - \mathcal{L}_0 \frac{\gamma_0}{c^2} \left( \diff{\varphi}{t} \right), \quad  \gamma_0 = \left(1 - \frac{v_0^{2}}{c^{2}} \right)^{-\frac{1}{2}}.
\end{equation}
Note that the assertion \ref{2.5.3.4} holds for any electromagnetic field in vacuum in which it is possible to define the oscillation center. For ordinary conditions, the effective mass $ m_{\mathrm{eff}} $ can be rewritten as
\begin{equation}
m_{\mathrm{eff}} = m_s \sqrt{1 + \frac{q_s A^2}{\alpha m_s^2 c^2}}
\end{equation}
with $ \alpha = 1 $ for circular and $ \alpha = 2 $ for linear polarization of the electromagnetic wave. 