Laser light in the plasma can be also absorbed by resonance absorption. It is a linear process in which an incident laser wave is partially absorbed by conversion into an electron wave at the critical density of plasma.

Resonance absorption takes place when a p-polarized laser pulse is obliquely incident on a plasma with an inhomogeneous density profile. A component of the laser wave electric field perpendicular to the target surface then resonantly excites an electron plasma wave also along the plasma density gradient, thus a part of the laser wave energy is transferred into the electrostatic energy of the electron plasma wave. This wave propagates into the underdense plasma and it is damped either by collision or collisionless damping mechanisms. Consequently, energy is further converted into thermal energy which heats the plasma.

In contrast to collisional absorption, resonance absorption is the main absorption process for high laser intensities and long wavelengths. The efficiency of resonance absorption can also be higher for hot plasma, low critical density, or short plasma scale-length.
