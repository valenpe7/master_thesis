Despite a wide variety of economic drives, the world energy consumption is continuously growing. Fossil fuels, crude oil and natural gas reserves are slowly shrinking and renewable resources alone will definitely not be able to meet the global energy demand. This energetic deficit might become a serious obstacle in the further sustainable development of human society. Therefore it is essential to find an alternative energy source; preferably one which could finally solve all of the aforementioned problems and, in addition, its utilization would be environmentally friendly.

During the 1950s, the demand for alternative and more feasible energy sources for the near future has led to the intensive research in the field of nuclear power engineering. Particularly, scientists became interested in a peaceful use of thermonuclear fusion. Firstly, they have tried to make required conditions for the fusion plasma using convenient geometry of magnetic fields. Early success, which has been expected by the scientific community, however, has not been achieved. Several years later, this fact also contributed to the idea of using lasers, at that time an entirely new source of intense radiation, to ignite thermonuclear fusion. It soon turned out, however, that by using lasers, a wide range of phenomena which might seriously complicate the ignition is inevitable as well.

One of the most negative effects in terms of inertial confinement fusion is the generation of hot electrons in a coronal plasma, in which the laser energy is transmitted to the kinetic energy of plasma. These electrons significantly preheat the core of the fuel target, which makes the required compression of plasma more difficult. Therefore, the laser-plasma interactions in this context have been intensively studied \cite{tikhonchuk}. At the same time, new, more sophisticated methods, whereby laser energy is deposited into the target as efficiently as possible and with a minimal production of hot electrons, have been investigated as well \cite{batani}.

The following work is focused on the interaction of laser radiation with plasma for the conditions according to current experiments in the Prague Asterix Laser System (PALS) facility, where the possibilities of fuel ignition are studied using a high-power iodine laser. More specifically, the interaction in terms of laser energy absorption efficiency, hot electron production and laser light scattering in a regime relevant to the shock ignition scheme, which has been proposed recently \cite{betti}, have all been studied.

The work is structured as follows: the first chapter provides a brief description of the thermonuclear fusion, including the conditions of its ignition for both basic approaches. The major part is devoted to inertial fusion. The second chapter summarizes the elementary knowledge of plasma physics and physics of laser-plasma interaction. The third chapter describes numerical simulation as an essential tool in modern science and engineering. Especially, one of the most popular methods in plasma physics, particle-in-cell (PIC), is thoroughly discussed. The characteristics of the code EPOCH\cite{bennett}, which has been used for the simulations within this work, can be found in the last section of this chapter. The last chapter demonstrates the results and benefits of this work. It contains the implementation of the boundary conditions in the code EPOCH and the results of performed two-dimensional simulations.

Although the most convenient unit system for most plasma applications is the Gaussian cgs system, throughout this work the SI (System Internationale) units are used, unless explicitly stated.

Symbols in bold represent vector quantities, and symbols in italics represent scalar quantities, unless otherwise indicated.
