Since the first demonstration of pulsed laser in 1960 \cite{Maiman1960}, the intensive research and development in the field of laser technology have seen a tremendous progress. Pulse compression and amplification techniques, such as CPA \cite{StricklandMourou1985}, OPCPA \cite{Dubietis1992} and lately RBA \cite{Malkin1999}, have enabled a generation of ultra-short laser pulses with intensities exceeding $ 10^{22} \ \mathrm{W/cm^{2}} $. Pulses in this regime provide an unprecedented capability for basic research (e.g. high-energy density physics, warm dense matter, plasma optics, laboratory astrophysics) as well as a broad range of groundbreaking applications in diverse fields (e.g. coherent diffractive imaging, X-ray diffraction and spectroscopy, production of compact sources for radiotherapy, inertial confinement fusion).

The peak laser intensities are typically increased by enhancing the produced laser properties, either by lowering the pulse duration or increasing the pulse energy. This approach comes at great cost since it requires a higher level of complexity for the laser chain \cite{Fuchs2014}. More effective way to increase the laser intensity is to reduce the focal spot size. However, the conventional solid state optics are inappropriate in the case of tight-focusing (expensive, susceptible to damage from solid target debris, sensitive to small misalignments). Nevertheless, it seems that many drawbacks might be in future overcome by using a plasma-based focusing optics. Therefore, the interaction of tightly focused laser beams with matter is currently attracting much attention \cite{Popov2008, Popov2009, Lifschitz2016, Yan2005}. 

This work is structured as follows: the first chapter provides a brief introduction to the classical electromagnetic field theory, including the mathematical derivation of the paraxial Gaussian beam formula. The second chapter summarizes the elementary knowledge of plasma physics and physics of laser-plasma interaction. In the third chapter, one of the most popular numerical methods in plasma physics, particle-in-cell (PIC), is thoroughly discussed. The characteristics and features of the code EPOCH \cite{bennett}, which has been used for the simulations within this work, can be found in the last section of this chapter. The fourth chapter is devoted to the tight-focusing of laser pulses, including a description of algorithm for rigorous calculation of electromagnetic fields at boundaries of simulation domain \cite{Thiele2016}. This chapter also contains the details about the implementation of the algorithm as well as thorough evaluation of its correctness and the correctness of the implementation. At the end, one may find the overview of currently used experimental methods for tight-focusing. The last chapter demonstrates the results of several large-scale two-dimensional simulations employing tightly focused laser beams interacting with solid targets.

Although the most convenient unit system for most plasma applications is the Gaussian cgs system, throughout this work the SI (System Internationale) units are used, unless explicitly stated. Symbols in bold represent vector quantities, and symbols in italics represent scalar quantities, unless otherwise indicated.
