Since the first demonstration of pulsed laser in 1960 \cite{Maiman1960}, intense research and development in the field of laser technology have seen a tremendous progress. Pulse compression and amplification techniques, such as chirped-pulse amplification (CPA) \cite{StricklandMourou1985}, optical parametric chirped-pulse amplification (OPCPA) \cite{Dubietis1992} and lately backward Raman amplification (BRA) \cite{Malkin1999}, have enabled generation of ultra-short laser pulses with intensities exceeding $ 10^{21} \ \mathrm{W/cm^{2}} $ \cite{Danson2015}. Pulses in this regime provide an unprecedented capability for basic research (e.g. high-energy density physics, relativistic plasma physics and optics, laboratory astrophysics \cite{Council2003, Graziani2014, Lebedev2007, Bridgman1958, Krehl2008, Andreev2006, Weber2013, Bulanov2015, Zakharov2003}) as well as a broad range of groundbreaking applications in diverse fields (e.g. coherent diffractive imaging, X-ray diffraction and spectroscopy, pulse radiolysis, production of compact sources for radiotherapy \cite{Zewail2010, Bulanov2004, Malka2004}).

The peak laser intensities are typically amplified by increasing the input energy. This approach comes at great cost since it requires a higher level of complexity for the laser chain \cite{Fuchs2014}. A more effective route would be focusing to sub-wavelength level (tight-focusing) since the laser intensity is proportional to the square of the inverse of the focal spot size \cite{Kon2010}. When using a conventional solid state optics, however, the laser beam diameter has to be relatively large in order to keep the energy density on the optical components below the damage threshold. Furthermore, additional care has to be taken to protect the optics from the target debris due to their short focal length \cite{Liu2011}. The solid state optics are thus inherently inappropriate in this case. Nevertheless, it seems that many drawbacks might be in future overcome by using a plasma-based focusing optics. In addition, curved plasma mirrors may enhance the spatial and temporal contrast ratio of laser pulse which is crucial for many applications of laser-matter interaction \cite{Fuchs2014}. Therefore, the tight-focusing and the interaction of tightly focused laser beams with matter are currently attracting much attention \cite{Popov2008, Popov2009, Lifschitz2016, Yan2005}. 

This work is structured as follows: the first chapter provides a brief introduction to the classical electromagnetic field theory, including the mathematical derivation of the paraxial Gaussian beam formula. The second chapter summarizes the elementary knowledge of plasma physics and physics of laser-plasma interaction. In the third chapter, one of the most popular numerical methods in plasma physics, particle-in-cell (PIC), is thoroughly discussed. The characteristics and features of the code EPOCH \cite{bennett}, which has been used for the simulations within this work, can be found in the last section of this chapter. The fourth chapter is devoted to the tight-focusing of laser pulses, including a description of algorithm for rigorous calculation of electromagnetic fields at boundaries of simulation domain \cite{Thiele2016}. This chapter also contains the details about the implementation of the algorithm as well as thorough evaluation of its correctness and the correctness of the implementation. At the end, one may find the overview of currently used experimental methods for tight-focusing. The last chapter demonstrates the results of several large-scale two-dimensional simulations employing tightly focused laser beams interacting with solid targets.

Results...

Although the convenient unit system for most plasma applications is the Gaussian cgs system, the SI (System Internationale) units are used throughout this work, unless explicitly stated. Symbols in bold represent vector quantities, and symbols in italics represent scalar quantities, unless otherwise indicated.
