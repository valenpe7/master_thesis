In this section, general properties of the electromagnetic wave propagation in magnetized plasmas are described. Particularly, consider in some detail the waves traveling parallel to and perpendicular to magnetic field. 

First, the dispersion relation is derived from the hydrodynamic plasma equations. Since one assume plasma response to a high frequency field, the ions are treated as a stationary, neutralizing background. Thermal motion of particles is also ignored, thus the pressure term in \ref{2.2.3} can be neglected. According to previous assumptions, one shall solve the following set of equations,
\begin{equation}
\label{2.3.1}
\diffp{n_e}{t} + \nabla \cdot \left(n_e \vec{u}_e \right) = 0,
\end{equation}
\begin{equation}
\label{2.3.2}
m_e n_e \left[ \diffp{\vec{u}_e}{t} + \left(\vec{u}_e \cdot \nabla \right) \vec{u}_e \right] = - e n_e \left(\vec{E} + \vec{u}_e \times \vec{B} \, \right).
\end{equation}
The symbol $ e $ denotes the elementary charge. 

To obtain the wave equations for the oscillating electric and magnetic field, Faraday's law (\ref{1.3}) and Ampere's law (\ref{1.4}) are also needed. Next, the system of equations will be linearized by using the methods of perturbation theory. Consider a small perturbations from the stationary state,
\begin{equation}
\label{2.3.4}
n_{e} = n_{e0} + \delta n_{e}, \quad \vec{u}_{e} = \vec{u}_{e0} + \delta \vec{u}_{e}, \quad \vec{B} = \vec{B}_{0} + \delta \vec{B}, \quad \vec{E} = \vec{E}_{0} + \delta \vec{E},
\end{equation}
where $ \vec{u}_{e0} $ and $ \vec{E}_{0} $ are obviously identically equal to zero vector. After substituting perturbed quantities (\ref{2.3.4}) into initial system of equations and performing Fourier transform one obtains
\begin{equation}
\label{2.3.5}
\delta n_{e} = \mathrm{i} \frac{n_{e0}}{\omega} \vec{k} \cdot \delta \vec{u}_{e},
\end{equation}
\begin{equation}
\label{2.3.6}
\delta \vec{u}_{e} = - \mathrm{i} \frac{e}{m_{e} \omega} \delta \vec{E} - \mathrm{i} \frac{e}{m_{e} \omega} \delta \vec{u}_{e} \times \vec{B}_{0},
\end{equation}
\begin{equation}
\label{2.3.7}
\delta \vec{B} = \frac{1}{\omega} \vec{k} \times \delta \vec{E},
\end{equation}
\begin{equation}
\label{2.3.8}
\delta \vec{E} = - \frac{1}{\varepsilon_{0} \mu_{0} \omega} \vec{k} \times \delta \vec{B} + \mathrm{i} \frac{e n_{0}}{\varepsilon_{0} \omega} \delta \vec{u}_{e}.
\end{equation}

Note that the equation for density perturbation \ref{2.3.5} may be ignored. Eliminating $ \delta \vec{B} $ and $ \delta \vec{u}_{e} $ from the equation \ref{2.3.8} one gets the equation for perturbation of the electric field,
\begin{equation}
\begin{split}
\label{2.3.9}
& \left( \omega^{2} - \omega_{pe}^{2} - c^{2} k^{2} \right) \delta \vec{E} + \mathrm{i} \frac{\omega_{ce}}{\omega} \left( \omega^{2} - c^{2} k^{2} \right) \delta \vec{E} \times \vec{e}_{B} \: + \\[5pt]
& \qquad + c^{2} \left( \vec{k} \cdot \delta \vec{E} \right) \vec{k} + \mathrm{i} \frac{\omega_{ce}}{\omega} c^{2} \left( \vec{k} \cdot \delta \vec{E} \right) \vec{k} \times \vec{e}_{B} = 0.
\end{split}
\end{equation}
Here $ \vec{e}_{B} $ is a unit vector in the direction of the magnetic field,
\begin{equation}
\label{2.3.10}
\vec{e}_{B} = \frac{\vec{B}_{0}}{B_{0}}.
\end{equation}
If one choose, without the loss of generality, the coordinate system where $ \vec{B}_{0} = (0, 0, B_{0}) $ and $ \vec{k} = (k \sin \alpha, 0, k \cos \alpha) $, one obtains the equation $ \mathbb{M} \cdot \delta \vec{E} = 0 $ with the matrix
\begingroup
\renewcommand*{\arraystretch}{1.8}
\begin{equation}
\label{2.3.11}
\mathbb{M} =  \begin{pmatrix}
 \omega^{2} - \omega_{pe}^{2} - c^{2} k^{2} \cos^{2} \alpha & \mathrm{i} \dfrac{\omega_{ce}}{\omega} \left( \omega^{2} - c^{2} k^{2} \right)  & c^{2} k^{2} \cos \alpha \sin \alpha \\
 - \mathrm{i} \dfrac{\omega_{ce}}{\omega} \left( \omega^{2} - c^{2} k^{2} \cos^{2} \alpha \right) & \omega^{2} - \omega_{pe}^{2} - c^{2} k^{2} & - \mathrm{i} \dfrac{\omega_{ce}}{\omega} c^{2} k^{2} \cos \alpha \sin \alpha \\
 c^{2} k^{2} \cos \alpha \sin \alpha & 0 & \omega^{2} - \omega_{pe}^{2} - c^{2} k^{2} \sin^{2} \alpha
 \end{pmatrix}.
\end{equation} 
\endgroup
The system of equations has non-trivial solution if and only if $ \det \left( \mathbb{M} \right) = 0 $. This condition leads to the desired dispersion relation for an arbitrary angle $ \alpha $,
\begin{equation}
\begin{split}
\label{2.3.12}
& \ \ \left[ \left( \omega^{2} - \omega_{pe}^{2} - c^{2} k^{2} \cos^{2} \alpha \right) \left( \omega^{2} - \omega_{pe}^{2} - c^{2} k^{2} \alpha \right) - \left(  \frac{\omega_{ce}}{\omega} \right)^{2} \left( \omega^{2} - c^{2} k^{2} \right) \left( \omega^{2} - c^{2} k^{2} \cos^{2} \alpha \right) \right]\\[5pt]
& \left( \omega^{2} - \omega_{pe}^{2} - c^{2} k^{2} \sin^{2} \alpha \right) - c^{4} k^{4} \cos^{2} \alpha \sin^{2} \alpha \left[ \left( \omega^{2} - \omega_{pe}^{2} - c^{2} k^{2} \right) - \left( \frac{\omega_{ce}}{\omega} \right)^{2} \left( \omega^{2} - c^{2} k^{2} \right) \right] = 0.\\[5pt]
\end{split}
\end{equation}

Now, it is necessary to find dispersion relations for the two simplest cases, propagation along and perpendicular to the magnetic field. For the waves propagating along $ \vec{B_{0}} $ is $ \alpha = 0 $ and the dispersion relation \ref{2.3.12} gets relatively simple form,
\begin{equation}
\label{2.3.13}
\left( \omega^{2} - \omega_{pe}^{2} \right) \left[ \left( \omega^{2} - \omega_{pe}^{2} - c^{2} k^{2} \right)^{2}  - \left( \frac{\omega_{ce}}{\omega} \right)^{2} \left( \omega^{2} - c^{2} k^{2} \right)^{2} \right] = 0.
\end{equation}
The equation \ref{2.3.13} has three solutions. The first describes plasma oscillations at frequency $ \omega = \omega_{pe} $. The second and third solutions give right-handed (R) and left-handed (L) circularly polarized waves,
\begin{equation}
\label{2.3.14}
N_{R}^{2} = 1 - \frac{\left( \omega_{pe} / \omega\right)^{2} }{1 - \omega_{ce} / \omega}, \qquad N_{L}^{2} = 1 - \frac{\left( \omega_{pe} / \omega\right)^{2} }{1 + \omega_{ce} / \omega}.
\end{equation}
The symbol $ N $ stands for the index of refraction, which is more useful for describing how the waves propagate through medium.

In a similar manner, for the waves propagating perpendicular to $ \vec{B_{0}} $ is $ \alpha = \pi/2 $ and the dispersion relation \ref{2.3.12} has the following form,
\begin{equation}
\label{2.3.15}
\left( \omega^{2} - \omega_{pe}^{2} - c^{2} k^{2} \right) \left[ \left( \omega^{2} - \omega_{pe}^{2} \right) \left( \omega^{2} - \omega_{pe}^{2} - c^{2} k^{2} \right) - \omega_{ce}^{2} \left( \omega^{2} - c^{2} k^{2} \right) \right] = 0.
\end{equation}
The equation \ref{2.3.15} has two solutions, which give ordinary (O) and extraordinary (X) waves,
\begin{equation}
N_{O}^{2} = 1 - \left( \frac{\omega_{pe}}{\omega} \right)^{2}, \qquad N_{X}^{2} = 1 - \left( \frac{\omega_{pe}}{\omega} \right)^{2} \frac{1 - \left( \omega_{pe} / \omega\right)^{2}}{1 - \left( \omega_{pe} / \omega\right)^{2} - \left( \omega_{ce} / \omega\right)^{2}}.
\end{equation}
The ordinary wave corresponds to a linearly polarized wave with electric field lying along the magnetic field direction, so that the motion remains unaffected. The extraordinary wave has the electric fields that are perpendicular to magnetic field, but with components both perpendicular and parallel to the wave vector.

The important properties of these waves are distinguished by their cut-offs ($ N \rightarrow 0 $) and resonances ($ N \rightarrow \infty $). In the vicinity of the resonance there is a total absorption, at a cut-off frequency there is a total reflection of incident waves. All of the cut-offs and resonances of waves (including ions) are listed in the table \ref{2.3.16}.
\begingroup
\renewcommand*{\arraystretch}{2.5}
\begin{table}[h!]
\centering
\begin{tabular}{ c | c | c }
Wave & Cut-offs & Resonances \\ \hline \hline
R & $ \omega_{R} = \dfrac{1}{2} \omega_{ce} + \dfrac{1}{2} \sqrt{\omega_{ce}^{2} + 4 \omega_{pe}^{2} } $ & $ \omega_{ce} = \dfrac{e B_{0}}{m_{e}} $ \\ \hline
L & $ \omega_{L} = - \dfrac{1}{2} \omega_{ce} + \dfrac{1}{2} \sqrt{\omega_{ce}^{2} + 4 \omega_{pe}^{2}} $ & $ \omega_{ci} = \dfrac{Z e B_{0}}{m_{i}} $ \\ \hline
O & $ \omega = \sqrt{\omega_{pe}^{2} + \omega_{pi}^{2}} $ & - \\ \hline
X & $ \omega = \omega_{R}, \quad \omega = \omega_{L} $ & $ \omega_{lh} = \sqrt{\omega_{ce} \: \omega_{ci}}, \quad  \omega_{uh} = \sqrt{\omega_{pe}^{2} + \omega_{ce}^{2}}  $ \\
\end{tabular}
\caption{Summary of cut-offs and resonances for all the principal waves. Note that $ \omega_{lh} $ and $ \omega_{uh} $ are so-called lower hybrid frequency and upper hybrid frequency, respectively.}
\label{2.3.16}
\end{table}
\endgroup
