Because of the relatively large ion mass, currently achievable laser intensities are not strong enough to accelerate protons or heavier ions directly to sufficiently high energies. However, ions typically respond on slowly varying electric fields in plasma arising from the strong charge separations induced by various phenomena that take place during the interaction of intense laser beam with matter. As one may see later, ions can be either accelerated in the vicinity of the laser focal spot at the front side of the target as well as in the vicinity of the target-vacuum boundary at the rear side. In the last two subsections of this chapter, the two main mechanisms for ion acceleration in laser-plasma interactions are briefly described.