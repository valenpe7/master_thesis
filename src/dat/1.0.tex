The energy contained in an atomic nucleus can be obtained in two different ways. The first one consists in the splitting of heavy atomic nuclei into lighter elements. However, fission reaction entails some safety risks. In addition, the products of this reaction themselves are often unstable, and therefore give off relatively large amounts of radiation which cause some serious difficulties with their long term storage.

The second approach is nuclear fusion. It is the exact opposite process in which two or more atomic nuclei come very close and then join together to form a new type of atomic nucleus. If the mass of the product is less than the sum of the masses of the initial fusing nuclei, an amount of energy corresponding to this mass deficit is released in accordance with Einstein's famous relation. This chapter provides a brief insight into the challenges of nuclear fusion, particularly to one of the main approaches to achieve it - inertial confinement fusion.