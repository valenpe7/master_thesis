The laser beam filamentation and self-focusing result from the same physical processes. These instabilities occur for densities less than the critical density where the laser beam couples to an ion acoustic perturbation. The filamentation and self-focusing correspond to the growth of zero-frequency density perturbations and are caused by variations in either the laser intensity across the beam regions or plasma density.

In the higher laser intensity region, plasma is pushed aside in the radial direction of the beam due to the ponderomotive force. This reduces the density locally and increases the index of refraction of the plasma. The resulting index of refraction is seen by the laser pulse as a focusing lens, thus prevents it from spreading. Consequently, the laser intensity increases there still further, which completes the unstable feedback loop.

Notice that in addition to ponderomotive driven filamentation and self-focusing, there is a variety of mechanism that leads to a change of the refractive index in laser-plasma interactions. These include the collisions, a thermal, or relativistic effects.

The undesirable effects of filamentation and self-focusing are induced instabilities which can produce hot electrons that preheat the fuel target core or reduce laser energy absorption. It is, of course, a serious problem in laser fusion research.