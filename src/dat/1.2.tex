The first-order partial differential Maxwell's equations can be effectively converted to a smaller number of second-order equations by introducing electrodynamic potentials. Hence, one can express the electric and magnetic field as follows,
\begin{equation}
\label{1.12}
\vec{E} = -\grad{\Phi} - \diffp{\vec{A}}{t},
\end{equation}
\begin{equation}
\label{1.13}
\vec{B} = \rot{\vec{A}},
\end{equation}
where $ \Phi\left(\vec{r}, t \right) $ is the scalar potential and $ \vec{A}\left(\vec{r}, t \right) $ is the vector potential of the corresponding fields. One can clearly see that using the definitions \ref{1.12}, \ref{1.13}, six vector components are replaced by only four potential functions and two Maxwell's homogeneous equations (\ref{1.8}, \ref{1.9}) are fulfilled identically. 

However, by definitions \ref{1.12}, \ref{1.13}, $ \Phi\left(\vec{r}, t \right) $ and $ \vec{A}\left(\vec{r}, t \right) $ are not defined uniquely, thus an infinite number of potentials which lead to the same fields may be constructed. To avoid that, one has to impose a supplementary condition, for example
\begin{equation}
\label{1.14}
\div{\vec{A}} + \mu \varepsilon \diffp{\Phi}{t} = 0.
\end{equation}
The condition \ref{1.14} is called the Lorenz gauge. Lorenz gauge is commonly used in electromagnetism because of its independence of the coordinate system. Furthermore, it leads to the following uncoupled equations,
\begin{equation}
\label{1.15}
\laplace{\Phi} - \mu \varepsilon \diffp[2]{\Phi}{t} = -\frac{\rho}{\varepsilon},
\end{equation}
\begin{equation}
\label{1.16}
\laplace{\vec{A}} - \mu \varepsilon \diffp[2]{\vec{A}}{t} = -\mu \vec{J},
\end{equation}
that are in all respects equivalent to the Maxwell's equations and in many situations much simpler to solve.

Equations \ref{1.15}, \ref{1.16} correspond to the inhomogeneous wave equations for scalar potential $ \Phi\left(\vec{r}, t \right) $ and vector potential $ \vec{A}\left(\vec{r}, t \right) $. Their general solutions are given by the following expressions,
\begin{equation}
\label{1.17}
\Phi\left(\vec{r}, t \right) = \frac{1}{4 \pi \varepsilon} \int \frac{\rho\left(\vec{r^{\: \prime}}, t^{\: \prime} \right)}{\norm{\vec{r} - \vec{r^{\: \prime}}}} \mathrm{d} V,
\end{equation}
\begin{equation}
\label{1.18}
\vec{A}\left(\vec{r}, t \right) = \frac{\mu}{4 \pi} \int \frac{\vec{J}\left(\vec{r^{\: \prime}}, t^{\: \prime} \right)}{\norm{\vec{r} - \vec{r^{\: \prime}}}} \mathrm{d} V,
\end{equation}
where $ \mathrm{d} V $ is a volume element and $ \norm{.} $ stands for the standard Euclidean norm. Note that the solutions \ref{1.17}, \ref{1.18} are dependent only on charge and current densities at position $ \vec{r^{\: \prime}} $ at so-called retarded time $ t^{\: \prime} = t - \sqrt{\mu \epsilon} \norm{\vec{r} - \vec{r^{\: \prime}}} $ which takes into account the finite velocity of the wave. In other words, the fields at the observation point $ \vec{r} $ at the time $ t $ are proportional to the sum of all the electromagnetic waves that leave the source elements at point $ \vec{r^{\: \prime}} $ at the retarded time $ t^{\: \prime} $.