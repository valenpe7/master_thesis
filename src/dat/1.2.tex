The first attempts to make the fusion reaction in terrestrial conditions are heading into the 1930s. The idea was to collide deuteron beams with heavy water vapor using a particle accelerator. This approach is referred to as beam fusion \cite{oliphant}. The resulting reaction led to the release of a large number of particles including among other a helium nuclei. However, it was soon revealed that this would not be a feasible way for power production. Even though the process also releases heat, much more energy is consumed for accelerating the particles. The deuteron beams lose their energy by the ionization or elastic Coulomb collisions with target nuclides whose cross-section is several orders of magnitude higher. Thus the probability of nuclear fusion is almost negligible.

After the worldwide declassification of fusion research in 1956, J. D. Lawson submitted his article for publication \cite{lawson}. In it, the conditions that have to be fulfilled in order to reach ignition in thermonuclear reactor were demonstrated. As originally formulated, the Lawson criterion gives a minimum required value for the product of the plasma density and the confinement time. This value depends on the plasma temperature which has to be significantly high for all kinds of reactants. It is obvious, since the involved particles have to acquire sufficient kinetic energy to overcome electric repulsion, as mentioned before, and thus increase the rate of fusion reactions.

Even though the Coulomb barrier is penetrated by quantum tunneling, allowing the process to proceed at lower temperatures, the vast majority of substances are in the state of a fully or partially ionized plasma under that conditions. However, exposing any solid material to hot plasmas can damage the material, which then emit impurities contaminating the plasma and decreasing its temperature and confinement stability. Hence, the fundamental question is how to confine such a fusion plasma in the terrestrial conditions.

Taking into account that plasma consists of charged particles, one can exploit a magnetic field of appropriate geometry to shape and confine them. The charged particles will then follow helical paths encircling the magnetic lines of force. However, the intensity of the magnetic field is limited by the structural characteristics of matter, thus it is possible to confine only low density plasmas. From Lawson's criterion, one can clearly find out that in this case fusion reactions have to take place for a relatively long time period.

The second route to confine hot plasmas is to take advantage of the inertia of the mass itself. The fusion burn can occur before the plasma escapes into the surrounding environment. Mass inertia keeps it together for a certain moment given by the time depending on the velocity of an acoustic wave, thus it involves no external means of confinement. The confinement time is obviously very short in this approach, therefore it is necessary to compress the solid fuel to extremely high densities.