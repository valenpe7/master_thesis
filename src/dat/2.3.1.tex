To derive the dispersion relation of electromagnetic wave propagating through unmagnetized plasmas, the hydrodynamic approach is exploited. Since one assumes plasma response to a high frequency field, the ions are treated as a stationary, neutralizing background. Thermal motion of particles is also ignored, thus the pressure term in \ref{2.2.3} can be neglected. One also needs the wave equation for the electric field \ref{1.34}. However, it is now necessary to include the current density due to motion of charged particles in plasma. The hydrodynamic equations are coupled with the Maxwell's equations via $ \vec{J} = q_s n_s \vec{u}_s $, thus one shall solve the following set of equation,
\begin{equation}
\label{2.3.1.2}
m_e n_e \left[ \diffp{\vec{u}_e}{t} + \left(\vec{u}_e \cdot \nabla \right) \vec{u}_e \right] = - e n_e \vec{E},
\end{equation}
\begin{equation}
\label{2.3.1.3}
\laplace{\vec{E}} - \frac{1}{c^{2}} \diffp[2]{\vec{E}}{t} = - \mu_0 e n_e \diffp{\vec{u}_e}{t}.
\end{equation}
The system of equations above will be linearized using the methods of perturbation theory. Consider a small perturbations from the stationary state,
\begin{equation}
\label{2.3.1.4}
\vec{u}_{e} = \vec{u}_{e0} + \delta \vec{u}_{e}, \quad \vec{E} = \vec{E}_{0} + \delta \vec{E},
\end{equation}
where $ \vec{u}_{e0} $ and $ \vec{E}_{0} $ are obviously identically equal to zero vector. After substituting perturbed quantities \ref{2.3.1.4} into initial system of equations and performing Fourier transform one obtains
\begin{equation}
\label{2.3.1.6}
\delta \vec{u}_e = -\i \frac{e}{\omega m_e} \delta \vec{E},
\end{equation}
\begin{equation}
\label{2.3.1.7}
\delta \vec{E} = \i \frac{\omega \, e \, n_{e0}}{\varepsilon_0 \left(\omega^2 - c^2 k^2 \right)} \delta \vec{u}_e.
\end{equation}

Eliminating $ \delta \vec{u}_{e} $ from the equation \ref{2.3.1.7} one gets the equation for perturbation of the electric field,
\begin{equation}
\label{2.3.1.8}
\left(\omega^2 - \omega_{pe}^2 - c^2 k^2 \right) \delta \vec{E} = 0. 
\end{equation}
The equation \ref{2.3.1.8} is the dispersion relation of the electromagnetic wave in plasma. To describe how the electromagnetic waves propagates through given medium, it is useful to introduce the index of refraction $ N \left( \omega \right) =  c k \left( \omega \right) / \omega $. Consequently, the dispersion relation may be rewritten as
\begin{equation}
\label{2.3.1.9}
N^{2} = 1 - \left(\frac{\omega_{pe}}{\omega}\right)^2.
\end{equation} 
The important properties of the waves are distinguished by their cut-offs ($ N \rightarrow 0 $) and resonances ($ N \rightarrow \infty $). In the vicinity of the resonance there is a total absorption, at a cut-off frequency there is a total reflection of incident waves.

In the case of the electromagnetic wave propagating through unmagnetized plasma, it might be clearly seen that there are no resonances. On the other hand, the equation \ref{2.3.1.9} exhibits cut-off and the corresponding frequency (including ions) is given by the following expression,
\begin{equation}
\label{2.3.1.10}
\omega = \sqrt{\omega_{pe}^{2} + \omega_{pi}^{2}}
\end{equation}
The condition \ref{2.3.1.10} occurs at the so-called critical plasma density $ n_c \left( \omega\right) $. Note that the electromagnetic wave with frequency $ \omega $ passing through plasma with densities larger than $ n_c \left( \omega\right) $ is exponentially damped.