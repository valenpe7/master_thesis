One of the most important mechanism of laser light absorption in plasma is collisional absorption, also called inverse bremsstrahlung. It is a process in which an electron absorbs a photon while colliding with an ion or with another electron, and it leads to the heating of all particles in the interaction region. This way electromagnetic energy of the laser wave is transferred into the kinetic energy of plasma.

The change in laser intensity I, passing through plasma in the direction of x-axis, is given by
\begin{equation}
\label{2.4.1.1}
\diff{I}{x} = -\kappa I,
\end{equation}
where $ \kappa $ is the spatial damping rate of the laser energy caused by collisional absorption. For a slab of plasma of length $ L $, the absorption coefficient $ \alpha_{\mathrm{abs}} $ is given by
\begin{equation}
\label{2.4.1.2}
\alpha_{\mathrm{abs}} = 1 - \frac{I_{\mathrm{out}}}{I_{\mathrm{in}}} = 1 - \exp \left(-\int\limits_0^L \kappa \: \mathrm{d} x \right).
\end{equation}
Here $ I_{\mathrm{out}} $ and $I_{\mathrm{in}} $ are the outgoing and the incoming laser intensities, respectively. The absorption coefficient for a linear and exponential density profiles is given by solving the integral (\ref{2.4.1.2}) \cite{eliezer},
\begin{equation}
\label{2.4.1.3}
\alpha_{\mathrm{abs}} = 1 - \exp\left(-\frac{32}{15} \frac{\nu_{ei}(n_c) L}{c}\right) \qquad \mathrm{for} \qquad n_e = n_c \left(1 - \frac{x}{L} \right),
\end{equation}
\begin{equation}
\label{2.4.1.4}
\alpha_{\mathrm{abs}} = 1 - \exp\left(-\frac{8}{3} \frac{\nu_{ei}(n_c) L}{c}\right) \qquad \mathrm{for} \qquad n_e = n_c \exp\left(- \frac{x}{L} \right).
\end{equation}
where $ \nu_{ei}(n_c) $ is the collision frequency evaluated at the critical density $ n_c $, which is given by formula
\begin{equation}
n_{c} = \frac{\varepsilon_{0} m_{e} \omega}{e^{2}}.
\end{equation}
Notice that a significant fraction of the collisional absorption is from the region near the critical density. Collisional absorption is the preferred absorption mechanism for driving matter ablatively with laser beams. In the case of long laser pulse duration with relatively low intensity, collisional absorption can be very efficient.