Within the work, two large scale simulations of laser plasma interaction in the two-dimensional geometry have been performed. Both simulations have been computed using a massively parallel relativistic electromagnetic PIC code EPOCH mentioned in the previous chapter. The laser beam parameters have been set according to the high-power iodine laser system PALS in Czech Academy of Sciences. More specifically, the parameters correspond to the attributes of laser beam at its fundamental wavelength, which is $ \lambda = 1.315 \: \mathrm{\mu m} $. At this wavelength, the laser is capable to deliver energy of $ \mathrm{E} = 1 \: \mathrm{kJ} $. The intensity of laser light was $ \mathrm{I} = 1 \cdot 10^{20} \: \mathrm{W/m^{2}} $ for both simulations, thus the maximum laser dimensionless potential is $ \mathrm{a_0} = 0.1123 $. The beam profile is Gaussian in both, space and time, focused to the distance of $ \mathrm{x} = 50 \: \mathrm{\mu m} $ from the entry boundary. The $ \mathrm{FWHM} $ was equal to $ 10 \: \mathrm{\mu m} $ and the pulse duration has been set to $ \mathrm{t} = 20 \: \mathrm{ps} $. These parameters do not correspond to the laser PALS. The real parameters have been decreased because of our limited computational resources.

Initial parameters of plasma have been taken from the hydrodynamic simulations, which have been performed recently. The electron and ion densities were equal in both simulations, described by exponential profile with the scale length $ \mathrm{L} = 90 \: \mathrm{\mu m} $. The critical density is $ \mathrm{n_{c}} = 6.46 \cdot 10^{26} \: \mathrm{m^{-3}} $, which is the maximum density in both simulations. The simulations differ in chosen temperatures which are constant throughout the target. The first one has been performed with the electron temperature $ \mathrm{T_e} = 0.5 \: \mathrm{keV} $ and the temperature of ions $ \mathrm{T_i} = 0.5 \: \mathrm{keV} $, while the second one has $ \mathrm{T_e} = 2.5 \: \mathrm{keV} $ and $ \mathrm{T_i} = 0.5 \: \mathrm{keV} $.

To include the density profile up to the critical density, the simulation domain has the size of $ 381.35 \: \mathrm{\mu m} \times 52.6 \: \mathrm{\mu m} $. The size of individual cells has to be comparable to the minimum Debye length in the simulation box. However, in order to reduce the computational time, the size of the cells has been chosen a little bit larger, $ 50 \: \mathrm{nm} \times 50 \: \mathrm{nm} $, while the required accuracy of computations is still maintained. This corresponds to the number of cells in both directions $ 7540 \times 1040 $, thus the total number of cells in simulations was approximately $ 8.3 \cdot 10^{6} $. The simulations are performed on the time scale of $ 20 \: \mathrm{ps} $, one simulation time step was approximately $ \tau = 0.1 \:\: \mathrm{fs} $ to fulfill the CFL condition, thus the total number of time steps was about $ 2\cdot 10^{5} $. Simulations contain two species of particles - electrons and ions. Both species have density $ 8 $ particles per cell, thus there is around $ 1.3 \cdot 10^{8} $ computational particles in total.

In both simulations, Convolutional Perfectly Matched Layer (CPML) boundary conditions have been used. On each boundaries, absorbing conditions were specified. The laser has been attached to the left boundary, and the beam is propagating along the $ x $ direction. The thickness of the CPML boundary conditions is $ 26 $ cells. For the simulation with $ \mathrm{T_e} = 0.5 \: \mathrm{keV} $, the special boundary condition to absorb the kinetic energy flux into the target, which is described in the beginning of this chapter, has been used. 

\floatsetup[figure]{style=plain, subcapbesideposition=center}
\begin{figure}[h!]
	\sidesubfloat[]{\includegraphics[width=1.0\linewidth]{./img/ne_05keV_10ps.eps}}\\
	\sidesubfloat[]{\includegraphics[width=1.0\textwidth]{./img/ne_25keV_10ps.eps}}
	\caption{Electron density at the time 10 ps for the case of simulation with electron temperature 0.5 keV \textbf{(a)} and for the case of simulation with electron temperature 2.5 keV \textbf{(b)}. Figure shows cavities in the density profile and the continuous expansion of the plasma.}
	\label{fig:ne}
\end{figure}

\begin{figure}[h!]
	\centering
	\subfloat[]{{\includegraphics[width=0.5\linewidth]{./img/cav_05keV.eps} }}%
	\subfloat[]{{\includegraphics[width=0.5\linewidth]{./img/cav_25keV.eps} }}%
	\caption{Cavities formed at quarter of critical density at the time 10 ps for the case of simulation with electron temperature 0.5 keV \textbf{(a)} and for the case of simulation with electron temperature 2.5 keV \textbf{(b)}. Density cavities contribute to the total absorption of the incident laser light. Electron plasma waves are also observed.}%
	\label{fig:cav}%
\end{figure}

\floatsetup[figure]{style=plain, subcapbesideposition=center}
\begin{figure}[h!]
	\sidesubfloat[]{\includegraphics[width=1.0\linewidth]{./img/Bz_05keV_10ps.eps}}\\
	\sidesubfloat[]{\includegraphics[width=1.0\textwidth]{./img/Bz_25keV_10ps.eps}}
	\caption{Z-component of the magnetic field at the time 10 ps for the case of simulation with electron temperature 0.5 keV \textbf{(a)} and for the case of simulation with electron temperature 2.5 keV \textbf{(b)}. Figure shows the filamentation of the incident beam.}
	\label{fig:Bz}
\end{figure}

The collisions of plasma particles are accounted for both simulations. Using formulas (\ref{2.1.6}) and (\ref{2.4.1.4}), one can theoretically estimate their effect. For the simulation with electron temperature $ \mathrm{T_e} = 0.5 \: \mathrm{keV} $, the average time between two collisions evaluated at the critical density is $ 1/\nu_{ei} = 1.1 \: \mathrm{ps} $ and the absorption coefficient calculated for the initial conditions is $ \alpha_{abs} = 0.55 $. For the simulation with electron temperature $ \mathrm{T_e} = 2.5 \: \mathrm{keV} $, the average time between two collisions at the critical density is obviously higher, $ 1/\nu_{ei} = 8.2 \: \mathrm{ps} $, thus the absorption coefficient is only about $ \alpha_{abs} = 0.09 $. However, since the time scale of simulations is relatively long, the contribution of collisions to absorption cannot be neglected for both cases.

The threshold for the laser stimulated instabilities increases as the laser wavelength decreases. Thus one can expect, that the parametric instabilities might play important role, since wavelength of the incident laser light is relatively long. Scattered radiation on the frequency of incident laser beam corresponds to the reflection at the critical density and stimulated Brillouin scattering. The dependency of Brillouin scattering frequency on the plasma density is governed by following relation \cite{klimo3},
\begin{equation}
	\omega = \omega_0 - 2 k_0 (c_s + u), \quad k_0 = \frac{\omega_0}{c} \sqrt{1 - \frac{n_e}{n_c}},
\end{equation}
where $ u $ is the velocity of expanding plasma. Thus the frequency shift of the stimulated Brillouin scattering is caused by ion acoustic wave and Doppler effect.

\begin{figure}[h!]%
	\centering
	\subfloat[]{{\includegraphics[width=0.5\linewidth]{./img/fflux-in_05kev.eps} }}%
	\subfloat[]{{\includegraphics[width=0.5\linewidth]{./img/fflux-out_05kev.eps} }}%
	\caption{Intensity of the incident laser light \textbf{(a)} and scattered radiation \textbf{(b)} passing through the entry boundary for the case of simulation with electron temperature 0.5 keV.}%
	\label{fig:fflux_05kev}%
\end{figure}

\begin{figure}[h!]%
	\centering
	\subfloat[]{{\includegraphics[width=0.5\linewidth]{./img/fflux-in_25kev.eps} }}%
	\subfloat[]{{\includegraphics[width=0.5\linewidth]{./img/fflux-out_25kev.eps} }}%
	\caption{Intensity of the incident laser light \textbf{(a)} and scattered radiation \textbf{(b)} passing through the entry boundary for the case of simulation with electron temperature 2.5 keV.}%
	\label{fig:fflux_25kev}%
\end{figure}

Radiation scattering takes place as well on the half of the frequency of the incident laser light, which corresponds to stimulated Raman scattering. The weak radiation might be also registered on the quarter of the fundamental frequency and can be interpreted as secondary Raman scattering of the reflected electromagnetic wave. Raman scattering is dependent on the instantaneous intensity of the incident electromagnetic wave and it can be described as a space localized instability, which is either absolute or convective. In the absolute case, the temporal growth of the instability is localized while in the convective case the most unstable region is propagating with the laser wave. In the corresponding area, significant changes of the plasma parameters may occur.

\begin{figure}[h!]%
	\centering
	\subfloat[]{{\includegraphics[width=0.5\linewidth]{./img/distr_05keV.eps} }}%
	\subfloat[]{{\includegraphics[width=0.5\linewidth]{./img/distr_25keV.eps} }}%
	\caption{Distribution function of electrons going towards the target core and back in front of the implemented boundary condition for the case of simulation with electron temperature 0.5 keV \textbf{(a)} and for the case of simulation with electron temperature 2.5 keV \textbf{(b)}. Simulation data are approximated with Maxwellian distribution functions.}%
	\label{fig:distr}%
\end{figure}

The Figure \ref{fig:ne} shows the electron density at the time 10 ps. Plasma is expanding much more in the case of simulation with $ \mathrm{T_e} = 2.5 \: \mathrm{keV} $ (Figure \ref{fig:ne}-b), because the initial temperature is relatively high. Until the time 10 ps, plasma is continuously expanding, but the scale length of its profile remains unchanged during the whole simulation. After this moment, the intensity of the incident laser pulse reaches its maximum, thus the ponderomotive force weakens and the expansion is faster. As mentioned before, the density cavities are created due to the Raman scattering on the quarter of the critical density. The cavities can be seen even better in Figure \ref{fig:cav}, which shows the electron density profile in the detail around quarter of the critical density. Emphasize, that this might not correspond to reality because the cavitation is three-dimensional phenomena \cite{klimo6}. The part of the electromagnetic field is caught in the cavities and cannot escape. Caught radiation is continuously converted into the kinetic energy of plasma. Thus this can also be one of the important absorption mechanisms of the incident laser light.

The total absorption of the incident laser energy in plasma for the simulation with $ \mathrm{T_e} = 0.5 \: \mathrm{keV} $ has been 42.4 \%, for the case of simulation with $ \mathrm{T_e} = 2.5 \: \mathrm{keV} $, the total absorption rate was significantly lower, about 33.1 \%. According to the previous theoretical calculations, the Coulomb collisions between electrons and ions should have been the dominant mechanism of absorption in the first case, whilst in the second case, the major part of laser energy should have been absorbed by the density cavities, filamentation and parametric instabilities. These calculations, however, do not have to be accurate because they do not describe the interaction completely; there might be a wide range of other processes that lead to the absorption or scattering of the incident laser light. In order to distinguish the rate of absorption due to individual processes in plasma, further investigation of the simulation results has to be done.

Figure \ref{fig:Bz} shows the z-component of the magnetic field at the time 10 ps. Here, one can clearly observe the filamentation of the incident laser light behind $ \mathrm{x} = 250 \: \mathrm{\mu m} $ for both cases, which corresponds to the half of the critical density. In this region, the laser intensity is above the threshold for the relativistic self-focusing. Not only the incident but also the scattered wave is filamented.

Figures \ref{fig:fflux_05kev} and \ref{fig:fflux_25kev} show the flux of intensity of the incident laser beam and the scattered radiation through the entry boundary. The peak intensity of scattered radiation is almost three times higher than the intensity of the incident laser light for the case of the simulation with $ \mathrm{T_e} = 0.5 \: \mathrm{keV} $. For the simulation with  $ \mathrm{T_e} = 2.5 \: \mathrm{keV} $, the laser beam intensity is increased in plasma approximately two times. 

Energy spectrum of electrons propagating through the imaginary border in the place of the beginning of the boundary condition is plotted in Figure \ref{fig:distr}. According to the recent simulations \cite{klimo2}, it seemed that the temperature of hot electrons does not depend on the intensity of the incident laser pulse and can be estimated in order of several tens of keV. In addition, recent studies showed, that these hot electrons themselves might drive a strong spherical convergent shock wave \cite{tikhonchuk}. Thus the excessive preheat of the compressed fuel caused by these electrons in the late phase of compression would not have to be significant, since the major contribution to preheat of the core comes from electrons with energy higher than 100 keV which are able to penetrate the compressed shell and the amount of such electrons is relatively small. This fact could completely change the point of view on the parametric instabilities in the context of the shock ignited inertial fusion.

The favorable results have been confirmed in the presented simulations of laser system PALS. The temperature of the hot electrons propagating in forward direction into the target has been estimated to 36 keV for the case of simulation with electron temperature $ \mathrm{T_e} = 0.5 \: \mathrm{keV} $ (Figure \ref{fig:distr}-a) and 25 keV for the case of simulation with $ \mathrm{T_e} = 2.5 \: \mathrm{keV} $ (Figure \ref{fig:distr}-b). By integrating the corresponding Maxwellian functions, one can estimate how large are the fractions of hot electrons. In the first case, the fraction is about 0.024 \% of all electrons in the simulation box. In the second case, the fraction is higher, about 0.037 \%, which is consistent with previous theoretical absorption rate estimates and points out the presence of other non-linear absorption processes. Nevertheless, the number of hot electrons in both cases is relatively low; taking into account their energy, these electrons probably would not pass into the target core, which is in accordance with recent studies \cite{tikhonchuk}. 

Finally, the temperature of electrons moving back into the interaction domain is in both simulations only a little bit higher than the initial temperature and does not contain any significant high energy tail. Therefore the boundary conditions behave in accordance with our requirements to suppress the return current of hot electrons. These and the previous results are also important for the interpretation of ongoing experiments in the laser facility PALS.