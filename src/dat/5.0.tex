\section{Algorithm}

\begin{enumerate}
	\item Calculate $ \hat{\vec{E}}_{0, \bot}^{ijn} $ via discrete Fourier transforms in time:
	\begin{equation}
	\omega^n = \frac{2 \pi}{N_t \delta t} \left( -\frac{N_t}{2} + n \right),
	\end{equation}
	\begin{equation}
	\hat{\vec{E}}_{0, \bot}^{ijn} = \frac{\delta t}{2 \pi} \sum_{l=1}^{N_t} \vec{E}_{0, \bot}^{ijl} \e^{\i \omega^n t^l}, \quad n \in \left\lbrace 1, \dots, N_t \right\rbrace.
	\end{equation}
	\item Calculate $ \bar{\vec{E}}_{0, \bot}^{ijn} $ via two-dimensional discrete Fourier transforms in transverse space:
	\begin{equation}
	k_x^i = \frac{2 \pi}{N_x \delta x} \left( - \frac{N_x}{2} + i\right), \quad k_y^j = \frac{2 \pi}{N_y \delta y} \left( - \frac{N_y}{2} + j\right),
	\end{equation}
	\begin{equation}
	\bar{\vec{E}}_{0, \bot}^{ijn} = \frac{\delta x \delta y}{(2 \pi)^2} \sum_{l, m = 1}^{N_x, N_y} \hat{\vec{E}}_{0, \bot}^{lmn} \e^{- \i \: \left(k_x^i x^l + k_y^j y^m \right)}, \quad i, j \in \left\lbrace 1, \dots, N_{x, y} \right\rbrace.
	\end{equation}
	\item Calculate transverse electric field components at the boundary $ z = z_\mathrm{B} $:
	\begin{equation}
	k_z^{ijn} = \Re \sqrt{\frac{(\omega^n)^2}{c^2} - (k_x^i)^2 - (k_y^j)^2},
	\end{equation}
	\begin{equation}
	\bar{\vec{E}}_{\mathrm{B}, \bot}^{ijn} =
	\begin{cases} \bar{\vec{E}}_{0, \bot}^{ijn} \e^{\i k_z^{ijn}(z_\mathrm{B} - z_0)} & \text{for} \ k_z^{ijn} > 0 \\ 0 & \text{for} \ k_z^{ijn} = 0 \end{cases}.
	\end{equation}
	\item Calculate longitudinal electric field components at the boundary $ z = z_\mathrm{B} $:
	\begin{equation}
	\bar{E}_{\mathrm{B}, z}^{ijn} = \begin{cases} -\frac{k_x^i \bar{E}_{\mathrm{B}, x}^{ijn} + k_y^j \bar{E}_{\mathrm{B}, y}^{ijn}}{k_z^{ijn}} & \text{for} \ k_z^{ijn} > 0 \\ 0 & \text{for} \ k_z^{ijn} = 0 \end{cases}.
	\end{equation}
	\item Calculate the magnetic field at the boundary $ z = z_\mathrm{B} $:
	\begingroup
	\renewcommand*{\arraystretch}{1.7}
	\begin{equation}
	\mathbb{R}^{ijn} =  \begin{pmatrix}
	-k_x^i k_y^j & (k_x^i)^2 - (\omega^n)^2/c^2 \\
	(\omega^n)^2/c^2 - (k_y^j)^2 & k_x^i k_y^j \\
	-k_y^j k_z^{ijn} & k_x^i k_z^{ijn} 
	\end{pmatrix},
	\end{equation} 
	\endgroup
	\begin{equation}
	\bar{\vec{B}}_{\mathrm{B}}^{ijn} = \begin{cases} (\omega^n k_z^{ijn})^{-1} \mathbb{R}^{ijn} \bar{\vec{E}}_{\mathrm{B}, \bot}^{ijn} & \text{for} \ k_z^{ijn} > 0 \\ 0 & \text{for} \ k_z^{ijn} = 0 \end{cases}.
	\end{equation}
	\item Calculate $ \hat{\vec{E}}_{\mathrm{B}}^{ijn} $, $ \hat{\vec{B}}_{\mathrm{B}}^{ijn} $ via two-dimensional inverse discrete Fourier transforms:
	\begin{equation}
	\hat{\vec{E}}_{\mathrm{B}}^{ijn} = \frac{(2 \pi)^2}{N_x N_y \delta x \delta y} \sum_{l, m = 1}^{Nx, Ny} \bar{\vec{E}}_{\mathrm{B}}^{lmn} \e^{\i(k_x^l x^i + k_y^m y^j)},
	\end{equation}
	\begin{equation}
	\hat{\vec{B}}_{\mathrm{B}}^{ijn} = \frac{(2 \pi)^2}{N_x N_y \delta x \delta y}  \sum_{l, m = 1}^{Nx, Ny} \bar{\vec{B}}_{\mathrm{B}}^{lmn} \e^{\i(k_x^l x^i + k_y^m y^j)}.
	\end{equation}
	\item Calculate $ \vec{E}_{\mathrm{B}}^{ij}(t) $, $ \vec{B}_{\mathrm{B}}^{ij}(t) $ for any given time $ t \in [t^{1} - \frac{z_{\mathrm{B}} - z_{0}}{c}, t^{N_{t}}  - \frac{z_{\mathrm{B}} - z_{0}}{c}] $.
	\begin{equation}
	\vec{E}_{\mathrm{B}}^{ij} (t) = \frac{2 \pi}{N_t \delta t} \sum_{n = 1}^{N_t} \hat{\vec{E}}_{\mathrm{B}}^{ijn} \e^{-\i \omega^n t},
	\end{equation}
	\begin{equation}
	\vec{B}_{\mathrm{B}}^{ij} (t) = \frac{2 \pi}{N_t \delta t} \sum_{n = 1}^{N_t} \hat{\vec{B}}_{\mathrm{B}}^{ijn} \e^{-\i \omega^n t}.
	\end{equation}
\end{enumerate}

\section{Implementation}

One of the main goals of this work has been to implement the algorithm mentioned in the previous section, to evaluate its correctness in several test simulations and finally, to exploit resulting implementation for simulations of tightly focused Gaussian beams in laser-matter interaction. The main requirement on implementation has been easy to use with 2D version of particle-in-cell simulation code EPOCH. For this reason, several possible solutions has been taken into account.

The final decision has been to create a static library, which will be able to compute desired quantities and will provide functions for communication with the main simulation code. The essential advantage is that it could be basically linked with any laser-plasma simulation code. Also, since it is necessary to call only two additional functions, the instrumentation will be fast, easy and the main simulation code will not be excessively disturbed. Furthermore, the implementation itself come with the CMake support, which simplify the compilation process using platform and compiler independent configuration files.

The library has been written in C++ language and is object oriented so the algorithm can be easily extended to three dimensional geometry. In order to speedup the whole underlying computation, the algorithm has been parallelized using hybrid techniques. The time domain has been decomposed into the stripes corresponding to individual computational processes, the communication between these processes is ensured by MPI library. Furthermore, the computationally most expensive cycles are parallelized using OpenMP implementation of multi-threading. Later on, the speedup and parallel scaling performance will be briefly discussed.

Fourier transforms form the core of the computational process and their performance is crucial for the overall performance of the code. For this reason, many currently available libraries have been considered. Eventually, the Fourier transforms in the algorithm can be computed using FFTW library, Intel MKL library or it is also possible to directly evaluate the formulas without using any additional library. The user specifies his option at a compile time. Regarding both libraries, a threaded versions of 1D in-place complex fast Fourier transforms has been used throughout the code. According to several measures, there is no significant difference between the speed of both implementations.

One potential bottleneck could happen during the computation of spatial Fourier transforms since the arrays with spatial data are decomposed into different processes. The cluster versions of functions performing the Fourier transforms has been tried, however they did not bring any significant speedup. The reason is as follows. They require to have the global array in memory and use its own distribution which involves overlapping. Since the size of global arrays is usually not so large and since it is necessary to perform a lot of different Fourier transforms, the majority of computational time is spent rather for communication, mainly if many of computational cores are used.

This issue has been solved by gathering the data on master process, performing the Fourier transforms in space by only one processor and scattering the data back to corresponding processes. This is the reason why the code does not scale well, however, the time to compute all desired quantities is in most cases negligible in comparison with time required by main simulation cycle. Nevertheless, this issue could be improved in future.

Cely vypocet najednou, do pameti se nevejde -> nutnost ukladat data. The next critical part of the code is responsible for dumping the data.

\begin{itemize}
	\item 2D version of algorithm, Ey, Bx, Bz omitted (identically equal to 0) 
	\item Code written in C++, object oriented to be easily extended to 3D, compiled to static library
	\item Linked into EPOCH as a static library (in order not to disturb the code, for this reason also added support for CMake – machine independent)
	\item Parallelized using hybrid techniques (OpenMP + MPI – computation time in most cases negligible in comparison with the main simulation)
	\item Fourier transforms can be computed using Intel MKL library, FFTW library or without any library (compile time option)
	\item Computed fields dumped into shared files using binary coding (speed up output, save disk storage)
	\item Only transverse component of electric field (Ex) passed to the EPOCH at each time step (no significant slowdown or memory overhead), other fields computed by EPOCH
	\item All new parameters needed for tight-focusing (w0, focal length, etc.) may be specified via input file
	\item Implementation works generally regardless the number of lasers in the simulation or boundaries that they are attached to
\end{itemize}

\begin{lstlisting}[style=CXX, caption=Function performing forward fast Fourier transform using MKL library]
std::vector<std::complex<double>> fft::mkl_fft_forward(std::vector<std::complex<double>> in) {
	DFTI_DESCRIPTOR_HANDLE desc;
	MKL_LONG status;
	DftiCreateDescriptor(&desc, DFTI_DOUBLE, DFTI_COMPLEX, 1, static_cast<MKL_LONG>(in.size()));
	DftiCommitDescriptor(desc);
	status = DftiComputeForward(desc, in.data());
	if(status != 0) {
		std::cerr << DftiErrorMessage(status) << std::endl;
		abort();
	}
	DftiFreeDescriptor(&desc);
	return in;
}
\end{lstlisting}

\begin{lstlisting}[style=CXX, caption=Function performing backward fast Fourier transform using MKL library]
std::vector<std::complex<double>> fft::mkl_fft_backward(std::vector<std::complex<double>> in) {
	DFTI_DESCRIPTOR_HANDLE desc;
	MKL_LONG status;
	DftiCreateDescriptor(&desc, DFTI_DOUBLE, DFTI_COMPLEX, 1, static_cast<MKL_LONG>(in.size()));
	DftiCommitDescriptor(desc);
	status = DftiComputeBackward(desc, in.data());
	if(status != 0) {
		std::cerr << DftiErrorMessage(status) << std::endl;
		abort();
	}
	DftiFreeDescriptor(&desc);
	return in;
}
\end{lstlisting}

\begin{lstlisting}[style=CXX, caption=Function performing forward fast Fourier transform using FFTW library]
std::vector<std::complex<double>> fft::fftw_fft_forward(std::vector<std::complex<double>> in) {
	fftw_plan p = fftw_plan_dft_1d(in.size(), reinterpret_cast<fftw_complex*>(in.data()), reinterpret_cast<fftw_complex*>(in.data()), FFTW_FORWARD, FFTW_ESTIMATE);
	fftw_execute(p);
	fftw_destroy_plan(p);
	return in;
}
\end{lstlisting}

\begin{lstlisting}[style=CXX, caption=Function performing backward fast Fourier transform using FFTW library]
std::vector<std::complex<double>> fft::fftw_fft_backward(std::vector<std::complex<double>> in) {
	fftw_plan p = fftw_plan_dft_1d(in.size(), reinterpret_cast<fftw_complex*>(in.data()), reinterpret_cast<fftw_complex*>(in.data()), FFTW_BACKWARD, FFTW_ESTIMATE);
	fftw_execute(p);
	fftw_destroy_plan(p);
	return in;
}
\end{lstlisting}

\begin{lstlisting}[style=CXX, caption=Function performing forward discrete Fourier transform without using any library]
std::vector<std::complex<double>> fft::fft_forward(std::vector<std::complex<double>> in) {
	std::vector<std::complex<double>> out(in.size());
	for(auto j = 0; j < out.size(); j++) {
		for(auto l = 0; l < out.size(); l++) {
			out.at(j) += in.at(l) * exp(-2.0 * constants::pi * I * l * j / in.size());
		}
	}
	return out;
}
\end{lstlisting}

\begin{lstlisting}[style=CXX, caption=Function performing backward discrete Fourier transform without using any library]
std::vector<std::complex<double>> fft::fft_backward(std::vector<std::complex<double>> in) {
	std::vector<std::complex<double>> out(in.size());
	for(auto j = 0; j < out.size(); j++) {
		for(auto l = 0; l < out.size(); l++) {
			out.at(j) += in.at(l) * exp(+2.0 * constants::pi * I * l * j / in.size());
		}
	}
	return out;
}
\end{lstlisting}

\begin{lstlisting}[style=CXX, caption=Method for performing discrete Fourier transform in time]
void laser_bcs::dft_time(field_2d<std::complex<double>>& field) const {
#ifdef OPENMP
#pragma omp parallel for schedule(static)
#endif
	for(auto j = 0; j < this->domain->Nx; j++) {
#ifdef USE_MKL
		field.add_col(fft::mkl_fft_backward(field.get_col(j)), j);
#elif USE_FFTW
		field.add_col(fft::fftw_fft_backward(field.get_col(j)), j);
#else
		field.add_col(fft::fft_backward(field.get_col(j)), j);
#endif
	}
	field.multiply(this->domain->dt / (2.0 * constants::pi));
	return;
}
\end{lstlisting}

\begin{lstlisting}[style=CXX, caption=Method for performing inverse discrete Fourier transform in time]
void laser_bcs::idft_time(field_2d<std::complex<double>>& field) const {
#ifdef OPENMP
#pragma omp parallel for schedule(static)
#endif
	for(auto j = 0; j < this->domain->Nx; j++) {
#ifdef USE_MKL
		field.add_col(fft::mkl_fft_forward(field.get_col(j)), j);
#elif USE_FFTW
		field.add_col(fft::fftw_fft_forward(field.get_col(j)), j);
#else
		field.add_col(fft::fft_forward(field.get_col(j)), j);
#endif
	}
	field.multiply(2.0 * (2.0 * constants::pi) / (this->domain->Nt * this->domain->dt));
	return;
}
\end{lstlisting}

\begin{lstlisting}[style=CXX, caption=Method for performing discrete Fourier transform in space]
void laser_bcs::dft_space(field_2d<std::complex<double>>& field) const {
	std::vector<std::complex<double>> row_global(this->domain->Nx_global);
	std::vector<std::complex<double>> row_local;
	for(auto j = 0; j < this->domain->Nt; j++) {
		row_local = field.get_row(j);
		MPI_Gatherv(row_local.data(), this->domain->Nx, MPI_CXX_DOUBLE_COMPLEX, row_global.data(), this->domain->counts.data(), this->domain->displs.data(), MPI_CXX_DOUBLE_COMPLEX, 0, MPI_COMM_WORLD);
		if(this->domain->rank == 0) {
#ifdef USE_MKL
			row_global = fft::mkl_fft_forward(row_global);
#elif USE_FFTW
			row_global = fft::fftw_fft_forward(row_global);
#else
			row_global = fft::fft_forward(row_global);
#endif
		}
		MPI_Scatterv(row_global.data(), this->domain->counts.data(), this->domain->displs.data(), 	MPI_CXX_DOUBLE_COMPLEX, row_local.data(), this->domain->Nx, MPI_CXX_DOUBLE_COMPLEX, 0, MPI_COMM_WORLD);
		field.add_row(row_local, j);
	}
	field.multiply(this->domain->dx / (2.0 * constants::pi));
	return;
}
\end{lstlisting}

\begin{lstlisting}[style=CXX, caption=Method for performing inverse discrete Fourier transform in space]
void laser_bcs::idft_space(field_2d<std::complex<double>>& field) const {
	std::vector<std::complex<double>> row_global(this->domain->Nx_global);
	std::vector<std::complex<double>> row_local;
	for(auto j = 0; j < this->domain->Nt; j++) {
		row_local = field.get_row(j);
		MPI_Gatherv(row_local.data(), this->domain->Nx, MPI_CXX_DOUBLE_COMPLEX, row_global.data(), this->domain->counts.data(), this->domain->displs.data(), MPI_CXX_DOUBLE_COMPLEX, 0, MPI_COMM_WORLD);
		if(this->domain->rank == 0) {
#ifdef USE_MKL
			row_global = fft::mkl_fft_backward(row_global);
#elif USE_FFTW
			row_global = fft::fftw_fft_backward(row_global);
#else
			row_global = fft::fft_backward(row_global);
#endif
		}
		MPI_Scatterv(row_global.data(), this->domain->counts.data(), this->domain->displs.data(), MPI_CXX_DOUBLE_COMPLEX, row_local.data(), this->domain->Nx, MPI_CXX_DOUBLE_COMPLEX, 0, MPI_COMM_WORLD);
		field.add_row(row_local, j);
	}
	field.multiply((2.0 * constants::pi) / (this->domain->Nx_global * this->domain->dx));
	return;
}
\end{lstlisting}

\begin{lstlisting}[style=CXX, caption=Method for dumping data into shared file]
template <typename T>
void field_2d<T>::dump_to_shared_file(std::string name, int row_first, int row_last, int row_size_local, int row_size_global, int col_start) const {
	MPI_File file;
	MPI_Offset offset = 0;
	MPI_Status status;
	MPI_Datatype local_array;
	int col_size = row_last - row_first;
	const int ndims = 2;
	std::array<int, ndims> size_global = {col_size, row_size_global};
	std::array<int, ndims> size_local = {col_size, row_size_local};
	std::array<int, ndims> start_coords = {0, col_start};
	MPI_Type_create_subarray(2, size_global.data(), size_local.data(), start_coords.data(), MPI_ORDER_C, MPI_DOUBLE, &local_array);
	MPI_Type_commit(&local_array);
	std::vector<double> real_part(col_size * row_size_local);
	for(auto i = std::make_pair(row_first, 0); i.first < row_last; i.first++, i.second++) {
		for(auto j = 0; j < row_size_local; j++) {
			real_part[i.second * row_size_local + j] = std::real(this->data[i.first * row_size_local + j]);
		}
	}
	MPI_File_open(MPI_COMM_WORLD, name.data(), MPI_MODE_CREATE|MPI_MODE_WRONLY, MPI_INFO_NULL, &file);
	MPI_File_set_view(file, offset, MPI_DOUBLE, local_array, "native", MPI_INFO_NULL);
	MPI_File_write_all(file, real_part.data(), col_size * row_size_local, MPI_DOUBLE, &status);
	MPI_File_close(&file);
	MPI_Type_free(&local_array);
	return;
}
\end{lstlisting}

\begin{lstlisting}[style=CXX, caption=Extern C++ function to fill Fortran arrays with laser fields dumped in binary file]
void populate_laser_field_on_boundary(double* field, int* id, const char* data_dir, const char* name, int* timestep, int* size_global, int* first, int* last) {
	double num = 0.0;
	std::string laser_id = std::to_string(*id);
	std::string output_path(data_dir);
	std::string filename(name);
	std::ifstream in;
	in.open(output_path + "/" + filename + laser_id + ".dat", std::ios::binary);
	if(in.is_open()) {
		in.seekg(((*timestep) * (*size_global) + (*first) - 1) * sizeof(num));
		for(auto i = 0; i < *last - *first + 1; i++) {
			in.read(reinterpret_cast<char*>(&num), sizeof(num));
			field[i] = num;
		}
		in.close();
	} else {
		std::cout << "error: cannot read file " << output_path + "/" + filename + laser_id + ".dat" << std::endl;
	}
	return;
}
\end{lstlisting}

\begin{lstlisting}[style=FORTRAN, caption=Fortran interfaces for C++ library functions]
INTERFACE

SUBROUTINE compute_laser_fields_on_boundary(rank, nproc, laser_start, laser_end, fwhm_time, t_0, omega, pos, amp, w_0, id, L_min, L_max, L_focus, T_min, T_max, T_ncells, cpml_thickness, t_end, T_cell_size, L_cell_size, dt, output_path) bind(c)
	USE, INTRINSIC :: iso_c_binding
	IMPLICIT NONE
	INTEGER(c_int), INTENT(IN) :: rank, nproc, id, T_ncells, cpml_thickness
	CHARACTER(kind=c_char), DIMENSION(*), INTENT(IN) :: output_path
	REAL(c_double), INTENT(IN) :: laser_start, laser_end, fwhm_time, t_0, omega, pos,    &
	amp, w_0, L_min, L_max, L_focus, T_min, T_max, t_end, T_cell_size, L_cell_size, dt
END SUBROUTINE compute_laser_fields_on_boundary

SUBROUTINE populate_laser_field_on_boundary(field, laser_id, output_path, field_name, timestep, size_global, first, last) bind(c)
	USE, INTRINSIC :: iso_c_binding
	IMPLICIT NONE
	INTEGER(c_int), INTENT(IN) :: laser_id, timestep, size_global, first, last
	CHARACTER(kind=c_char), DIMENSION(*), INTENT(IN) :: output_path, field_name
	REAL(c_double), DIMENSION(*), INTENT(OUT) :: field
END SUBROUTINE populate_laser_field_on_boundary

END INTERFACE
\end{lstlisting}

\begin{lstlisting}[style=FORTRAN, caption=Fortran subroutines for Maxwell consistent computation of laser fields on boundaries]
SUBROUTINE Maxwell_consistent_computation_of_EM_fields
  
  TYPE(laser_block), POINTER :: current
  
  current => laser_x_min
  DO WHILE(ASSOCIATED(current))
	  CALL compute_laser_fields_on_boundary(rank, nproc, current%t_start, current%t_end, current%fwhm_time, current%t_0, current%omega, current%pos, current%amp, current%w_0, current%id, x_min, x_max, current%focus, y_min, y_max, ny_global, cpml_thickness, t_end, dy, dx, dt, TRIM(data_dir)//C_NULL_CHAR)
	  current => current%next
  ENDDO
  
  current => laser_x_max
  DO WHILE(ASSOCIATED(current))
	  CALL compute_laser_fields_on_boundary(rank, nproc, current%t_start, current%t_end, current%fwhm_time, current%t_0, current%omega, current%pos, current%amp, current%w_0, current%id, x_min, x_max, current%focus, y_min, y_max, ny_global, cpml_thickness, t_end, dy, dx, dt, TRIM(data_dir)//C_NULL_CHAR)
	  current => current%next
  ENDDO
  
  current => laser_y_min
  DO WHILE(ASSOCIATED(current))
	  CALL compute_laser_fields_on_boundary(rank, nproc, current%t_start, current%t_end, current%fwhm_time, current%t_0, current%omega, current%pos, current%amp, current%w_0, current%id, y_min, y_max, current%focus, x_min, x_max, nx_global, cpml_thickness, t_end, dx, dy, dt, TRIM(data_dir)//C_NULL_CHAR)
	  current => current%next
  ENDDO
  
  current => laser_y_max
  DO WHILE(ASSOCIATED(current))
	  CALL compute_laser_fields_on_boundary(rank, nproc, current%t_start, current%t_end, current%fwhm_time, current%t_0, current%omega, current%pos, current%amp, current%w_0, current%id, y_min, y_max, current%focus, x_min, x_max, nx_global, cpml_thickness, t_end, dx, dy, dt, TRIM(data_dir)//C_NULL_CHAR)
	  current => current%next
  ENDDO
  
END SUBROUTINE Maxwell_consistent_computation_of_EM_fields
\end{lstlisting}

\begin{lstlisting}[style=FORTRAN, caption=Fortran subroutines for populating laser sources on boundaries]
SUBROUTINE get_source_x_boundary(source1, source2, laser_id) 
  REAL(num), DIMENSION(:), INTENT(INOUT) :: source1, source2
  REAL(num), DIMENSION(ny) :: laser_ex, laser_ey
  INTEGER, INTENT(IN) :: laser_id
  INTEGER :: i
  CALL populate_laser_field_on_boundary(laser_ex, laser_id, TRIM(data_dir)//C_NULL_CHAR, "e_x"//C_NULL_CHAR, step, ny_global, ny_global_min, ny_global_max)
  laser_ey = 0.0_num
  DO i = 1, ny
	  source1(i) = source1(i) + laser_ex(i)
	  source2(i) = source2(i) + laser_ey(i)
  ENDDO
END SUBROUTINE get_source_x_boundary
  
SUBROUTINE get_source_y_boundary(source1, source2, laser_id)
  REAL(num), DIMENSION(:), INTENT(INOUT) :: source1, source2
  REAL(num), DIMENSION(nx) :: laser_ex, laser_ey
  INTEGER, INTENT(IN) :: laser_id
  INTEGER :: i
  CALL populate_laser_field_on_boundary(laser_ex, laser_id, TRIM(data_dir)//C_NULL_CHAR, "e_x"//C_NULL_CHAR, step, nx_global, nx_global_min, nx_global_max)
  laser_ey = 0.0_num
  DO i = 1, nx
	  source1(i) = source1(i) + laser_ey(i)
	  source2(i) = source2(i) + laser_ex(i)
  ENDDO
END SUBROUTINE get_source_y_boundary
\end{lstlisting}

\newpage
\section{Evaluation}

To evaluate the correctness of the algorithm presented in the previous section of this chapter as well as to demonstrate the drawbacks of the paraxial approximation, several test simulations in 2D geometry have been performed. In the following text, a two limit cases are presented. The first pair of simulations employs tightly focused Gaussian laser beam, with the size of the focus comparable with the center laser wavelength, whilst the second one shows the case of the Gaussian beam with the size of the focus one order of magnitude larger than the center laser wavelength, where both approaches should return identical results. Note, that all simulations have been computed using 2D version of particle-in-cell code EPOCH instrumented with library for tight focusing.

First, let us have a look at the simulation of a tightly focused Gaussian beam. The p-polarized laser pulse with center wavelength $ \lambda = 1 \: \mathrm{\mu m} $ propagates from left hand side boundary to the right. Its duration has been chosen to $ \tau = 20 \: \mathrm{fs} $ in FWHM and amplitude $ E_0 = 1 \cdot 10^{15} \: \mathrm{V/m} $. The beam width in focus $ w_0 = 0.7 \: \mathrm{\mu m} $ is shorter than the laser wavelength, which implies that non-negligible parts of $ \bar{\vec{E}}_{0, \bot}(k_x, \omega) $ are evanescent. The focus is located at a distance $ x = 8 \: \mathrm{\mu m} $ from the boundary that the laser is attached to.

The size of the simulation domain is $ 16 \lambda \times 16 \lambda $, with 100 cells per laser wavelength in both directions, thus $ \Delta x = \Delta y = \lambda/100 = 10 \: \mathrm{nm} $. The one simulation time step is according to CFL condition $ \Delta t = \sqrt{2} \lambda/ 100 c \approx 0.05 \: \mathrm{fs} $, the whole simulation time is then $ t = 150 \: \mathrm{fs} $. The pulse propagates in vacuum in order to get rid of all effects that could be potentially caused by plasma.

In the following paragraph, the results of the first simulation are discussed in a more detail. Fig. \ref{fig:1} shows transverse and longitudinal electric field components at their maximal intensity in the focus for both cases, laser beam propagating under the paraxial approximation (Fig. \ref{fig:1} - a, b) and according to the approach consistent with Maxwell equations (Fig. \ref{fig:1} - c, d). In the case of paraxial approximation, one can clearly see strong distortions and asymmetry in the shape of both electric field components. In addition, the focus location is shifted about $ 1 \: \mathrm{\mu m} $ closer to the left boundary and the corresponding amplitude in focus is less than half the required value. In contrast, the fields produced by the simulation using Maxwell consistent calculation of laser fields at boundary are symmetric with respect to the focus and without any distortions. Furthermore, the focus location as well as the amplitude fulfills the initial requirements precisely.

\floatsetup[figure]{style=plain, subcapbesideposition=top}
\begin{figure}[h!]
	\centering
	\sidesubfloat[]{{\includegraphics[width=0.44\linewidth]{./img/parax/Ey_focus.pdf}}}
	\hspace{2mm}
	\sidesubfloat[]{{\includegraphics[width=0.44\linewidth]{./img/parax/Ex_focus.pdf}}}\\
	\sidesubfloat[]{{\includegraphics[width=0.44\linewidth]{./img/lbcs/Ey_focus.pdf}}}
	\hspace{2mm}
	\sidesubfloat[]{{\includegraphics[width=0.44\linewidth]{./img/lbcs/Ex_focus.pdf}}}
	\caption{Transverse ($ E_{y} $) and longitudinal ($ E_{x} $) electric laser field components captured at the time step of their maximal intensity in the focal spot. The cases \textbf{(a)}, \textbf{(b)} correspond to the laser pulse propagating under the paraxial approximation, whilst \textbf{(c)}, \textbf{(d)} come from the simulation where the beam propagation has been resolved within the Maxwell consistent approach. In the case of paraxial approximation, both components reveal strong distortions and asymmetry, their focal spot is located about $ \mathrm{1 \mu m} $ closer to the left boundary than specified and the corresponding amplitude is significantly lower. The laser has been attached to the left hand side boundary.}
	\label{fig:1}
\end{figure}

Fig. \ref{fig:2} shows transverse and longitudinal slices of transverse electric field component in focus for the case of laser beam propagating under the paraxial approximation (Fig. \ref{fig:2} - a, b) as well as for the case where the beam propagation has been resolved within the Maxwell consistent approach (Fig. \ref{fig:2} - c, d). For the case of paraxial approximation, one can clearly see the asymmetry of the field shape in the longitudinal slice (Fig. \ref{fig:2} - b), which consequently leads to a decrease of the amplitude in focus and to the strong side-wings in the spatial beam profile, as might be seen from the transverse slice (Fig. \ref{fig:2} - a). On the other hand, Maxwell consistent approach calculates fields of perfect symmetry with respect to the focus (Fig. \ref{fig:2} - c) and no side-wings or distortions are present (Fig. \ref{fig:2} - d).

In Fig. \ref{fig:3} one can examine the time evolution of transverse (Fig. \ref{fig:3} - a) and longitudinal (Fig. \ref{fig:3} - b) electric field components at boundary as computed using Maxwell consistent approach. Note, that one have to chose carefully the transverse size of the domain, since the beam width at boundary may be much larger than in focus because of a diffraction.

\floatsetup[figure]{style=plain, subcapbesideposition=top}
\begin{figure}[h!]
	\centering
	\sidesubfloat[]{{\includegraphics[width=0.4\linewidth]{./img/parax/Ey_focus_trans.pdf}}}
	\hspace{2mm}
	\sidesubfloat[]{{\includegraphics[width=0.4\linewidth]{./img/parax/Ey_focus_long.pdf}}}\\
	\sidesubfloat[]{{\includegraphics[width=0.4\linewidth]{./img/lbcs/Ey_focus_trans.pdf}}}
	\hspace{2mm}
	\sidesubfloat[]{{\includegraphics[width=0.4\linewidth]{./img/lbcs/Ey_focus_long.pdf}}}
	\caption{Transverse \textbf{(a)}, \textbf{(c)} and longitudinal \textbf{(b)}, \textbf{(d)} slices of the transverse electric laser field ($ E_{y} $) at the time step when it reaches maximal intensity in the focal spot. The cases \textbf{(a)}, \textbf{(b)} correspond to the laser pulse propagating under the paraxial approximation, whilst \textbf{(c)}, \textbf{(d)} come from the simulation where the beam propagation has been resolved within the Maxwell consistent approach. In the case of paraxial approximation, one can clearly see strong side-wings in the spatial beam profile \textbf{(a)} as well as the asymmetry of the field in the longitudinal line-out \textbf{(b)}.}
	\label{fig:2}
\end{figure}

To evaluate a correctness of the beam propagation using Maxwell consistent approach, several criteria has been defined. The correctness of the amplitude and beam width at focus as well as the right focus location has already been verified. Additional criteria has been set on a beam symmetry. In Fig. \ref{fig:4}, one can find a comparison of the transverse electric laser field component when it achieves its maximal intensity at front and rear boundary. One can clearly see the exact match between the field shapes at different time steps of the simulation in transverse (Fig. \ref{fig:4} - a) and longitudinal (Fig. \ref{fig:4} - b) slices. Moreover, all the aforementioned criteria has been fulfilled also in other simulations with different input parameters that are not presented here. Consequently, these observations prove the correctness of the propagation at least of the tightly focused Gaussian laser beams.

For the second simulation, all the input parameters remained the same except the beam width in focus. Now, the parameter $ w_0 = 5 \: \mathrm{\mu m} $, which is about the limit case for the beam propagating under the paraxial approximation. Thus, one would expect that the simulation results will be almost identical.

\floatsetup[figure]{style=plain, subcapbesideposition=top}
\begin{figure}[h!]
	\centering
	\sidesubfloat[]{{\includegraphics[width=0.44\linewidth]{./img/lbcs/Ey_boundary_time.pdf}}}
	\hspace{2mm}
	\sidesubfloat[]{{\includegraphics[width=0.44\linewidth]{./img/lbcs/Ex_boundary_time.pdf}}}
	\caption{The time evolution of transverse ($ E_{y} $) \textbf{(a)} and longitudinal ($ E_{x} $) \textbf{(b)} electric laser field components at the boundary that the laser is attached to. Both components has been calculated according to the Maxwell consistent approach.}
	\label{fig:3}
\end{figure}

\floatsetup[figure]{style=plain, subcapbesideposition=top}
\begin{figure}[h!]
	\centering
	\sidesubfloat[]{{\includegraphics[width=0.44\linewidth]{./img/lbcs/Ey_boundary_trans.pdf}}}
	\hspace{1mm}
	\sidesubfloat[]{{\includegraphics[width=0.44\linewidth]{./img/lbcs/Ey_boundary_long.pdf}}}
	\caption{Transverse \textbf{(a)} and longitudinal \textbf{(b)} slice of the transverse electric laser field ($ E_{y} $) when it reaches its maximal intensity at the front (blue) and rear (red) boundary. The results come from the simulation where the Maxwell consistent approach for laser propagation has been used. For better comparison, the field at the rear boundary in \textbf{(b)} has been horizontally flipped. The exact match between the field shapes at a different time steps of simulation proves the correctness of the laser beam propagation.}
	\label{fig:4}
\end{figure}

Similarly as in Fig. \ref{fig:1}, Fig. \ref{fig:5} shows again transverse and longitudinal electric field components at their maximal intensity in the focus for both cases, laser beam propagating under the paraxial approximation (Fig. \ref{fig:5} - a, b) and according to the approach consistent with Maxwell equations (Fig. \ref{fig:5} - c, d). Here, one cannot register any difference between the results corresponding to both approaches.

\floatsetup[figure]{style=plain, subcapbesideposition=top}
\begin{figure}[h!]
	\centering
	\sidesubfloat[]{{\includegraphics[width=0.45\linewidth]{./img/parax/Ey_focus_5mic.pdf}}}
	\hspace{2mm}
	\sidesubfloat[]{{\includegraphics[width=0.44\linewidth]{./img/parax/Ex_focus_5mic.pdf}}}\\
	\sidesubfloat[]{{\includegraphics[width=0.44\linewidth]{./img/lbcs/Ey_focus_5mic.pdf}}}
	\hspace{2mm}
	\sidesubfloat[]{{\includegraphics[width=0.43\linewidth]{./img/lbcs/Ex_focus_5mic.pdf}}}
	\caption{Transverse ($ E_{y} $) and longitudinal ($ E_{x} $) electric laser field components captured at the time step of their maximal intensity in the focal spot. The cases \textbf{(a)}, \textbf{(b)} correspond to the laser pulse propagating under the paraxial approximation, whilst \textbf{(c)}, \textbf{(d)} come from the simulation where the beam propagation has been resolved within the Maxwell consistent approach. The size of the focus has been chosen to be one order of the magnitude larger than the center laser wavelength. One can clearly see, that there is no significant difference between the shapes of the electric field components.}
	\label{fig:5}
\end{figure}

Also, the transverse slice of the transverse electric laser field component in focus (Fig. \ref{fig:6} - a) shows the correct shape and amplitude for both cases. The longitudinal slice (Fig. \ref{fig:6} - b), however, points out the fact that the location of the focus is still little bit shifted closer to the left hand side boundary. Nevertheless, this difference could be in practice neglected. At the end of the day, for the Gaussian beams propagating under the paraxial approximation, the beam diameter at focus should be at least one order of magnitude larger than the center laser wavelength.

\floatsetup[figure]{style=plain, subcapbesideposition=top}
\begin{figure}[h!]
	\centering
	\sidesubfloat[]{{\includegraphics[width=0.44\linewidth]{./img/parax/Ey_focus_trans_comparison.pdf}}}
	\hspace{1mm}
	\sidesubfloat[]{{\includegraphics[width=0.45\linewidth]{./img/parax/Ey_focus_long_comparison.pdf}}}
	\caption{Transverse \textbf{(a)} and longitudinal \textbf{(b)} slices of the transverse electric laser field ($ E_{y} $) at the time step when it reaches maximal intensity in the focal spot. Red lines correspond to the laser pulse propagating under the paraxial approximation, whilst blue lines come from the simulation where the beam propagation has been resolved within the Maxwell consistent approach. The size of the focus has been chosen to be one order of magnitude larger than the center laser wavelength. In the case of paraxial approximation, the focus is slightly shifted closer to the left boundary \textbf{(b)}, otherwise the size of the focus as well as the amplitude is correct for both cases \textbf{(a)}.}
	\label{fig:6}
\end{figure}

In conclusion, one should be aware that propagation of tightly focused laser pulses cannot be described by paraxial approximation. Above, it has been shown that for the beams focused to a spot with the size comparable to a center laser wavelength, paraxial approximation leads to a shifted location of the focus, asymmetric laser field profiles with distortions and lower amplitude. These deviations are far from negligible and have without any doubt strong impact on the laser-matter interaction results. On the other hand, the propagation of tightly focused Gaussian laser beams prescribed at boundaries according to the Maxwell consistent approach has been proven to be correct.
