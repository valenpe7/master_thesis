There exist three different approaches to plasma physics. The particle theory, the kinetic theory and the hydrodynamic theory. Each approach has some advantages and limitations that stem from the simplified assumptions appropriate only for certain phenomena and time scales. Note that the coupling with the Maxwell's equations is usually straightforward, therefore the corresponding relations are not mentioned here. In the following three subsections, all the approaches are briefly discussed. 

The time evolution of the system containing charged particles in plasma is influenced by the electromagnetic fields formed due to the motion of particles as well as the external fields (e.g. laser). The particle approach is based on solving the equations of motion for all the particles in the system. In this case, the motion of particles is governed by the Newton's equations with the Lorentz force,
\begin{equation}
\label{2.2.5}
m_s \diff[2]{\vec{r}}{t} = q_s \left(\vec{E} + \vec{v} \times \vec{B} \, \right).
\end{equation}
Note that the equation \ref{2.2.5} holds only for non-relativistic particles. 

The particle theory is useful for tracking a single particle motion in prescribed electromagnetic field. Obviously, the general solution for the system of charged particles using this approach can be complicated since the plasma typically consist of a large number of such particles interacting with the self-consistent field. However, the full analysis for certain cases may be applied with the help of computing infrastructures and particle simulation codes.

%\subsection{Kinetic theory}
The plasma kinetic theory takes into account the motion of all charged particles in the system as well. However, the evolution of such system is not described by the exact motion of the particles, but only via certain average properties. The kinetic theory is based on a set of equations for the distribution function $ f_s \left(\vec{x}, \vec{v}, t \right) $ of particles of species $ s $ in plasma. The distribution function may be interpreted as a statistical description of a large number of interacting particles in the system \cite{nicholson}. If collisions can be neglected (for example in hot plasmas), the distribution function is governed by the collisionless Vlasov equation,
\begin{equation}
\label{2.2.1}
\diffp[]{f_{s}}{t} + \vec{v} \cdot \nabla f_s + \frac{q_{s}}{m_{s}}\left( \vec{E} + \vec{v} \times \vec{B} \right) \cdot \diffp[]{f_s}{\vec{v}} = 0.
\end{equation}
The Vlasov equation \ref{2.2.1} is obtained only by making the assumption that the particle density is conserved in the phase space, such that the time rate of change in a phase-space volume is equal to the flux of particles in or out of that volume \cite{nicholson}. Due to its simplicity, the equation \ref{2.2.1} is probably the most commonly used equation in the kinetic theory. However, the assumption to neglect collisions in a plasma is not valid generally. If it is necessary to take them into account, the collision term can be approximated using several methods \cite{Smirnov2008, nicholson, Chen1984, Drake2006, Boyd2003}.
 
%\subsection{Hydrodynamic theory}
In spite of the fact that the hydrodynamic theory is the roughest approximation for the description of plasma, it is sufficiently accurate to describe the majority of observed phenomena. The velocity distribution of each species is assumed to be Maxwellian everywhere, so the dependent variables are functions of only space and time coordinates \cite{Chen1984}. In other words, the fluid equations are the first three moments of the Vlasov equation (\ref{2.2.1}). These yield the following hydrodynamic equations for the density, momentum and the energy,
\begin{equation}
\label{2.2.2}
\diffp{n_s}{t} + \nabla \cdot \left(n_s \vec{u}_s \right) = 0,
\end{equation}
\begin{equation}
\label{2.2.3}
m_s n_s \left[ \diffp{\vec{u}_s}{t} + \left(\vec{u}_s \cdot \nabla \right) \vec{u}_s \right] + \nabla \cdot \mathbb{P}_s = q_s n_s \left(\vec{E} + \vec{u}_s \times \vec{B} \, \right),
\end{equation}
\begin{equation}
\label{2.2.4}
\diffp{}{t} \left(\frac{1}{2} n_s m_s u_s^2 + e_{s} \right) + \nabla \cdot \left(\frac{1}{2} n_s m_s u_s^2 \vec{u}_s + e_{s} \vec{u}_s + \mathbb{P}_s \vec{u}_s + \vec{Q}_s \right) = q_s n_s \vec{u}_s \cdot \vec{E}.
\end{equation}
The zeroth-order moment (\ref{2.2.2}) yields the continuity equation, where $ \vec{u}_s\left(\vec{x}, t \right) $ is the velocity of the fluid of species $ s $. This equation essentially states that the total number of particles is conserved. The first-order moment (\ref{2.2.3}) leads to a momentum equation. Here, $ \mathbb{P}_s\left(\vec{x}, t \right) $ is the pressure tensor. This comes about by separating the particle velocity into the fluid and a thermal component of velocity. The thermal velocity then leads to the pressure term. Finally, the second-order moment (\ref{2.2.4}) corresponds to the energy equation, where $ e_{s} $ is the density of the internal energy and $ \vec{Q}_s $ describes the heat flux density.

It might be clearly seen that the moment equations do not form a closed system. For the system of equations \ref{2.2.2} - \ref{2.2.4} to be complete, it has to be supplemented by the equation of state, which describes the relation between pressure and density in the plasma. However, the equations of state are well defined only in local thermodynamic equilibrium. Otherwise, the system cannot provide sufficiently exact description and thus the fluid equations \ref{2.2.2} - \ref{2.2.4} are not appropriate. Note that if the magnetic field is dominant in plasma, the set of hydrodynamic equations \ref{2.2.2} - \ref{2.2.4} should be replaced by the equations of magnetohydrodynamics \cite{Boyd2003}.