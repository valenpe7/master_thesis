The equation \ref{1.1} is Gauss's law of electrostatics in the differential form. 
The way in which charges and currents interact with the electromagnetic field is described by Lorentz force.
Electric and magnetic fields can be regarded as a forces produced by distribution of charge and currents
essential to electrodynamic is the speed of light in vacuum (universal constant), given by ...
Electromagnetic fields - forces produced by distribution of charge and currents - can exist in regions of space where there are no sources (they can carry energy, momentum , have existence totally independent of charge and currents) 

In this section, the properties of a Gaussian beam will be presented.
Propagation along the z axis and a stationary
focus at the origin of a Cartesian coordinate system will be
assumed.
Since in ... electromagnetic radiation is described by its propagation in the free space.
In this case E and B fields may be expressed by A alone. 
In general, the forms of laser beams can be usefully deduced from a vector potential that
has a single Cartesian coordinate.
Linearly polarized beams result from a vector potential with only Ax or Ay nonzero
Radially polarized beam results from having only Az nonzero
The solution of wave equation for vector potential A
Assume A is polarized in transverse direction 
The geometry of a focused, cylindrical beam is expressed in terms
of the three parameters (beam waist at focus w0, Rayleigh range - depth of focus zr, beam diffraction angle theta)
theory starts with the exact Maxwell equations and expands the electric field vector in powers of w0/l, where w0 and l are the scaling parameters for the beam waist and diffraction length, respectively.

For many purposes the above form is a good enough approximation

Paraxial approximation seems to be sufficient as long as one is interested in the region close to the beam axis and the focusing is not too tight.