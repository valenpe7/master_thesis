The energy of laser beam may be absorbed by resonance absorption in the plasma. It is a linear process in which an incident laser wave is partially absorbed by conversion into an electron wave at the critical density of plasma $ n_c (\omega) $.

Resonance absorption takes place when a linearly p-polarized laser pulse is obliquely incident on a plasma with an inhomogeneous density profile. A component of the electric laser field perpendicular to the target surface then resonantly excites an electron plasma wave also along the plasma density gradient, thus a part of the laser wave energy is transferred into the electrostatic energy of the electron plasma wave \cite{eliezer}. This wave propagates into the underdense plasma and is damped either by collision or collisionless absorption mechanisms. Consequently, energy is further converted into the thermal energy which heats the plasma and may possibly produce hot electrons.

Particularly, resonance absorption is the main absorption process for the laser beams of higher intensities and longer wavelengths. The efficiency of the resonance absorption may also be higher in plasma with high temperature, low critical density, or short scale length of the density profile.
