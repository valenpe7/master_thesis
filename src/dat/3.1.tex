Laser-plasma interaction involves a collective behavior of particles and electromagnetic fields which is in general complex and strongly non-linear problem to solve. The investigation of such systems cannot be usually carried out only through the theoretical and experimental work. Since a large amount of simultaneous processes with many degrees of freedom may be present there, analytical modeling seems to be impractical. On the other hand, many of the significant details of laser-plasma interaction may be extremely difficult or even impossible to obtain experimentally. Hence, other tools and techniques are required for the further progress in this field of research \cite{jaroszynsky}.

Numerical simulations help researchers to develop models covering a wide range of physical scenarios as well as to investigate their properties. With the advent of powerful computational systems, numerical simulations now play an important role as an essential tool in developing theoretical models and understanding experimental results in various fields of modern science \cite{pang}. These so-called computer experiments are often faster and much cheaper than a single real experiment in laboratory \cite{gould}. In addition, one numerical code can solve a broad range of tasks only by modification of its initial and boundary conditions.

Nowadays, it is clear that detailed understanding of the physical mechanisms in the laser-plasma interaction can be achieved only through the combination of theory, experiment and simulation. Development of parallel algorithms that are able to provide sufficiently exact solutions in reasonable time, however, belongs to the most challenging fields of modern science.

The PIC method refers to a technique that has been used to solve a certain class of partial differential equations. The method has been proposed in the mid-fifties and it has early gained a great popularity in the field of plasma physics \cite{birdsall, hockney, Verboncoeur2005}. It is based on the kinetic approach, thus the evolution of the system is described in principle via the motion of the charged particles.

However, real systems are often extremely large in terms of the number of particles they contain. In order to make simulations computationally tractable, so-called macro-particles are used. The macro-particle is a finite-size computational particle representing a group of physical particles that are near each other in the phase space. Notice that it is allowed to rescale the number of particles, because the Lorentz force depends only on the charge to mass ratio, which is invariant to this transformation. Thus, the macro-particle will follow the same trajectory as the corresponding real particles would \cite{hockney}.

Although this approach significantly reduces the number of computational particles in simulation, the binary interactions for every pair in the system cannot be taken into account. The cost would scale quadratically as the number of particles increases and that makes the computational effort unmanageable in the case of larger systems. Fortunately, many of the phenomena occur in high-temperature plasmas where the collective behavior dominates and the collisional effects are very weak (see \ref{2.1.6}), thus one can neglect them. Otherwise, one has to exploit other techniques to estimate the collisional effects \cite{lapenta}.

In the collisionless plasma, the phase space particle distribution function  $ f_{s} \left(\vec{r}, \vec{v}, t\right) $ for a given species $ s $ is governed by the Vlasov equation (\ref{2.2.1}). The mathematical formulation of the PIC method is obtained by assuming that the distribution function $ f_{s} \left(\vec{r}, \vec{v}, t\right) $ is given by the sum of distribution functions for macro-particles $ f_{p, s} \left(\vec{r}, \vec{v}, t\right) $,
\begin{equation}
\label{3.1.1}
f_{s} =  \sum_{p} f_{p, s}.
\end{equation}
Index $ p $ denotes hereafter the quantities attributable to macro-particles. The distribution function for each macro-particle is further assumed to be
\begin{equation}
\label{3.1.2}
f_{p, s}\left(\vec{r}, \vec{v}, t \right) = w_{p} S_{r}\left(\vec{r} - \vec{r}_{p}\left(t\right) \right)  S_{v}\left(\vec{v} - \vec{v}_{p}\left( t\right) \right),
\end{equation}
where $ w_{p} $ is a weight depending on the number of physical particles represented by each macro-particle, and $ S_{r}\left(\vec{r} - \vec{r}_{p}\left(t\right) \right) $, $ S_{v}\left(\vec{v} - \vec{v}_{p}\left(t\right) \right) $ are the so-called shape functions for the spatial and velocity coordinate, respectively.

\begin{figure}[h!]
	\centering
	\tikzstyle{empty} = [rectangle, fill=white, text width=12em, text badly centered, node distance=4em]
	\tikzstyle{block} = [rectangle, draw, thick, fill=white, text width=12em, text centered, rounded corners, minimum height=4.5em]
	\tikzstyle{line} = [draw, -triangle 45]
	\begin{tikzpicture}[node distance = 2cm, auto]
	
	\node [block] (nahore) {Particle pusher\\[2mm] $ \vec{F}_p \rightarrow \left(\vec{x}, \vec{v}\,\right)_{p} $};
	
	\node [empty, below of=nahore, node distance=2.5cm] (uprostred) {};
	
	\node [block, right of=uprostred, node distance=4.5cm] (vpravo) {Particle weighting\\[2mm] $ \left(\vec{x}, \vec{v}\right)_{p} \rightarrow \left(\rho, \vec{J}\,\right)_{i, j, k} $ };
	
	\node [block, left of=uprostred, node distance=4.5cm] (vlevo) {Field weighting \\[2mm] $ \left(\vec{E}, \vec{B}\,\right)_{i, j, k} \rightarrow \vec{F}_p $};
	
	\node [block, below of=uprostred, node distance=2.5cm] (dole) {Field solver\\[2mm] $ \left(\rho, \vec{J}\,\right)_{i, j, k} \rightarrow \left(\vec{E}, \vec{B}\,\right)_{i, j, k} $};
	
	\path [line] (dole) -| (vlevo);
	\path [line] (vlevo) |- (nahore);
	\path [line] (nahore) -| (vpravo);
	\path [line] (vpravo) |- (dole);
	\end{tikzpicture}
	\caption{Computational cycle of the particle-in-cell method}
	\label{3.1.4}
\end{figure}

The shape functions cannot be chosen arbitrarily. They have to fulfill several special properties. Let $ S_{\xi} $ be the shape function of the phase space coordinate $ \vec{\xi} $. Then:
\begin{enumerate}[nolistsep, topsep=5pt]
\item The support of the shape function is compact, $ \exists R > 0, \: \mathrm{supp} \: S_{\xi} \subset \left(-R, R\right) $.
\item Integral of the shape function is unitary, $ \int\limits_{-\infty}^{+\infty} S_{\xi}\left(\vec{\xi}\right)  \mathrm{d} \xi = 1 $.
\item The shape function is symmetrical, $ S_{\xi}\left(\vec{\xi}\right) = S_{\xi}\left(-\vec{\xi}\right) $.
\end{enumerate}

While these restrictive conditions still offer a wide range of options, the standard PIC method is essentially determined by the choice of the shape function in the velocity coordinate as a Dirac delta function and in the spatial coordinate as a m-th order b-spline basis function $ b_{m} $.
%\begin{equation}
%\label{3.1.3}
%S_{v}\left(\vec{v} - \vec{v}_{p}\left(t\right)\right) = \delta\left(\vec{v} - %\vec{v}_{p}\left(t\right)\right), \qquad S_{x}\left(\vec{x} - \vec{x}_{p}\left(t\right)\right)  = %b_{m}\left(\frac{\vec{x} - \vec{x}_{p}\left(t\right)}{\Delta p}\right),
%\end{equation}
%where $ \Delta p $ is the size of the support of the computational particles, typically the same as simulation grid cell. 
Note that the choice of the shape functions has strong impact on the stability and accuracy of the simulation. Higher-order basis functions result in less numerical noise and reduce non-physical phenomena in simulations, obviously at the cost of increased computational time.

Substituting the discretization \ref{3.1.1} into the Vlasov equation \ref{2.2.1} and taking into account the properties of shape functions mentioned above, one obtains the set of equations of motion for macro-particle $ p $ as the first moments,
\begin{equation}
\label{3.1.5}
\diff{\vec{r}_{p}}{t} = \vec{v}_{p}, \qquad \diff{\vec{p}_{p}}{t} = \vec{F}_{p},
\end{equation}
where $ \vec{p}_{p} $ is particle momentum and $ \vec{F}_p \left( t\right) $ represents a spatial average of the force acting on the macro-particle,
\begin{equation}
\label{3.1.6}
\vec{F}_p = q_s \left(\vec{E}_p + \vec{v}_p \times \vec{B}_p \right).
\end{equation}
The fields at the macro-particle position $ \vec{E}_p \left( t\right) $ and $ \vec{B}_p \left( t\right) $ in \ref{3.1.6} are given by the spatial shape function, which implies some form of interpolation. This will be closer discussed in the fourth section of this chapter.
%\begin{equation}
%\label{3.1.7}
%\vec{E}_p\left( t\right) = \int \vec{E}\left( \vec{r}, t\right) S_{r}\left(\vec{r} - %\vec{r}_{p}\left(t\right) \right) \mathrm{d} \vec{r}, \qquad \vec{B}_p\left( t\right) = \int %\vec{B}\left( \vec{r}, t\right) S_{r}\left(\vec{r} - \vec{r}_{p}\left(t\right) \right) \mathrm{d} %\vec{r}
%\end{equation}

The PIC method combines Lagrangian and Eulerian frame of reference. It means that the macro-particles are tracked in continuous phase space, whereas the electromagnetic fields are calculated only on stationary grid points. Therefore, it is necessary to perform the discretization of the spatial coordinates $ \vec{r} \rightarrow \vec{r}_{i j k} $ where $ (i,j,k) \in \mathbb{Z}^{3} $ are grid indices. It is also necessary to perform discretization of the temporal coordinate $ t \rightarrow t^{\,n} $, where $ n \in \mathbb{N} $ is the time level index. The algorithm will be outlined for a three-dimensional equidistant rectangular grid, thus $ \vec{r}_{i j k} = \left(i \Delta x, j \Delta y, k \Delta z\right) $, where $ \Delta x, \Delta y, \Delta z $ are the spatial steps in each direction and $ t^{\,n} = n\Delta t $, where $ \Delta t $ is the time step. Each quantity $ A \left(\vec{r}_{i j k}, t^{\,n} \right) $ will be hereafter denoted as $ A_{ijk}^{\,n} $.

The computational cycle of the PIC method is shown in Figure \ref{3.1.4}. Individual steps are closer described in several following sections. The influence of the choice of the time and spatial step on the stability and accuracy of the PIC method will be demonstrated as well.